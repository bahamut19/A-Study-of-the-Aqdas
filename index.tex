% Options for packages loaded elsewhere
\PassOptionsToPackage{unicode}{hyperref}
\PassOptionsToPackage{hyphens}{url}
%
\documentclass[
  11pt,
  letterpaper,
]{scrbook}

\usepackage{amsmath,amssymb}
\usepackage{iftex}
\ifPDFTeX
  \usepackage[T1]{fontenc}
  \usepackage[utf8]{inputenc}
  \usepackage{textcomp} % provide euro and other symbols
\else % if luatex or xetex
  \usepackage{unicode-math}
  \defaultfontfeatures{Scale=MatchLowercase}
  \defaultfontfeatures[\rmfamily]{Ligatures=TeX,Scale=1}
\fi
\usepackage{lmodern}
\ifPDFTeX\else  
    % xetex/luatex font selection
\fi
% Use upquote if available, for straight quotes in verbatim environments
\IfFileExists{upquote.sty}{\usepackage{upquote}}{}
\IfFileExists{microtype.sty}{% use microtype if available
  \usepackage[]{microtype}
  \UseMicrotypeSet[protrusion]{basicmath} % disable protrusion for tt fonts
}{}
\makeatletter
\@ifundefined{KOMAClassName}{% if non-KOMA class
  \IfFileExists{parskip.sty}{%
    \usepackage{parskip}
  }{% else
    \setlength{\parindent}{0pt}
    \setlength{\parskip}{6pt plus 2pt minus 1pt}}
}{% if KOMA class
  \KOMAoptions{parskip=half}}
\makeatother
\usepackage{xcolor}
\ifLuaTeX
  \usepackage{luacolor}
  \usepackage[soul]{lua-ul}
\else
  \usepackage{soul}
  
\fi
\setlength{\emergencystretch}{3em} % prevent overfull lines
\setcounter{secnumdepth}{6}
% Make \paragraph and \subparagraph free-standing
\ifx\paragraph\undefined\else
  \let\oldparagraph\paragraph
  \renewcommand{\paragraph}[1]{\oldparagraph{#1}\mbox{}}
\fi
\ifx\subparagraph\undefined\else
  \let\oldsubparagraph\subparagraph
  \renewcommand{\subparagraph}[1]{\oldsubparagraph{#1}\mbox{}}
\fi


\providecommand{\tightlist}{%
  \setlength{\itemsep}{0pt}\setlength{\parskip}{0pt}}\usepackage{longtable,booktabs,array}
\usepackage{calc} % for calculating minipage widths
% Correct order of tables after \paragraph or \subparagraph
\usepackage{etoolbox}
\makeatletter
\patchcmd\longtable{\par}{\if@noskipsec\mbox{}\fi\par}{}{}
\makeatother
% Allow footnotes in longtable head/foot
\IfFileExists{footnotehyper.sty}{\usepackage{footnotehyper}}{\usepackage{footnote}}
\makesavenoteenv{longtable}
\usepackage{graphicx}
\makeatletter
\def\maxwidth{\ifdim\Gin@nat@width>\linewidth\linewidth\else\Gin@nat@width\fi}
\def\maxheight{\ifdim\Gin@nat@height>\textheight\textheight\else\Gin@nat@height\fi}
\makeatother
% Scale images if necessary, so that they will not overflow the page
% margins by default, and it is still possible to overwrite the defaults
% using explicit options in \includegraphics[width, height, ...]{}
\setkeys{Gin}{width=\maxwidth,height=\maxheight,keepaspectratio}
% Set default figure placement to htbp
\makeatletter
\def\fps@figure{htbp}
\makeatother

\makeatletter
\@ifpackageloaded{bookmark}{}{\usepackage{bookmark}}
\makeatother
\makeatletter
\@ifpackageloaded{caption}{}{\usepackage{caption}}
\AtBeginDocument{%
\ifdefined\contentsname
  \renewcommand*\contentsname{Table of contents}
\else
  \newcommand\contentsname{Table of contents}
\fi
\ifdefined\listfigurename
  \renewcommand*\listfigurename{List of Figures}
\else
  \newcommand\listfigurename{List of Figures}
\fi
\ifdefined\listtablename
  \renewcommand*\listtablename{List of Tables}
\else
  \newcommand\listtablename{List of Tables}
\fi
\ifdefined\figurename
  \renewcommand*\figurename{Figure}
\else
  \newcommand\figurename{Figure}
\fi
\ifdefined\tablename
  \renewcommand*\tablename{Table}
\else
  \newcommand\tablename{Table}
\fi
}
\@ifpackageloaded{float}{}{\usepackage{float}}
\floatstyle{ruled}
\@ifundefined{c@chapter}{\newfloat{codelisting}{h}{lop}}{\newfloat{codelisting}{h}{lop}[chapter]}
\floatname{codelisting}{Listing}
\newcommand*\listoflistings{\listof{codelisting}{List of Listings}}
\makeatother
\makeatletter
\makeatother
\makeatletter
\@ifpackageloaded{caption}{}{\usepackage{caption}}
\@ifpackageloaded{subcaption}{}{\usepackage{subcaption}}
\makeatother
\ifLuaTeX
  \usepackage{selnolig}  % disable illegal ligatures
\fi
\usepackage{bookmark}

\IfFileExists{xurl.sty}{\usepackage{xurl}}{} % add URL line breaks if available
\urlstyle{same} % disable monospaced font for URLs
\hypersetup{
  pdftitle={Achieving True Liberation},
  pdfauthor={Joshua W. Adams},
  hidelinks,
  pdfcreator={LaTeX via pandoc}}

\title{Achieving True Liberation}
\usepackage{etoolbox}
\makeatletter
\providecommand{\subtitle}[1]{% add subtitle to \maketitle
  \apptocmd{\@title}{\par {\large #1 \par}}{}{}
}
\makeatother
\subtitle{An Exploration of the Kitab-i-Aqdas}
\author{Joshua W. Adams}
\date{2026-02-08}

\begin{document}
\frontmatter
\maketitle

\renewcommand*\contentsname{Table of contents}
{
\setcounter{tocdepth}{0}
\tableofcontents
}
\mainmatter
\bookmarksetup{startatroot}

\chapter*{Cover}\label{sec-cover}
\addcontentsline{toc}{chapter}{Cover}

\markboth{Cover}{Cover}

\begin{figure}[H]

{\centering \includegraphics{images/Achieving True Freedom Cover.png}

}

\caption{Generated Using GPT5.2}

\end{figure}%

\newpage{}

\bookmarksetup{startatroot}

\chapter*{Introduction}\label{sec-introduction}
\addcontentsline{toc}{chapter}{Introduction}

\markboth{Introduction}{Introduction}

Praise be to God, the Creator of and Source of all inspiration,
knowledge, and wisdom!

Allah'u'abha!

The Báb and Bahá'u'lláh, from 1844 to 1892, ushered in a new era for all
mankind. The culmination of their revelations, the Bayán by the Báb in
1848 and the Kitáb-i-Aqdas by Bahá'u'lláh in 1873, offer a framework to
develop individuals, communities, and nations in a newly spiritual way.
The Bayán, which was unfinished prior to the Báb's execution in 1850,
was symbolically completed by the Kitáb-i-Aqdas. This book began with a
desire to study the Kitáb-i-Aqdas and to understand what it is truly
offering humanity, not only in isolated passages, but as a complete and
integrated work.

When I first began writing, I had a narrower vision of what the book
would become. I expected to remain close to the surface of the text,
speaking about themes and laws in a straightforward way. As the chapters
unfolded, the Aqdas drew me further inward and further outward at the
same time. It drew me inward into belief, prayer, and the reality of the
soul. It drew me outward into relationships, community life, leadership,
and peace. Writing this book became a process of discovery, where the
Kitáb-i-Aqdas increasingly revealed itself as a composition of spiritual
life, rather than a list of isolated commands.

This book aims to place the Kitáb-i-Aqdas in the exalted position it
deserves to be. To do this, we need to understand the Kitáb-i-Aqdas is
more than a book of laws. It is a framework for God's will, which is to
liberate souls, to create distinct spiritual communities, and to bring
about unity among the people of the world. It does not speak only to
what we should do, but to who we are, what we can become, and what kind
of world can emerge when human beings learn to live in justice and
mercy. It is not meant to crush the soul beneath obligation, but to
awaken the soul into dignity, purpose, and spiritual maturity.

The Bayán will be used as a primary supplement, as well as other
writings by Bahá'u'lláh. No other sources will be considered. This is
not because other voices have no value, but because I wanted the
Scriptures to speak with their own force and internal coherence. It is a
technique which is rarely used, to use Scripture to explain Scripture.
The Kitáb-i-Aqdas makes no claim of starting completely from scratch, to
not have any prior origins or inspirations. It also does not deny the
reality of the world in which it was created. The world was already deep
into the Industrial Revolution and on the cusp of the Modern age.
Societies were progressing in certain qualities more than others, and
each society carried strengths and weaknesses.

What the Kitáb-i-Aqdas does is act like a composer of music, taking
instruments, notes and chords which already exist, and using them to
create an entirely new composition. This composition is unique in
itself, but it should not be terribly strange to anyone. When listened
to in its entirety, you may feel its vibrations within your soul, a soul
which does yearn to have a relationship with the divine. This
composition is itself a technique able to create more heavenly
compositions, where the compositions are the various organizations of
society. It refines what already exists so that mankind can be prepared,
guided, and inspired to continue its progress into perpetual tomorrow, a
tomorrow of hope, love, and if so desired, peace.

I make no claim of expertise, divinity, or scholarly prowess. I make no
claim where the contents of this book is the only way to interpret the
Kitáb-i-Aqdas. My perspective is but one perspective and my hope is more
authors offer their own perspectives on this magnificent work. The
Kitáb-i-Aqdas is worthy of scholarship and a place on every bookshelf,
coffeetable, and handbag.

This book is divided into four parts. The first part is the pillars of
belief and serves as the foundation for the rest of the book. The second
part focuses on the relationship we are capable of having with God along
with the virtues this relationship develops. The third part starts to
explore the practical application of the Kitáb-i-Aqdas on our private
lives and relationships, outlining a set of rights and responsibilities
from birth to maturity. The fourth part explores leadership in the civil
and faith-based spheres, culminating in a vision of the future we can
all be a part of.

Different readers will approach this book with different needs. Some may
be seeking clarity of belief. Some may be seeking a devotional path that
is practical and sustaining. Some may be recovering from deep emotional
wounds and need a gentle way to return to God without fear. Some may be
looking for a way to teach others without coercion, and to support
growth without control. Some may be drawn to the chapters on leadership
and peace, because they sense that the future of mankind depends on
spiritual principles being applied with wisdom. The book is written to
hold all of these paths together, because the Kitáb-i-Aqdas itself holds
them together. Inner life and outer life cannot be separated for long
without injury to one or the other.

Whether you currently believe in Bahá'u'lláh or not, I feel the
Kitáb-i-Aqdas has something to offer for everyone. Liberation is for
everyone, to include you and your loved ones. I am grateful you have
given this book a chance to be read. Thank you!

\newpage{}

\part{Part 1: The Pillars of Belief}

\chapter{Belief in God}\label{belief-in-god}

\section{Chapter Introduction}\label{chapter-introduction}

The Kitáb-i-Aqdas never explicitly says you must believe in God, but the
mention of God is everywhere within it. Belief in God is foundational to
everything else in the Kitáb-i-Aqdas. The book is rarely considered a
mystical treatise, nor is it often recommended as a text for learning
how to grow your spirit. In reality, the Kitáb-i-Aqdas is written for
the growth of the human spirit through a constant interweaving of
mystical teaching, theology, and straightforward commands. One way
Bahá'u'lláh enables this journey is by ensuring God is present in every
single command and teaching. He does this by calling on God's various
names and attributes. I counted 110 different names used for God within
the Kitáb-i-Aqdas. This is not every name of God, and of course names
can exist within every language. Both the Báb and Bahá'u'lláh teach that
God is inclusive of all names and is the source of all names. Rather
than studying all 110 names one by one, the next section groups these
names and attributes into 19 groups. I chose 19 due to the significance
of the number in representing unity. These 19 groups begin with one
concept and then progress to the next. Each group name also includes a
number showing how often the names in that group appear in the
Kitáb-i-Aqdas.

\section{19 Groups of Names and
Attributes}\label{groups-of-names-and-attributes}

\subsection{Subtlety (2)}\label{subtlety-2}

Subtlety is the quality of being difficult to analyze or even detect.
When combined with the name Inaccessible, we can start to imagine God's
mysterious and hidden nature. God may seem unable to be directly
observed, or to some, lacking in direct evidence or proof. Yet these
qualities are highly nuanced. Subtlety can be viewed as an interweaving
of words, meanings, ideas, and feelings which language may not be able
to fully convey, or whose conveyance can be completely relative to the
contexts of time, space, and perspective. In one moment you may have an
understanding, and in the next, you gain a different understanding.

There are hundreds of traditions which attempt to explain who God is
through art, music, dance, and poetic stories. They are filled with
metaphor, exaggerated expressions, and other methods in an attempt to
express the mystery which is God. Each is an expression of the subtlety
of God, which, taken together, are a divine tapestry whose threads are
interwoven in truth and a shared experience along a spiritual journey.
This also means there is more than one way to experience God. The
Kitáb-i-Aqdas provides the framework to open up these experiences, to
find the different pathways to explore the hidden nature of God. All one
must do is seek, to be a seeker.

The next 18 groups include the quality of Subtlety.

\subsection{Manifestation (2)}\label{manifestation-2}

One way God assists the seeker is by manifesting signs that hint at His
presence. The purpose is to make the inaccessible more accessible, even
if it isn't accessibility in God's entirety. These signs might not be
obvious like a stop sign, but could be more like being nearsighted as
you approach a highway sign pointing to multiple paths. It may be
something not immediately clear, or if you are looking in a different
direction, the sign is something you may not observe. Signs are full of
subtlety.

These signs could come from the natural world and the laws of creation
in which we are a part of. They could come from others like ourselves
through their actions, thoughts, and feelings. They could even come from
our own imagination or subconscious, such as with a dream. A seeker may
not observe every sign. When the seeker observes a sign, they may not
understand the significance of the sign completely. A person who is not
seeking may not even be aware there are signs. Yet, they are there for
everyone with great consistency.

God can be made Manifest in different forms, including through humanity.
A human Manifestation of God is a reflection and embodiment of God
Himself. This subject will go into deeper detail regarding
Manifestations of God in Chapter 6, ``Belief in the Manifestation of
God.''

The next 17 groups include the quality of Manifestation.

\subsection{Knowledge (46)}\label{knowledge-46}

The Kitáb-i-Aqdas uses the names such as All-Knowing, All-Informed,
Wise, and the Knower of All Things frequently. These words convey God's
own infinite knowledge, a knowledge that, when manifested, is full of
Subtlety. Sometimes the knowledge may seem inaccessible or difficult to
understand. This is an acceptable perspective. Sometimes knowledge is
entirely based on perspective, especially within space and time. What a
person knows in one moment as true can change as facts change,
conditions change, or new knowledge becomes known. What is true within
the Kitáb-i-Aqdas is God possesses all knowledge.

Sometimes this knowledge is viewed as a knowledge of all facts. This
could be knowledge of what is within our hearts. It could be knowledge
of all possible outcomes. This could be a knowledge of all which has
been manifested. This could be a knowledge of all which could be
manifested. This knowledge is derived from the Manifestations of the
Signs.

As names such as All-Knowing exist with Subtlety, these infinite modes
of potential clarity and understanding are interwoven within each other,
completely nuanced, and are all pathways to the name Manifest. This
knowledge leads to God being the Wise, knowing how to apply knowledge in
the best way possible.

The next 16 groups include the quality of Knowledge.

\subsection{Creation (6)}\label{creation-6}

God's names include the Dawning Place of Signs and the Creator of the
Earth and Heavens. Creation is manifested through God's Knowledge. God's
role as Creator begins to become evident, showing how His knowledge
shapes and sustains the universe. ``Every beginning is from God and unto
Him it returns. He is indeed the source of creation and the ultimate
destination of all the worlds.''\footnote{Kitáb-i-Aqdas \#144}

With God being the Creator, that also demonstrates God's eternal
preexistence. God is the Creator, not the created. God is the Cause and
Creation is the Effect. Through Knowledge, God is able to express His
purpose, and through Creation He is able to fulfill this purpose. As God
is the Creator of the Earth and Heavens, He is the Creator of all we are
able to observe and all which we are unable to observe. God is the
Creator of the Seen and the Unseen. God is the creator of all we feel
and experience. God is the creator of all purpose, of all action, of all
energy, and of all matter. God is the creator of you and I, and all
which we care about.

Creation demonstrates Subtlety in that even seemingly simple things,
such as a human cell, are complex and quite mysterious. The How and the
Why a cell originates and functions demonstrate the subtle craftsmanship
of a Creator, whereupon so far the answers to why remain quite
inaccessible. All of Creation maintains this Subtlety.

The next 15 groups include the quality of Creation.

\subsection{Power (36)}\label{power-36}

The Power group includes the second most frequent of names in the
Kitáb-i-Aqdas. These names include the Almighty, All-Powerful,
All-Capable, All-Subduing, and All-Sufficient. Each of these names
express a different mode or aspect of God as Power. With God being the
Creator, an awareness of God's Power and omnipotence follows.

Might can be synonymous with strength, especially with the traditional
notions of being able to exert a great amount of force. Power can be
viewed in how strength can be projected, or the ability to exert
influence. Being capable is the ability to get things done, especially
through skill. Subduing is being able to express one's Will over
another. Being All-Sufficient means being qualified and competent in
using power to achieve the required goals and to fulfill purpose.

These names all demonstrate God's ability to govern and sustain all
which has been created. Notice how Power comes after other attributes
such as being All-Knowing. Knowledge leads to Power. Power itself does
not lead to being All-Knowing. This shows, in a way, how being Almighty
or All-Subduing can be quite nuanced by Subtlety. Skill in using might
and power involves knowing what tools to use, or exactly the right
amount of energy to achieve the purpose within creation. The use of
power might be a challenge to understand, as is understanding God's
Will, which is a Will that subdues all of creation. The Will of God
could be quite difficult to comprehend or fathom, also being a part of
the greater mystery of God.

There is hidden strength, such as how fine-tuned the laws of physics are
or how life seemingly came from inert matter. Power does not need to be
coercive but instead influential, such as in the role to guide hearts or
establish those natural laws. The rotation of the planets or the growth
of a seed show how God can invisibly utilize power. There doesn't need
to be extravagant, one-time shows of might for power to be manifested.
Because God manifests the Signs, power does not need to be a show of
domination, but a gentle way to empower creation to flourish. Dramatic
interventions are always possible with God if it serves a purpose, and
whether they are dramatic or subtle, the power is always transformative.
``He can seize the world with a single word of His
presence.''\footnote{Kitáb-i-Aqdas \#132}

The next 14 groups include the quality of Power.

\subsection{Lordship (51)}\label{lordship-51}

The Kitáb-i-Aqdas includes many different ways in describing the various
ways God is Lord. The title of Lord is used to describe concepts such as
the Lord of Beauty, of all Religion, of Grandeur, of the Seen and
Unseen. Lordship can be defined as God's ultimate authority. This
authority would be over all creation and is derived from God's knowledge
and power. Lordship can discuss authority and governance. This authority
is absolute, encompassing everything which is seen and unseen. Lordship
is universal, transcending all divisions, stations, ranks, and beliefs.

God is the ruler over all creation, although as God is Subtlety,
Lordship can also be a mystery. Lordship might be observed through the
Manifestation of Signs, understood through Knowledge, witnessed through
Creation, and felt through Power.

The next 13 groups include the quality of Lordship.

\subsection{Justice (6)}\label{justice-6}

Justice is expressed through the names Judge, Just, and Reckoner.
Justice can often be seen as the determination between right and wrong.
It can also be viewed as the balance expressed through God's Lordship.
Although justice can often be immediately evident, Subtlety allows for
justice to be precise, operating beyond the surface. This includes
realities which can seem hidden, such as intentions, unseen
consequences, or other spiritual conditions.

Justice can be witnessed through the various signs manifested,
especially in how perfectly ordered creation is. Justice requires the
omniscience of God to truly be balanced, informed of every action,
intention, and consequence. Justice is the fulfillment of the purpose of
creation in an equitable way. Justice requires God's power to be upheld.
Power without justice can become tyranny. Justice ensures power is fair
and equitable. All of creation benefits from justice.

God as an All-Capable Reckoner is able to calculate and account for all
actions, intentions, and consequences, to include those seemingly
unknown. God as the All-Knowing Judge determines the best rulings
according to divine principles. Judgment and reckoning are required for
justice to be implemented. As Lord, God's justice becomes evident as
does order within His governance.

The next 12 groups include the quality of Justice.

\subsection{Exaltation (9)}\label{exaltation-9}

God's justice demonstrates God's exalted nature. The names of Most
Exalted, Most High, Great, Majestic, Most Glorious, and the Greatest
Infallibility come from God's role as Just. Without justice, God's
Power, Lordship, and Knowledge would be unworthy of exaltation. However,
because of God's Justice, nothing else in creation can be more exalted
than God.

Given God's Knowledge and Power, the Name of Greatest Infallibility
demonstrates the fairness and righteousness of justice, showing that God
does not and will not error. God's Subtlety can make His Exaltation seem
veiled, especially when compared to more superficial things we can place
above ourselves, or when we place ourselves above everything. Subtlety
is what creates awe and acknowledges the extraordinary nature, despite
being incomprehensible. The Manifestation of the Signs and of God help
reveal different aspects of God's exalted nature.

The belief in God's Exaltation is not because God feels jealousy if you
do not, nor needs to be exalted to feel good. The belief in God's
Exaltation is a function of God's Lordship and Justice above any other
form of justice and authority.

The next 11 groups include the quality of Exaltation.

\subsection{Independence (9)}\label{independence-9}

The names of God in the group Independence include Self-Sufficient,
Self-Subsisting, and Independent of All Worlds. It should be noted the
Power group utilized the Name All-Sufficient. All-Sufficient
demonstrates how God can suffice all things, whereas Self-Sufficient
shows how God can suffice Himself without requiring any support from
creation itself.

God has complete autonomy. God is independent of our own belief or
exaltation. It also means God does not need anything in return in being
Just, Almighty, All-Knowing, and Inaccessible. God acts only due to His
own purpose. This purpose, this Cause, is solely for creation's benefit.
``Evil deeds do not harm Us, good deeds do not benefit Us''\footnote{Kitáb-i-Aqdas
  \#59} and due to this, God is free of corruption.\footnote{Kitáb-i-Aqdas
  \#64}

The next 10 groups include the quality of Independence.

\subsection{Command (6)}\label{command-6}

The Kitáb-i-Aqdas describes God as being the Commander, Ordainer, and
Fulfiller. These names serve as the foundation for legislation and the
execution of this legislation. These roles in one way serves as the Word
through which the entire framework of order, laws, directives, and
intentions are used to shape the destiny of creation.

It would seem justice would require the command in practice. God cannot
judge without establishing a standard to judge against. This would
definitely be correct. With Commander being a Name of God, it is more
about the attribute. In order for belief in God's command to exist, one
must believe in the attribute of Justice first. Otherwise, a person
would not follow the command.

Subtlety is still the foundation here within Command. Even a simple
outward command could have significant and profound spiritual meaning.
Believing in the Command thus, isn't just merely acting on the command,
but believing the Command itself has an inner essence full of mystery,
spiritual progress, and deeper purpose within the grand scope of
creation. The Command is bound within God's All-Knowing nature and
serves as the framework not just for people, but for all creation.
Command acts through God's Lordship and serves the balance of Justice.
The Command is Exalted over any other command and is independent of
everything other than God Himself.

The next 9 groups include the quality of Command.

\subsection{Counsel (6)}\label{counsel-6}

The group counsel is informed by the names Counselor, Speaker, and
Source of Inspiration. Unlike Command which is more about legislation,
Counsel is more about helping navigate complex moral, spiritual, and
practical challenges which may not have a clear prohibition. This
complements Command.

Belief in God as a Counselor means you believe God can provide you with
advice and wisdom. This originates from God being Wise and All-Informed.
God can do this as the Speaker, using His Word or Manifestations of His
Word to provide guidance at anytime. This guidance could be sought or
offered voluntarily. These counsels, advice, and wisdom act within God's
name Source of Inspiration. The inspirations inspired could be inner
understandings or renewed energies to act.

Counsel interacts with Subtlety by allowing for a process of
self-discovery through the hidden, inner dimensions of truth which may
Manifest through Signs and Manifestations of God. While inspiration
itself is a creation, effective counsel can help ensure a person may
find harmony and balance within the framework of creation itself.
Counsel could be highly contextual and individualized, which requires
God's knowledge to ensure relevance and timeliness. God's counsel is
authoritative as Lord and is another pathway to justice, as His counsel
will be most fair and equitable. The Exalted nature of God ensures a
sense of awe and respect for His wisdom. The counsel is effective
through God's independence, ensuring the counsel is solely for the
benefit of creation, not from any need or want.

The next 8 groups include the quality Counsel.

\subsection{Faithfulness (3)}\label{faithfulness-3}

Faithfulness is a name or attribute rarely attributed to God, or at
least in what I read and hear. Usually faithfulness is used only in the
context of a person in relation to God or a significant loved one. This
group is highlighted by the names in the Kitáb-i-Aqdas of Trustworthy
and Faithful Guardian.

Belief in God as being Trustworthy is the result of believing God's
counsel and command. You can trust God's command and counsel is not only
wise, but upheld. They are not arbitrary or motivated in negative
qualities such as selfishness. God can be relied upon in good times and
in bad times to exist within all the names expressed. God upholds His
promises and fulfills His roles as Lord and Creator justly.

God is the Faithful Guardian. Being faithful, God is unwavering in His
promises and in upholding justice. He is loyal to creation and being
independent, is not affected by human actions or decisions. God being a
guardian demonstrates an active role within creation. While God being
Inaccessible may seem God is impersonal, this is quite far from the
truth. Inaccessible is solely due to His exalted nature, not due to a
lack of care or concern for creation. God is our guardian. Creation
could face challenges, and God is able to protect it and guide it,
ensuring it flourishes. Within humanity, God is like a shepherd.

Both these names are a testament to the faithfulness of God, especially
to each individual person although not exclusive to humanity. There is a
steadfast commitment to our guidance, protection, and well-being. This
should help encourage a sense of hope, trust, and gratitude in our own
walks of faith.

The next 7 groups include the quality Faithfulness.

\subsection{Praise (4)}\label{praise-4}

God's names in this group include All-Praised, Praised, and
Praiseworthy. This demonstrates how created entities, not just humanity,
praise God and His names. God is worthy of adoration due to how all of
these prior names are manifested within creation.

Praise can be for God's Subtlety, awe, and wonderment. The hidden
aspects of God's nature invite reflection and admiration. Praise can be
for God's manifestations, helping to connect humanity to the divine.
Praise can be for God's omniscience, for understanding all things
perfectly for the benefit of creation. Praise can be for creation
itself, for its beauty, complexity, artistry, and magnificence. Praise
can be for God's power and Lordship, for the ability to govern, sustain,
and transform creation into constantly better versions. Praise can be
for God's purpose and justice, for the assurance there is fairness and
moral accountability. God deserves praise for being the Most High and
Most Glorious, showing a deep reverence for His supreme nature. God can
be praised for needing nothing but giving everything selflessly, being
fully independent of creation. Praise is for God's command, providing a
moral and spiritual framework for all. Praise for laws which promote
growth, harmony, justice, and peace. Praise can be for God's counsel
which ensures humanity can navigate life's challenges in a wise and
caring way. Praise is for God's faithfulness in His consistent care and
fulfillment of promises, for being steadfast in being an active part in
developing our spiritual and material needs.

The next 6 groups include the quality of Praise.

\subsection{Love (11)}\label{love-11}

Love is probably the most often stated attribute of God. It would seem
natural to place the Love grouping much earlier in this. Bahá'u'lláh
says ``I was in my ancient essence and eternal being when I knew my love
for you; therefore, I created you, cast my image upon you, and revealed
my beauty to you.''\footnote{The Hidden Words in Arabic \#3} It appears
love was the beginning, but the ancient essence of Subtlety is the
beginning.

Believing in God the Subtlety can teach God's presence is real, even
when unseen. This is a foundation to prepare a believer to recognize
love even in those troublesome and mysterious times. Love itself often
seems hidden, which requires God to Manifest it. The Manifestation is
the expression of love, which appears in signs, prophets, and
scriptures. Manifestations help make love seem tangible instead of being
inaccessible. As we see in the Hidden Word, God's love is intentional,
born through God's knowledge. Love is purposeful and directed towards
the well-being of all creation. Creation teaches our existence is an
expression of love. Everything is the gift. Love is born from Power, as
it is protective and able to sustain creation. Love without power is
ineffective. Love has authority derived from God's Lordship, expressed
with authority. God's love can lead one wisely. If Love wasn't a result
of Justice, it would not be perfectly balanced. God does not favor in
Love, nor is inequitable.

Discipline is not out of anger or jealousy, but is a method to guide and
protect those He loves. The belief in God's Exaltation also leads to
feeling God's Love is also Exaltation. It is higher than human love and
is limitless. Love is not bound by any limitation and it is
unconditional and constant. Independence teaches God loves not because
God needs something from us, but because the Love is True. God is not
dependent on our love, which means His Love is completely free. The
Command of God demonstrates how Love provides direction through laws,
teachings, and boundaries. Love is not merely only a feeling but God's
Love includes guidance. God's Love is not silent and He can speak and
inspire, providing loving counsel. Faithfulness teaches Love is reliable
and any unreliable love is not love. Love is eternal, just as God is
eternal. God will not abandon or forget His creation.

Once recognizing God's greatness through these various names and
attributes, Praise teaches God's greatness naturally leads to Love.
Feeling and understanding how wonderful God is, in all of these
attributes, there is no other result to also love Him. Love itself is
the highest form of praise. As an example of God's Love, His names
include Beloved, Loving, and Gracious.

Grace is an attribute which encapsulates God's Love succinctly. It is
personal and relational. When we started with God as being inaccessible,
God's love is deeply personal. ``I knew my Love for You.'' Grace is
intimate. Grace is the foundation of spiritual aid. Grace is Love in
action. Grace is given even if it is not deserved. Every person, being
created out of Love, is a recipient of Grace.

The next 5 groups include the quality of Love.

\subsection{Forgiveness (15)}\label{forgiveness-15}

Forgiveness is one of the ultimate expressions of love. With the names
of God including All-Forgiving, Ever-Forgiving, and Oft-Forgiving, we
see forgiveness can include forgiving all transgressions, lasting
eternally, and can occur with great frequency. There are no limitations
on what can be forgiven for God and any forgiveness lasts forever. What
is forgiven today will not be held against you tomorrow.

Believing in the Forgiveness of God does require believing in the
qualities of God's Love. Once we know love is not transactional and
completely selfless, we can also believe forgiveness is all not
transactional. Forgiveness shows unconditional care and devotion. The
mistakes and failings of a person will not separate them from God. God,
being the embodiment of Faithfulness, is steadfast in His Love and
Forgiveness. No person is unworthy of forgiveness. Belief in God's
forgiveness also helps strengthen and secure one's belief in His love.
It helps feel the love is constantly being expressed and renewed, whose
purpose is to uplift creation. God's command and counsel is not about
creating a fear of punishment, but it is to deepen the relationship we
have with God.

The next 4 groups include the quality of Forgiveness.

\subsection{Mercy (29)}\label{mercy-29}

Mercy builds on forgiveness and is still sourced from God's Love and His
other attributes. Mercy includes the names All-Merciful, Most Merciful,
and Most Compassionate. Forgiveness is a reaction to something gone
wrong such as a mistake. Mercy is a proactive support for the future.
Believing God is All-Merciful means you believe God knows, through His
Knowledge, you will need help and assistance. You will be imperfect.
Mercy is gentleness, patience, and protection. Mercy will help ensure we
are guided away from that which harms us.

God being Most Compassionate shows God has sympathy and compassion for
all of His creation, to include us. Mercy is the action used to express
this compassion. For example, a parent can see their child wanting to
learn how to walk. They see the difficulty the child has and compassion
moves the parent to feel for the struggle. These feelings compel the
parent to want to comfort and encourage the child. Mercy is helping the
child up if they did fall, making sure the child will not be harmed too
much by falling, and guiding them so they can keep trying to walk.
Eventually, this expression of love and nurture will lead to the success
and growth of the child.

God's Mercy is eternal, unconditional, powerful, and subtle. It can
sometimes seem difficult to understand His Mercy when we are going
through difficult times and His Mercy seems hidden. Yet, it is always
there, invisibly ensuring our progress and success.

The next 3 groups include the quality of Mercy.

\subsection{Generosity (22)}\label{generosity-22}

Generosity includes the names All-Bountiful, Bestower, Most Generous,
and Most Bountiful. These three attributes highlight a natural
progression of God's love from Mercy. God being the Most Generous means
He will give freely with love with no expectations of anything in return
nor any limitation. God being All and Most Bountiful shows God will give
with great abundance and encompasses all of creation. God being the
Bestower demonstrates another aspect of a personal relationship. He will
deliberately give blessings according to our needs and capacity. The
abundance is intentional and specific. ``God enriches who He wills
through the means of the heaven and earth.''\footnote{Kitáb-i-Aqdas \#66}

Generosity can be quite Subtle as He can give in ways which seem hidden
or unnoticed. It can require perception and an open mind and open heart.
These gifts are manifested in every aspect of life and are tangible and
real. Knowledge can influence the bounty and bestowals by God knowing
what to give, when to give, and how much to give. This generosity will
be in ways which benefits all of creation, which itself is an act of
generosity. As God's Power is absolute, there is no constraints by need
or scarcity. God can give without any limitations. Lordship is reflected
because generosity as He is Lord of all Things, and all things are
sustained under His care and authority. As God is Just, He will not give
unequally or unfairly. The gifts may not be the same for all, but
everyone will receive what they need. As God is exalted over all, His
generosity is also spiritual in nature, not just material. It helps us
connect with the divine. True generosity will not seek anything in
return, showing God's Independence. One of the greatest gifts God can
provide is knowledge in how to live a righteous life through His command
and counsel. These gifts allow us to thrive, not just merely survive.
God's generosity is faithful and unwavering. Even though belief in God's
praiseworthiness allows us to recognize His love and generosity, the
generosity also gives us more reasons to offer praise to Him. We are
moved to glorify Him. Forgiveness itself is an act of generosity,
offering a fresh start and renewal.

The next 2 groups include generosity.

\subsection{Sovereignty (4)}\label{sovereignty-4}

Belief in God as the Sovereign, Lord of Dominion, the Ruler, and the
Master. It is a natural expansion of Lordship. A Lord focuses on the
care and provision of creation, ensuring it is sustained and guided. A
sovereign ruler has absolute rule and authority. The belief of God's
love, forgiveness, mercy, and generosity leads to believing in God's
ability to control and rule over all of existence.

Believing in God the Sovereign means you believe His authority is
absolute and independent of any other authority. As a Ruler, there is
belief He actively governs and ensures order. With God as the Master, He
has personal ownership and control over all things. God expresses
mastery over all matters. God's Dominion is both the spiritual and
physical worlds. This sovereignty is built upon all the other beliefs,
names, and attributes expressed so far. It is continuous and never ends.
Much like how God loves, it is built on faithfulness, forgiveness,
mercy, and generosity according to the capacities of creation,
capacities which can be ever-changing.

Recognizing God's sovereignty crowns this progression, affirming His
ultimate control over all realms.

\subsection{Purity (9)}\label{purity-9}

Purity includes the names of Purest, True, and One. This is the pinnacle
of belief in God. These names describe God's transcendence,
incorruptibility, and oneness. Being the Purest, God is beyond all
imperfection and is entirely flawless. Nothing can limit or alter Him.
Being True, God is constant and beyond any falsehood, while all else in
creation is temporary. Being One, God is indivisible and without equal.
There is no multiplicity of His essence and He is inclusive of all
reality and of all creation.

In our journey through the groups, building our foundations of belief,
we have an opportunity in believing in the true and transcendent nature
of God. Without this journey, belief in God's Purity or oneness might
seem shallow. For example, what might be the consequence of believing
God is One but lack belief in the other names of God?

\begin{itemize}
\tightlist
\item
  Belief in God's Oneness without belief in God's Subtlety might view
  God as being remote, causing a possible agnostic or deistic
  perspective.
\item
  Belief in God's Oneness without belief in God's Manifestation would
  prevent the person from recognizing or connecting with the divine,
  causing a possible rejection of prophecy, scriptures, and religious
  teachings. Faith could be reduced to a mere philosophy instead of a
  living reality.
\item
  Belief in God's Oneness without belief in God's Knowledge causes one
  to feel God lacks wisdom and His rulings imperfect and arbitrary. This
  could cause a fatalistic or impersonal view of God where He is not
  aware of individual lives.
\item
  Belief in God's Oneness without belief in God as Creator would remove
  God's role as the source of existence. This would lead to pantheism or
  atheism.
\item
  Belief in God's Oneness without belief in God's Power means His
  oneness is meaningless as He cannot act. This could lead to polytheism
  or dualism where other forces compete with God's will.
\item
  Belief in God's Oneness without belief in God's Lordship would mean
  God is unable to sustain or guide His creation. This could lead to
  nihilism.
\item
  Belief in God's Oneness without Justice might make it seem God is
  unconcerned with fairness or morality. This could lead to tyranny,
  cruelty, or oppressive religious views where oneness is used as the
  justification.
\item
  Belief in God's Oneness without Exaltation might make God seem too
  human, leading to anthropomorphism where God is viewed as a human-like
  figure.
\item
  Belief in God's Oneness without Independence would cause belief God
  needs something to exist, leading to a belief in paganism where others
  share divine authority.
\item
  Belief in God's Oneness without belief in His Command would lead to
  moral relativism where there are no consequences or ultimate justice,
  possibly no right or wrong.
\item
  Belief in God's Oneness without His Counsel would have one believe
  there is no divine wisdom guiding human affairs, leading to despair or
  confusion and potentially directionless.
\item
  Belief in God's Oneness without His Faithfulness could make God seem
  fickle and unreliable in fulfilling promises. This would lead to
  distrust in religion and a fear God is not dependable.
\item
  Belief in Oneness but not belief in God's praiseworthiness would mean
  worship has no meaning. God would not inspire love or devotion,
  leading to spiritual emptiness.
\item
  Belief in Oneness with belief in God's love makes oneness cold and
  rigid. It leads to fear-based religion, where obedience is motivated
  by dread.
\item
  Belief in God's Oneness without His forgiveness make God seem
  unapproachable and terrifying. This would lead to a legalistic
  religion where perfection is the only goal to prevent eternal
  condemnation.
\item
  Belief in God's Oneness without His Mercy is rigid and oppressive,
  leading to hopelessness and possibly depression, where a sinner may
  feel there is no way back to God or salvation.
\item
  Belief in God's Oneness without His generosity is a belief where
  creation is an accident and good things are merely a matter of luck.
  This could lead to pessimism where people believe God does not care
  about their well-being.
\item
  Belief in God's Oneness without His sovereignty could mean God has no
  impact and other forces could challenge Him. This also could lead to
  polytheism or deism where God's rule is questioned, challenged, or
  replaced by others.
\end{itemize}

Belief in God's Purity leads to an understanding which there is no
partner with God as the Greatest Infallibility\footnote{Kitáb-i-Aqdas
  \#47}. No one else has been given a share in this station. This belief
means God never errs and is absent of imperfection. Truth is never
compromised. The essence of infallibility is oneness, as God cannot be
divided against Himself.

As you can see, a lack of belief in any one of these names or attributes
compromises belief in God's purity, truth, and oneness. If any are
missing, God cannot be truly be One and God cannot be True. With any
missing, instead of religion we develop some other type of corrupt
ideology which lack the completeness of truth, spirit, and experience.

\emph{Reflection: Which name or attribute felt most real to you, and
which felt most distant?}

These names teach us who God is, but they also point beyond themselves,
toward the purpose behind what has been revealed.

\section{The Purpose of God}\label{the-purpose-of-god}

``Blessed is the one who understands the purpose of God in what has been
revealed from the heaven of His will, which reigns supreme over all the
worlds.''\footnote{Kitáb-i-Aqdas \#125} What is the purpose of God?
Within the 19 groups of names and attributes, we learn about who or what
God is, but we do not necessarily learn why God has acted as He has.
There are various answers throughout the writings of the Báb and
Bahá'u'lláh. One answer by the Báb is ``Your purpose for me is that I
recognize You in all things, so that I may not be ignorant of You in
anything.''\footnote{Persian Bayán, Váḥid 2 Gate 3} Bahá'u'lláh also
says ``Whoever has known Me has known the intended purpose.''\footnote{Kitáb-i-Aqdas
  \#138}

Belief in these names and attributes of God are a pathway to
understanding what this purpose is. The purpose of God is not solely the
commandments and prohibitions\footnote{Lawḥ-i-Sirāj} and never will be.

\emph{Reflection: If the purpose is to recognize God in all things, what
is one place in your life you rarely look for Him?}

The purpose of God is not merely stated, it is carried, and one way it
is carried is through what Bahá'u'lláh calls the Breath of God.

\section{The Breath of God}\label{the-breath-of-god}

The breath of God is a mode in how God expresses His purpose. It ``wafts
through all else''\footnote{Kitáb-i-Aqdas \#158} and ``passes through
all created things.''\footnote{Kitáb-i-Aqdas \#111} The breath has
fragrance\footnote{Lawḥ-i-Aqdas} and causes action and movement. If
there is no movement due to the breath of God, the person or object is
considered dead.\footnote{Súríy-i-Haykal \#167} Thus the breath of God
is what gives life. Having both fragrance and life also means this
breath can permeate and be deeply felt. Fragrance can attract others,
but often fragrance can be hidden by other smells.

The breath of God cannot be confined, as itself is movement causing
movement. It is spreading to those with a receptive nose or receptive
soul. It is a sign of beauty and renewal. A fragrance does not
discriminate and is able to permeate or be sensed by anyone open to it.
Believing in God also means believing in the Breath.

\emph{Reflection: What in your life feels like fragrance and movement,
and what feels like dead air?}

The Breath of God describes God's influence, and the covenant describes
our relationship to that influence, and what it asks of us.

\section{Covenant of God}\label{covenant-of-god}

The Kitáb-i-Aqdas describes the covenant of God. A covenant is a formal
agreement God has made with humanity. It defines expectations, rewards,
and responsibilities between Himself and us. This ties into God's
Faithfulness but also includes our own faithfulness. This section will
not go into various details of this covenant but it is important there
is belief in this covenant.

\emph{Reflection: If God is faithful, what is one act of faithfulness
you can offer in return?}

With these foundations set, we can now gather the thread of this chapter
before turning toward the soul.

\section{Chapter Summary}\label{chapter-summary}

This chapter established why belief in God is foundational to everything
else in the Kitáb-i-Aqdas, even though the Kitáb-i-Aqdas never
explicitly says you must believe in God. We began with the simple
observation that God is present everywhere within the book, and that
Bahá'u'lláh weaves mystical teaching, theology, and straightforward
commands together for the growth of the human spirit. From there, we
explored how Bahá'u'lláh calls on God's names and attributes as a way to
keep God present in every command and teaching, and I grouped these
names into 19 groups which begin with one concept and progress to the
next. We then moved from the question of who God is to why God has acted
as He has, and how the purpose of God is not solely commandments and
prohibitions. This led into the breath of God as a mode of expression
which wafts through all else, gives life, has fragrance, and spreads to
receptive souls. Finally, we closed with belief in the covenant of God,
as a formal agreement defining expectations, rewards, and
responsibilities between God and humanity, and without these beliefs we
do not develop religion but instead develop some other type of corrupt
ideology lacking the completeness of truth, spirit, and experience. This
foundation is also a step toward true liberation, where faith is chosen
freely and fear loses its power. With this foundation set, Chapter 2
turns to the second foundation of this book: belief in the human soul,
its reality, its unique nature, its inner struggles, and its fate.

\newpage{}

\chapter{Belief in the Human Soul}\label{belief-in-the-human-soul}

\section{Chapter Introduction}\label{chapter-introduction-1}

The Kitáb-i-Aqdas regularly describes the human soul and the station of
humans and humanity. Bahá'u'lláh does not explicitly state that belief
in the human soul is required in the Kitáb-i-Aqdas or other writings,
but like belief in God, it is strongly implied. This will be the second
foundation of this book upon which the Kitáb-i-Aqdas rests.

The soul is an idea that appears across languages, eras, and peoples.
Even when cultures are distant, people still speak of an inner reality
that remembers, chooses, struggles, and longs for meaning. In Chapter 1,
we looked outward---toward God's names and attributes, and toward the
Breath of God as the active influence that gives life and fragrance.
Here we turn inward, toward the receiver of that Breath: the human soul.
This shared human question becomes one more doorway for honest
conversation, because whatever else we are, we are equally human.

This chapter will be divided into four main sections: the Reality of the
Human Soul, the Unique Nature of the Soul, the Inner Struggles of the
Soul, and the Fate of the Soul. These sections are derived from the
Kitáb-i-Aqdas as the primary source document. Any supplementation from
other sources from the Báb and Bahá'u'lláh will be noted.

\emph{Reflection: When you say ``I,'' what part of you is
speaking---body, mind, or something deeper?} We first need to establish
what it means to say the soul exists at all.

\section{The Reality of the Human
Soul}\label{the-reality-of-the-human-soul}

The first belief is the fact the human soul exists and is real. All
living things have a soul, but the human soul is distinct to humans. The
human soul is capable of self-awareness, moral choice, and a conscious
which can turn towards God.

The human soul in its form did not eternally pre-exist but was created
by God, uniquely for each individual. Bahá'u'lláh says, simply, that we
are created from water.\footnote{Kitáb-i-Aqdas \#148} This water is both
the physical foundation of the body and a symbol for the flowing grace
of the Divine Will. Bahá'u'lláh says water is one of the key components
of all creation, celestial and terrestrial, along with fire, air, and
earth as they combine to make heat, moisture, cold, and
dryness.\footnote{Tablet of the Light Verses and Disjointed Letters} He
also describes various kinds of water used in the creation of spirit,
soul, and body in the same tablet. Bahá'u'lláh describes the water of
semen as being pure, not something itself which needs
purified.\footnote{Kitáb-i-Aqdas \#74} Water, regardless of the source,
is the foundation of all life.

Bahá'u'lláh says that we are brought to life with a spirit from
God\footnote{Kitáb-i-Aqdas \#73}, through the divine breath.\footnote{Kitáb-i-Aqdas
  \#158} All living things are animated by this breath of God, which is
spirit itself. Spirit is the animating breath of life; the soul is the
inner life of a living thing; and the human soul is the unique form of
soul in humanity. This breath works as the catalyst upon water and the
other elements, giving life to everything living. This includes organic
entities such as animals, plants, bacteria, and other things biology
classifies as living. This spirit also is within other non-organic
objects such as the sun, the moon, earth, and other objects whose
lifespan endures for millions and billions of years. The spirit animates
the body of every living thing, and within that living animation each
creature has its own soul; in humans, this is the human soul.\footnote{Suriy-i-Rais
  (Tablet to the Chief) - This tablet provides a very good sermon about
  the nature of the soul} The soul is not created at the moment of
conception, as it does require the body to be the vessel and for its
elements to reach a stage of maturity. The body is the soul's womb.
Bahá'u'lláh does not define exactly when this occurs but this could
occur during embryonic of fetal development.

The human soul is the innermost part of ourselves. The human soul
provides the ``direction of the spirit towards one orientation over
others.''\footnote{Suriy-i-Rais} The soul can be ignited by fire of the
love of God or by one's own whim. This expresses the soul itself has
free will. Despite God's names and attributes of power and knowledge,
God has imparted every thing made of spirit, and every soul within it,
the freedom to move as it wishes, according to its temperament and
direction.

Bahá'u'lláh expresses all souls, whether in animals or in humans, and
all humans are Signs of the divine reality.\footnote{Kitáb-i-Aqdas \#72}
In Chapter 1, we learned how these signs are subtle and when manifested,
are a foundation for us in understanding God and His purpose. The soul
is also a sign, functioning as a mirror capable of reflecting God's
reality. This is true for all people.\footnote{Kitab-i-Badi - This book
  offers an in-depth explanation of what it means to be a mirror} The
names and attributes are manifested in the human soul.\footnote{Kitab-i-Iqan
  (Book of Certitude) - Another great source about the potentials of the
  soul} This mirrorship is a way to recognize God within us.\footnote{The
  Qur'an: ``We shall show them Our signs in the horizons and within
  themselves until it becomes clear to them that it is the truth.'' Also
  repeated in the Persian Bayan, the Seven Valleys, and the
  Kitab-i-Iqan.} These manifestations of Signs can occur in the same way
God's names and attributes can also be believed in, perhaps in stages or
through periods of continued realization.

\emph{Reflection: What makes you irreplaceable, even if no one else can
see it?} From existence, we can move to the soul's distinct qualities
and capacities.

\section{The Unique Nature of the
Soul}\label{the-unique-nature-of-the-soul}

Bahá'u'lláh describes the body as being a temple, and the collection of
all people as being temples of humankind.\footnote{Kitáb-i-Aqdas \#96}
The figurative heart serves as a malleable lens within the body which
serves to perceive divine truth. This helps a person with receptivity to
God, which in turn helps a human soul determine its destiny and
spiritual path.

The Suriy-i-Haykal (Surah of the Temple) describes how Bahá'u'lláh,
being a Manifestation of God, is a sacred vessel in the highest form.
Every human body, although temporary, is also a sacred vessel although
in a lower station than that of the Manifestation of God. The temple is
temporary but is a place of worship and the development of spiritual
qualities, according to its capacity as a mirror. This means the body is
also capable of being used to nurture the human soul, not just through
material means but through being used for spiritually purposeful conduct
and expression of divine attributes.

The human soul is capable of having an exalted and noble
station.\footnote{Kitáb-i-Aqdas \#120} There are various stations and
ranks a soul is capable of achieving. The soul itself was born
noble\footnote{The Arabic Hidden Words \#13} but requires effort to
remain so by turning one's sight inward. There is no higher station than
a person bearing witness to God.\footnote{Persian Bayan, Vahid 1 Gate 2}
The human soul's natural station is higher than that of animals, but can
descend to the level of an animal's soul if it is neglected.\footnote{Persian
  Bayan, Vahid 5, Gate 14}

Each person and each human soul has its own capacity.\footnote{Kitáb-i-Aqdas
  \#176} Not every person's soul will express the divine attributes and
names equally or in the same exact way. They are expressed according to
one's capacity.\footnote{Kitab-i-Iqan} This means every soul may have
its own distinct strengths and weaknesses, bound by the human temple and
its position in time and space. These different combinations of
capacity, reflections of divine names and attributes, and effects of the
body ensures every soul is distinct, much like the iris of the eye or
the lines of the palm of the hand.

\emph{Reflection: What inner habit most often steals your
freedom---fear, suspicion, or pride?}

Once the soul's potential is clear, we can be honest about the conflicts
that dim it.

\section{The Inner Struggles of the
Soul}\label{the-inner-struggles-of-the-soul}

Bahá'u'lláh describes a suspicious soul\footnote{Kitáb-i-Aqdas \#119} as
an opposition to one who uses reason. This suggests that the soul and
reason are meant to be allies. When we are `headless' or suspicious, we
aren't just losing faith; we are losing the proper use of our intellect.
We begin to use our minds to build walls (idols) instead of bridges to
the truth.

A suspicious soul could be someone who is ruled by doubt and is not sure
who or what to trust. This doubt could be about God, God's purpose, or
even their own purpose in life. This skepticism may exist even if truth
or a path forward is evident. They may not understand the good in
certain situations, understand the potential wisdom in difficult
situations, and due to this, lack discipline to truly submit to God.

This lack of discipline could lead to self-destructive acts and habits,
which can lead a person further down the road of headlessness and
suspicion. For example, a person may seek externally induced mystical
experiences by depriving themselves of their mental faculties such as
reason. A suspicious soul could be someone who is deceptive, also known
as hypocritical. They may outwardly portray belief but inwardly or in
private has other beliefs and actions. With the example of externally
induced experiences, the mirror of the soul isn't reflecting the light
of God; it is merely caught in the glare of its own internal chemistry.

Bahá'u'lláh teaches these suspicions act as idols.\footnote{Lawh-i-Dunya
  (Tablet of the World)} These idols are the cause of humiliation and
loss and keep a soul from ascending to the station they are rightful to
inherit and claim. However, the Kitáb-i-Aqdas does include the most
simple perspective to help alleviate these inner struggles the soul may
face.

Bahá'u'lláh says all are servants.\footnote{Kitáb-i-Aqdas \#72} The
soul's fundamental purpose is in relation to the divine. All souls are
servants. Belief not only in the soul, but in the soul's purpose as a
servant of God. This establishes trust and humility, eliminating the
symptoms of a suspicious soul such as distrust, skepticism, arrogance,
or hypocrisy. Belief in servitude to God is the source of freedom for
the soul\footnote{Kitáb-i-Aqdas \#125} and reveals our true human
rank.\footnote{Kitáb-i-Aqdas \#120}

\emph{Reflection: In your recent moments of doubt, was your reason
acting as a lamp to find the truth, or an idol to justify your own
whim?}

From struggle, the chapter turns to the question every soul eventually
faces: its departure and destination.

\section{The Fate of the Soul}\label{the-fate-of-the-soul}

One matter which is not a matter of belief is the fact death of the body
is inevitable. Death is the singular inescapable certainty of the
material condition. It is also absolutely true once the body has died,
the energy which used to be within the body no longer exists within it.
It is also a fact that the body's heat and motion no longer remain in
the same way once life has departed. In the material world, what we call
energy changes form rather than vanishing, and this can be a useful
reminder that endings are not always simple. This is not offered as
proof of the soul, but as a way to keep the mind from collapsing death
into mere emptiness. Where does that life go?

Bahá'u'lláh reminds us we will return to dust.\footnote{Kitáb-i-Aqdas
  \#148} The temple of the human body will end. The human soul will take
flight.\footnote{Kitáb-i-Aqdas \#97} Taking flight symbolizes the soul's
transcendence over material life, to a real place higher than where it
was. The soul is also capable of transcending the human body even before
the body has perished, if a person so desires and believes. ``If you are
a person of mystery and longing, take flight with the wings of the
saints' aspirations so that you may see the secrets of the beloved and
reach the light of the loved one. `Indeed we belong to Allah, and indeed
to Him we will return.'\,''\footnote{Haft Vadi (The Seven Valleys)}
Bahá'u'lláh throughout His teachings reminds us the importance of the
soul being allowed to take flight while the body is alive while
constantly being aware of the inevitable flight when the body no longer
exists. The first is not inevitable but desired by God, and the second
is inevitable and desired for a believer and God. A suspicious soul
might show fear for the second flight, fearing the end of everything. A
soul in belief is prepared for flight. A soul prepared for flight is a
liberated soul.

The desired journey of the human soul is onboard the Crimson
Ark.\footnote{Kitáb-i-Aqdas \#84} Crimson often represents the blood
from sacrifice and the victory of divine authority. The Crimson Ark
represents salvation and refuge, much like how the Noah's Ark led the
people of the time to salvation. Belief in the salvation of the soul
during the life of the body and after is important, especially within
the belief of all the names and attributes of God. The journey in the
Crimson Ark requires belief in God, in all the names and attributes of
God. The journey in the Crimson Ark requires belief that you have a
soul. The journey in the Crimson Ark requires belief that every person
has a soul, even a person you feel might be an enemy. The fate of the
soul doesn't require only consideration of your own salvation, but in
the salvation of others.\footnote{BH00093 (Letters to the Land of Kha)}
The fate of the soul requires a belief that other souls deserve
salvation. All souls deserve liberation. All souls are capable of
transcending the bounds of self and body. All souls are able to take
flight.

\emph{Reflection: If death is certain, what would it mean to live as if
your soul is preparing for flight?}

This chapter has treated belief in the soul as more than a comforting
idea. It is a lens for reading the Kitáb-i-Aqdas itself: commands become
training for the inner self, virtues become capacities the soul can grow
into, and struggle becomes a sign that the soul is alive and capable of
change. Belief in the soul is one of the roots of true liberation,
because it insists that you are not trapped inside appetite, habit, or
fear, and that your life has meaning beyond what can be measured.

The next chapter will discuss the potential destinations of the soul,
the worlds of God.

\newpage{}

\chapter{Belief in the Worlds of God}\label{belief-in-the-worlds-of-god}

\section{Introduction}\label{introduction}

Religious cosmology has always been a subject I have always found
fascinating. Every culture in the world has a creation story where the
world(s) come into existence, Earth is formed with the major elements
from which life begins, and humanity is created. All of this creation
has a divine aspect to it. There are elements of truth within all of
these stories, whether you believe them to be legend, myth, or absolute
sources of fact. They help shape identity, purpose, and help us
understand our place within creation. You will discover the cosmology
expressed by Bahá'u'lláh within the Kitáb-i-Aqdas and further elaborated
upon in His other writings has elements which may seem familiar. There
will also be elements which seem strange, depending on your cultural
perspective. I feel the Zoroastrian and Hindu traditions will find the
closest relationship to the cosmology of Bahá'u'lláh, but even then,
what is expressed by Bahá'u'lláh (as well as the Báb) is completely
unique. Belief in the worlds of God, while again not being explicitly
expressed as required by the Kitáb-i-Aqdas, is vital to the belief in
the Cause of God. The journey of the soul must go somewhere.
Pre-existence must have an origin. These worlds may not be easily
recognized in a physical sense, but are nevertheless real. In this
chapter, I will explore these worlds as layered realms of meaning and
existence, and how they contain kingdoms, cities, schools, and living
beings. Each of these realities helps describe how the human soul can
mature beyond its current form, and why the spiritual journey does not
end at the boundaries of this world. These are some of the deepest
mysteries of creation, full of subtlety. I hope this chapter will help
unlock some of these mysteries and fortify the belief in the worlds of
God. This belief is another foundation upon which the spiritual journey
of individuals and our systems of organization rests upon. Our journey
depends on this belief.

\section{Nature of the Worlds}\label{nature-of-the-worlds}

The Kitáb-i-Aqdas describes many worlds specifically. There is the
lesser world, the greater world, spiritual worlds, heaven, and various
kingdoms of these worlds. To introduce the nature of the worlds, I want
to start with a long passage from the Súríy-i-Vafá (Tablet to Vafá).

\begin{quote}
As for what you asked concerning the worlds, know that God has worlds
without end, infinite beyond comprehension, and no one encompasses them
except His All-Knowing, All-Wise Self. Reflect on sleep, for it is a
great sign among people if they were to ponder. For example, you see
something in your sleep at night, and you find it exactly as you saw it
after a year, or two, or more, or less. Even though the world in which
you saw what you saw in your sleep is not the same as the world you are
in, what you saw in your sleep must exist in this world at the time you
saw it in your sleep, and you are among the witnesses.

Moreover, you see something that did not exist in the world, yet it
appears afterward. This confirms that the world in which you saw what
you saw is another world that has neither a beginning nor an end. If you
say that this world is in your mind and is balanced within it by a
command from an All-Mighty, All-Powerful source, this is true. And if
you say that the spirit, when it is detached from attachments during
sleep, is directed by God into a world that is concealed within the
mystery of this world, this is also true.

Indeed, God has world after world and creation after creation, and in
each world, He has ordained what none can enumerate except His
All-Accounting, All-Knowing Self. Reflect on what We have cast upon you
so that you may understand the intention of your Lord and the Lord of
all worlds. Within it lies a treasure of the mysteries of wisdom, and We
have not elaborated on it due to the grief that has overwhelmed Me from
those who were created by My word, if you are among those who hear.
\end{quote}

Bahá'u'lláh says God reigns over all worlds.\footnote{Kitáb-i-Aqdas \#39}
With the Tablet to Vafá above, we can understand these worlds are
infinite in number. Some may view worlds as being planets, but the usage
suggests entities such as the universe or plane of existence. Dreams may
not be an actual world, but are a sign of these various worlds. There
are various theories which seem to address this type of reality, such as
multiverses and quantum superposition. I personally find quantum
superposition an interesting theory, as it expresses how particles exist
in multiple states simultaneously until observed. What if consciousness
and spiritual awareness work similarly? Just as observation collapses
quantum possibilities into a single actualized state, perhaps spiritual
seeking allows the soul to access worlds that are always present but not
always perceived, existing simultaneously in the same location but on
different planes of existence.

This leads to an infinite number of potentialities. As Bahá'u'lláh
discusses the existence of spiritual worlds that cannot be expressed by
words nor pointed to by signs,\footnote{Kitáb-i-Aqdas \#116} these
worlds could exist in the same location as us but on a different plane
of existence. They could also exist other locations which we sometimes
pass through as Earth is on its own celestial journey, rotating around a
Sun which rotates around the center of the Milky Way Galaxy, which
rotates around the center of the universe, which might rotate around
something else. We are never in the same spot we once were and most
likely, never be in that exact same spot ever again. Our physical world
is called the lesser world.\footnote{Kitáb-i-Aqdas \#55} There is also a
greater world.\footnote{*Kitáb-i-Aqdas \#55} In the writings of
Bahá'u'lláh, the greater world seems to represent the very structure of
existence itself. It's a vast spiritual reality which includes all
divine realms.

\subsection{Kingdoms Within the
Worlds}\label{kingdoms-within-the-worlds}

The Kingdom of Names bears witness God desires to rule over our
hearts.\footnote{Kitáb-i-Aqdas \#83} This Kingdom of Names is manifested
in all worlds, spiritual and lesser. In the lesser world we may be aware
of this Kingdom through the various signs, just as we were in learning
about God. Every name and attribute is manifested from this Kingdom, and
every name and attribute returns to this Kingdom.\footnote{Tafsir-i-Hu
  (Interpretation of the Name He)} It could also be possible some of
these spiritual worlds solely manifest a single Name. Imagine the
possibility of a particular day, we happen to pass through a spiritual
world which reflects the name Just. We could be working on some task and
suddenly come to a realization about a separate matter which seemed
unjust or unfair. This realization leads to an inspired solution. Could
this inspiration have come solely from our own imagination? Or could
this inspiration actually have come from God and the Kingdom of Justice
in a greater world? The Kingdom of the Lord\footnote{Kitáb-i-Aqdas \#79}
exists within all worlds. This kingdom represents God's sovereignty, one
of the names of God. There is also the Kingdom of Command\footnote{Kitáb-i-Aqdas
  \#91}, representing God's names of commander, ordainer, and fulfiller.

\subsection{Cities and Schools Within the
Kingdoms}\label{cities-and-schools-within-the-kingdoms}

These Kingdoms have cities of names where the Concourse on High and
other dwellers exist.\footnote{Kitáb-i-Aqdas \#132} If the worlds of God
are countless, and God encompasses all names, it would also stand to
reason these cities could also be infinite in number. Many scriptures
refer to various cities in the worlds of God. Many Christians are
familiar with the concept of Zion or the New Jerusalem. Hindus mention
many celestial cities such as Amaravati and Dwarka, the birthplace of
Lord Krishna.

These cities can also represent various stages of the maturation of the
soul, such as the City of Love and Rapture.\footnote{Gems of Divine
  Mysteries} Aiding this development are schools within these cities.
The School of God is where the Tablets are revealed.\footnote{Kitáb-i-Aqdas
  \#176} The School of God predates the creation of the Earth and
heavens\footnote{Kitáb-i-Aqdas \#177} and encompasses all the knowledge
God the All-Knowing has. As we experienced in the chapter ``Belief in
God,'' belief in the knowledge of God is essential to belief in God the
Creator. The Tablet may not necessarily be what is revealed in
revelation, but what is revealed within the soul's personal journey. The
School of God is something we could think of as a greater institute
which itself has branches. These branches can focus on being sources of
knowledge teaching a soul a particular value a soul can develop to
progress along the divine names and attributes. The Kitáb-i-Aqdas
mentions the School of Detachment and the School of Meanings and
Exposition while the Kitáb-i-Badí` mentions the School of Knowledge.

Imagine the various combinations which could exist within these schools,
cities, and kingdoms. Say you enter the School of Servitude, either here
in the lesser world by chance, through meditation, or through a glimpse
through one of the greater spiritual worlds of God. This School may help
teach virtues such as humility or the fear of God. These lessons within
the School of Servitude may help you enter the City of Submission which
is when the soul is ready to surrender to God's Will. This School helps
refine the soul and prepares it to reflect potential names such as the
Almighty or the Lord of Lords. Eventually in Part 2 of this book
regarding the relationship with God, I will present up to 39 virtues
identified within the Kitáb-i-Aqdas.

\emph{Reflection: If you imagined your next trial as a lesson in a
``School,'' what virtue would you hope you are being taught?}

With the structure of worlds, kingdoms, cities, and schools in view, we
can now ask what all of this is for, and what the soul is meant to
become.

\section{The Soul's Purpose Within These
Worlds}\label{the-souls-purpose-within-these-worlds}

Belief in the worlds of God is essential to those who believe they have
a soul. The purpose of the soul is to find pathways beyond the lesser
world, even if at first it is merely just glimpses into the greater
world. The Bayan says the soul progresses degree by degree, especially
those who ponder upon the realm of the infinite.\footnote{Persian Bayan
  Vahid 2, Gate 12} As the soul is designed to reflect the divine names
of God, the greater worlds manifest the divine names more purely than in
the lesser world. The soul though, even when bound by the temple of the
body, is able to experience the greater world. Think in a sense nirvana
as taught by Buddha. For a soul not seeking, they may consider an
outcome or event to be a matter of destiny or fate. However, for a
seeking soul who reflects upon the spiritual realities, they may see an
event or outcome as a reflection of a City of God, a Kingdom of God, a
manifestation of a Name, and something guided by the divine hand.

This journey is fundamentally about liberation by freeing the soul from
the limitations of the lesser world and the veils that separate it from
God. Just as Buddhist philosophy speaks of liberation from suffering and
Hindu tradition speaks of moksha (release from the cycle of rebirth),
Bahá'u'lláh's cosmology presents liberation as the soul's progressive
freedom from attachment to the material world and its movement toward
reunion with the divine. This liberation isn't a single moment of escape
but an ongoing process of spiritual unfoldment across infinite worlds,
and this is part of what I mean by achieving true liberation.

The eventual destination is Heaven. Heaven is often used in the
Kitáb-i-Aqdas as a spiritual world itself, designating both a source of
blessings and a destination for the soul. This paradise also reflects
the names and attributes of God, such as the Heaven of Bounty\footnote{Kitáb-i-Aqdas
  \#55} and Heaven of Revelation.\footnote{Kitáb-i-Aqdas \#64 and \#105}
Bahá'u'lláh says ``Paradise, it is a reality beyond doubt, and today in
this world, it is my Love and My pleasure. Whoever attains it, God will
assist them in this world, and after death, He will admit them into a
Paradise whose expanse is like that of the heavens and the
earth.''\footnote{Súríy-i-Vafá} Paradise is also ``my Love, your
heavenly home, reunion with me.''\footnote{The Hidden Words of Arabic
  \#6} The purpose of the soul is reunion with God in the heavenly
paradise.

A person who is a seeker and goes on a journey similar to that as
described by Bahá'u'lláh in the Seven Valleys can attain heaven even
while on earth. It is the condition of the soul which is near to God.
The beautiful thing which I personally love with this cosmology is even
if I do not attain this station now, I will forever have opportunities
after my body gives way. The condition of being distant from God, also
known as Hell, is not a condition which lasts forever. I would highly
recommend reading the Bayan's Vahid 2, Gate 16 for a wonderful sermon
about paradise and hell.

The Kitáb-i-Aqdas does teach a powerful tool a person can use to elevate
their own soul and that of others, reciting the Words of God in a
melodious tone.\footnote{Kitáb-i-Aqdas \#116, \#150, \#160} This
practice mimics the melodies which come from the greater worlds.
Bahá'u'lláh often describes these melodies as from the ``nightingales of
the Supreme Paradise.''\footnote{The Words of Paradise} It can also be
the Words of God, no matter who sings them, are also heaven.

\emph{Reflection: If Heaven can be a condition of nearness, what would
``one step nearer'' look like in your inner life today?}

To see how the worlds of God are not empty space but living reality, we
now turn to who dwells in them and how they aid the soul.

\section{Inhabitants of the Greater
Worlds}\label{inhabitants-of-the-greater-worlds}

Earlier in this chapter the Celestial Concourse was mentioned. A
concourse is a meeting place or a location where multiple paths merge.
This could be similar to a convention center, transportation station,
hotel, mall, or other places such as this. The Concourse on High would
be the grouping of those who dwell in the greater worlds. They are
inhabitants of the Eternal City.\footnote{The Persian Hidden Words \#71}
These are the souls who have achieved the exaltation of paradise, of
heaven, and of nearness to God. These souls wail when God restrains any
knowledge, even for a moment, and they tremble and sway when He releases
the knowledge.\footnote{BH00230 (Unnamed Writing)} They lament the fire
of hatred\footnote{Kitáb-i-Aqdas \#89} and have submitted to the Cause
of God.\footnote{Kitáb-i-Aqdas \#132}

The Hosts of the Celestial realm and the nearest angels will aid anyone
who rises to assist the Cause of God.\footnote{Kitáb-i-Aqdas \#53}
Angels are celestial beings whose purpose is to do God's will. They send
His Tablets\footnote{Surah of Our Name The Messenger} and are able to
traverse the greater worlds into our lesser world.\footnote{Tablet of
  the Birth} They will announce with a trumpet the arrival of a
Messenger with a Message.\footnote{Kitáb-i-Íqán (Book of Certitude)}
Angels also announce the death of a person, bring about wrath and
punishment for a soul which had not progressed.\footnote{Surah Fu'ad}

A soul which has been purified can achieve a station as
angels.\footnote{Kitáb-i-Íqán (The Book of Certitude)} Human souls
remain distinct from angels in nature, but can reach similar levels of
spiritual development and serve similar purposes, offering aid and
assistance from the spiritual worlds. These purified souls, which could
include loved ones from here on Earth who have passed away, might be
offering aid and assistance. Belief in these inhabitants helps us
understand we are not alone on our spiritual journeys. God has devoted
an incredible amount of energy within creation to help us and sustain us
through our good times and bad times.

As you can see, there is this unity between all of creation. What we do
today has effect with the beings of the greater worlds, not just here in
our physical world. What happens in the greater worlds can also have
effect here in our lesser world. We are all entangled. The best way to
understand this interwoven unity, this entanglement, is through the
Sidrat al-Muntahá.

\emph{Reflection: If unseen beings can aid a soul, what kind of help
would you be willing to receive without controlling how it arrives?}

To make sense of how the worlds touch, and how divine assistance can
cross their boundary, we now turn to the image of the Sidrat al-Muntahá.

\section{Sidrat al-Muntahá}\label{sidrat-al-muntahuxe1}

The Sidrat al-Muntahá means the Tree of the Farthest Extent. It is also
known as the Divine Lote Tree. If we imagine we are looking out from a
high mountain or maybe on a beach looking out to the ocean, we can see
where the land and the sky meet. No matter how far on Earth we journey,
we could never discover the physical point where the Earth and sky
actually meet. The Sidrat al-Muntahá represents the point where the
lesser world and the greater worlds meet. It is not something you can
physically reach, but it is the furthest extent our terrestrial
knowledge will get us before we need our spiritual knowledge to
progress. The Kitáb-i-Aqdas says ``the Lote-Tree of the farthest limit
proclaims: `There is no God but Me'\,''\footnote{Kitáb-i-Aqdas \#100}

The Sidrat al-Muntahá can call out and proclaim. It has Branches which
provide shade and salvation.\footnote{The Tablet of the Branch} As this
is within the horizon where the worlds meet, it is also the point where
light appears and disappears. The Divine Lote-Tree is the place of Dawn
for the Day of Resurrection. The Breaker of Dawn, who speaks between the
heavens and the earth\footnote{Kitáb-i-Aqdas \#117} is also a celestial
being often known as the Holy Spirit, or in Bahá'u'lláh's words the Holy
Maiden. Bahá'u'lláh describes her this way:

\begin{quote}
I heard the most beautiful, the sweetest voice above my head. When I
turned, I saw a maiden, the mention of my Lord's name suspended in the
air, aligned with the sun's head, and I saw her rejoicing in herself. As
if the embroidery of pleasure appears from her face and the light of
mercy declares from her cheek. She was speaking between the heavens and
the earth with a call that attracts hearts and minds and gives all my
limbs, both outward and inward, the good news that my soul was rejoiced
with, and honored servants rejoiced from it.\footnote{Súratu'l-Haykal
  (Surah of the Temple)}
\end{quote}

What is spoken from the Sidrat al-Muntahá is the Mother Book, which is
the hidden treasure.\footnote{Kitáb-i-Aqdas \#103} The Mother Book
includes everything spoken by God, and things which people are not ready
to hear. It contains all knowledge and also manifests all of God's names
and attributes.

The Holy Spirit also appears from the Sidrat al-Muntahá. Much like the
Holy Maiden, revelation appears from them. While the Holy Maiden may be
a personal apparition for Bahá'u'lláh, the Holy Spirit has descended
from Heaven to animate the missions of those such as Jesus Christ. The
Holy Spirit has its own language.\footnote{Súratu'l-Haykal (Surah of the
  Temple)} The Holy Spirit can also act as an intermediary between the
Manifestations of God and us, or between God and us. It calls from the
Kingdom of God\footnote{Kitáb-i-Aqdas \#80} and animates and inspires
life within the soul, even if the soul seems inert or dead. The Holy
Spirit is a source of grace.\footnote{The Persian Hidden Words \#58}

Belief in the Sidrat al-Muntahá is important as it helps a person
understand there are pathways for divine assistance available to us.
These can come when we need it, even if we don't seek this assistance.
These can come if we choose to approach the Sidrat al-Muntahá on our own
accord. It represents a constant hope for the soul and constant pathway
for God's names and attributes to have real effect in our lives.

\emph{Reflection: Where do you sense the ``horizon'' in your own life,
the place where reason reaches its limit and the soul must learn trust?}

Having looked at the worlds themselves, their inhabitants, and the point
where worlds meet, we can now gather what this belief changes in the
life of the soul.

\section{Summary}\label{summary}

Belief in the worlds of God, and all which exists within them, is vital
to the progression of the soul. These realms and entities are not
abstractions and are signs of God's names and attributes and help
manifest these realities in a way which can elevate the soul, whose
purpose is to be near to God in the condition of paradise. This also
demonstrate we are not alone in our journey.

Sometimes life here in this lesser world can be difficult. There is
suffering, pain, and seemingly insurmountable challenges. All of these
experiences are real. Are these experiences supposed to keep us down, or
help us rise to greater heights? We are never alone as long as we
believe we are not alone. Belief in angels as real entities, not merely
as symbols, is important. Belief those prior to us who loved us may be
offering a helping hand within the worlds of God is deeply heartfelt.
Belief in the Holy Spirit as a real entity who animates life within the
soul inspires living with purpose.

Belief that the worlds of God are infinite ensures our imagination can
be alive, never bound by our primary senses. Infinite worlds equal
infinite possibilities. The spiritual journey unfolds endlessly across
dimensions unseen. This is where unity comes from, the manifestation of
God's Oneness. It may not necessarily be the manifestation of sameness,
but through a real understanding every soul, every world, every being,
and all of creation are woven together like a divine tapestry. If any of
this is ever in doubt, be open to dreaming. Be willing to remember those
dreams. The Kitáb-i-Íqán and books like the Surah of Joseph in the
Qur'an, or even the Ahsan al-Qasas by the Báb (The Best of Stories)
which further explains the Surah of Joseph, testify to the value and
symbology which are embedded within dreams. I can personally testify I
first started this path of my journey to the Báb and Bahá'u'lláh due to
a series of dreams I experienced. Dream big. Be imaginative. Your soul
is ready for an amazing journey. The next chapter will discuss belief in
the Day of God, also known as the Day of Resurrection.

\newpage{}

\chapter{Belief in the Day of
Resurrection}\label{belief-in-the-day-of-resurrection}

\section{Introduction}\label{introduction-1}

Belief in God, all of God's names and attributes, the human soul, the
soul's purpose, the infinite worlds of God, and the inhabitants of these
worlds set the stage for establishing belief in the Day of Resurrection.
The Day of Resurrection is a theme which occurs in many religious
traditions. Zoroastrianism and Hinduism teach there are cyclical periods
which culminate in the triumph of good over evil. The Abrahamic faiths
teach of the establishment of the Kingdom of God on earth, through a
Messianic figure, where bodies rise from the grave with new life. The
Mayans devoted much of their astronomical sciences towards understanding
these spiritual cycles of return, rebirth, and the victory of good.

Belief in the Day of Resurrection is a belief the Kitáb-i-Aqdas assumes,
although it does not provide a detailed explanation. Bahá'u'lláh devoted
the Kitáb-i-Íqán (The Book of Certitude) entirely to discussing the Day
of Resurrection and is an incredible resource for understanding. The
Persian Bayán by the Primal Point (the Báb) discusses the Day with many
layers and perspectives, where it could be understood in potentially
infinite ways. The Day is a foundation in the religion of God. This
chapter will lightly touch upon what the Kitáb-i-Aqdas has to say about
the Day and how it ties into the beliefs established in the first three
chapters. The Day of Resurrection, the Day of Judgment, the Day of God,
and the Day of the Covenant describe different aspects of one divine
reality. We begin with the vivid images Bahá'u'lláh uses to show the Day
in motion, not as an abstract idea.

\section{Moses and Jesus}\label{moses-and-jesus}

Verse \#80 of the Kitáb-i-Aqdas uses powerful imagery to describe what
the Day of Resurrection is, without itself being an explanation which
required volumes to write. He describes two different prophetic figures
from the past to illustrate the Day. With the first, Bahá'u'lláh
describes Moses as attaining the lights of the Ancient One. The Ancient
One is most commonly referred to as God within the Kitáb-i-Aqdas. In the
Book of Exodus, Moses is described as climbing Mount Sinai, where God
appeared to Him as the burning bush. The Ten Commandments were first
revealed on Sinai. Why would Moses now have attained the lights of God,
when He already done so in Exodus? The answer is in the next sentence of
verse \#80.

Moses drank the pure waters of reunion. He attained the presence of God.
Did Moses attain this reunion with His physical body? This reunion is a
soulful one, a reunion of a completely spiritual nature. On each Day of
Resurrection, all are called upon. Reunion can also refer to the fact
Moses is among us and we are reunited with Moses. Just as Moses first
appeared to Pharaoh as a baby on the waters of the Nile river, the same
waters which set Egypt ablaze, again history repeats itself. Fire, while
seemingly a source of destruction, is also a source of renewal. Just as
the forests, grasslands, and marshes of the Americas required fire to
renew, so too do all of us. Fire is not an eternal punishment, but more
like a symbol of the Phoenix or a Zoroastrian temple. From this reunion,
Mount Sinai, which used to be the source of Revelation in the first days
of Moses, now circles around the new Source of Revelation and the new
Source of Divine Command.

The Spirit, which usually refers to Jesus Christ in Bahá'u'lláh's
writings, calls from the Kingdom ``Come, O sons of vanity!'' If we
consider what vanity is, which is those who are prideful and do things
for the purpose of being noticed, not because they are sincere in their
action. The call to come could be a pathway to restore the person's
self-worth through humility, a pathway of salvation. The Resurrection is
not only for humans, but for all the hosts of God in the celestial
worlds. These angels, saints, and other beings longed to meet God on
this Day. It is an example all should be willing to follow. The city of
Zion, which is a destination of paradise, also speaks about the promise
of God which was always ordained in the Tablets of God.

\emph{Reflection: When a new Day dawns, what changes in the way we
understand past Revelation?}

From these images of reunion, we now turn to the promise behind reunion,
which is the Covenant itself.

\section{The Covenant}\label{the-covenant}

In Chapter 1, we briefly alluded to belief in the Covenant of God as a
required belief. This covenant is entirely about the Day of Resurrection
where all are reunited with God. This promise has existed throughout
many Scriptures and many faith traditions. The Qur'an expresses the
trumpet and the blast. The trumpet and blast happened when prior
Prophets brought a new Revelation, such as with Ṣáliḥ to the Thamúd
people.\footnote{The Holy Qur'an, Surah Al-Haqqah \#5} The trumpet is
the announcement and the blast is the effect. There are two future
trumpets and blasts.\footnote{The Holy Qur'an, Surah Az-Zumar \#68} The
first blast causes universal death and the second blast resurrects all
souls for judgement.

Outside of the Abrahamic faiths, this promise for the return of God
remains strong. The faith of the Inca people taught the hopeful return
of Viracocha. Viracocha is a name of God as Creator who departed across
the ocean with a promise of return. The Iroquois believed in the return
of Deganawida, a name of God which means Great Peacemaker. Deganawida is
to return in a time of great turmoil to restore peace and unity. The
Bhagavad Gītā (4:7-8) says about the return of Krishna, a manifestation
of Vishnu ``Whenever righteousness declines and unrighteousness
prevails, I manifest Myself. To protect the righteous, destroy the
wicked, and establish dharma, I appear in every age.'' There are
countless examples of the return of God. Most faiths teach this return,
regardless of their historical connections to each other.

\emph{Reflection: If the promise of return echoes across religions, what
would fulfillment feel like when it arrives?}

Next, we look at how Bahá'u'lláh describes the shaking of forms that
happens when the promise becomes present.

\section{Fulfillment}\label{fulfillment}

The trumpet blast is often called the Great Announcement.\footnote{\textbf{\emph{Kitáb-i-Aqdas
  \#167}}} This announcement is ``the Day of God; none is to be
mentioned in it but His own Self, the Sovereign over all that is in the
heavens and the earth. This is a Cause by which all that you hold as
illusions and forms has been shaken.'' This verse expresses God's
sovereignty but also alludes it should be a day where everything which
is associated with God, such as all of those names and attributes which
are a part of His own Self. By doing so, everything that we know can and
will be changed. In regards to faith and religion, this could be
traditions, false beliefs which are created by others, and other things
we are attached to such as names. By reminding us God is the Sovereign,
God is reminding us on the Day of Resurrection, anything is possible
within the fulfillment of the Covenant.

Bahá'u'lláh describes the moon as being cleft asunder.\footnote{Kitáb-i-Aqdas
  \#81} The moon, being it reflects the light of the Sun high in the
nighttime sky, signifies its station as a Mirror of God, the God who
brings light of day upon us. In the Kitáb-i-Íqán, the moons can signify
saints or scholars who reflected the cause of God from a revelation
past, it could refer to the traditions and sciences attributed to the
leaders and scholars of that prior age, and it could even refer to the
practices derived from the laws of the past revelation. Splitting the
moon, much like in the night vision of Muhammad, represents the act of
God in changing all of these things for a new Day. What was current is
now old. What was once today is now yesterday. Imagine how all of these
forms would shake, like a great earthquake, by such an act of power from
God the All-Powerful. With every irrevocable matter being clearly
distinguished\footnote{Kitáb-i-Aqdas \#81} we are able to discern what
was true from the past and what was false. This causes a radical
transformation within the world itself, to include the religious, moral,
and social structures of all humankind. The Day of Resurrection itself
is an irrevocable matter, as it is the greatest promise God makes with
us. God promises change.

``The light has shone from the horizon of manifestation, and the
horizons have been illuminated as the Owner of the Day of the Covenant
has come.\footnote{Lawh-i-Ishraqat (Tablet of Splendors)}

\emph{Reflection: If the Day shakes inherited forms, what is the soul
supposed to do with that shaking?}

Bahá'u'lláh answers by calling the believer to act, not to watch.

\section{Seize the Day}\label{seize-the-day}

Bahá'u'lláh refers to what the Báb had prior said in verse 135 in the
Kitáb-i-Aqdas: ``if you attain what We reveal, you will ask from the
grace of God for Him to reign upon your innermost selves, for that is a
glory beyond reach. For Him to drink a cup of water with you is greater
than for every soul to drink of His own existence, indeed greater than
all things.''

This cup of grace, this cup of water, is what purifies the soul. This
water gives life and nourishes the innermost part of ourselves, much
like it purified Moses or purified Jesus after His baptism. The Day,
then, is born out of love for us, our salvation, and the awakening of
the soul. There is no greater honor for a soul than for this.
Bahá'u'lláh says to seize the Day. Belief in the Day of Resurrection is
not a passive act. It requires an active belief in God, all the names
and attributes of God, the human soul and its potentials, all the worlds
of God, and the belief everything within the material world is subject
to renewal and transformation.

\emph{Reflection: If belief is meant to be active, what does the Day
reveal about God's character?}

To answer, we now return to the Names and attributes from Chapter 1 and
see how the Day makes them visible.

\section{The Day Manifests God's Names and
Attributes}\label{the-day-manifests-gods-names-and-attributes}

The Day of Resurrection exemplifies and magnifies God's names and
attributes. The following demonstrates how any believer can explore this
connection using the framework established in Chapter 1.

\subsection{The Day as Unveiling}\label{the-day-as-unveiling}

The Day of Resurrection reflects Subtlety, as the Day and Hour is hidden
until it is revealed. ``The hour has come, the call has been raised, the
cry has appeared, and the mountains have passed, but the people do not
perceive it.''\footnote{BH00010, Letters to Tehran through Ali-Akbar}
This demonstrates the soul must not rely on the physical senses derived
from the body, but from its spiritual capacities.

The Day reflects Manifestation, as the Day unveils what was hidden. ``We
have opened the doors of the kingdom for you. Do you shut the doors of
your houses in My face? This is indeed a great error. Say: He has come
from the heaven as He came the first time.''\footnote{The Lawh-i-Aqdas
  (Most Holy Tablet) aka Tablet to the Christians} What was subtle is
fully manifested and truths which were once veiled become clear and
undeniable. The doors of the Kingdom, the spiritual worlds of God, are
open and revealed. This Manifestation has happened before, is happening
today, and will happen again tomorrow.

The Day reflects Knowledge, as the Day separates truth from falsehood.
``Thus does God teach you what will save you from doubt and perplexity
and deliver you in this world and the next.''\footnote{Kitáb-i-Aqdas
  \#163} Past misunderstandings are corrected, hidden wisdom is brought
forth, and those who seemed wise are humbled before God. Divine
knowledge is fully revealed.

The Day reflects Creation, as the Day is when a new reality is born.
``He created all things by His command and sovereignty and decreed for
every matter a time in His Book.''\footnote{BH00054: Unnamed Tablet
  about the annihilation of all things} With every time, the last leads
to the first, the end leads to the new. ``Everything will set down its
burden; how blessed is the sender of this favor that encompasses all
things, visible and hidden. This is how we created the cosmos anew
today''\footnote{Suriy-i-Haykal}

The Day reflects Power, as the Day establishes divine authority.
``\ldots the signs of the Resurrection, such as ``the splitting of the
sky,'' ``the breaking of the earth,'' ``the leveling of the mountains,''
``the rising of the dead from their physical graves,'' and similar
expressions that signify the signs of the Last Day.''\footnote{Kitab-i-Badi}
No force can resist the unfolding of the divine decree and God's power
is made evident.

\subsection{The Day as Judgment and
Renewal}\label{the-day-as-judgment-and-renewal}

The Day reflects Lordship, as the Day affirms God's Sovereignty. ``Thus
commands you the Lord of the Day of Judgment. Reside in the fortress of
My love.''\footnote{The Epistle to the Son of the Wolf} All past rulers
and authorities are weighed against this balance, even if they have
acted according to the law of yesterday. The Day confirms God's rule
over all of the worlds and every soul and being within them.

The Day reflects Justice, as the Day brings perfect judgment. ``Beware,
oh people, do not deprive yourselves of the seas of meanings and do not
follow every rebellious devil. Then look with the corner of holiness to
the balance of God, so that you may know His true, straight balance.
Say, today it is the right of every soul to purify its heart from
attachment to what was created between the heavens and the
earth.''\footnote{Suriy-i-Bayán (The Chapter of Paradise)} No injustice
will remain hidden and all souls will stand accountable for their deeds.
There are no exceptions. The balance is set and no soul will be wronged.

The Day reflects Exaltation, as the Day magnifies God's Glory (Baha).
``Today is a day that cannot be equaled by the previous centuries for
even a single hour. This most holy and exalted day has always been and
will forever be set apart and is referred to in the divine
books.''\footnote{BH00093 (Letters to the Land of Kha)} All of the
worlds bear witness. On the Day, God's supreme station is affirmed.

The Day reflects Independence, as the Day reveals that God needs no one.
``All things will appear from the possibility of the earth and will
return to Him. God will be alone and unique in His holy residence, free
from space, time, remembrance, statement, indication, description,
praise, exaltation, and diminution, and no one knows it but He and those
who have knowledge of the Book with Him.''\footnote{Kitab-i-Sultan} No
force could hasten nor delay the Day. It does not come from human
effort, nor from human interpretations and traditions.

The Day reflects Command, as the Day implements divine law. ``This is
the Day of Resurrection and the Lord of all worlds. This is the Day of
the Call and My immutable decree.''\footnote{BH00141 (Unnamed Tablet to
  Abu'l-Hasan)} Past laws and dispensations are weighed and fulfilled,
while God's Will is executed without resistance even if it seems He is
being resisted.

The Day reflects Counsel, as the Day reveals God's wisdom. ``I have not
ceased advising this creation and planning for their acceptance of God,
their Lord, and their faith in God, their Creator. And that they believe
in Him on the day of His Manifestation, all that is on earth. If so, My
being will be pleased, for everyone has reached the pinnacle of their
existence, attained the appearance of their Beloved, and realized the
possible manifestations of their purpose. Otherwise, My heart is not
saddened, for I have nurtured everything for that.''\footnote{Epistle to
  the Son of the Wolf} The wisdom can remove all veils, bring clarity to
all who attain this counsel. The Day is not merely a reckoning, but a
moment of divine instruction.

\subsection{The Day as Mercy and
Fulfillment}\label{the-day-as-mercy-and-fulfillment}

The Day reflects Faithfulness, as the Day fulfills God's promises.
``This is a reminder from Us to whomever turns to God and hears His
sweetest voice, which rose between earth and heaven, that they may give
thanks to their Lord, the Ever-Watchful, the Self-Subsisting. The
promise has been fulfilled, and the Promised One has come.''\footnote{BH00091}
Prophecies and promises reach their climax, assuring the faithful with
certainty that their devotion was not in vain.

The Day reflects Praise, as the Day inspires worship. ``O assembly of
sages in the land of Há, the Supreme Pen declares: Today is a day of
remembrance and praise, a day of praiseworthy qualities. God willing,
you should quench the thirst of the parched with the Euphrates of divine
mercy, and guide the homeless to their true homeland. With utmost spirit
and joy, bring glad tidings of the All-Merciful's favor to the
inhabitants of the world and arise to serve with wisdom.''\footnote{BH00141}
All throughout the worlds, such as the hosts of heaven, glorify God on
this Day. The Day reveals majesty, evoking praise with spirit and joy.
Even those who doubted are overwhelmed by awe.

The Day reflects Love, as the Day is an expression of divine love.
``This is a day on which fire speaks in all things; the Beloved of the
worlds has come.''\footnote{Suriy-i-Rais (Chapter of the Chief)} The Day
isn't merely about judgment, it is about God's love for His creation.
The fire which speaks in all things is the fire of
transformation,\footnote{The Tafsir-i-Hu (Tafsir of the Name He)} an
opportunity for every soul to embrace nearness to God.

The Day reflects Forgiveness, as the day offers redemption. ``If you
abandon your deeds and believe in Him in those days, perhaps Allah will
forgive your misdeeds. Indeed, He is the Mighty, the
Generous.''\footnote{Suriy-i-Haykal} It is a Day when barriers between
the soul and God are removed. Those who seek forgiveness, it is granted
to full measure.

The Day reflects Mercy, as the day is a triumph of divine mercy. ``the
effulgent lights of the dawn of meanings that have shone forth in these
merciful days.'\footnote{Kitab-i-Badi} Judgment is tempered by mercy and
even those who rejected the truth might still find grace.

The Day reflects Generosity, as the day bestows diving gifts. ``I
commanded the breezes of spring, and opened the gates of heaven, and the
clouds rained. Blessed is the one who succeeded in recognizing Allah in
his days.''\footnote{Lawh-i-Istinaq} Spiritual capacities are unlocked
for those who accept them. The gates of divine bestowal have opened.
This bounty demonstrates the Day is not merely about accountability, but
it is about bounty.

The Day reflects Sovereignty, as the day confirms who rules. ``The sighs
of the highest heaven rose, and the tears of the angels and the Spirit
descended. Say, if you have anything greater than what has come from the
Dominion of Will, bring it forth and do not follow every ignorant,
rejected one. Say, come so I may show you what you have neglected in
this day attributed to God, the Mighty, the Beloved. This is the day in
which the fragrance of the Merciful has spread, the breaths of
revelation have wafted, and the Nightingale of the Cause has sung upon
the branches. The kingdom belongs to God, the Master of
kings.''\footnote{BH00010 (Letters to Tehran through Ali-Akbar)} On this
Day, all those who claim power are shown their claims are false, as God
is the true ruler of existence. The Dominion belongs to none but God.

The Day reflects Purity, as the day confirms the absolute triumph of
truth. ``The Pen of the Most High declares in this Most Great
Manifestation: ``This is the Day of God, if you would but recognize it.
And this is the Day of Manifestation, if you would but witness it. On
this day, none is mentioned except God alone, if you would but perceive
it. He has come with the Truth, sanctified from all witness and
witnessed, and purified from all that has flowed from the Pen and
appeared from the tongue, if you would but know it.''\footnote{BH00074}
The day separates truth from illusion. No falsehood, corruption, or
impurity can remain. The Day is a testament of Oneness. It is a Day of
Unity.

The Day is therefore historical through the appearance of a new
Manifestation, and it is inward through the resurrection and
transformation of the soul.

\emph{Reflection: If the Day transforms both the world and the soul, how
does the soul move through the worlds of God when it recognizes the
Day?}

We now return to the worlds described in Chapter 3 and describe the Day
as a migration of the soul.

\section{The Day as the Soul's Migration through the
Worlds}\label{the-day-as-the-souls-migration-through-the-worlds}

The Day of Resurrection is the ``Great Return'' where the Human Soul
transcends the limitations of its private Self to navigate the Worlds of
God with newfound clarity. On this Day, the soul is resurrected from the
grave of the World of Nasut (the physical realm of shadows and
limitations) and is invited into the World of Malakut (the Kingdom of
signs). This is not a physical movement, but a shift in the mirror of
the soul. When the soul recognizes the Manifestation, it ceases to be a
suspicious soul lost in the headless state of doubt and becomes a
sanctified Mirror capable of reflecting the light of the World of
Jabarut (the Celestial Might). The Resurrection is thus the graduation
day of the soul, where the lessons learned in the School of Servitude
are finally applied as the soul enters the City of Submission.

Furthermore, the balance of the Resurrection serves as the bridge
between the soul's internal foundational virtues and its external
environment. As the soul moves through the worlds of God, the Day of
Resurrection acts as a spiritual climate shift. The breath of the
Merciful mentioned in the Kitáb-i-Aqdas transforms the atmosphere of the
soul's residence. The return is seen as the soul reclaiming its original
identity by aligning its attributes with the Divine Names manifested on
this Day. By drinking from the pure waters of reunion, the soul
effectively migrates from the isolation of the individual world into the
unified Kingdom of Affirmation. This transition ensures that the
Resurrection is not merely an external historical event, but an internal
advancement where the soul finally realizes its purpose across all the
worlds of God.

\emph{Reflection: If the first law of the Kitáb-i-Aqdas begins with
recognition, what does that recognition complete?}

\section{The First Paragraph of the
Kitab-i-Aqdas}\label{the-first-paragraph-of-the-kitab-i-aqdas}

In the first verse of the Kitáb-i-Aqdas, Bahá'u'lláh says

\begin{quote}
``The first thing that God has prescribed for His servants is the
recognition of the dawning place of His Revelation and the source of His
command, which is the station of His own Self in the realm of command
and creation.''
\end{quote}

The next two chapters will discuss belief in the Command of God (Chapter
5) and belief in the Manifestation of God (Chapter 6), which complete
the covenant of God. These two beliefs compromise the foundation of the
Kitáb-i-Aqdas, derived from belief in the Day of Resurrection. Belief in
these two things help reinforce belief in God, the station of the soul,
and the worlds of God. All of this is a cycle which is an endless loop
of self-discovery and transformation. All of this is embedded within the
very first law of the Kitáb-i-Aqdas. As you see, the Kitáb-i-Aqdas is
not merely a book of laws. It is the Most Holy Book. It is the
fulfillment of the Covenant of God.

Resurrection reveals the Day, Command establishes its authority, and the
Manifestation makes it known.

\newpage{}

\chapter{Belief in the Commands of
God}\label{belief-in-the-commands-of-god}

\section{Introduction: The Divine Command as the Source of
Order}\label{introduction-the-divine-command-as-the-source-of-order}

Belief in the Command of God is central to the Kitáb-i-Aqdas and the
entire revelation of Bahá'u'lláh. It has been a central tenet of all
religion. It is a direct reflection of God's names as Commander,
Fulfiller, and Ordainer. As established in Chapter 1, the name of God as
Commander becomes clear and actionable once we understand God as the
Just Reckoner, the Greatest Infallibility, and the Independent. The
Command is built upon nine prior layers of attributes before serving as
the pivot to the nine other layers that proceed from the Command. Let us
explore what the Command is and what it means for our liberation.

\section{Command of God Before and
After}\label{command-of-god-before-and-after}

``This is the command of God, before and after, and with it He has
adorned the scriptures of former generations. This is the mention of
God, before and after, with which He has embroidered the fabric of the
Book of Existence, if you are of those who perceive. This is the command
of God, before and after.''\footnote{Kitáb-i-Aqdas \#138}

Bahá'u'lláh affirms this is the religion of God from before and
after.\footnote{Kitáb-i-Aqdas \#182} The Book of Existence encompasses
all of creation, not just the Command. However, the Command is derived
from the Book of Existence and is the standard for creation. The Command
from before does not have everything within the Book of Existence,
neither does the current, and neither does the future Command.
Bahá'u'lláh says ``Regarding the question about the heavenly names, the
pulse of the world is in the hands of the knowledgeable physician. He
sees the ailment and with His wisdom, He treats it. Each day has its own
secret, and every matter has its own sound. The remedy for today's pain
is one thing, and for tomorrow, another. Be mindful of today and speak
of today's matters.''\footnote{Tablet of Mánikchí Ṣáḥib}

Basically there are many chapters in the Book of Existence, an eternal
reality which precedes and outlasts creation itself. God does not
promise any chapter will be the only chapter forever, but through the
covenant expressed within each chapter, there is also the promise of
future chapters which are more adapted to the needs and aspirations of
humankind. Not everything in a chapter is completely new, but each prior
chapter establishes the foundations for the next. Any archaeologist will
testify history is built in layers. Religion is a part of this history.

\subsection{Revelation}\label{revelation}

Each chapter of the Book of Existence is unveiled through divine
revelation, adapting to the spiritual needs of each age. Revelation is
the disclosure of truth or knowledge from God. It is the process by
which God's command is unveiled, renewing itself in every age. Given one
of God's attributes is Subtlety, it could be difficult for a person to
believe in revelation when it happens. There are many traditions out
there which teach their revelation is the final revelation for the rest
of time. There are other traditions where it is taught because the
recent revelation is true, prior revelations could not have been true.

This is the Revelation around which evidence and proof
circle.\footnote{Kitáb-i-Aqdas \#136} God does what He wills and decrees
what He desires. This is a test of faith for many. ``They do not
consider the manifestations of divine Oneness impossible in the realm of
dominion, and if a soul deems it impossible, what difference is there
between him and those who considered God's hand to be tied? And if they
regard God, exalted be His mention, as free to choose, they must accept
every command that emanates from the source of His decree, without
escape or refuge. There is no sanctuary or shelter except in Him. The
matter that requires presenting evidence and proof of a claim is not
contingent upon the opinions of people, whether knowledgeable or
ignorant, and it never has been and never will be.''\footnote{Súriy-i-Haykal
  Verses 241 \& 242} Revelation itself is the proof of divine authority,
not whether people accept it.

The Kitáb-i-Aqdas describes this revelation as the most exalted and most
wondrous. ``In every subsequent manifestation, the appearance of God is
the most exalted.''\footnote{The Persian Bayán, Váḥid 4, Gate 12} This
does not take away from the station of prior revelations, and future
revelations will also be most exalted and wondrous when compared to this
current one.

\subsection{The Book}\label{the-book}

Throughout the Kitáb-i-Aqdas, Bahá'u'lláh refers to the Book.
Bahá'u'lláh uses the word Book to describe various aspects of creation
and the knowledge God has. Often times in the Kitáb-i-Aqdas, the Book
without any other description refers to the Kitáb-i-Aqdas itself. ``This
blessed verse was mentioned: His Exalted Majesty said, ``Refer what you
do not understand from the Book to the branch that has branched from
this mighty root.'' The `Book' refers to the Kitáb-i-Aqdas, and the
`branch that has branched off from this mighty root' refers to the
Aghṣán.''\footnote{A Tablet to Varqá (Letters to Yazd) and BH00057}

Belief in the Command of God means believing in the Kitáb-i-Aqdas as the
Book, the Most Exalted and Wondrous Revelation, and an update in the
single religion of God. The Kitáb-i-Aqdas is central to the continuity
of God's favor upon us, the latest expression of Divine Will. It is a
Mercy for us and a proof of divine authority. It is the source of true
freedom. The Kitáb-i-Aqdas is all of this and a source of the divine
ordinances and laws. The rest of the chapter will expand upon these
themes further.

\emph{Reflection: Do I treat divine command as an intrusion, or as an
expression of love that gives shape to life?}

To see why the Command matters now, we must first see how it has always
moved through history.

\section{The Continuity of Divine Command Across
Dispensations}\label{the-continuity-of-divine-command-across-dispensations}

When the King of Austria (a Christian) visited the al-Aqṣá Mosque
(Muslim) in Jerusalem (Jewish Holy Place), Bahá'u'lláh tells him he
passed by the Light of Divine Unity coming from the prison of
'Akká.\footnote{Kitáb-i-Aqdas \#85} There was an opportunity to
experience the Oneness of God. There was an opportunity to experience
the continuity of divine command across revelatory dispensations. The
King had an opportunity to believe in unity. Believing in the command of
today would mean believing in the command of yesterday. It encompasses
all prior commands.

The prior commands and prior books span across history, but not all
words claiming to be God's book is God's book. For example, Bahá'u'lláh
says the Qur'án, the Gospel, the Psalms, the Torah, and the
Bayán\footnote{The Kitáb-i-Badí`} are revelations from God. If we look
at the Bible, the Gospel would include the words and actions of Jesus
Christ as documented in the Gospels of Matthew, Mark, John, and Luke.
This does not include the sermons of Paul who wrote or inspired the
writings of Acts, Romans, and other New Testament books. There is a
distinction in the command coming through a Manifestation of God versus
the inspirations of the leader immediately following the Manifestation.
Paul was not the Word but was subservient to the Word. Elsewhere,
Bahá'u'lláh testifies to the truths of Zoroastrian and Hindu teachings
as being relevant to the relative position of people in time and
condition.\footnote{Tablet to Mánikchí Ṣáḥib} This command could be
inclusive of all cultures which discuss a deity, worship, and social
guidance. Belief in the command does not necessarily mean spending a
great effort determining what exactly from the past was from God and
what was from the inspirations of normal people like you and I. Belief
in the command means to acknowledge God is the source of all truth
before, during, and after today. Bahá'u'lláh warns in the Kitáb-i-Aqdas
``The books of the world and all the scriptures therein avail you
nothing today, except by this Book, which speaks at the pivot of
creation.''\footnote{Kitáb-i-Aqdas \#168} Through the name of God
All-Encompassing, the Command of Today encompasses and surpasses all the
Commands of Yesterday. This is a vital belief on the path of Divine
Unity.

\subsection{The Bayán as Paradise Awaiting Its
Lord}\label{the-bayuxe1n-as-paradise-awaiting-its-lord}

The Persian and Arabic Bayán were revealed by the Báb around 1848.
Together they are known simply as the Bayán, whose very name means
``Paradise.'' It reveals how divine Command flows from dispensation to
dispensation. He revealed only 11 Unities in the Arabic, and the Persian
ended at Unity 9 Gate 10. The Arabic Bayán consists of short commands,
while the Persian expands on those same commands with the deeper
theology of why, explaining why paradise awaits its Lord.

The Báb structured His revelation with intentional incompleteness. It
was designed for 19 Unities of 19 Gates (361 sections total). Yet what
paradise could be complete without its Lord? It is the Persian Bayán
which prepares a garden whose gates would only open when ``He Whom God
Shall Make Manifest'' arrived. From beginning to end, the Bayán serves
less as a final destination than as a prepared dwelling, a paradise
awaiting its Beloved. The Báb warned His followers not to use His own
book as grounds for rejecting the Promised One, knowing that those who
clung to the garden's gates might miss the Lord of the garden Himself.
This makes the Bayán unique. It is scripture that teaches believers how
to move beyond itself, a paradise that exists primarily to welcome its
Sovereign. The Báb says: ``I counsel all followers of the Bayán: If, at
the appearance of Him Whom God will make manifest, you are all granted
success to attain the greatest paradise and the supreme encounter, then
blessed are you, thrice blessed!''\footnote{Persian Bayán Váḥid 2, Gate
  16}

The laws of paradise are not complete until the Master of paradise gives
them life. The Báb declared: ``In Him Whom God will make manifest, all
the ordinances of the Bayán are fulfilled. He is none other than the
Point of the Bayán.''\footnote{Persian Bayán Váḥid 4, Gate 7} Every
Bayánic law was like a seed planted in paradise's soil, not because the
Báb's authority was incomplete, but because divine wisdom knows which
seeds must grow and which must transform when the divine Gardener
arrives. When Bahá'u'lláh appeared, He both honored the Bayán's
preparation and cultivated its garden according to the season's needs,
bringing some plants to flower while planting new varieties for a
maturing world. The pattern is clear. Command flows continuously from
age to age, each paradise preparing the ground for its successor, each
garden teaching souls how to recognize the next Gardener. This is why
recognition of the Manifestation and belief in His Command are
inseparable. For what use is paradise without the One for whom it was
made?

``The fruit of this is that, on the Day of Resurrection, no soul should
remain veiled from the commands of He Whom God shall make manifest. For
if He commands over all existence, His command is the command of
God---yesterday, today, and forever.''\footnote{The Persian Bayán Váḥid
  8, Gate 4} The Lord of paradise has arrived. Let us turn to the
Kitáb-i-Aqdas.

\subsection{The Most Great Command}\label{the-most-great-command}

Bahá'u'lláh describes the Command of today as the Most Great Order, the
Most Great Law, the Most Great Proof, and the Most Great Balance. He
says ``The order has been disturbed by this Most Great Order, and the
arrangement has been altered by this wondrous Revelation, the like of
which the eye of creation has never witnessed.''\footnote{Kitáb-i-Aqdas
  \#181} This is a testament to the fact there are good things from the
past worth keeping, which can provide a positive contribution to the
future. It is also a testament to the fact there is a need for the old
to be rearranged and altered in a new and substantial way. Progress
cannot happen if we hold onto the old ways of doing things. Progress
requires a belief that God's religion is progressively unfolding and
adapting to the needs of different ages.

When the Kitáb-i-Aqdas declares that `the order has been disturbed by
this Most Great Order', it signals an integrated design rather than a
collection of isolated commands. In this work, I attempt to map that
integrated Order as I perceive it within the text. The names and
structures I employ are tools of reference and indexing; they are not
claims of authority. Other mappings are possible.

The Most Great Proof testifies to the truthfulness of this claim. ``This
is the Balance of Guidance for those in the heavens and on the earth,
and the Most Great Proof, if only you knew. Say: Through it, every proof
has been established throughout the ages, if only you were certain. Say:
Through it, every poor soul has been enriched, every learned one has
been taught, and whoever desires to ascend unto God has been lifted up.
Beware lest you differ concerning it.''\footnote{Kitáb-i-Aqdas \#183}
The proof can be witnessed through the development of the ages, such as
through the Islamic Golden Age or the rise of powerful Hindu kingdoms
throughout Southeast Asia. New revelation brings new prosperity for
those who believe and implement the Most Great Command. Just as the
Islamic Golden Age ended and the Hindu Kingdoms of the Khmer gave way to
Buddhist cultures, we must also allow God to continue guiding us to
better ways of living and being.

This is because of the Command being the Most Great Balance and Most
Great Law. The balance cannot be measured by prior rules and sciences
which existed yesterday.\footnote{Kitáb-i-Aqdas \#99} The Balance
fulfills the prophecies of old,\footnote{Lawḥ-i-Ishrāqāt (Tablet of
  Splendors)} the Covenant of God, and is the balance of justice. When
the balance of justice is set up on the Day of Resurrection, everyone
will be given their due.\footnote{Súriy-i-Mulúk (Súriy-i-Mulúk)} Who has
status and honor yesterday may not have status and honor today. A law
which was raised to the heights of heaven yesterday may be annulled
today. The Command is both immutable and variable depending on what we
need. The Most Great Law is for all in heaven and on earth,\footnote{Kitáb-i-Aqdas
  \#6} not just for a select few. It is the standard of judgment for all
today, not only for tomorrow.

``Everything is realized by His decisive command when the Sun of
ordinances rises from the horizon of explanation. All are to follow it,
even if it be a command that rends asunder the hearts of the adherents
of religions. He does as He wills and is not questioned about what He
wills.''\footnote{Kitáb-i-Aqdas \#7}

\emph{Reflection: Do I recognize divine authority as living guidance, or
do I only accept it when it agrees with what I already prefer?}

If the Command is truly from God, it must also illuminate the mind and
awaken understanding.

\section{The Command is Manifestations of
Light}\label{the-command-is-manifestations-of-light}

The Kitáb-i-Aqdas utilizes many forms of symbolism to describe the
Command and Revelation of God. Many of these names and symbols focus on
how the Command manifests the Light, such as is associated with the Day
of Resurrection. The Command is the Sun of Wisdom and the Sun of
Explanation. Within the laws and counsel, there are pathways to
understanding, the application of knowledge, and explanations and
interpretations.

The Sun emerges from the Sidrat al-Muntahá at dawn. The Command is the
dawning place of the knowledge of God and is adorned with the Seal of
the Breaker of the Dawn, who speaks between the heavens and the
Earth.\footnote{Kitáb-i-Aqdas \#117} This affirms the Command is
authentically divine and a true pathway away from darkness. Just as God
had bestowed upon Muhammad the title Seal of the Prophets, the seal
authenticates what came prior, it also unseals divine mysteries. As a
seal could also denote finality, we know the Sun and the command are
unalterable. They are forever true, even when the Sun arises again in
the future. As the Sun rises for the new Day, the Command can also be
known as the Book of Origin.\footnote{Kitáb-i-Aqdas \#121} While final
and unalterable, it is also new and as fresh as the first days of
springtime.

Even outside of the revelation being the dawn of a new day, the command
also serves as a lamp, lighting a pathway for us on our spiritual
destination. The Kitáb-i-Aqdas describes the command as a Lamp of God's
Care, a Lamp of Wisdom, a Lamp of Success, and a Lamp of Eternity. These
lamps serve to guide us, providing direction, nurture, and a promise of
success for our souls. Much like how God led the Jews from Egypt to the
promised land of Canaan, the Kitáb-i-Aqdas will lead people to their
spiritual destiny.

``This is a Book that has become the Lamp of Eternity for the world and
its most upright Path among the peoples. Say: It is the Dawning-Place of
the knowledge of God, if you but knew, and the Rising-Point of the
commandments of God, if you but recognized.''\footnote{Kitáb-i-Aqdas
  \#186}

\emph{Reflection: Do I see the Command as something that forms me, or as
something I merely study from a distance?}

Light is not only meant to be seen, but to be walked in.

\section{The Command Nurtures Us}\label{the-command-nurtures-us}

``O Greatest Sea! Sprinkle upon the nations that which you have been
commanded by the Ancient Lord, and adorn the temples of humankind with
the fabric of the laws through which hearts may rejoice and eyes be
brightened.''\footnote{Kitáb-i-Aqdas \#96}

When you look at a fabric, you can see the finest fabrics have a high
density of interwoven threads which provide strength, durability, and
protection. The fine fabric provides comfort and beauty, being able to
help a person uniquely express their own personality and character. The
fabric of laws serves a similar purpose. They are to adorn the temple of
humankind, the same temple used to elevate the souls to the heights of
heaven. One must beware not to remove threads from the fabric. The
fabric could seem weak and dull. It will not cause hearts to rejoice nor
brighten the eyes of those who witness it. All of the laws are like an
ornament\footnote{Kitáb-i-Aqdas \#98} helping others to recognize the
beauty of the Command.

``From My laws, the fragrance of My garment is diffused.''\footnote{Kitáb-i-Aqdas
  \#4} Bahá'u'lláh expresses fragrance regularly in the Kitáb-i-Aqdas
and elsewhere. Fragrance is a pleasant smell emanating from an object,
in this case God's garment. The laws are a source of this diffusion. As
a pleasant smell attracts a lover to their beloved, or a bee to a
flower, the laws can also attract the hearts of a spiritual seeker.
``Blessed is the lover who has inhaled the fragrance of the Beloved from
this word, from which the breezes of bounty have wafted in an
indescribable manner. By My life, whoever drinks the nectar of justice
from the hands of grace will circle around My commands which have shone
forth from the horizon of creativity.''\footnote{Kitáb-i-Aqdas \#4} The
nectar is also described as the nectar of life.\footnote{Kitáb-i-Aqdas
  \#150} Anyone who wears this fabric of laws will be examples of a
loving and nurturing justice and life.

Bahá'u'lláh tells us not to consider the Command and the Kitáb-i-Aqdas
as merely laws, but as the choice sealed wine.\footnote{Kitáb-i-Aqdas
  \#5} Drinking from this wine, much like how the disciples of Jesus did
in the last supper, will cause every bone to be set in motion with
life.\footnote{Kitáb-i-Aqdas \#173} Imagine the fragrance of such a
wine. Who would not want to inhale it and drink it? Who would not want
others to inhale it and drink it?

The Command is also often referred to in terms of water. Water has
always been considered a purifying element. The Báb says ``it symbolizes
the radiance of the Sun of His bounty.''\footnote{The Persian Bayán
  Váḥid 5, Gate 14} Those who act in accordance with the Command are
drinking from the Kawthar (abundant river from paradise) of
life.\footnote{Kitáb-i-Aqdas \#73} These pure and flowing waters from
the clear stream provide prosperity when drunk with the belief and
remembrance of God.\footnote{Kitáb-i-Aqdas \#50} The water may be
provided as showers of grace\footnote{Kitáb-i-Aqdas \#55} poured down
from heaven as a favor from God to us. These waters act as a salvation
and a great gift to us.

We are also nurtured by the Command coming forth from the Most Great
Ocean. Within the ocean are pearls of knowledge and wisdom.\footnote{Kitáb-i-Aqdas
  \#180} The ocean serves as a great metaphor for the bounties of God.
Depending on the depths one is able to dive to, there is incredible
amounts of knowledge, wisdom, and mysteries waiting to be attained.
These pearls are valuable, but as they can only be had through diving,
it requires effort and work to reach. These are not just free gifts.

``Beware that compassion does not prevent you from carrying out the laws
of God. Act according to what you have been commanded by a compassionate
and merciful Lord. We have nurtured you with the whips of wisdom and
laws for your own protection and the elevation of your station, just as
parents nurture their children.''\footnote{Kitáb-i-Aqdas \#45} As this
nurture is from God in the role of Divine Parent out of love for us, we
must also ``carry out My ordinances out of love for My
beauty.''\footnote{Kitáb-i-Aqdas \#4}

Belief in the commands of God is belief in the love of God. Belief in
the commands of God also requires a love for God and the beauty of God's
creation.

\emph{Reflection: Do I mistake freedom for the absence of restraint, or
do I see freedom as the power to live with dignity and purpose?}

To understand liberation, we must first understand what the Command is
protecting.

\section{The Command Provides True
Liberation}\label{the-command-provides-true-liberation}

The laws of the Command serve mostly as bounds.\footnote{Kitáb-i-Aqdas
  \#148} These boundaries serve as a framework whereupon the soul may be
elevated and the world within which we live can better manifest the
spiritual qualities of heaven. This does not mean we should expect some
type of fraudulent utopia. All of this requires work and requires skill
navigating these boundaries. These boundaries are not particularly
strict and are the cause of absolute freedom.\footnote{Kitáb-i-Aqdas
  \#125}

In this small sermon about freedom from the Kitáb-i-Aqdas, Bahá'u'lláh
says:

\begin{quote}
123 Freedom ends in consequences that lead to discord, whose fire cannot
be extinguished---thus informs you the Reckoner, the All-Knowing. Know
that the sources and manifestations of freedom are found in animals. For
humanity, it is necessary to be under laws that protect them from the
ignorance of their own selves and the harm of the deceitful. Freedom
removes a person from the realm of courtesy and dignity, reducing them
to the lowest of the low.

124 Observe the people; they are like sheep, in need of a shepherd to
protect them---this is indeed an absolute truth. We affirm this in
certain contexts but not in others, for We are all-knowing.

125 Say: True freedom lies in following My commandments, if you are of
those who know. If people were to follow what We have revealed to them
from the heaven of divine revelation, they would find themselves in
absolute freedom. Blessed is the one who understands the purpose of God
in what has been revealed from the heaven of His will, which reigns
supreme over all the worlds. Say: The freedom that benefits you is found
in servitude to the True God, and whoever has tasted its sweetness will
not trade it for the kingdom of the heavens and the earth.
\end{quote}

True freedom is derived from the protection the Command provides. The
command is a key to God's mercy,\footnote{Kitáb-i-Aqdas \#3} helping us
unlock new doors of realization and possibility derived from God's love.
These possibilities from true freedom are not solely for the individual.
``The ordinances of God are the greatest means for the order of the
world and the preservation of nations.''\footnote{Kitáb-i-Aqdas \#3} The
scope of the Command is great. The Command is not merely for individual
belief, but is also designed for the order of the world.

Within this wider scope, the Command establishes a pattern of
trusteeship rather than domination. Just as a shepherd is accountable
for the care, safety, and nourishment of the flock, authority under the
Command carries responsibility before it carries power. The sheep are
not owned as objects but entrusted as lives, each with dignity, limits,
and protection owed to them. In this sense, the Command defines both
rights and responsibilities at once: the right of souls and communities
to be safeguarded from harm and confusion, and the responsibility of
those who act in God's name to serve, preserve, and guide without
exploitation. Order is therefore not imposed for its own sake, but
upheld as a trust, measured by care, justice, and fidelity to the
purpose for which the Command was given.

``\,`There is none other God but Me, the One, the All-Knowing, the
All-Informed.' This is a station God has designated for this most
wondrous, most exalted Revelation. This is from the grace of God, if you
are of those who know. This is from His irrevocable command, His
greatest name, His supreme word, and the dawning place of His most
excellent names, if you are of those who understand. Indeed, by it, all
the rising and setting points are made manifest. Reflect, O people, on
what has been revealed in truth, and ponder it, and do not be of the
transgressors.''\footnote{Kitáb-i-Aqdas \#143}

\emph{Reflection: Do I want the benefits of divine order without
accepting the trust and responsibility that make it possible?}

To close this chapter, we can gather its images and claims into a single
view of what the Command is, and why it liberates.

\section{Conclusion}\label{conclusion}

With the Kitáb-i-Aqdas, the Command of God has been made new. It is more
than laws. It is fabric adorning souls, fragrance attracting seekers,
wine enlivening bones, water purifying hearts, and light guiding
humanity. Command is not restriction but revelation, not burden but
bounty, not constraint but freedom's foundation.

Belief in the Command reflects belief in God, the soul, and the worlds
of God. The Command flows from the Book of Existence like eternal
chapters. Each complete in itself, each preparing for what follows. From
Torah to Gospel, from Gospel to Qur'án, from Qur'án to Bayán, from Bayán
to Aqdas, paradise prepares for its Lord.

Yet Command cannot exist without Commander. To believe in the Command is
to believe in the unfolding of God's will, to see divine order in all
things, and to recognize that belief in the Manifestation of God Himself
is the foundation upon which all Command rests.

\newpage{}

\chapter{Belief in the Manifestation of
God}\label{belief-in-the-manifestation-of-god}

\section{The Station of the
Manifestation}\label{the-station-of-the-manifestation}

In Chapter 1, we explored the various names and attributes of God to
help understand what belief in God can encompass. The second group is
Manifestation. The ability for God to manifest Himself to His creation
is fundamental for us. Manifestation turns the inaccessible and subtle
into more tangible and easier to discern aspects of reality. One of the
ways God manifests Himself is through the human flesh. While this flesh
is bound by the material world, the soul within it reflects divinity in
such a way the other names and attributes of God also appear. While the
historical terms of prophet, messenger, avatar, or apostle help us
understand aspects of this divinity and purpose, the Manifestation
serves as the living bridge between our human soul and divine reality.
We also explored in Chapter 3 the symbol of the Sidrat al-Muntaha, the
tree of the farthest extent. The Manifestation of God serves as the
Dawning Place, a title used regularly throughout the Kitáb-i-Aqdas. The
Manifestation is the Dawning Place of Revelation, of the Light of
Divinity, of God's Most Excellent Names, of Grandeur, of God's Most
Radiant Cause, and of Oneness. This Dawn itself marks the fulfillment of
the Day of Resurrection. ``By Him the Hour has come.''\footnote{Kitáb-i-Aqdas
  \#81} He appears ``in the Most Glorious attire''\footnote{Kitáb-i-Aqdas
  \#82} as if He were a Monarch. The Manifestation is also the means by
which the human soul is able to understand God's will, purpose, and
those other names and attributes more fully. ``Whoever has turned to Me
has turned to the Worshiped One.''\footnote{Kitáb-i-Aqdas \#131} The
Manifestation is the focal point for the devotional life for God, much
like the Qiblih of Islam or the Temple of Judaism. Thus, all
Manifestations of God are the Sidrat al-Muntaha, with each being a
branch of this divine tree.

This whole purpose of the Manifestation is entirely for our soul. The
belief in the soul, as described in Chapter 2, requires belief in the
Manifestation as the primary means of progress towards God. They serve
not only as the Dawn for the worlds of God, but they also serve as the
Dawn of the transcendence of the soul as it prepares to navigate the
spiritual worlds of God. We express this belief by recognizing the
Dawning Place of Grandeur, their exalted nature over everything else on
Earth, and their high spiritual capacity and fragrance. We express this
belief by recognizing their Revelation, which is the Word of God, the
Command and Counsel which guides and organizes all with justice and
mercy. We practice this through submission and love, in fear and in awe,
in faithfulness and good deeds.

\emph{Reflection: What changes in your daily life when you treat
Revelation as the place where the hidden becomes knowable?}

This is where we step from God's attributes into God's appearance in
history.

\section{Countless Adams}\label{countless-adams}

This station of Manifestation has been fulfilled by many throughout
history. The Kitáb-i-Aqdas nor Bahá'u'lláh's writings describe all the
Manifestations throughout history, but God has spoken to the people of
the world at different places and at different times. The Báb had said
``From the appearance of Adam to the first manifestation and the Point
of the Bayán, only 12,210 years of this world's time have passed. Beyond
this, there is no doubt that God has created countless Adams, whose
number is known to none but Him.''\footnote{Persian Bayán Vahid 3 Gate
  13} The Adam described in the Book of Genesis of the Hebrew Torah
would have lived around 10,360BC, just as the world was emerging from
the last Ice Age and humankind was starting to transition to a more
agrarian lifestyle. There could have been countless Adams. These Adams
could have emerged in various places throughout the world to establish
the foundations of human civilization. A Garden of Eden would represent
the emergence of plants being cultivated, a Garden which emerged in
newly forming agrarian societies. These Adams could have also helped
guide mankind as we look back beyond 10,360BC. While Genesis says Adam
was the first man and Eve was the first woman, it is definitely clear
Adam represented a dawn of a new era of humanity from the Paleolithic
and Mesolithic periods into the Holocene. Humankind had emerged into a
new creation, where God was guiding the development of us throughout the
world.

\emph{Reflection: When you read sacred origins, do you look for a single
date, or for a single turning point in the soul of humanity?}

With that widening of history, we can return to the most immediate Dawn
for our age.

\section{The Báb, The Primal Point}\label{the-buxe1b-the-primal-point}

This brings us to the Báb. He was known by the various titles in the
context of Islam, particularly in Shí`a Islam such as the Qa'im.
However, there are two names and titles which I want to bring particular
attention to, given these two names are rather universal in nature. The
first title is the Primal Point.

My understanding of the Primal Point is that it signifies the ultimate
unity and reality of all things. It is the essence of the oneness in its
most pure form, if you recall Purity and One as the pinnacle of
understanding the Essence of God. The Primal Point is the point where
all creation returns to and from all creation emerges. I think of the
Primal Point as being as if a black hole, where all energy and mass
return to. I also think of the Primal Point as being as if the Big Bang,
where all energy and mass expand from a point and all reality of the
Universe emerges from. In this context, creation ended one cycle and a
new cycle emerged.

The most common name is the Báb. The word Báb means the Gate. A Gate is
never the ultimate destination. It is the entrance we pass through as we
arrive to our destination. In this context, the Gate is where we leave
one space, Adam and the Holocene, and emerge in a new space, Bahá'u'lláh
and the emerging Anthropocene epoch. The Primal Point and the Báb are
two distinct ways to express a complex but singular reality. The Báb
relates to this theme in the Persian Bayán by saying:

\begin{quote}
``The names and attributes are manifestations of the multiplicity of
that primal unity. Reflect upon the verbal letters of the Bayán: all
multiplicity originates from the first unity, even if it extends
infinitely. And in the multiplicity of the universal manifestations,
there arises a strength in the manifestation surpassing that of the
primal unity. Yet, all things are realized through Him, and all return
to Him, just as they originate from Him.''
\end{quote}

Even if we look at traditions which seem polytheistic, they are not. All
are Manifestations of God, expressions of God's attributes according to
the needs and cultures of their time.

\emph{Reflection: What is the difference between a destination you
possess, and a Gate that changes who you are as you pass through it?}

That Gate leads directly to the pattern the Báb placed at the center of
His Revelation.

\section{The Vahid, The Number 19}\label{the-vahid-the-number-19}

The Báb placed significant importance on the number 19. Some tie the
origin of its significance to the Qur'án's Surah al-Muddaththir, where
the Seal of the Prophets describes 19 angels who guard the gates of
Hell. The number is a marker of faith in God. The audience to whom the
Seal of the Prophets was speaking to in the early Meccan period was most
likely the Jewish population of Mecca, who observed the lunisolar
Metonic cycle. This cycle represents the 19 year cycles where the lunar
phases repeat at the same solar periods. Today's Easter observances
follow this Metonic cycle. Other ancient cultures also observed this
cycle, such as the Polynesians.

Mathematically 19 is a prime number, which is indivisible by any other
numbers. Within it, it includes the number one and the number 9, which
is the highest single digit whose value is inclusive of all other single
digits. In this context, 19 can represent the Bayánic verse of unity and
multiplicity shared in the last section. The Báb makes regular use of
the number 19 throughout the Bayán and other writings. As you recall,
the Bayán's intended structure was to be 19 Vahids of 19 Gates, 361
total Gates. A Vahid, using the Arabic abjad system of numerology,
represents the number 19. Vahid literally means unity. Thus the Bayán
would represent 19 separate unities, each multiplicities of 19, which
comprise one Paradise (Bayán). There are 19 gates to pass through for
each unity, much like the Báb Himself was a Gate.

When we refer back to the Báb's ministry, the first believers were
called the Letters of the Living. There were 18 of them and together
with the Báb, were 19 in total. Each had served as gates in their own
way and ushered in a period of considerable upheaval in Persia. All but
one were executed for apostasy from Islam, much like what the Qur'án was
alluding to in Surah al-Muddaththir. Before He was executed, He had
implemented the Bayánic Calendar of 19 months of 19 days and as you will
see throughout this book, 19 is a common feature of this cycle. How does
Bahá'u'lláh represent the Vahid, the Unity of the Number 19? Let's
explore who Bahá'u'lláh is and the connection to the Báb's Revelation.

\emph{Reflection: Where in your life do you feel unity most strongly,
and what pattern helps you return to it when multiplicity overwhelms
you?}

From the pattern of nineteen, we now turn to the Glory that fulfills it.

\section{Bahá'u'lláh, The Glory of
God}\label{bahuxe1ulluxe1h-the-glory-of-god}

Bahá'u'lláh is the most recent of these Manifestations of God who has
brought forth Revelation as their divinely ordained mission. Belief in
Bahá'u'lláh is belief in all the prior Manifestations of God, belief
their words and causes were true, but also belief that the prior
Manifestation's teachings are now superseded by Bahá'u'lláh's
revelation. It is the dawn of a new day with a new command, which will
serve us until the dawn of the next day with a next Manifestation.

In Bahá'u'lláh's writings, some Manifestations are referred to by
certain titles. For example, Jesus is often referred to as the
Spirit\footnote{Kitáb-i-Aqdas \#80} and Muhammad as the Messenger of
God.\footnote{The Súríy-i-Ra'ís (Súrah to the Chief)} In the Bayán, the
Báb had said everything in the Bayán is for He Whom God Shall Make
Manifest (HWGSMM), a title used at least 240 times in the Bayán.
Bahá'u'lláh refers to this title in the Kitáb-i-Aqdas \#137. This helps
signal to the Bábí community the station Bahá'u'lláh is claiming. HWGSMM
was said to be one to complete the Bayán.\footnote{The Testament of the
  Báb to Subh-i-Azal} The Kitáb-i-Aqdas serves as the completion of the
Bayán, with some laws abrogated and some laws confirmed. As we go
through this book, we will sometimes refer to these changes or instances
where a command from the Bayán might still apply. Understand though,
that while the Kitáb-i-Aqdas also refers to the Báb as Bahá'u'lláh's
Herald,\footnote{Kitáb-i-Aqdas \#137} the Báb was completely a
Manifestation of God. Bahá'u'lláh is also a Herald for a future
Manifestation of God who will come no earlier than 2873AD (1029 Badíʿ
Calendar).\footnote{Kitáb-i-Aqdas \#37}

Bahá'u'lláh was often referred to as the Greatest Name.\footnote{Kitáb-i-Aqdas
  \#51} The Greatest Name is reference to various Muslim traditions. In
one way, God can elevate any of His names to be the greatest, as
ultimately all names come from the word ``One''. In this dispensation,
the Greatest Name is Baha, which means Glory. Much like how Jesus was
given the name Son, these names are used to demonstrate a primary
purpose in the mission of the Manifestation. The Gospel of John says the
Gospel exists so all may become sons of God, so thus Jesus's mission was
to teach Sonship of God the Father. Simultaneously, Krishna means
all-attractive and His mission was to demonstrate what true devotion to
God and knowledge of the true Self, which are all manifestations of
Beauty and Love. The purpose of Bahá'u'lláh's mission then, is for our
souls to embody and believe in the quality of Glory. The revelation is
to demonstrate in every way the exaltation of God, His Majesty over all,
and only He is the Infallible. It is also our purpose to reflect this
Glory to the best of our abilities, which helps make the world around us
more beautiful, more magnificent, and more illustrious. This purpose
teaches us we all deserve these attributes and are also able to manifest
them, if we so believe. This Glory reigns supreme over all the
worlds.\footnote{Kitáb-i-Aqdas \#127}

The Kitáb-i-Aqdas does not go too deeply into the biography of
Bahá'u'lláh, which may matter to some. Belief in Bahá'u'lláh as the
Manifestation of God does not require knowledge of the biography, but
there are some aspects about Him which are mentioned. The Book says He
is unlettered,\footnote{Kitáb-i-Aqdas \#104} saying ``I have not entered
schools, nor have I studied scholarly works.'' Despite this, He is
unmatched in the fields of mystical insight and knowledge and none can
keep up with Him in the course of wisdom and expression.\footnote{Kitáb-i-Aqdas
  \#101} He Himself had entered various Schools of God.\footnote{Kitáb-i-Aqdas
  \#175 and \#176} Bahá'u'lláh first had His epiphany of station while
within the prison called the Black Pit of Tehran in the year 1852 (8
BE). He says this in the Súríy-i-Haykal:

\begin{quote}
So when I saw myself at the pole of affliction, I heard the most
beautiful, the sweetest voice above my head. When I turned, I saw a
maiden, the mention of my Lord's name suspended in the air, aligned with
the sun's head, and I saw her rejoicing in herself. As if the embroidery
of pleasure appears from her face and the light of mercy declares from
her cheek. She was speaking between the heavens and the earth with a
call that attracts hearts and minds and gives all my limbs, both outward
and inward, the good news that my soul was rejoiced with, and honored
servants rejoiced from it.

She pointed with her finger at my head and addressed those in the
heavens and the earth, by God, this is the beloved of the worlds, but
you do not understand. This is the beauty of God among you and His
authority within you, if you indeed know. This is the secret of God and
His treasure and the command of God and His dignity for those in the
dominion of command and creation, if you indeed comprehend. Indeed, this
is the one whom those in the realm of permanence long to meet, then
those who have settled behind the most splendid pavilion, but you turn
away from his beauty.
\end{quote}

In Baghdad, He revealed His mission during the festival of Ridván in
1863 (19 BE), ``when We manifested to those in existence with Our most
beautiful names and highest attributes.''\footnote{Kitáb-i-Aqdas \#75}
This mission began 19 years after the Báb began His ministry in May 1844
(0 BE). It continued until His passing in 1892 in 'Akká, 19 years after
Bahá'u'lláh revealed the Kitáb-i-Aqdas. The 10 year period from 1863 to
1873 was much like a pivot period, where the Bayán was still the Command
of God, but Bahá'u'lláh was declaring His mission and station to
believers of the Báb and prior Manifestations, to leaders in Asia and
Europe. The Command emerged through writings such as the Súríy-i-Haykal
(The Súrah of the Temple) which addresses four monarchs and the Pope,
and the Súríy-i-Mulúk (addressing Kings in general). The Bayán, which
was an incomplete paradise and like a Gate, was to be passed through
over one period of 19 years (1 Vahid), and Bahá'u'lláh was to live for
one Vahid of the Kitáb-i-Aqdas. This was no mere coincidence. The Vahid
is the design of this new post Adam cycle.

By 1873, both Baghdad and 'Akká were part of the Ottoman Empire. While
in 'Akká, Bahá'u'lláh was in the Most Great Prison, where the
Kitáb-i-Aqdas itself was revealed.\footnote{Kitáb-i-Aqdas \#132} Despite
being a prisoner of the Persian and Ottoman Empires for most of this 40
year period as a Manifestation of God, He says His love is able to burn
away veils\footnote{Kitáb-i-Aqdas \#132} which inhibit one's ability to
fully believe in God. This love and imprisonment serves one major
purpose.

\begin{quote}
Hasten to what you were promised in the Books of God, and do not follow
the ways of the ignorant. My body has been imprisoned for the liberation
of your souls. Turn towards the Face, and do not follow every tyrant and
obstinate one. He accepted the greatest humiliation for your
honor.\footnote{The Lawh-i-Aqdas (Most Holy Tablet)}
\end{quote}

Bahá'u'lláh's mission is the liberation of our souls from oppression. It
might seem somewhat similar to the liberation theology of the 1960's,
which used Christian doctrine as inspirations of revolution. Yet the
center here is not political power, but the soul's deliverance from
every form of tyranny, outward and inward.

You will see the rest of this book outlines how the Kitáb-i-Aqdas
fulfills this mission through the commands and revelation of Glory.
Every law, counsel, exhortation, and theological ruling should be
considered from the perspective of this mission, the mission born of the
love from God for all of us.

\emph{Reflection: When you think of liberation, do you imagine escape
from a place, or the awakening of a capacity you forgot you had?}

With that mission named, we can now look at how Bahá'u'lláh describes
His own roles in fulfilling it.

\section{How Bahá'u'lláh Fulfills His
Mission}\label{how-bahuxe1ulluxe1h-fulfills-his-mission}

The Kitáb-i-Aqdas is ripe with various symbols and expressions of the
way Bahá'u'lláh fulfills His Mission for us. These titles include the
Sun, the Pen, the Shepherd, the Sea, the Master, the Remembrance, the
Temple, the Judge, the Chief, the Reminder, the Tongue, the Book, and
We. These titles are often used in conjunction with other descriptions,
but as you can see, these titles are encompassing of many roles and
duties. All of these roles are framed within the name Glorious, and
Bahá'u'lláh's roles should be considered with respect of this exalted
station.

Belief in Bahá'u'lláh as the Manifestation of God for our age does not
mean to merely view Him as a philosopher or ethical reformer. He should
not be viewed as a Unitarian where all paths currently lead to the same
destination. Belief in Bahá'u'lláh as the Manifestation of God for our
age means to view Him as more glorious than any person. This station
lasts until the Day of His Revelation sets and a new Manifestation of
God appears to fulfill the next Covenant. Every word, every letter,
every breath which came from Bahá'u'lláh is the pathway for freedom.

\emph{Reflection: What would change if you treated every command as a
path toward freedom, rather than a burden placed upon you?}

That question is why Part 1 must end not with ideas alone, but with a
call to lived response.

\section{Conclusion of Part 1}\label{conclusion-of-part-1}

We began this journey in Chapter 1 exploring the names and attributes of
God from Subtlety to Purity, from Knowledge to Love, discovering that
God's essence remains forever inaccessible yet His purpose is for us to
recognize Him in all things. Yet how would we recognize any of this
without the Manifestation. Without the Manifestation, God's subtlety
would remain forever hidden rather than purposefully veiled, His love
would be an abstract concept rather than a tangible reality, and His
covenant would be mere speculation. The Manifestation is the voice of
the breath of God, carrying its fragrance to receptive souls.

In Chapter 2, we learned the human soul is a mirror capable of
reflecting divine attributes, born noble yet requiring cultivation to
remain so. But without the Manifestation's teachings, how would we know
the soul's divine origin, its capacity for flight, or its destiny aboard
the Crimson Ark. The Manifestation shows us what the polished mirror
looks like and demonstrates the soul's potential made visible in the
human temple.

Chapter 3 revealed infinite worlds beyond our lesser world including
kingdoms, cities, and schools where souls progress through stages of
spiritual maturation. Without the Manifestation, these worlds would
remain speculation, dreams without interpretation, mystery without
guidance. The Manifestation is the Sidrat al-Muntahá made knowable, the
point where heaven touches earth, where the greater worlds become
accessible to seeking souls.

In Chapter 4, we discovered the Day of Resurrection is happening now,
not as bodies rising from graves but as souls awakening to divine truth.
Yet without the Manifestation, how would we recognize this Day. How
would we know the trumpet has sounded, the moon has been cleft, and all
forms are being shaken. The Manifestation announces the Day, embodies
its reality, and calls us to seize it rather than wait passively.

Chapter 5 established that God's Command flows eternally through
revelations such as the Torah, Gospel, Qur'án, Bayán, and Aqdas, each
paradise preparing for its Lord. Yet Command cannot exist without
Commander. The Manifestation is the living authority who speaks the Most
Great Command, the Sun rising from the Sidrat al-Muntahá at dawn, the
lamp lighting our pathway toward true liberation. Belief in the
Manifestation completes the foundation we have built across these six
chapters.

The Manifestation is not simply another belief to add to our collection.
It is the answer to every question these chapters have raised, the lens
through which all other beliefs become clear and actionable. With this
foundation complete, a question naturally emerges. What do we do with
these beliefs? Belief alone, however complete, remains incomplete
without expression. The breath of God does not merely inform. It moves
and animates. The Commands we have received through the Manifestation
call us not simply to know, but to act, not merely to understand, but to
practice, not just to believe, but to become. Part 2 will explore how
these beliefs translate into a living relationship with God through
spiritual practices that transform knowledge into experience, conviction
into devotion, and understanding into love.

\newpage{}

\part{Part 2: The Relationship with God}

\chapter{Foundational Virtues}\label{foundational-virtues}

\section{Introduction to Virtues}\label{introduction-to-virtues}

Part 2 is going to focus on the relationship we are capable of
developing with God through spiritual practice. These spiritual
practices help the soul retain its naturally born nobility by developing
our virtues, which help us reflect the names and attributes of God. We
will focus on this process of inner transformation by describing innate,
foundational, and emergent virtues, and how these virtues emerge from
spiritual practice. Throughout each chapter we will use a short
fictional story as a vehicle to illustrate how virtues can illuminate
the soul of a protagonist in a difficult situation. This story is
designed to be used as one example in how God inspires inner
transformation, guides God-conscious awareness, and inspires actions
which reflect all virtues. The story will be told in this chapter, and
illuminated without any plot advancement from Chapters 8 to 13.

When virtues are often discussed, they can often seem to be absolutes.
Either you have a virtue or you do not. I don't necessarily believe that
this is the only way to view virtues. Throughout the Kitáb-i-Aqdas,
Bahá'u'lláh is often serving the role as Counselor instead of Commander.
It is important to be able to distinguish which is a command and which
is a counsel. A counselor will use wisdom to advise on the best course
of action, but the counsel is not necessarily binding. The person
receiving counsel must still decide what the final action will be, given
the context of the situation they find themselves in. When it comes to
virtues, these are not laws. They are counsels. The Kitáb-i-Aqdas, due
to this, is also a book of counsel.

Virtues are not fixed destinations, like train stations along a fixed
track. You cannot get on one train and reach a virtue, then go on
another train and reach another. I like to view virtues more like stars
in the nighttime sky. We are familiar with Polaris, the North Star that
was used by ancient people all over the world to know which direction
was north. Polaris is also part of the constellation known as Ursa Minor
(Little Bear) or the Little Dipper. Imagine a virtue being a star and
all virtues being part of a constellation. We can use these stars to
navigate daily life, while never reaching them as an absolute
destination. We should never just navigate using one star, but use the
entire constellation when we consider what actions we should take in a
given situation.

Virtues then, help us on our spiritual journey. They help us develop our
souls. They help us embody the names and attributes of God within us,
and bring us closer to Him. Each star can be reflected within us as we
are mirrors. Virtues also do not happen automatically. They appear and
develop with practice and patience. It takes considerable wisdom in
learning how to navigate the entire constellation of virtues, but we are
not alone in this journey. We can develop and refine our virtues through
regular spiritual practices which also enhance our relationship with
God. As God is independent of us, God counsels us to these practices and
virtues solely for us. God desires this relationship for us. This
relationship is vital to our liberation.

\emph{Reflection: Which virtues feel like guiding stars for you today,
and how do they change the way you interpret counsel as something lived
rather than something enforced?}

The next section defines the five daily practices that make the
constellation usable in real life.

\section{Five Spiritual Practices}\label{five-spiritual-practices}

There are five regular spiritual practices the next chapters will
discuss. These spiritual practices include prayer, remembrance,
recitation, reflection, and honoring God. All five of these practices
are designed to develop the soul in different ways. If one is missing,
we may also be missing opportunities to enhance our virtues and liberate
ourselves. Each spiritual practice is as if we are attending these
different spiritual schools within the cities and kingdoms of God. Yet,
each practice is vital for the other practices. For example, if we do
not practice reflection, our prayer may not fully be sincere or honest.
Our recitation would not seem personal. The way we honor God might be
contrary to our abilities and intentions. Remembrance cannot be fully
realized if we are unable to discover how God works within ourselves.

\emph{Reflection: If you had to choose one practice that most often gets
neglected, what virtues do you think weaken first when it is missing?}

To clarify what these practices restore, we begin with the virtues
Bahá'u'lláh calls innate.

\section{Five Innate Virtues}\label{five-innate-virtues}

There are some virtues which Bahá'u'lláh describes as being
innate.\footnote{Kitáb-i-Aqdas \#120} These virtues are piety, pure
truthfulness, courtesy, loyalty, and trustworthiness. This means all
people were born with the ability to have these virtues and apply them
in some way within their lives. Being innate, they do not require
education as they are a natural part of being human. But because they
are integrated into our DNA, it also takes considerable effort to
override them. I call these manual overrides.

Being innate does not mean they do not need to be practiced, but it can
be trusted every person was born with these initial virtues, whether or
not every person you meet is attempting to use them. To a person who
views virtue as absolute, these virtues might seem rare. To a person who
views virtue as existing on a spectrum, there are signs these virtues
are everywhere. The innate virtues can be viewed as foundational
virtues, from which other virtues can also emerge. A person who only has
these five virtues can accomplish great good in this world. The
spiritual practices help restore our noble right in cases we choose to
override the virtue born within us.

\subsection{Piety}\label{piety}

Piety is often viewed as being committed to religious practice,
especially those who spend considerable effort in promoting an image of
religiosity. When looking at the Kitáb-i-Aqdas, it does not feel as
though this is the context Bahá'u'lláh defines piety. Let's look at how
piety is used in the Kitáb-i-Aqdas.

In Kitáb-i-Aqdas \#64, we are commanded to piety after Bahá'u'lláh
forbids oppression and lewdness. These two acts are acts against others.
Again in Kitáb-i-Aqdas \#71, oppression and wrongdoing is the opposition
to piety. In Kitáb-i-Aqdas \#88, piety is paired with justice. In
Kitáb-i-Aqdas \#108, piety is described with sincerity to describe how
to do remembrance of God. In Kitáb-i-Aqdas \#148, piety is paired with
kindness to oppose contention and disputes.

It seems as though for Bahá'u'lláh, piety is not about outward religious
appearances. Piety is more about serving one's obligations and duties
towards others. These obligations could be towards parents, children,
your work, your community, and to yourself. Piety isn't a virtue to look
good, it is a virtue to guide how we go about our spiritual practice and
treat others. How we treat others is also a pathway in how we honor God.

I believe piety is innate, or part of our natural disposition, because
we always have a sense of obligation to others, especially when we start
our lives as children. Learning what we are supposed to do, how to help
our parents, and how to play with our siblings and friends are all ways
we naturally seek piety at an early age. It may not seem like spiritual
practice as children, but once we reach the age of maturity, piety is
important in how we navigate our place within society.

\subsection{Pure Truthfulness}\label{pure-truthfulness}

The Kitáb-i-Aqdas only mentions truthfulness once, and it is combined
with the adjective pure. We are not counseled to be merely truthful, but
to adorn our tongues with pure truthfulness.\footnote{Kitáb-i-Aqdas
  \#120} The common view is truthfulness is a strict adherence to fact
or expressing their opinions of a matter. While facts should never be
denied, nor opinions be avoided, pure truthfulness expresses a different
standard than what is common.

One way to consider this standard is to remain aware the person you are
communicating with has a soul which is trying to attain liberation,
whether or not this person is aware of it in the moment. Pure
truthfulness is a virtue which can guide how we express truth in a way
which uplifts and conveys more of the constellation of virtues. Being
truthful may seem as if it requires courage, but pure truthfulness is
not about courage at all. It is about grace, being able to say what is
needed in a way the soul can receive. There are no sharp edges which aim
to hurt another. There are no pedestals to place yourself above the
person. Pure truthfulness is a warm embrace which lets the person know
you are together in a shared experience. This also means it is a
truthfulness which is not judgmental.

There are millions of ways to share the truth. Millions of people may
also have their own understandings of what is true or untrue. Pure
truthfulness helps provide a higher standard which allows a healthy
relationship with others regardless of divergent perspectives. Pure
truthfulness is innate as every person \textbf{\emph{has}} a desire to
know what is true and real, while learning this in a loving and caring
way. Every person deserves this respect.

\subsection{Courtesy}\label{courtesy}

Bahá'u'lláh describes courtesy with dignity in opposition to
freedom.\footnote{Kitáb-i-Aqdas \#123} In the Lawḥ-i-Dunyá, Bahá'u'lláh
says ``O people of God, I adjure you by courtesy and good manners, for
the supreme seat of ethics is first and foremost. Blessed is the soul
that has been illuminated by the light of etiquette and adorned with the
trappings of truth. Possessing manners is possessing a high station.''

Courtesy is a virtue which focuses on the consideration and respect of
others. Courtesy can be polite and it can also be kind. I like to view
courtesy within the act of driving. When I am driving, I have a
destination and often a specific time I need to be at the destination.
If I am solely focused on my journey, I might do so recklessly and
endangering others who are also on the road. If I am driving with
courtesy, I am driving with the understanding every person on the road
is also attempting to attain a destination at a potentially scheduled
time. My journey must never hinder another person's journey.

When Bahá'u'lláh describes courtesy in opposition to freedom,
Bahá'u'lláh is making sure we understand freedom as a virtue can be
quite harmful, as it may cause individuals to compete to exert their
freedoms at the cost of other individual's freedom. Courtesy also
opposes coercion and aggression. The libertarian nonaggression principle
is one way to view courtesy on a large scale nonreligious practice.
While everyone desires freedom, everyone deserves courtesy. The Golden
Rule to treat others as you desire to be treated is innate to all of us.
This is part of the pathway to actual liberation.

\subsection{Loyalty}\label{loyalty}

Loyalty is a virtue which can purify a soul.\footnote{Lawḥ-i-Siráj}
Bahá'u'lláh often pairs loyalty with love and steadfastness to help
express the purpose of loyalty. The first loyalty is to God in
servitude.\footnote{Kitáb-i-Aqdas \#120} Loyalty is also in service to
anyone you make a commitment to. Loyalty is not something we demand from
others, as we must not oppress. Loyalty is only something we may offer
to another. We have opportunities to show loyalty to parents, to those
who lead us, whether we voluntarily or involuntarily are led.

Sometimes there are competing loyalties and this can be difficult to
navigate or discern. Imagine a scenario where two parents divorce and
each are competing in their authority for the child. It can be confusing
to navigate when the child shows obedience or loyalty. In the Epistle to
the Son of the Wolf, Bahá'u'lláh describes how competing loyalties
existed within His family. His future daughter-in-law, Fátimih Khánum
expressed her loyalty to her sister, but the sister was trying to
prevent Fátimih Khánum from expressing her loyalty to her fiancé, 'Abbás
Effendi (The Most Great Branch). Eventually the sister was loyal to
Mírzá Yaḥyá (Bahá'u'lláh's half-brother). In the last year of His life,
Bahá'u'lláh asked God to take care of the sister, despite her being
disloyal to Him.

Loyalty is a virtue which must be navigated within the full
constellation, and not as a star by itself. It takes considerable wisdom
to navigate. Loyalty must not violate other virtues. Yet, we should not
be scared of offering our loyalty. Loyalty is the expression of
devotion, not love. Loyalty is love through time, complete with the acts
and long-term service required to ensure a timeless relationship.

\subsection{Trustworthiness}\label{trustworthiness}

Bahá'u'lláh describes Himself as the trustworthy counselor.\footnote{Kitáb-i-Aqdas
  \#52} This is in opposition to describing those who are wolves in
sheep's garments. He enjoins responsibilities for the trustworthy to act
as trustees\footnote{Kitáb-i-Aqdas \#69} with their
obligations.\footnote{Words of Paradise} Trustworthiness eliminates
doubts in the affairs of the world.\footnote{Kitáb-i-Aqdas \#134}
Bahá'u'lláh joins trustworthiness with the lights of certainty,
steadfastness, and tranquility.\footnote{Lawḥ-i-Dunyá}

Trustworthiness is not exactly about being truthful, but it is a virtue
which guides when we must lead. We can be trusted to fulfill our
responsibilities and to navigate virtues even when there are no
witnesses to our actions. It is a virtue which allows the protection of
others, while also providing peace of mind and spirit for the person who
is trusting you. These obligations might be tangible, such as honoring
an agreement, supervising a team, or being a parent. Some of these
obligations may seem intangible, such as a friend sharing information
which may leave them vulnerable. A trustworthy person will react to
these obligations with a sense of duty, perhaps being guided by the
virtues of piety, courtesy, and loyalty. These virtues are also guided
by trustworthiness. No virtue can truly exist if a person is not
trustworthy.

Trustworthiness is innate because everyone wishes to be trusted, such as
the child wanting to do something without supervision. When
trustworthiness is compromised, it risks conflict in that moment and
into the future. It is a virtue which is difficult to restore. Yet, it
is not absolute. Trustworthiness is also based on having realistic
expectations, which are also a sign of trustworthiness. These
expectations should be steady, which is why steadfastness is often
joined with trustworthiness in Bahá'u'lláh's counsels to us. Shifting
expectations are impossible to be fulfilled and cannot be used to
express another's lack of trustworthiness. As we can see,
trustworthiness requires a delicate balance to be achieved between two
people, as it requires not only the intention of yourself, but the
perception of who observes you.

\subsection{Manual Overrides}\label{manual-overrides}

Manual overrides are required for a person to stop utilizing their
innate virtues. These could be conscious choices we make, but they could
also be imposed upon us by others. They hijack the human temple and
interfere with our soul's ability to orient towards the Sun of Truth.
Where the Sun provides liberation, these overrides tend to be sources of
suffering and self-imprisonment. I will categorize these manual
overrides in five ways. These five ways can easily interact with each
other, depending on the situation:

\begin{enumerate}
\def\labelenumi{\arabic{enumi}.}
\item
  Social and Traditional Overrides: These are external structures such
  as with family, clergy, or societal expectations. Religious and
  cultural practices such as blind imitation of traditions without any
  personal investigation overrides pure truthfulness. There could be
  family authority figures who attempt to govern another's soul,
  perpetuating different standards for girls than boys. This overrides
  courtesy.
\item
  Identity and Shame Overrides: These are internal overrides where we
  adopt a false narrative about ourselves. We may view a past mistake as
  a permanent stain on our worthiness, rather than opportunities of
  reconciliation and progress. This overrides piety. We may see others
  who excel at something we wish to excel at, and feel too ordinary or
  worse, worthless. This overrides trustworthiness by preventing us to
  spread our wings and believe we can be more or achieve more.
\item
  Sensory and Material Overrides: These are internal overrides often in
  relationships to external structures. We could choose to remain in
  emotionally comfortable places, while emotional depths seem too
  demanding. This overrides our potentials for loyalty. We might choose
  to constantly overload our senses, refusing moments of silence or
  self-awareness. This overrides courtesy to ourselves.
\item
  Intellectual and Ego Overrides: These are internal overrides focused
  on the self. We could have an over-reliance on our own senses and
  experiences as the primary sources of truth or evidence. Not only does
  this cause self-exaltation, but it overrides pure truthfulness by
  denying potential sources of truth. We could have cognitive rigidity
  where we define things in the most narrow way solely to fit narratives
  or normative views of the world. This overrides trustworthiness.
\item
  Relational Overrides: These are often the focus of traditional
  religious discourse. These overrides occur solely in the space of
  human connection, both internally and externally. The ego may attempt
  to dominate that of another. Sometimes we may seek instant
  gratification of desires, overriding both courtesy and piety. We could
  withhold difficult truths which prevent others from making
  knowledgeable and appropriate decisions, overriding pure truthfulness.
\end{enumerate}

I want to provide an example of how these overrides could interplay with
each other. Say a person belongs to a very conservative religious
community, where absolute chastity is a primary worth of a young woman.
Both young men and young women, due to the traditions and rituals of
their families and community, may not learn how to form healthy and
meaningful bonds out of fear of being led to temptation. They can only
see each other fearfully, despite the potential for a soulful
connection. This focus on temptation from desire ironically only allows
them to see each other as objects of temptation, increasing the taboo
desire they feel for each other. Eventually they fulfill their desire,
despite their fear of being ruled by demonic forces. Each uses each
other purely for an immediate pleasure, even if that soulful connection
had not yet been developed. Afterwards, both feel guilt and shame as
strongly as they had felt their initial desire. The young woman in
particular feels she has lost something she can never recover, worthless
not only to her future husband but in the eyes of God and her family.
Maybe both can keep their actions a secret, hidden from the eyes of
those who judge, but only if a new life was not created.

This situation covers all five manual overrides in various ways. One
could wonder if either the woman or man are truly liberated by these
manual overrides, which outwardly seem to be born of a belief in a
virtue, but at the cost of other virtues. I would argue both individuals
suffer greatly from a prison God desires us to be free from. If the
manual overrides served as clouds hiding the innate virtues, the five
spiritual practices are the ways we can allow the Sun to shine
gloriously through.

\emph{Reflection: Where do you recognize a manual override in yourself
that does not erase an innate virtue, but temporarily clouds it and
redirects it?}

Having named distortion, we now name the balancing principle that keeps
virtue from turning into excess.

\section{Moderation}\label{moderation}

All of these virtues require moderation. This moderation can exist
within the expression of individual virtues, but the other virtues
within the constellation also provide moderating influences to ensure a
single virtue does not overpower another. They are all part of the
Balance of God.

Bahá'u'lláh alludes to moderation in Kitáb-i-Aqdas \#43 when describing
a state between despair in calamities nor excessively rejoicing in
happiness. Bahá'u'lláh explicitly states ``Truly, I say, moderation in
all things is beloved. When it is exceeded, it leads to
harm.''\footnote{The Words of Paradise, Ninth Leaf} Nothing is excluded
within moderation.

Moderation can be seen as both a virtue and a practice. It is a virtue
because it is a guiding principle and by definition, can never be
absolute. It is a practice because it takes constant work and awareness
to discover what is the middle way. As we can observe with the five
innate virtues, each helps the others achieve the middle way. The closer
we are able to discover moderation, the closer we are able to achieve
liberation for ourselves and for others we care about.

Moderation is also a foundational principle for the two final virtues of
this chapter. These virtues are the fear of God and the love of God. If
one exists without the other, we are unable to have a healthy
relationship with God. Our spiritual practices will be unbalanced, our
spiritual compass could lead us in the wrong directions, and liberation
may seem as oppression from the eyes of the observer.

\emph{Reflection: When have you watched a virtue become harmful because
it lost moderation and began to overpower the rest of the
constellation?}

With moderation established, we can now see how fear and love function
as paired forces that orient the soul toward God without distortion.

\section{The Fear and Love of God}\label{the-fear-and-love-of-god}

Bahá'u'lláh says ``all will perish from a single Word from God'' without
the fear of God.\footnote{Kitáb-i-Aqdas \#40} Without this virtue, one
may become arrogant.\footnote{Kitáb-i-Aqdas \#148} This arrogance could
cause one to place themselves in the position of God or the
Manifestation of God. This fear of God is a method to understand God, in
all of His names and attributes, is Most Powerful, the Lord of All, and
is the Judge. If we do not follow the commands, we may struggle when we
pass away from Earth. If we follow the commands but make no attempt at
following the counsel of virtues, we also may struggle. The fear of God
also comes with the idea not to have fear of others. We should fear only
God and God alone.

Fear should not be the cause of anxiety, but instead should be filled
with an awe-filled admiration of God's majesty.\footnote{Kitáb-i-Badí`}
However, if we only fear God without the love of God, we could be
greatly handicapped.

The love of God balances out fear by providing purpose. It helps temper
the awareness of might and justice with the confidence of forgiveness,
mercy, and grace. Love is the ultimate aim\footnote{Hidden Words in
  Arabic \#5} and is the ultimate motivation for all good. The Seven
Valleys describes love as expansive of all horizons, able to transform
agony to ecstasy.

How fear and love complement each other is beautifully portrayed
elsewhere in the Seven Valleys.

\begin{quote}
At last, the tree of his hope bore the fruit of despair, and the fire of
his anticipation cooled down. Until one night, weary of life, he left
his home and went to the market. Suddenly, a watchman started following
him. He began to run, and the watchman gave chase until more watchmen
gathered, and from every side, they blocked the path of the restless
lover.

The poor man was weeping from his heart and running in fear, thinking to
himself, ``This watchman is my 'Azrá'íl (angel of death), so hastily
pursuing me, or perhaps an avenger, harboring enmity against the
devotees.'' That man, weary from the arrow of love, kept running and
wailing until he reached a garden wall. With great effort and hardship,
he managed to climb over the extremely tall wall, sacrificing his very
life, and threw himself into the garden.

He saw his beloved holding a lantern, looking for a lost ring. When the
heartbroken lover saw his heart-stealing beloved, he sighed and raised
his hands in prayer, saying, ``O God, grant dignity and wealth to these
watchmen and preserve them, for they were like Jabrí'íl, guiding this
weak soul, or like Isráfíl, giving life to this lowly one.''
\end{quote}

While this story utilizes the fear of being apprehended by a guard to
illustrate this story, Bahá'u'lláh refers to God as a Guardian in books
such as the Súriy-i-Haykal. The story is a great example in how the very
pressure of fear can deliver you to a spiritual reunion with your
Beloved. The man's gratitude towards the guard for causing fear shows
how intricately fear and love are meant to exist together, as moderating
forces upon the soul of a believer.

Now I want to share a story of my own and thereafter, show how the
virtues of the fear and love of God, plus moderation, can illuminate
this story.

\emph{Reflection: In a moment of moral pressure, what helps you keep
fear from becoming anxiety and love from becoming indulgence?}

With these two forces in place, we can now watch them operate inside a
single moment where power, truth, and responsibility collide.

\section{The Unknown Sister - A
Story}\label{the-unknown-sister---a-story}

The envelope wasn't physically heavy, but it made her pause before
opening it. The way things feel when they contain more than they should.
She set it on the small conference table and adjusted the strap of her
bag with her foot, nudging it farther under the chair so her daughter
wouldn't tip it over.

Her brother sat across from her, hands folded, posture careful. He had
insisted she read it first. You're the executor, he had said simply. I
trust you. Whatever it is, we'll handle it. He meant it, and that made
it harder, not easier.

Her daughter swung her legs beneath the table, humming to herself,
tracing the wood grain with one finger. She had brought her along
because childcare had fallen through and because, somehow, it had felt
wrong to do this without her. The child's presence grounded the room,
even as it made the moment feel slightly unreal.

The room was cool, the air conditioning running steadily in the
background. The lawyer sat at the head of the table, quiet and
professional, ready to answer questions when needed but offering nothing
unprompted.

She opened the envelope and unfolded the pages. The language was formal,
exacting. Names, dates, clauses that flattened a life into orderly
sections. She moved through the familiar parts first, personal effects,
instructions about the house, the residue of decisions her father had
already made peace with. Her eyes scanned automatically, the way they
always did when she was responsible for something important.

Then she stopped.

There was a name she didn't recognize.

She went back, reading it again, slower this time. The name appeared
once, then again, tied to a date of birth. Her mind did the math before
she could stop it. Fourteen. Born while her parents were still married,
years before her mother's long illness had taken her.

Her daughter leaned over and whispered, ``Mama, can I have a snack?''

``In a minute,'' she said softly, without looking up, trying to veil the
shakiness she felt deep within her.

She continued reading. The remaining assets were to be divided into
three equal parts. One to her. One to her brother. One to a sister she
had never known.

The room didn't change, but her body did. Her brother watched her face,
reading nothing from it, waiting. He didn't ask questions. He didn't
lean forward. He simply stayed where he was, steady and present, as if
that were the only thing he could responsibly offer.

She turned the page. There was an address. Another woman's name. No
explanation. No letter. No attempt to frame what this meant or why it
was here. Just the facts, laid out with the same neutrality as
everything else.

Her daughter slid off the chair and crouched on the floor, pulling a
small car from her bag and rolling it back and forth beneath the table.
``It's going to the store,'' she announced.

``That's nice,'' she said, her voice calm, practiced.

She was aware, suddenly, of how many things were happening at once. The
memory of her father's voice, steady and familiar. The image of a sister
she had never known, living an ordinary life somewhere else, believing
certain things to be true. The knowledge that she was now responsible
for turning this document into reality.

She glanced at her brother. He gave her a small nod, the kind that said
take your time, though he couldn't possibly know what she had just read.
Whatever he saw on her face, he didn't try to interpret it. He trusted
her to carry it first.

The air in the room felt thick now, as if the coolness had been replaced
by something denser.

She folded the pages carefully, aligning the corners. Her daughter
crawled back into her lap, warm and solid, pressing her cheek against
her arm. The child smelled faintly of apple slices and vanilla shampoo.

``Are we done?'' her daughter asked.

``Almost,'' she said.

She placed the will back into the envelope. She glanced at the lawyer,
who waited patiently, pen poised over a legal pad, ready to explain
whatever needed explaining. But she did not speak. There would be time
for questions later, legal ones, practical ones, but not yet. Right now
there was only the knowledge itself, newly uncovered, sitting between
her and the life she had thought she understood.

Outside the window, the day continued without reference to any of it.
Somewhere else, a teenaged girl was moving through her afternoon,
unaware that her name had been written into another family's future. And
here, in this room, with her brother waiting patiently and her child
tracing imaginary roads along the floor, she held the first quiet moment
of a truth that would not stay contained for long.

\emph{Reflection: When a truth places power in your hands, what inner
posture helps you keep responsibility from turning into control or
avoidance?}

To see how virtues illuminate action, we now return to the same moment
and observe it through fear, love, and moderation.

\section{Illuminations of the Unknown Sister - Foundational
Virtues}\label{illuminations-of-the-unknown-sister---foundational-virtues}

\subsection{Fear of God}\label{fear-of-god}

The fear of God doesn't tell her what to do, but it puts her back in
proportion. The will has placed power in her hands, and she feels how
easily she could confuse permission with righteousness, comfort with
justice, silence with peace. Awe steadies her. God alone is Judge, Most
Powerful, Lord of all outcomes. That fear is not panic. It is sobriety.
It frees her from fearing her brother's eventual reaction or the
lawyer's gaze, because she is not answering to them first. The fear
liberates her from a fear of losing her presumed inheritance, an
inheritance to share with her daughter.

\subsection{Love of God}\label{love-of-god}

The love of God softens what fear has steadied. It allows her to look at
the will without hardening, to see not only disruption but care reaching
beyond death. Love reminds her that justice is not only measured, but
meant to heal, and that mercy can exist without erasing truth. It gives
purpose to restraint and patience, assuring her that goodness is not
proved by control, but by a willingness to let compassion guide whatever
must come next. The love of God gives assurances that even when
provision is less than desired, provision exists in correct proportion.

\subsection{Moderation}\label{moderation-1}

Moderation steadies her between extremes. She does not collapse into
despair at what has been uncovered, nor does she rush toward relief by
forcing resolution. She resists the urge to freeze everything in place
and the opposite urge to act simply to end the tension. Moderation holds
fear and love in balance, keeping either from overpowering the other. It
reminds her that neither silence nor action is pure on its own, and that
liberation lies in remaining attentive, restrained, and awake to the
middle way as it slowly reveals itself.

\emph{Reflection: Which virtue in this illumination feels most difficult
to embody when your emotions are loud and the consequences are real?}

To close the chapter, we now set the trajectory for how each practice in
Part 2 will restore an innate virtue and help other virtues emerge.

\section{Summary}\label{summary-1}

In each chapter from Chapters 8 to 12, we will introduce a new spiritual
practice described in the Kitáb-i-Aqdas. An associated innate virtue
will be described as being restored by this spiritual practice. For each
innate virtue, other virtues will emerge giving the innate virtue
further form and application. We will start with prayer and piety,
seeing what God counsels us towards, and how piety illuminates the story
``The Unknown Sister.'' From there, each chapter will deepen the same
moral moment, so that by Chapter 13 we can see unity not as an abstract
ideal, but as a constellation we learn to navigate when truth arrives
with weight and responsibility.

\newpage{}

\chapter{Prayer}\label{prayer}

\section{Introduction}\label{introduction-2}

Prayer is the first spiritual practice this book presents. Bahá'u'lláh
dedicated a large part of the Kitáb-i-Aqdas to prayer. This chapter will
describe what the Book discusses regarding prayer, and some emergent
virtues from piety which are cultivated by the spiritual\_attachice of
prayer. There is obligatory prayer, which is a required daily practice.
There is also non obligatory prayer, which is personal prayer whenever
you feel the need or desire to communicate with God. Prayer is like
calibrating a compass before you walk into the day, because it helps the
soul remember which direction is truly north.

\emph{Reflection: When you think of prayer as daily calibration, what
would change in how you approach obligation, sincerity, and freedom?}

We begin by clarifying who is required to pray, because the boundaries
of duty reveal the mercy within the command.

\section{Who is Required to Pray}\label{who-is-required-to-pray}

The age of maturity is the age when a believer is first required to
pray, as well as other religious duties such as fasting.\footnote{Kitáb-i-Aqdas
  \#10} The age of maturity will be discussed more in depth in Chapter
15, but the age is defined as 11 for spiritual practices.\footnote{BB00083}

Prayer is also not required if any person is ill or if they are in old
age. Old age is not defined, but at this time is up to the believer. Any
person who is not required to pray may do so if they feel compelled to.
There is no requirement for a medical professional to determine if an
illness qualifies for prayer exemption, nor to determine if a person has
reached old age. Once again, no definitive universal number was
described. I feel this is also more of a stage of life which is
determined by certain prerequisites.

A woman who is menstruating does not have to perform obligatory prayer
nor fast\footnote{Kitáb-i-Aqdas \#13}. She is to perform the ablutions
and repeat the phrase ``Glory be to God, the Lord of Splendor and
Beauty'' 95 times in a 24 hour period starting at noon. This repetition
does not need to be done all at once, and there is no particular method
described. How she performs this spiritual task is entirely up to her.
This exclusion is not due to her being impure, but more of a mercy. As
the medium and long prayers include the physical acts of prostration,
the pain associated with prostrating might be distracting. I don't
believe God wants us to be unduly troubled during prayer. The act of
remembrance, covered more deeply in Chapter 9, is also equally important
as prayer. I hope a person would not feel a woman is spiritually weaker
for praying less than is required of a man and find remembrance to be as
vital to the soul as prayer is.

\emph{Reflection: Where do you see mercy operating inside religious
duty, especially where life stages and bodily realities make devotion
more demanding?}

With duty clarified, we can now look at how prayer begins, not with
complexity, but with simple preparation.

\section{Preparation for Obligatory
Prayer}\label{preparation-for-obligatory-prayer}

Ablutions are required in preparation for obligatory prayer.\footnote{Kitáb-i-Aqdas
  \#18} An ablution is the washing of your hands and face with water. If
ablutions cannot be performed, the short prayer should be offered to God
prior to the obligatory prayer ``In the Name of God, the Purest, the
Purest'' five times. The purpose of ablution is a ritualistic way to
symbolize purifying yourself prior to prayer. While ablutions is an
outer form of purifying, one must not neglect the aspects of purifying
inwardly. There are no hard rules, but I feel part of the purpose of the
obligatory prayer is to purify. The Ṣalát i Kabír (Great Obligatory
Prayer) is a great example.

There are no other preparations described. However, the prayer is meant
to be an individual act with the requirement of congregational prayer
being lifted.\footnote{Kitáb-i-Aqdas \#12} There should be no pressure
to pray with others.

For personal prayer, ablutions do not need to be performed although you
could if you want to.

\emph{Reflection: What would it look like to treat ablutions as a symbol
of inner cleansing rather than as a test of outward perfection?}

Once preparation is simple, the next question becomes practical: how
long must prayer be, and how long may it be?

\section{Length of Prayer}\label{length-of-prayer}

The length of the obligatory prayer was to consist of nine rak'ahs. A
rak'ah is a sequence of movements performed during the prayer, such as
prostrating and standing up. However, after the Kitáb-i-Aqdas was
revealed, Bahá'u'lláh provided three different obligatory prayers. One
can be chosen each day. The small prayer is quite short and can be
remembered and conducted quite easily. The midmost prayer follows the
structure most like the original obligatory prayer and can take a few
minutes to perform. The great prayer is designed more for a deep
meditation which can take at least fifteen minutes. All of the times can
vary depending on if you sing, take time to reflect on the words, or
even repeat some of the invocations.

For individualized non obligatory prayer, the length does not matter. I
feel the obligatory prayer, while designed to stand alone without other
prayers, can be a great way to lead into personal prayer. Personal
prayer is where you say what is within your heart to God. The obligatory
prayer can help set the stage in establishing sincerity, detachment, and
other spiritual qualities. Still, pray whenever you feel like it.

\emph{Reflection: If prayer is a daily foundation, how do you keep
flexibility from becoming neglect, and discipline from becoming burden?}

After length, the next practical question is timing, because daily
rhythm shapes what devotion becomes over time.

\section{Time of Day}\label{time-of-day}

The obligatory prayer was originally to be performed three times per
day. Once in the morning between sunrise until noon. The second time
would be between noon and sunset. The final time is from sunset until
two hours after sunset. These instructions follow with the midmost
prayer, which is designed most closely to the original prayer prescribed
in the Kitáb-i-Aqdas.

The small prayer is to be performed at noon, which is from noon until
sunset. The great prayer is to be performed anytime in the 24 hour
period.

For believers who are in high latitudes where the time of day can vary
greatly depending on the season, the use of watches and clocks to
determine prayer times are acceptable.\footnote{Kitáb-i-Aqdas \#10} For
example, in Calgary, Canada, sunset is close to 10pm during the summer
solstice and closer to 8pm during the spring equinox. In Quito, Ecuador,
sunset was at 6:21pm. In Quito, sunset is usually around the same time
due to its location at the equator. A person in Calgary could choose to
do the evening prayer closer to 8pm year round as that could be similar
to the spring equinox, or maybe they choose a random location near the
equator such as in Quito. Any of these alternatives are acceptable as it
can reasonably be sunset somewhere in that longitude.

Depending on perspective, one may feel a longer prayer will have more
merit than the other, but really Bahá'u'lláh did not say either is
preferred. The intention is to follow your own heart and express your
spirit depending on your own circumstances each day. It is important
that it is performed daily, as prayer is the foundation of a loving
relationship with God.

As God has also prescribed work as an act of worship, I would recommend
not getting too concerned about how to balance a busy lifestyle with the
devotional act of prayer. This will develop naturally as your
relationship grows. You will start to develop a sense of when you need
to pray, or which prayer is the best for that day. I have found the
great prayer difficult to recite word for word. When I get a little
lost, I paraphrase or just start expressing what feels right within
myself. Sometimes I also use pre recorded music to guide me.

No matter what, the important aspect is that prayer is done at least
once per day.

\emph{Reflection: When time becomes irregular, how do you preserve the
spirit of daily prayer without turning it into a scheduling anxiety?}

Once timing is understood, we can now consider orientation, because
direction becomes a physical reminder of what the heart is turning
toward.

\section{Direction of Prayer}\label{direction-of-prayer}

The direction of obligatory prayer is called the Qiblih, or the point of
adoration. As the point is currently the point where the Sun of Truth
and Explanation had set, this would be the mansion of Bahjí in present
day Acre, Israel.

Even though we are to turn to where Bahá'u'lláh was last manifested in
the flesh, it does not mean we are praying to Bahá'u'lláh or worshiping
Him. We are still praying to God. The purpose for praying towards
Bahá'u'lláh is due to His role as the Manifestation of God. Bahá'u'lláh
is directly branched from the Sidrat al Muntahá and reflects the names
and attributes of God. If we turn our souls towards Bahá'u'lláh, it
helps our souls to also reflect these names and attributes as mirrors.
Say, in one example, we turn to the Qiblih and by doing so, we turn to
God's name All Bountiful. Simultaneously, we are also turning away from
the symbols of wealth which might exist elsewhere near us, such as a
luxurious building, gold jewelry, or a TV promoting consumerism. We
eliminate potential idols each time we turn to God through the practice
of facing the Qiblih. It truly is the point of adoration.

Although we are mirrors and prayer helps us reflect the names and
attributes of God, we must not be thinking of any particular name or
attribute of God during prayer. The Báb says this in the Persian Bayán
``The worshiper, during prayer, should see none but their Beloved and
focus solely on God, One without Partner. If they bring to mind any name
or attribute in their worship, they become veiled, and their worship is
not accepted.''\footnote{The Persian Bayán, Váḥid 9 Gate 19}

Today, there are various apps which can be used to determine the most
correct direction for prayer. However, if you are travelling and don't
have a tool to use, it is acceptable to face the direction you feel is
the most right. As with everything in religion, intention is the most
important aspect. For prayers which are not obligatory, a believer can
face in any direction. God is in all directions.

As a simple sidenote, Acre is one of the world's longest continuously
lived settlements, going back to the Bronze Age. It also served as the
final stronghold for Christian Crusaders before they were defeated by
the Mamluks. Today Acre is a diverse small city with a well balanced mix
of people of various religions.

\emph{Reflection: When you physically turn toward the Qiblih, what inner
turning do you most need in that same moment?}

With direction established, we can now see why prayer is protected from
hierarchy, so that worship remains addressed to God alone.

\section{Congregational Prayer}\label{congregational-prayer}

The Báb had prohibited congregational prayer except for the prayer for
the deceased.\footnote{The Persian Bayán Váḥid 9, Gate 9} The Báb
explains none may worship God in a station beneath God. He continues by
saying any designated prayer leader, such as an Imam, represents a
letter of affirmation but present themselves as a letter of negation
when a new revelation exists. The prayer leader's affirmation is only an
outward appearance instead of a spiritual condition. They are no longer
allowed to lead. This keeps our prayer focused on God.

Bahá'u'lláh continues the prohibition in the Kitáb-i-Aqdas.\footnote{Kitáb-i-Aqdas
  \#12} He provides some changes to the Prayer of the Dead.

\emph{Reflection: What changes in your own prayer when there is no
religious hierarchy between you and God?} With congregational prayer
limited, the one exception becomes a window into how a community can
honor a soul with unity and humility.

\section{The Prayer for the Dead}\label{the-prayer-for-the-dead}

The prayer for the dead provides for two options. The first option is to
recite the full prayer revealed. There is a version for the male
deceased and another for the female deceased. The second option exists
if the believer has not remembered the full prayer. This second option
is to say Alláhu Akbar (God is Great) six times. The prayer is performed
during Qunút, which is based on the Muslim tradition of praying in a
standing position, with the palms facing outward towards the sky. The
Qunút is also a part of the obligatory prayers and signifies a position
and condition of praise towards God. No other requirements exist in the
prayer for the dead.

The Prayer for the Dead is the only prayer allowed to be performed in a
congregation. This is a beautiful way for a community to honor the life
of a loved one. With the repetition of the six verses 19 times, this can
create a powerfully emotional experience as the departed journeys
towards God.

The person should be buried with spirit and fragrance.

\emph{Reflection: How does praying together for the deceased reveal
unity without creating spiritual rank among the living?}

From burial prayer, we turn to accessibility, because Bahá'u'lláh
removes many barriers that once turned purity into exclusion.

\section{No Restrictions on Hair and
Clothing}\label{no-restrictions-on-hair-and-clothing}

A person can wear their hair as they wish.\footnote{Kitáb-i-Aqdas \#9}
Bahá'u'lláh refers to bones and the like as not being able to invalidate
the prayer. The clothing materials worn during prayer also do not
invalidate prayer. These are items which religious scholars and leaders
had often described as making a person impure, or their prayer to have
less effect. It is a primary reason why you might see a Muslim woman put
on a clean and white outer garment before she prays. Bahá'u'lláh states
these rules were not in the Qur'án and in the Kitáb-i-Aqdas, He is
explicitly stating a believer can wear their hair and clothing as they
wish. The clothes worn for prayer must be clean, in keeping with the
idea of being purified through ablutions.

I believe the main purpose of this is to demonstrate what truly matters
in prayer is the spirit and sincerity of the person. Bahá'u'lláh and God
are for the reformation of hearts, and the liberation of the soul.
Prayer is spiritual communion with God. With prayer also meant to be
done in private, it also helps demonstrate the fact that God does not
see us as these sexual creatures, or with sexual eyes. He sees our
spirit. He feels our soul. The rules by scholars who say otherwise are
attributing desire on God, giving God human and animal attributes
falsely.

A female believer can also interpret these lack of restrictions as she
please. Could a woman, in private, pray without any clothing? Could a
woman wear athletic shorts and a sports bra? Could she wear a burqa?
This is entirely left up to her and her only.

Bahá'u'lláh does regularly use garments as a symbol of our character. An
example is in a letter to 'Alí Akbar, where Bahá'u'lláh hopes all the
friends of God may be adorned with the garment of sanctity.\footnote{BH00010:
  Letters to Tehran Through Ali-Akbar} The garment is what others are
supposed to see from us. In this example, people who are or who are not
Bahá'í will be able to easily see a Bahá'í as holy and inviolable in
their belief and conduct. Perhaps saintly. This outer garment is more
important than the actual garment, and this outer garment is portrayed
through good deeds and morals. It is not portrayed through words or
wishful thinking.

\emph{Reflection: When outward restrictions fall away, what inner
sincerity remains, and how do you keep it from becoming performance?}

With clothing freed, the next reminder is simple and physical, because
even prostration asks only for cleanliness and reverence.

\section{Clean Flooring}\label{clean-flooring}

For prostration, Bahá'u'lláh allows for any clean surface. This does
mean the surface must be clean. A person may use a clean prayer rug, but
a prayer rug is not required. You could be on an old wooden flooring and
it is acceptable if it is clean. You could be on the finest marble and
it is acceptable if it is clean. Once again, the idea is for prayer to
be accessible and focused on the spirit, not the material.

\emph{Reflection: What does it teach you about God when prayer is made
accessible on any clean surface, regardless of status or setting?}

From ordinary surfaces, we move to extraordinary moments, because fear
and awe can become doorways for remembrance rather than panic.

\section{In Times of Exceptional Natural
Phenomena}\label{in-times-of-exceptional-natural-phenomena}

The Báb had outlined a specific prayer for the signs, or natural
phenomena. Bahá'u'lláh has abrogated this and says we only need to say,
````Greatness belongs to God, the Lord of what is seen and what is
unseen, the Lord of all the worlds.''

The verse is a great reminder in times of fear and distress. Sometimes
in the greatness of human potential, nature offers humbling and
dangerous experiences. Being able to remember that whatever we are
experiencing, no matter how great or awe inspiring it seems, it is not
as great as God. God is the Lord of what we observed and the Lord of
what we have not yet observed. Hopefully these words can provide solace
in those times of exceptional natural phenomena, the signs of God.

\emph{Reflection: In moments when nature overwhelms you, what helps you
turn awe into worship instead of anxiety?}

From extraordinary moments, we move to ordinary disruption, because
travel tests whether devotion is flexible without becoming optional.

\section{During Travel}\label{during-travel}

During travel, obligatory prayer may be replaced while taking rest. The
first option is to do a single prostration on a clean surface and recite
``Glory be to God, the Lord of Grandeur and Majesty, of Bounty and
Grace.'' If one is unable to find a clean surface or is unable to
prostrate for any reason, the second option is to say ``Glory be to
God.''

After the prostration, sit in the position of unity. I interpret this as
a cross legged position common to many religions, but you may have
another way to sit which demonstrates reverence. While seated, say
``Glory be to God, the Lord of Dominion and Sovereignty'' 18 times. This
totals to 19 repetitions of Glory.

It should be noted that after the Kitáb-i-Aqdas was released, the
obligatory prayer had changed to include the short prayer. This can, at
a minimum, take just a minute although the length can vary depending on
how you recite the prayer.

\emph{Reflection: When life becomes mobile or disrupted, what makes
prayer feel like a living relationship rather than a fragile routine?}

With the practice established, we now return to the Unknown Sister and
observe how prayer draws piety into new forms.

\section{Illuminations of The Unknown Sister - Emergent Virtues from
Piety}\label{illuminations-of-the-unknown-sister---emergent-virtues-from-piety}

\subsection{Detachment}\label{detachment}

Prayer loosens her grip on what the inheritance represents: security,
fairness, even family identity. In turning toward God, she feels the
subtle release of needing outcomes to protect her sense of order. The
unknown sister is no longer a threat to what she thought she owned, but
a reality she does not control. Detachment does not make the situation
disappear; it simply removes the illusion that it belongs to her to
possess or manage for her own sake.

\subsection{Humility}\label{humility}

In prayer, she feels how little she truly knows. The facts of the will
do not grant her insight into motives, histories, or consequences.
Humility quiets the instinct to believe her perspective is sufficient.
She recognizes that being named executor does not elevate her above
error or blind spots. Prayer places her back into dependence, reminding
her that clarity is not earned by authority, but received slowly, if at
all.

\subsection{Lowliness}\label{lowliness}

Prayer draws her into an awareness of smallness. Smallness is not
insignificance, but limitation. She is one person standing at the edge
of lives she cannot see and outcomes she cannot predict. Lowliness frees
her from the pressure to be equal to the moment. She does not need to
carry the weight of resolution yet. In accepting her lowliness, she
allows space for guidance beyond her own strength or cleverness. In her
daughter, she witnesses how there was a time she was on the floor
playing as her parents were responsible over the challenges of
parenthood and life.

\subsection{Reverence}\label{reverence}

Through prayer, the situation takes on gravity. This is no longer merely
a legal matter or family complication; it is a moment touching human
lives shaped by forces larger than intention. Reverence restrains casual
thinking. It slows her inner pace. She senses that how this truth is
handled matters, not because she is being watched, but because she
stands before something sacred: the fragile intersection of truth, life,
and consequence.

\subsection{Thankfulness}\label{thankfulness}

Prayer awakens gratitude in unexpected places. She becomes aware of what
remains steady: her child's warmth, her brother's trust, the fact that
truth surfaced while there was still time to respond thoughtfully.
Thankfulness does not erase the difficulty, but it keeps bitterness from
taking root. Even here, she can acknowledge that she has been given
time, awareness, and the chance to approach what lies ahead with care.
Her life exists through the spirit of God, sparked by the love her
mother and father shared, even if that love was not perfect.

\subsection{Innate Virtue Piety}\label{innate-virtue-piety}

In this moment, piety is her turning toward God rather than toward
control. Detachment expresses piety by loosening her attachment to
money, inheritance, and the version of family she assumed. Humility
expresses piety by admitting her knowledge is partial and her authority
does not make her right. Lowliness expresses piety by accepting her
limits and refusing the impulse to play savior or judge. Reverence
expresses piety by treating the situation as weighty and sacred, not
merely legal. Thankfulness expresses piety by recognizing mercy within
trial, and receiving time and guidance as gifts.

\emph{Reflection: Which of these emergent virtues would most change the
next action you take when a truth disrupts your assumptions?}

To keep this chapter grounded, we now look to Bahá'u'lláh Himself, not
as an abstract model, but as an exemplar of praying as a servant.

\section{The Example of Baha'u'llah}\label{the-example-of-bahaullah}

The writings of Bahá'u'lláh are full of examples of Him praying to God
as a servant. In this condition, Bahá'u'lláh is showing us the framework
of how prayer can be. We do not necessarily have to say exactly what
Bahá'u'lláh would say, as most of these prayers were for the time,
space, and context of the particular moment the prayer was revealed. We
can see what Bahá'u'lláh prayed for, to whom, and maybe see if God
answered these prayers and if so, in what way. We must always remember
that while Bahá'u'lláh was a Manifestation of God, Bahá'u'lláh was still
human. I interpret this human condition as He was not omniscient nor
omnipresent, but had capacities to serve a specific mission God had
ordained for Him. When Bahá'u'lláh was praying to God, this was not
necessarily revelation as God would not pray to God. It is important to
be mindful of Bahá'u'lláh's human condition and not to confuse Him as
God.

With that said, the answering of prayers is entirely up to God. The
purpose of prayer is not for wishes to be granted, but for the will of
us to merge with that of the will of God. Often Bahá'u'lláh would pray
that a person would attain certain spiritual qualities or be assisted in
doing so. I believe God would always provide this assistance. Yet, it is
still up to the individual person to choose to embody those qualities.
This means if Bahá'u'lláh prayed for a person, or prayed for the
qualities and assistance of a person, it does not automatically mean
this person attained that quality or station.

This is why emergent virtues such as detachment, humility, lowliness,
reverence, and thankfulness are all vital to the practice of prayer. No
matter what the outcome, whether we recognize it or not, and whether
another person reflects our wish or not, we must always be thankful for
the assistance, mercy, and bounty God does provide us. We should do our
best to pray like Bahá'u'lláh did, according to the time, space, and
context of our individual and unique lives. This practice will help the
innate piety which exists within us to further emerge and express itself
more fully within our lives.

This is one of the pathways to liberate our souls, just as Bahá'u'lláh's
soul was liberated. As we proceed through these spiritual practice
chapters, I invite you to look for how Bahá'u'lláh was an exemplar of
His teachings. They are truly inspiring.

\section{Summary}\label{summary-2}

Prayer is the foundation of daily devotion because it teaches the soul
how to turn, how to humble itself, and how to receive guidance without
demanding control. When prayer is practiced with sincerity, it invites
piety to express itself through detachment, humility, lowliness,
reverence, and thankfulness, so that even unsettling truths can be
carried with steadiness. In Chapter 9 we will turn to recitation, and
see how the words we repeat shape the words we speak, and how pure
truthfulness can begin to emerge through this practice.

\newpage{}

\chapter{Recitation}\label{recitation}

\section{Introduction}\label{introduction-3}

The next spiritual practice Bahá'u'lláh describes in the Kitáb-i-Aqdas
is recitation. Recitation is a practice which can go hand in hand with
the obligatory prayers, as a person must be able to recite them to
completely fulfill the obligation. Recitation is also a practice which
exists outside the obligatory prayers, and is the primary means to make
the Word of God known. You can say recitation is a conduit of divine
power. If God's Word is reflective of the animating power of God's
Breath, recitation is also us animating our souls and the world around
us with this animating and creative power of God. Recitation also helps
develop emergent virtues derived from the innate virtue loyalty. Before
we discuss what to recite, we first clarify the rhythm of the
obligation, because morning and evening create a daily structure that
shapes loyalty over time.

\section{Who is Required to Recite and
When?}\label{who-is-required-to-recite-and-when}

Recitation is a practice, which unlike the obligatory prayers, is not
restricted by age or health. All believers are commanded to ``recite the
verses of God every morning and evening. He who does not recite has not
fulfilled the covenant of God and His testament.''\footnote{Kitáb-i-Aqdas
  \#149} This is a universal obligation without exceptions. Every
person, however, may fulfill this obligation according to their own
capacities and it will vary in form and effort.

There is no definitive time designated as morning and evening. Simply it
could mean the beginning of your day and the end of the day.
Traditionally this would mean from dawn or sunrise until noon for the
morning, and when light decreases until bedtime for the evening. For
people who must work non traditional schedules, this would be left up to
your own circumstances and conscience. For example, if you wake up in
the afternoon to work, the afternoon might be your morning and the
period after midnight is your evening. What would not change is there
are two distinct periods to recite the verses of God, establishing a
daily rhythm.

\emph{Reflection: What would shift in your inner life if you treated
morning and evening recitation as two anchors that hold the day
together?}

With the rhythm established, we can now ask what is worthy of being
recited, because the content of recitation defines what power we are
welcoming into the soul.

\section{What Should Be Recited?}\label{what-should-be-recited}

Bahá'u'lláh instructs us to recite the verses revealed by God. In
Chapter 5, we defined what the Word of God is, which are the verses.
Today, this includes everything by Bahá'u'lláh. He says ``Whoever reads
a verse from My verses, it is better for him than reading the books of
the former and latter generations. This is the declaration of the
Merciful, if you are of those who listen. Say: This is true knowledge,
if you are of those who recognize.''\footnote{Kitáb-i-Aqdas \#138}

While a person can recite the verses of the Gospel or other former
Scriptures from God, we are counseled to use Bahá'u'lláh's revelation as
our primary focus. This does not prohibit reading prior Scriptures, as
reading is not exactly recitation. What is excluded from this command?
Anything which is not God's words. For example, this book is not from
God so you should not use this book for recitation. Any leader after
Bahá'u'lláh who is not a Manifestation of God should not be used for
recitation. No scholar deserves to be recited, nor the most beautiful
poet. Recitation is reserved for God's Word and God's Word only.

Bahá'u'lláh's writings include the Kitáb-i-Aqdas and many other tablets,
letters, and prayers. You can recite in the original Arabic or Persian,
but you should do so in a way which you can comprehend. If this means
translating to your native language, do so. In today's age, there are
many tools capable of translating such as OpenAI's ChatGPT, which was
used for the translations used within this book.

\emph{Reflection: When you choose what to recite, are you selecting what
is easiest, what is most beautiful, or what most directly reforms your
character?}

Once the content is clear, the next question is method, because
recitation can either enliven the soul or overburden it.

\section{How Should One Recite?}\label{how-should-one-recite}

Bahá'u'lláh offers a few guidelines in how a believer should recite the
verses of God. The first counsel from Bahá'u'lláh is not to overburden
yourself in reciting the verses of God. Recitation should come with
spirit and joy\footnote{Kitáb-i-Aqdas \#149} and it is welcome for one
who is ``enraptured by the love of My Name, the Merciful.''\footnote{Kitáb-i-Aqdas
  \#150} Reciting one verse with spirit and joy is more beneficial for
the soul than to recite numerous verses to the point of weariness or
listlessness.

I feel this guidance is exceptionally inclusive, as it allows a person
to determine what is best for them. Say you struggle with literacy and
can only read and understand a few simple words. That's ok! What if you
are neurodivergent? Do what you can! Did you just have a long day? God
understands. Will I commit a day off to recite for an hour? Yes, if you
are able to do so with spirit and joy! Bahá'u'lláh only wants what is
best for you, the individual. God is Most Merciful.

Bahá'u'lláh also says we should not expect other souls to be
overburdened in this spiritual practice. In general, we should not
overburden any soul with anything which makes them heavy and lethargic.
If a person is reciting with spirit, joy, and love, if they recite in
melodious tones, it is one of the most powerful effects known. He says
``they perceive from them that which no dominion of the heavens and the
earth can equal, and through them they discover the fragrance of My
worlds, which today are known only to those endowed with vision from
this exalted station. Say, it draws pure hearts to spiritual worlds that
cannot be expressed by words nor pointed to by signs. Blessed are those
who listen.''\footnote{Kitáb-i-Aqdas \#116}

\emph{Reflection: When you think about reciting with spirit and joy,
what helps you keep devotion from turning into pressure or performance?}

With method established, we can now ask why recitation exists at all,
because purpose determines whether recitation becomes merely vocal or
truly transformative.

\section{Purpose of Recitation}\label{purpose-of-recitation}

This brings us to the purpose of recitation. Recitation is to be
heartfelt. This isn't solely to touch our own heart, but to potentially
touch the hearts of a soul which is listless. Heartfelt recitation of
God's verses is like a beautiful flower alone in a garden, attracting
not only our eyes and noses, but also that of butterflies, bees, and
other creatures animated by God's spirit to its fragrance. The verses of
God permeate the visible and invisible, inside and outside of us.

This also means recitation can be both a private and public practice.
Bahá'u'lláh describes recitation as one of the practices of the
Mashriq'ul-Adhkar. Setting the verses of God to music and singing them
would seem to be an important community and cultural practice of
believers, according to the style and instrumentation they feel is best.
One doesn't have to be a professional singer. Whatever is a most
melodious tone for you and heartfelt is what is important. The cool
aspect of any spiritual practice is that it is a practice, meant to be
practiced. The practice helps develop a skill, such as memorizing,
embodying, singing, and feeling the Word of God. No one will ever be
perfect but with any practice, time and patience is key.

\emph{Reflection: If recitation is meant to carry fragrance into the
world, what would it look like for your recitation to be both private
nourishment and public service?}

With the purpose stated, we return to the Unknown Sister and watch how
loyalty becomes visible when recitation guards truth against distortion.

\section{Illuminations of The Unknown Sister - Emergent Virtues from
Loyalty}\label{illuminations-of-the-unknown-sister---emergent-virtues-from-loyalty}

\subsection{Fidelity}\label{fidelity}

Through recitation, she feels drawn back to what must remain true, even
when everything else feels unstable. Fidelity steadies her against
distortion, against reshaping the facts to ease discomfort or soften
their weight. The Words she recites remind her that truth has its own
sound, and her task is not to improve it or muffle it, but to remain
faithful to it without embellishment or avoidance.

\subsection{Morality}\label{morality}

Recitation sharpens her inner sense of rightness without accusation. It
does not condemn her uncertainty, but it does not excuse indifference
either. The verses quiet emotional noise and help her distinguish moral
clarity from emotional relief. Morality here is not about rules, but
about refusing to let fear, convenience, or sentiment replace
conscience. Morality is the Word allowing the discernment of herself.

\subsection{Righteousness}\label{righteousness}

As the words pass through her, righteousness becomes less about
appearing just and more about inward alignment. She senses that whatever
lies ahead must reflect integrity whether witnessed or unseen.
Righteousness does not demand immediate action; it demands that future
action not betray what she already knows to be upright, even if no one
else ever learns the struggle behind it.

\subsection{Servitude}\label{servitude}

Recitation shifts her posture from ownership to offering. She feels
herself loosen the claim that this situation must resolve in a way that
protects her comfort or preserves control. Servitude expresses itself as
a willingness to carry responsibility without dominance, to serve truth
without mastering it, and to accept that obedience to God may feel
heavier before it feels freeing.

\subsection{Steadfastness}\label{steadfastness}

The verses give her endurance rather than answers. Steadfastness forms
quietly as she realizes this moment will not be resolved quickly or
cleanly. Recitation anchors her so she does not retreat into avoidance
or rush toward premature resolution. It prepares her to remain present,
faithful, and oriented toward God even as uncertainty stretches forward.

\subsection{Loyalty (Innate Virtue)}\label{loyalty-innate-virtue}

In this situation, loyalty is her decision to let God's Word remain
supreme over her own instincts, narratives, or fears. Recitation keeps
her allegiance clear: not to convenience, not to appearances, not even
to inherited expectations, but to a truth that liberates by being spoken
first within the soul before it is ever carried into the world.

\emph{Reflection: Which emergent virtue in this illumination most
directly challenges your instinct to edit reality for comfort or
safety?}

To close the chapter, we now return to your reminder that virtue is a
spectrum, and we connect that spectrum to the next practice of
remembrance.

\section{A Gentle Reminder}\label{a-gentle-reminder}

To close this chapter, I want to offer a reminder that these virtues are
not absolute. One is not 100\% righteous nor 0\% righteous, for
examples. As we express our loyalty to God, we must do so in ways which
does not betray our loyalty to His creation, which can be nurtured
through our steadfast servitude. Yes, there are definitely clear
commands of what is right and what is wrong. There are also situations
where God seems relatively silent for some reason or another. Morality
and righteousness must be practiced to increase. They are not automatic
for any person, even if your desire is to seem like a saint. In an
orchestra, a clarinet can play a melody and then a trumpet. Even though
they play the same notes, the sound and its vibrations will be
different. When both instruments are played at the same time with the
same melody, the sound and its vibrations are amplified and again sounds
different.

I offer this prayer Bahá'u'lláh revealed at the end of the Lawḥ-i-Aḥbáb
(Tablet to the Friends):

\begin{quote}
Glory be to You, my God. You know that I am in prison, calling Your
beloved to a share of Your gifts, purely for Your sake. When the
idolaters surrounded me from all sides, I remembered You, O Master of
names and attributes. I ask You to grant Your servants success in
supporting Your cause and elevating Your word, then strengthen them in
what manifests the sanctification of Your Essence among Your creatures,
and the glorification of Your commands among Your creation. O Lord,
enlighten the eyes of their hearts with the light of Your knowledge, and
adorn their forms with the embroidery of Your Most Beautiful Names in
the realm of creation. Indeed, You are capable of what You will; there
is no god but You, the Mighty, the Wise.
\end{quote}

\section{Summary}\label{summary-3}

Recitation establishes a daily rhythm of loyalty because it keeps the
Word of God close to the tongue, close to the heart, and close to the
moment when a choice must be made. When recitation is practiced with
spirit and joy, it can draw fidelity, morality, righteousness,
servitude, and steadfastness into clearer form, so that truth is not
softened by fear or reshaped by convenience. In Chapter 10 we will turn
to remembrance, and see how what we repeat outwardly becomes what we
carry inwardly, and how the soul learns to remain turned toward God even
when no words are being spoken.

\newpage{}

\chapter{Remembrance}\label{remembrance}

\section{Introduction}\label{introduction-4}

The next spiritual practice is Dhikr, or the remembrance of God. Before
exploring the formal aspects of dhikr, I want to linger on the nature of
remembrance itself. What does it feel like to remember anything at all?

Sometimes remembrance is a conscious act. I choose to revisit a memory,
like returning to an old house where every doorway leads to a different
feeling. Perhaps I'm alone, walking through its corridors quietly, or
perhaps I open the door for someone else, inviting them in through a
story. It may be lighthearted, like the hope of laughter on a stage, or
personal, shared only with one close listener. Whatever the reason,
remembrance is not just to recall, it is to relive. I'm not just telling
a story, I'm feeling it again. And the story may not even be the same
one I told last time, because I am no longer the same person remembering
it.

Then there are the memories that come unbidden, like visitors in the
middle of the day. A flash of color, a familiar scent, a sudden voice
causing something to stir. I see a bluebird glide into a patch of grass,
and suddenly I'm a child again in my grandparents' yard. I hear an
accent, a cadence, and a long-buried grief awakens. These are not
thoughts I chased, but echoes that found me. Through sight, sound,
smell, touch, and taste, these gateways of senses, the soul is stirred.
Something subtle passes through, like a breeze carrying fragrance from a
distant garden.

Remembrance also comes in dreams. The real and unreal merge. A face that
no longer walks this world appears and says something you didn't know
you needed to hear. You wake with a strange emotion lingering, one part
truth, one part mystery. Sometimes these dreams feel like a letter from
another world. Did it arrive by chance, or did a divine hand guide it?

All these are ways we remember ourselves, including our past, our
longings, our unspoken questions. But how do we remember the One who is
veiled in every veil, whose Name echoes behind the curtain of all
things? How do we remember God, who is the Most Hidden, the Most Subtle?

This chapter will explore both formal and informal ways to remember God,
the purpose remembrance fulfills in the soul's journey, and the virtues
that blossom in its light.

\emph{Reflection: Where does remembrance already visit you without
effort, and what might change if you learned to treat those moments as
invitations rather than interruptions?}

With the inner texture of remembrance established, we can now ask what
the Kitáb-i-Aqdas tells us to remember about God.

\section{What Are We Remembering About
God?}\label{what-are-we-remembering-about-god}

The Kitáb-i-Aqdas does describe some memories we should use when
remembering God. We are to remember God among His creation.\footnote{Kitáb-i-Aqdas
  \#117} Creation, as we learned in Chapters 2 and 3, is everything
within us and outside of us. Creation is seen and unseen. We are to
remember God's bounty\footnote{Kitáb-i-Aqdas \#112} provided to us
through this Creation. Within this creation, we can remember God's
mercy\footnote{Kitáb-i-Aqdas \#14} and His greatness and
power.\footnote{Kitáb-i-Aqdas \#11} If creation seems scary or
intimidating, we can always remember to seek refuge in God\footnote{Kitáb-i-Aqdas
  \#167} such as with the prayer of the signs.

We are to remember His mighty and wondrous Name.\footnote{Kitáb-i-Aqdas
  \#50} This Name could refer to Bahá'u'lláh, which uses God's name of
Glorious, or it could also refer to every name of God. We learned a lot
of God's names in Chapter 1. We do not need to burden ourselves with
remembering every name each time, but we should also be careful not to
neglect any name over time. Nothing is excluded from the virtue
moderation. When we remember Bahá'u'lláh, we remember He is not God, but
the Manifestation of God and the dawning place of His most excellent
names and the supreme Word (Revelation).\footnote{Kitáb-i-Aqdas \#143}

Given we are to remember God, we are also to remember the Book
(Kitáb-i-Aqdas)\footnote{Kitáb-i-Aqdas \#185} and what was revealed from
Him,\footnote{Kitáb-i-Aqdas \#172} which is the entirety of
Bahá'u'lláh's Revelation.

\emph{Reflection: Which aspect of God do you most naturally remember?
Which do you most often neglect over time?}

With the content of remembrance clarified, we can now distinguish the
two modes by which remembrance comes to us: what arrives unbidden, and
what we choose deliberately.

\section{Subconscious Remembrance}\label{subconscious-remembrance}

Subconscious remembrance is not something we do. It is something we
allow. It is the state of being open to what God is constantly offering.

\begin{quote}
\textbf{O Essence of Heedlessness}\\
Alas, that a hundred thousand spiritual tongues are embodied in one
speaking tongue, and a hundred thousand hidden meanings appear in one
melody---yet there is no ear to hear, nor heart to grasp a single
word.\footnote{The Hidden Words in Persian \#16}
\end{quote}

If there are a hundred thousand spiritual tongues, perhaps from those
infinite spiritual worlds we have opportunities to pass through, there
could also be a hundred thousand spiritual ears for every ear, or a
hundred thousand spiritual hearts for every heart which grasps a single
word.

The Kitáb-i-Aqdas \#185 offers a simple blueprint for subconscious
remembrance. Bahá'u'lláh says ``This is the counsel of God, if you are
among those who hear. This is the grace of God, if you are among those
who turn toward Him. This is the remembrance of God, if you are among
those who feel. This is the treasure of God, if you are among those who
know.''

\subsection{Receiving Through Our
Senses}\label{receiving-through-our-senses}

Subconscious remembrance is not actively remembering, but allowing
memories and reminders enter our being. When Bahá'u'lláh mentions our
hearing, He wants us to be able to hear God. This could be the rhythm of
verses recited, the calls of birds in moments of silence, or the subtle
stirrings within. We hear a voice without a voice. Hearing often calls
for action. When you hear a siren, you respond. In the same way, hearing
the counsel of God leads us to live it.

To know is to open the mind. Hearing is one path to knowing, but so are
reading, witnessing, reflecting. Our minds process endless streams of
experience, and within those may be hidden names or attributes of God
waiting to be recognized. If we approach knowledge with honesty rather
than bias, we allow the truth to lead. Bahá'u'lláh calls knowing the
treasure of God. And like any treasure, when uncovered, it changes us.

To feel is to awaken the heart. Bahá'u'lláh says hearts are capable of
enlightenment.\footnote{Kitáb-i-Aqdas \#31} Yet we often avoid feeling
to avoid pain. Some pursue drugs to feel what they cannot otherwise
access, or to hide what they can no longer bear. But everything we
encounter evokes feeling. And what if the very feeling we fear is the
doorway to insight? Perhaps it's at least a path to empathy.

When our senses, mind, and heart are open in this way, the soul can
begin to turn. Subconscious remembrance is that turning, not as a
command, but as a quiet alignment. We may not always be fully oriented
toward God in our daily lives, but remembrance begins in our willingness
to receive. We don't say, ``I am a mirror.'' We simply become one.

\emph{Reflection: What is one sense---sight, sound, smell, touch, or
taste---that most often opens your heart, and how could you let it
become a doorway to remembrance rather than distraction?}

With subconscious remembrance described, we can now turn to conscious
remembrance, where the soul chooses to remember God with intention, joy,
and fragrance.

\section{Conscious Remembrance}\label{conscious-remembrance}

Conscious remembrance is the intentional act of turning the heart and
soul towards God. The Kitáb-i-Aqdas provides guidance in how this
remembrance should be carried out, allowing much of it to be done openly
or in secret. Much like how Bahá'u'lláh did not want a person to be
burdened by tiresome recitation, the Báb taught how remembrance should
be done with spirit and fragrance. Bahá'u'lláh encourages us to exalt,
magnify, and glorify their Lord with joy and gladness.\footnote{Kitáb-i-Aqdas
  \#16} We should thank Him with joy and spiritual fragrance.\footnote{Kitáb-i-Aqdas
  \#74}

The purpose is never to perform remembrance for its own sake. Just as we
use our senses to receive from God, we use our faculties to give back to
God and others. We use our voices to help others hear. We use our minds
to help others know. We use our hearts to help others feel. Our souls
can help others turn towards Him. To do so, you have to actually feel
the spirit and fragrance. It has to come from deep within you.
Remembrance is felt, not merely spoken.

\subsection{Ways to Remember}\label{ways-to-remember}

We are also to rejoice in the joy of the Greatest Name (Bahá'u'lláh), by
which hearts are enraptured and the minds of the near ones are
attracted.\footnote{Kitáb-i-Aqdas \#51} This is not an act of worship
towards Bahá'u'lláh Himself, but a celebration this name exists, that it
has manifested in the form of the temple of Bahá'u'lláh, and that it is
a sign of God's mercy. Bahá'u'lláh is the point of adoration (Qiblih).

One way we can conduct remembrance is through music. Music is a ladder
for the ascent of souls to the highest horizon.\footnote{Kitáb-i-Aqdas
  \#51} Bahá'u'lláh warns us by saying ``do not make it the wings of
self and desire'' and to ``beware your listening does not lead you away
from the path of dignity and reverence.'' Ladders can be used to ascend
or descend. For a soul to be liberated, it must ascend.

Bahá'u'lláh also says ``do not conduct remembrance in the streets or
marketplaces. Do so in a place designated for remembrance or in your
home. This is closer to sincerity and piety.''\footnote{Kitáb-i-Aqdas
  \#108} If we feel inspired to remember God in the streets and
marketplaces, this would be one of those times to practice secret
remembrance, done silently or in a whisper. Sometimes I imagine a
musical where random people join me for a choreographed dance and
singing, but it's really only in my head. Remembrance of God can also
joyously occur with spiritual fragrance even in silence.

\subsection{A Warning}\label{a-warning}

True remembrance is designed so we can feel the spirit and share the
spirit. The Bayán says remembrance has no value if it keeps us from
recognizing God and His Manifestations. Denying a Manifestation is the
same as denying God. Bahá'u'lláh echoes this warning, telling us
``beware\ldots that remembrance veils you from this Most Mighty
Remembrance.'' This warning is incredibly important. What if our
remembrance is contrary to what God desires? What if we end up not
recognizing the future Manifestation of God, cause future generations
from recognizing the future Manifestation of God, or do so in a way
which prevents people of today from recognizing Bahá'u'lláh as the
Manifestation of God? Say for example, we express the remembrance of
God's name All-Merciful and we respond to a person with apathy,
indifference, or cruelty. Is this response in remembrance? Will a person
come to love God or love Bahá'u'lláh or will they be driven away?

\subsection{The Recitation of
Alláh-u-Abhá}\label{the-recitation-of-alluxe1h-u-abhuxe1}

``Each day, the recitation of Alláh-u-Abhá (God is Most Glorious) is to
be done 95 times while facing the Qiblih. Ablutions need to be done
prior.\footnote{Kitáb-i-Aqdas \#18} This is a daily practice of
remembrance which helps teach us one way to remember a Name of God. This
recitation can be done in secret or in the open, depending on where you
are. You can perform this all at once or spread it out during the day.
However, if you spread it out, ablutions would need to be performed each
time. You can use prayer beads or other tool to keep count, but do be
mindful of the fact you are not remembering the count. You are
remembering God.

This practice is derived from the Persian Bayán in Váḥid 5, Gate 17.
``In this gate, it is decreed that from sunrise to sunset, every soul is
permitted to recite ninety-five times phrases such as \emph{``God is
Most Glorious'' (Alláh-u-Abhá),} \emph{``God is Most Great''
(Alláh-u-A`ẓam),} \emph{``God is Most Manifest'' (Alláh-u-Aẓhar),}
\emph{``God is Most Radiant'' (Alláh-u-Anwar),} \emph{``God is Most
Exalted'' (Alláh-u-Akbar),} or similar exalted expressions.'' There
would be nothing wrong in adding any of these statements of praise to
your remembrance, as long as you are still capable of doing so with
spirit, joy, and fragrance. I could envision a worship service of the
People of Baha where nothing is happening but using music to
collectively sing these names and praises of God. Wouldn't that truly be
glorious?

\emph{Reflection: When remembrance is meant to be joyful and fragrant,
what helps you keep it sincere when you are tired, distracted, or
surrounded by others?}

With remembrance described in both its quiet and deliberate forms, we
now return to the Unknown Sister to see what courtesy looks like when it
is fueled by remembrance.

\section{Illuminations of The Unknown Sister - Emergent Virtues from
Courtesy}\label{illuminations-of-the-unknown-sister---emergent-virtues-from-courtesy}

\subsection{Dignity}\label{dignity}

Through remembrance, she feels no need to justify herself inwardly or
outwardly. The discovery does not diminish her worth, nor does her role
enlarge it. Dignity settles her posture. She is upright but uninflated.
She does not rush to explain, apologize, or defend. Remembering God
frees her from measuring herself against imagined judgments, allowing
her to stand quietly within the weight of what she carries.

\subsection{Fairness}\label{fairness}

Remembrance keeps her from narrowing the situation to what is easiest or
most familiar. As God's justice comes to mind, she becomes aware that
every person touched by this truth has a claim beyond her preferences.
Fairness does not yet demand action, but it reshapes perception,
preventing her from unconsciously privileging one life, one comfort, or
one narrative over another.

\subsection{Kindness}\label{kindness}

In remembrance, the unknown girl becomes more than an abstract
complication. She is a soul, loved by God, living an ordinary life with
its own tenderness and vulnerability. Kindness emerges not as sentiment,
but as restraint---an unwillingness to think carelessly or harshly about
anyone involved. Remembering God softens her inner language before any
outer words are spoken.

\subsection{Purity}\label{purity}

Remembrance clarifies her intention. She notices how easily curiosity,
fear, or a desire for control could contaminate her thoughts. Purity
keeps her from rehearsing explanations or imagining outcomes that serve
her own relief. By returning to God inwardly, she keeps the situation
from becoming a stage for ego, even privately within her own mind.

\subsection{Radiance}\label{radiance}

As remembrance steadies her, a quiet calm begins to show. Not
cheerfulness, not resolve, just presence without agitation. Radiance
here is subtle: the absence of sharpness, the easing of tension in the
room, the way her stillness reassures her child and does not alarm her
brother. It is light without display.

\subsection{Refinement}\label{refinement}

Remembrance trains her in care. She becomes attentive to small things
which matter. Her tone, her silences, the way she folds the pages, the
way she answers her child are all meaningful. Refinement does not mean
elegance; it means avoiding harm through roughness or haste. Even not
speaking becomes an act shaped by consideration rather than avoidance.

\subsection{Courtesy (Innate Virtue)}\label{courtesy-innate-virtue}

In this moment, courtesy is remembrance made visible. By remembering
God, she remembers others as souls rather than obstacles. Courtesy
governs her presence, restraining impulse and sharpening awareness. It
does not resolve the situation, but it ensures that whatever follows
will not be careless, dismissive, or self-serving.

\emph{Reflection: Which of these emergent virtues---dignity, fairness,
kindness, purity, radiance, or refinement---most often fails first when
you feel pressure, and why?}

To end the chapter, we now turn to a warning, because remembrance is not
proven by speech but by the way power is used.

\section{In Closing}\label{in-closing}

Here is a prophetic warning Bahá'u'lláh sent to Napoleon III and his
fake claim of remembrance and how it violated the innate virtues of
courtesy by betraying the people of Europe:

\begin{quote}
O King, We heard a word from you when the King of Russia asked you about
the judgment of war. Your Lord is the All-Knowing, the All-Informed. You
said, I was asleep in the cradle, the call of the oppressed woke me
until they were drowned in the Black Sea. Thus We heard, and your Lord
is a witness to what I say. We testify that it was not the call that
woke you but desire, for We tested you and found you secluded. Recognize
the tone of speech and be of the discerning.

We do not wish to return to you a bad word to preserve the station We
granted you in the visible life. We chose courtesy and made it a habit
of the close ones. It is a garment that suits every soul, young and old.
Blessed is the one who made it the adornment of his body, and woe to the
one who is deprived of this great favor. If you were the possessor of
the word, you would not have cast the Book of God behind your back when
it was sent to you from the Mighty, the Wise. We tested you with it, and
We did not find you as you claimed. Arise and make up for what you
missed. The world will perish and what you have, and the dominion will
remain for God, your Lord, and the Lord of your forefathers. You should
not limit matters to what your desire wants. Beware the sighs of the
oppressed, protect him from the arrows of the oppressors. What you did
will cause matters to differ in your kingdom, and the dominion will
leave your hand as a result of your actions. Then you will find yourself
in manifest loss, and earthquakes will seize all tribes there unless you
arise to support this cause and follow the Spirit in this straight
path.\footnote{Suríy-i-Haykal (Súrah of the Temple)}
\end{quote}

True remembrance is not proclaimed but it is proven. May our remembrance
never be a veil, but a mirror turned toward the Sun, that all who
witness us may remember Him. Remembrance deepens prayer and recitation
by teaching the soul how to stay turned toward God even when no words
are being spoken. By learning to receive through the senses and to
consciously magnify God with joy and fragrance, courtesy emerges in
forms that protect others from harm: dignity, fairness, kindness,
purity, radiance, and refinement. In Chapter 11 we will turn to
reflection, and see how the mind learns to listen inwardly so that
remembrance becomes insight rather than mere emotion.

\newpage{}

\chapter{Reflection}\label{reflection}

The next spiritual practice Bahá'u'lláh describes in the Kitáb-i-Aqdas
is reflection. Regularly we are commanded to reflect throughout the
revelation. Reflection as a spiritual practice is not something you
commonly hear about, especially in Abrahamic religious circles. I
typically viewed reflection as an act of mindful contemplation, perhaps
thinking about how Bahá'u'lláh teaches ``O Son of Being! Bring yourself
to account each day before you are summoned to account, for death will
come suddenly, and you will stand in judgment over your own
self.''\footnote{The Hidden Words of Arabic \#31} Bringing to account
had always seemed to be about asking myself if I have done what I was
supposed to do, such as obeying Bahá'u'lláh's laws. What if there is
more to this accounting ledger?

Reflection, if I had to provide a definition, is the process of
integrating prayer, recitation, and remembrance into one's soul so as to
prepare the soul to honor God. Reflection serves as the bridge towards
increased God consciousness within the self. Many religious traditions
define this process as the core of their religion. Islam calls it
submission to God's will although reflection is missing from the
traditional pillars of faith. Reflection is recognizing that our
accounting ledger is about how our inner self measures against God's
Self.

We aren't judging based on the standards of humankind, but we are
comparing how we align with God's desire for us. If Muhammad is an
example, he regularly retreated to the cave of Hira to reflect, and it
was here when he first received revelation. After Jesus was baptized, he
spent 40 days in the desert wilderness of Galilee, reflecting and
preparing himself for His mission. Moses spent 40 days on Mount Sinai
whereupon the law of the Torah eventually emerged. Bahá'u'lláh, before
publicly announcing His mission from God, spent 2 years in Sulaymaniyah
as a Naqshbandi Sufi detached from all He was attached to. Reflection is
how God shines through us.

\emph{Reflection: When you ``bring yourself to account,'' what do you
actually audit? Is there anything you tend to avoid?}

With reflection defined as integration, we can now ask what purposes
reflection serves, because purpose determines what kind of understanding
is being sought.

\section{Purposes of Reflection}\label{purposes-of-reflection}

Reflecting is for understanding\footnote{Kitáb-i-Aqdas \#5} through
discovering hidden pearls\footnote{Kitáb-i-Aqdas \#136} such as pearls
of mysteries from the Ocean.\footnote{Kitáb-i-Aqdas \#137} What are we
to understand?

\subsection{The Choice Sealed Wine}\label{the-choice-sealed-wine}

One of the hidden pearls of mystery is the Choice Sealed Wine itself.
The Kitáb-i-Aqdas is not merely a book of laws. Imagine you are
practicing recitation of a verse of the Kitáb-i-Aqdas and you take some
time to reflect upon the verse. Depending on which spiritual world you
are passing through, you might gain an understanding and on a different
day, gain another understanding due to passing through a different
spiritual world. The hidden pearls are plural, which means there could
be countless forms of wisdom to be gained.

\subsection{The Bayán}\label{the-bayuxe1n}

There is a section of the Kitáb-i-Aqdas addressed specifically towards
the People of Eloquence, also known as the Bábís. Bahá'u'lláh says
whoever ``reflects on these verses (from the Bayán) and discovers the
hidden pearls within them, by God, will find the fragrance of the
Merciful emanating from the direction of the prison, and his heart will
hasten toward it with longing---nothing will stop him, not the armies of
the heavens or the earth.''\footnote{Kitáb-i-Aqdas \#136} The Primal
Point says ``for nothing has been ordained except for the sake of the
Day of the Manifestation of He Whom God shall make manifest, so that if
a soul stands before God, there should be no condition within them
except refinement---that nothing may be witnessed in them except the
love of God.''\footnote{The Persian Bayán Váḥid 8 Gate 6} The Bayán is a
love letter to Bahá'u'lláh and to reflect the Bayán, is to discover
hidden pearls of love for Bahá'u'lláh.

\subsection{The Qiblih}\label{the-qiblih}

Bahá'u'lláh says to ``reflect regarding the Qiblih\footnote{Kitáb-i-Aqdas
  \#137} pearls of mysteries from the ocean.'' To continue the theme of
love and adoration, what would it mean to reflect regarding the Qiblih?
Reflection, such as in this case, is not merely to acknowledge a fact,
but to feel the nuances from the ocean of meanings. Today, what am I
turning towards? Tomorrow, what am I turning away from? Reflection
regarding the Qiblih may offer opportunities to identify what we love,
and from what we love, what do we turn to more than Bahá'u'lláh? Do we
have multiple points of adoration?

\subsection{Dawn and Evening on the Mercy and Favors of
God}\label{dawn-and-evening-on-the-mercy-and-favors-of-god}

Bahá'u'lláh tells us to ``reflect, during the dawn and evening, on the
mercy and favors of God. Give thanks.''\footnote{Kitáb-i-Aqdas \#33}
This could be a practice we take during the obligatory prayer. Maybe
during the prayer, we take moments to pause and reflect on the portion
of the prayer recited. In those moments, we are meditating on a mercy
God had bestowed on you. You might not be in prayer, but curled up in
bed or your sofa, reflecting on a favor. This is also a practice to help
you, even on a burdensome day with some struggle or pain, to identify
something good within it. This type of reflection is a difficult
practice, but one which may help provide perspective to certain
feelings. The heart is a great place for hidden pearls to be unveiled.

\subsection{Your Lifecycle}\label{your-lifecycle}

The Kitáb-i-Aqdas tells us to ``reflect on what you were created from;
indeed, all were created from a base fluid.''\footnote{The
  Suríy-i-Haykal (To the King of Paris)} Bahá'u'lláh also says to
``reflect on your end, don't be unjust.''\footnote{Kitáb-i-Aqdas \#148}
He provides counsel for how we live our lives between our creation and
our end. He says ``the days of your life pass as the winds blow, and
your glory will be rolled up as was the glory of those before you.
Reflect, O people, where are your past days, and where are your vanished
years? Blessed are the days that were spent in the remembrance of God,
and the hours devoted to His praise, the All-Wise.''\footnote{Kitáb-i-Aqdas
  \#40}

Reflection here is combined with remembrance. What of God do we
remember? How is a name of God manifesting within my life? Is there too
much dust on my soul's mirror? Do I praise something more than God?
Maybe reflecting on the beginning, knowing all people were created from
an egg, a sperm, and the breath of life from God gives us new insights
and understandings in how we apply our constellation of virtues.
Awareness of the inevitable end may keep a regular perspective for the
unique situations we find ourselves in.

\subsection{The World and the Conditions of Its
People}\label{the-world-and-the-conditions-of-its-people}

Reflection is also a bridge between the individual's spiritual life and
how we use our life to honor God within the world. Bahá'u'lláh says to
``reflect upon the world and the condition of its people.''\footnote{The
  Suríy-i-Haykal (To the King of Paris)} Maybe you are reading a
political editorial or having a conversation with a friend regarding a
social issue. Emotions are high as there are reasons to feel strongly.
What if you reflect during such moments? Your prayers, recitation, and
remembrance have prepared you to reflect and how to respond. You may
view the world using a map, or view it from an image taken from space.
You may notice or feel its vastness, or its smallness, depending on your
reflection. Both can be true. You may feel the joys and pains of people
near or far from you. You might feel aspects of oneness and aspects of
difference, and see the truths of both aspects. We may observe what we
should not do, and the potentials harms which occur when people lack
faith, lack love for God, or lack the fear of God.

Reflection then, is not solely about how our souls reflect God for our
own liberation, but reflection is also about the potential liberation of
the world and its people. Reflection is what helps make prayer,
recitation, and remembrance active within the world. Reflection is the
initiator of the Cause of God.

\emph{Reflection: Which purpose of reflection draws you most---hidden
pearls, love of the Bayán, discernment of the Qiblih, gratitude at dawn
and evening, mortality, or the condition of the world---and why?}

With the purposes established, we can now move from why we reflect to
how we reflect, because method determines whether insight becomes
transformation.

\section{How Do We Reflect?}\label{how-do-we-reflect}

\begin{quote}
An example is set forth in the supreme word, upon which the entirety of
religion is established. Through its utterance, the foundation of all
religion is confirmed. In the hour of death, all will speak this word
and return to Him. The reflections within mirrors inevitably return to
their origin. When the mirrors reflect the image of the sun, they return
to it, for their existence began with it. The mirrors' purpose lies
solely in their capacity as mirrors, reflecting the sun from which they
originated.\footnote{The Persian Bayán: Váḥid 1 Gate 9}
\end{quote}

The quote above from the Báb encapsulates perfectly what it means to be
a mirror. We are what we reflect, in this life and thereafter. If the
purpose of the mirror of the soul is to reflect the light of God, what
inhibits the ability to reflect?

It could be the orientation of our mirror. Reflection of divine light
could be anywhere from 0 to 100\%, merely based on the direction our
mirror is facing. Our ability to reflect could be influenced by objects
getting between the mirror and the Sun. These objects can be veils,
created by ourselves or by others, which are used to keep a person from
identifying the light. When a veil exists, what direction will a mirror
turn? Other objects can be accumulated dust. This dust can represent a
mirror which reflected the light radiantly, but stopped its spiritual
practice. Maybe the mirror thought once it attained a spiritual station,
that the station was permanent, mistakenly doing what it wanted. Each
day which passes allows more dust to settle on the mirror. Eventually,
no light reflects as the dust absorbs the light.

How do we reflect to ensure our best reflection, so that our soul may
attain the liberation of paradise?

\subsection{Detach and Purify}\label{detach-and-purify}

Bahá'u'lláh says in the Book of Certitude: ``the door mentioned in the
statement that the servants will not reach the shore of the sea of
knowledge except by completely cutting off from everything in the
heavens and the earth. Purify yourselves, O people of the earth, so that
you may reach the position that God has destined for you, and enter into
a pavilion that God has raised in the sky of explanation.''

We learned earlier how prayer can help increase the virtue of detachment
and how remembrance can help increase the virtue of purity. These
practices help identify what is important for our souls, but we can also
learn a lot by what is missing from the obligatory prayers, the verses
of God, and the names and attributes of God.

For example, we do not see any description of human race or differences
of races. There is not a single writing from the Báb or Bahá'u'lláh
discussing race. Yet, during the European Enlightenment emerging in the
15th century, the idea of race and the subsequent ideologies of race
supremacy spread throughout the world as Europe colonized various parts
of the globe. By the time Bahá'u'lláh emerged from the prison of Akka in
the late 1870's, nearby Beirut was an intellectual hotspot where race
highly influenced the sociopolitical discourse. Beirut was the most
common port of entry for pilgrims to Bahá'u'lláh. It was where his son,
Abbas Effendi, sought medical care. Yet, race was notably absent from
the Revelation of God through Bahá'u'lláh. This begs the question: Is
race real?

Being able to set aside certain assumptions about what is real or
unreal, despite there being common ideas and ideologies current among
the people of the age or people of a culture, is vital to reflection.
When Bahá'u'lláh says certitude can only happen by separating ourselves
from these assumptions, we are no longer allowing those assumptions to
be chains weighing us down. These chains are often described by
Bahá'u'lláh as vain imaginings and idle fancies. A vain imagining is
something we create in our own head which is not real and serves no
positive purpose in our world of existence. Reflection helps use the
spiritual practices to help identify what is real, what is actually
true, versus what people may tell us is real and true. This discernment
is incredibly important.

Other vain imaginings which people often attach themselves to can
include political ideologies or parties, religious identity and labels,
conditions of the ego, or even social and economic outcomes. Detach from
all save God. To purify in these types of context could include
identifying what we love or hate and understanding why. Do we allow our
feelings to judge where we shouldn't? Do we blindly follow what is
popular or what our families commanded us to follow? Do we get caught up
in our own self doubts and insecurities? These things act to turn our
mirrors in directions which reflect corruption, not the radiant light of
God. Even if we proclaim ``I believe'' and still remain attached to
things which may not even be real or conducive to the liberation of
self, soul, and society, we accumulate a substantial amount of dust on
our mirrors.

God is the creator of all, but maybe Satan is merely the corrupter.

\subsection{Cleanse the Mirror}\label{cleanse-the-mirror}

So far I had avoided discussing meditation. Meditation is not explicitly
commanded by the Báb or Bahá'u'lláh and it is not a spiritual practice
by itself. Yet, meditation can be a tool to help a person reflect. When
we have identified what to detach from or purify from, it can be
beneficial to take time to be quiet.

\begin{quote}
The essence of faith is to speak little and to act abundantly; and if
one's words exceed his deeds, know that his nonexistence is better than
his existence, and his demise is better than his survival. The
foundation of well-being is silence, consideration of the consequences,
and withdrawal from the people.\footnote{Asl-i-Kullu'l-Khayr (Essence of
  All Good)}
\end{quote}

Reflection requires silence, stillness, and a sincere desire to look
both inward and while looking outward, deeply. Our spiritual eyes must
be piercing. Tools such as meditation or yoga can be useful, as long as
the practice themselves are not distracting. The purpose is not to empty
the mind, but instead to allow the mind, heart, and soul to align with
what God wills. Within my own reflection practice, I might curl up on a
couch and watch birds eat, sing, and play. Hiking and camping are
incredible times to reflect, even if you are with a loved one navigating
the depths of your relationship. The most socially isolating time I am
able to take is a road day trip, listening to music both vocal and
instrumental. I've learned more during these 3 periods of time than I
ever have only reading books, in school, or doing the common roles of
life. The times of reflection helps consolidate all of these sources of
knowledge, feeling, and experience into a form greater than the sum of
their parts.

Through this process we can identify a truth about every aspect of our
lives, and allow spiritual discernment to increasingly act as our
compass. While imperfect, we strive for the process of perfection.

\subsection{Reorient}\label{reorient}

The eventual goal is for the soul to be reoriented, facing away from
these attachments, imaginings, and worldly affairs and turned towards
God. Bahá'u'lláh asks us ``O Son of the Cloud I call you to eternal
life, yet you seek annihilation. Why have you turned away from what I
love and turned toward what you desire?''\footnote{The Hidden Words of
  Arabic \#23}

I will close this section with the first paragraphs of Bahá'u'lláh's
Seven Valleys describing the Valley of the Annihilation of Self:

\begin{quote}
Upon ascending the lofty stages of bewilderment, the seeker enters the
valley of true poverty, the principal annihilation. This stage is marked
by the annihilation of the self and the subsistence in God\ldots. For
when the sincere lover and the concordant beloved reach the meeting of
the loved one and the lover, they ignite a fire from the radiance of the
loved one's beauty and the heart's fire of the lover. This fire burns
all the curtains and veils, even burning all that is with them, down to
their core and shell, until nothing remains but the beloved\ldots.This
is because what is with people is limited to their own limitations, and
what is with God is holy beyond that.
\end{quote}

\emph{Reflection: When you notice ``dust'' on the mirror, what is your
most common form of dust and what helps you cleanse it without
self-condemnation?}

With method established, we can now return to the Unknown Sister and see
how reflection gives pure truthfulness its shape, so truth becomes
alignment rather than weapon.

\section{Illuminations of The Unknown Sister - Emergent Virtues from
Pure
Truthfulness}\label{illuminations-of-the-unknown-sister---emergent-virtues-from-pure-truthfulness}

\subsection{Eloquence}\label{eloquence}

Through reflection, she senses that truth does not need force to be
real. Eloquence begins as restraint, which is the understanding that
when the time comes, truth must be expressed clearly, without excess or
evasion. Reflection teaches her that how truth is spoken may matter as
much as what is spoken, and that beauty and care can prevent truth from
becoming a wound.

\subsection{Heedfulness}\label{heedfulness}

Reflection awakens her to responsibility. She can no longer treat this
as an abstract fact or a future problem. Heedfulness makes her aware
that lives are already being shaped by what she now knows, even in
stillness. This awareness does not rush her, but it prevents numbness,
keeping her soul awake rather than sheltered by delay.

\subsection{Perception}\label{perception}

As she reflects, she begins to see multiple truths at once without
forcing them into competition. The situation is not singular but
layered: legal, familial, emotional, spiritual. Perception allows her to
hold these realities together, recognizing that more than one truthful
path may exist, each refracting the same light differently.

\subsection{Wisdom}\label{wisdom}

Reflection tempers insight with patience. Wisdom does not urge her
toward immediate resolution, but toward fitting application. She
understands that knowing what is true does not mean knowing when or how
it should be acted upon. Wisdom grows as she senses the difference
between correct action and premature action.

\subsection{Reason}\label{reason}

Reflection brings coherence to her thoughts. Emotion no longer dominates
the landscape unchecked. Reason helps her see consequences unfolding
beyond the present moment---how words spoken now may echo for years, how
silence may also shape futures. It does not replace compassion; it
steadies it.

\subsection{Sincerity}\label{sincerity}

In reflection, she examines her motives without flinching. She notices
the subtle pull of self-protection, the desire to appear composed or
benevolent. Sincerity strips these away gently, aligning her concern
with love rather than image. Whatever she may eventually do, she wants
it to arise from care, not from fear or convenience.

\subsection{Pure Truthfulness (Innate
Virtue)}\label{pure-truthfulness-innate-virtue}

In this moment, truthfulness is not disclosure but alignment. Reflection
draws her soul into harmony with what is real, freeing her from
illusion, denial, or self-deception. She becomes less concerned with
managing appearances and more concerned with living honestly before God,
trusting that truth, when carried with love, will eventually find its
rightful expression.

\emph{Reflection: What are various things which can impair the emergent
virtues from pure truthfulness? How does reflection overcome these
impairments?}

\section{Summary}\label{summary-4}

Reflection integrates prayer, recitation, and remembrance so that the
soul becomes capable of discerning hidden pearls, cleansing the mirror,
and reorienting toward God. It is both inward accounting and outward
discernment, shaping truthfulness into eloquence, heedfulness,
perception, wisdom, reason, and sincerity. In Chapter 12 we will turn to
honoring God, and see how reflection becomes action, and how the inner
alignment of the soul is translated into how we live.

\newpage{}

\chapter{Honoring God}\label{honoring-god}

\section{Introduction}\label{introduction-5}

In the opening of the Kitáb-i-Aqdas, which frames the purpose of the
Book, Bahá'u'lláh says ``The tongue of My power has spoken in the
dominion of My greatness, addressing My creation: ``Carry out My
ordinances out of love for My beauty.'' Blessed is the lover who has
inhaled the fragrance of the Beloved from this word, from which the
breezes of bounty have wafted in an indescribable manner.'' This places
an important context for our spiritual practice. Every command is to be
carried out of love.

This love is not one-way. In Chapter 1 we learned how one of God's names
is Love and Beloved. God's love not only burns away the veils which
prevent the soul from being a true mirror, it is also the key to the
hidden treasure. The hidden treasure is the liberation of the soul.
Created noble, the soul is born already bearing honor, a sign of God's
hope that each one may rise to the seat of divine nobility.

Honor is the condition of being uplifted by God's love. God desires that
every soul may turn in their hearts to a seat of honor and nobility.
Although God does not need us to uplift Him, there are ways we can honor
our love of the Beloved. When this book transitions to the various
ordinances, these are all pathways to honor God's love for us and to
express love for God. These pathways are ways to live in alignment with
divine truth, pathways to that hidden treasure. Some of these pathways I
feel are rather spiritual in nature and key components to our spiritual
practice, hence naming this spiritual practice as honoring God.

This chapter will introduce various aspects of honoring God through
individual and social acts of devotion. Some are woven into the rhythm
of the Bayanic calendar, a sacred cycle of 19 months, fasts, and
festivals, while others are more spontaneous acts of devotion. These are
practices which form the bedrock of the spiritual identity of a
community. In honoring God, we begin to walk the path of law not as
burden, but as love made visible.

\emph{Reflection: When you think of ``honoring God,'' do you feel
burden, love, or something in between, and why?}

Now we will begin with the most communal and visible act of devotion,
the building of places dedicated to remembrance.

\section{Build the
Mashriq'u'l-Adhkár}\label{build-the-mashriqul-adhkuxe1r}

The Mashriq'u'l-Adhkár means the Dawning Place of the Remembrance of
God. It is every house built for the remembrance in cities and villages.
These houses are to be built in the name of the Lord of all religions.
They are to be built as perfectly as possible. They are to be adorned
with that which befits them, not with images and likenesses. The Dawning
Place has chambers, where children may recite the verses in the best
melodies. They should be established with wisdom, so that its building
does not cause conflict. For example, Bahá'u'lláh advised the community
of Tehran and some others to wait to establish the Mashriq'u'l-Adhkár.
Bahá'u'lláh says:

\begin{quote}
``Blessed is the hive, the house, the station, the city, the heart, the
mountain, the cave, the sanctuary, the wilderness, the land, the sea,
the island, and the dwelling where the mention and praise of God have
been raised.''
\end{quote}

While being perfect, they also do not need to be extravagant. For
example, Jinab-i-'Aziz and Mirza Muhammad Kazim established
Mashriq'u'l-Adhkárs at their houses. Their perfection was not the wealth
used to build them nor world renowned architects. The perfection was
from the love, devotion, sincerity, and purity of intention used in
their creation and use.

Bahá'u'lláh says we should ``turn towards the Dawning Place at dawn,
reflecting, remembering, reflecting, and seeking forgiveness.
Remembrance should be with spirit and fragrance. When we enter, we
should sit in silence, listening to the verses of God.'' This implies
the presence of someone reciting the verses of God aloud.

With Bahá'u'lláh himself being the breaker of dawn, the
Mashriq'u'l-Adhkár is designed to represent the dawn within each of us
who enter it, and perhaps the dawn of God's love in every city and
village where one is built. The Mashriq'u'l-Adhkár also is a symbol of
the houses where revelation started, such as the Báb's house in Shiraz
and Bahá'u'lláh's house in Baghdad. Each visit, in a way, can be as a
pilgrimage.

The Mashriq'u'l-Adhkár is the foundation of the spiritual life of the
community, honoring God as they feel within their liberating souls. If
many people were to attend at dawn, you would have an opportunity for
congregational remembrance. As remembrance is to be in melodious tones,
there could be music. Bahá'u'lláh ``remember(s) those who gathered or
will gather to serve God, that they may rejoice and be among the
thankful. Upon them are My glory, My remembrance, and My praise.''

As there would be chambers, there are also opportunities for private
remembrance, recitation, reflection, or prayer. The choice could be
yours, depending on how your spirit needs that day. Remembrance in the
Dawning Place is not required in the evenings, but I could see if a
person wanted to do their evening remembrance in a Dawning Place, that
should be an option. The Mashriq'u'l-Adhkár is the heartbeat of a
community's spiritual life, built not to impose rituals, but to invite
liberation. Whether for solitary prayer or congregational praise,
whether sung in melody or whispered in longing, it honors God in a way
each soul recognizes as home.

\emph{Reflection: If you entered a Mashriq'u'l-Adhkár at dawn, what
would you hope to leave behind, and what would you hope to carry out?}

From communal spaces of remembrance, we now turn to the most personal
journey of devotion, pilgrimage.

\section{Pilgrimage}\label{pilgrimage}

The Kitáb-i-Aqdas enjoins pilgrimage to the Sacred House. I like to
think of the Sacred House as a recurring station in sacred history, much
like how the title Manifestation of God has been fulfilled by many
people. The Sacred House is where revelation first dawned. When the
revelation of the Báb was first proclaimed, the House was the Báb's home
in Shiraz, Iran. When the revelation of Bahá'u'lláh was first announced,
the House was Bahá'u'lláh's home in Baghdad, Iraq. Bahá'u'lláh revealed
two pilgrimage tablets for both Sacred Houses.

The purpose of pilgrimage is a way to physically portray the spiritual
journey to God. As such, this way to honor God is quite sacred. There is
no time frame, no schedule, no permission, nor any tour guides to point
the way or tell you what to do. Pilgrimage is not an act of tourism. It
is solely an act of devotion. I like to think of it that God desires to
welcome you to His house and you are the honored guest. As an act of
hospitality, God will provide for whatever you actually need for the
nourishment and comfort of the soul.

The pilgrimage is for only those who are able to. Any able-bodied man
who is able to perform pilgrimage should as long as it does not cause
financial hardship. Women are exempt. This does not mean a woman cannot
do pilgrimage, but there is no penalty or judgment by God if an
able-bodied woman with financial means does not do pilgrimage.

Bahá'u'lláh also allowed a person who desired to conduct pilgrimage, but
was unable to, for Himself to fulfill the pilgrimage requirement. Today,
visiting the Sacred House is not fully possible. Both houses no longer
stand and the land is used for other purposes. However, pilgrimage is
still partially possible. I have not done so yet, but it is my hope to
visit one of locations. In the tablets Bahá'u'lláh revealed for
pilgrimage, the Suriy-i-Hajj I (Shiraz) and II (Baghdad), there are
prayers and acts one can take leading up to arriving to the location
where the Houses stood. If I am unable to, I do hope that I may do so in
a dream or in some other meditative state.

\emph{Reflection: What would make a journey feel like devotion rather
than tourism for you, even if no one else understood it?}

From sacred places and journeys, we now turn to sacred resources and the
purification of wealth.

\section{Rights of God
(Ḥuqúqu'lláh)}\label{rights-of-god-ux1e25uquxfaqulluxe1h}

The Rights of God, also known as Ḥuqúqu'lláh, is a way to honor God
through the purification of wealth. The Báb had initially introduced the
Rights of God in Vahid 5, Gate 19 of the Bayan. In the Bayan, the Rights
of God were assigned to He Whom God Shall Make Manifest, and fulfilled
through Bahá'u'lláh. Bahá'u'lláh says the purpose is to bring people to
higher stations. Bahá'u'lláh describes the payment as mandatory, saying
not to withhold ourselves from this great bounty. If the Ḥuqúq was not
paid with joy and radiance, Bahá'u'lláh would not accept it.

The Rights of God is paid solely from wealth, not income. If one's
wealth reaches 100 mithqáls (425g) of gold, 19 mithqáls (80.75g) thereof
belong to God. Basically it is a 19\% contribution. In a letter to
Jinab-i-Samandar Bahá'u'lláh explains there is a minimum amount upon
which the Ḥuqúq is paid, which is 19 mithqáls (80.75g). Wealth is
similar to a balance sheet of a business. You take assets, subtract
liabilities, and you end up with total equity. Once the Ḥuqúq has been
paid on a portion of wealth, that portion is not subject to payment
again. When a new 19 mithqáls is reached, a new 19\% payment is
required. Endowments for charity are also a part of the Ḥuqúqu'lláh.

Bahá'u'lláh would use the Rights of God to take care of the affairs of
the believers, as these things require money. These affairs included
teaching, taking care of the poor, prisoners, travel, and other affairs.
He had instructed an unspecified woman to give two-thirds of the
Ḥuqúqu'lláh for good works, and one-third to the Holy Court, which was
to Bahá'u'lláh. In another letter, He says all of it can be paid for
charity. It seems there was no rigid methodology in how Ḥuqúqu'lláh was
used. The Kitáb-i-Aqdas also specifies endowments can be used for
elevated places, which could be the two Houses to be established in
every city, such as the Mashriq'u'l-Adhkár and the House of Justice.

For those who collected the Rights of God, He warned not to betray the
Rights of God. This means not to use it for personal gain or any other
action which goes against the Revelation. Bahá'u'lláh required a
collector to provide a receipt and to keep records about how each Ḥuqúq
payment was dispersed. Financial accountability is guaranteed.
Eventually, Bahá'u'lláh enjoins the Trustees of the Houses of Justice to
receive and present the Rights of God.

This way to honor God is a pathway to honor the rest of the People of
Baha, as it is the primary way to be able to afford the care of each
other and the care of the Cause of God.

\emph{Reflection: When money is framed as ``trust'' rather than
``possession,'' what changes in your sense of responsibility?}

From the purification of wealth, we now move to the daily purification
of effort through work.

\section{Engaging in an Occupation}\label{engaging-in-an-occupation}

Related to the Rights of God, Bahá'u'lláh wants us to honor God through
engaging in an occupation. Working is equivalent to worship. The purpose
is to engage in activities which benefit yourself and others. The
purpose is not necessarily to gain wealth solely for your own pleasures
and comforts. He says in BH10890 ``Blessed is he who beareth burdens but
causeth none to bear his own. Let him engage in craft and profession. A
single coin earned thereby is, in the sight of God, more beloved than
the treasure that is gathered unrightfully and made ready.''

Working is a responsibility. When Bahá'u'lláh says a blessed person does
not cause burdens for another, this can be viewed in various ways. For
example, maybe you own a business and employ a person. Paying a wage
which keeps the employee in poverty could be a burden. Another way to
read this is you do not force another to take financial responsibility
for you when you are able to work.

\emph{Reflection: What would it look like for your work to feel like
worship?}

From labor as devotion, we now turn to giving as devotion through
regular charity.

\section{Zakát}\label{zakuxe1t}

The Kitáb-i-Aqdas enjoins upon us the payment of zakát for whatsoever is
beneath the value of Ḥuqúqu'lláh. In 1873 when the Kitáb-i-Aqdas was
revealed, the minimum was only going to be specified if God wills. A few
times, Bahá'u'lláh tells the believers to follow the zakát teachings of
the Qur'an.

``Alms are for the poor and the needy, and those employed to administer
the (funds); for those whose hearts have been (recently) reconciled (to
Truth); for those in bondage and in debt; in the cause of Allah; and for
the wayfarer: (thus is it) ordained by Allah, and Allah is full of
knowledge and wisdom.''

The Qur'an does not specify an exact amount, but it states in multiple
places to be regular in charity, or who are active in deeds of charity.
Zakát's purpose is very similar to the Rights of God, except there is no
minimum threshold of wealth a person must achieve to pay. There is also
no set amount to pay. The only command is regularity. How one conducts
zakát is completely discretionary and it relies entirely on one's
conscience and unique circumstances.

For example, in the days of Bahá'u'lláh a person could pay a slave's
debt, or take another financial measure to free them. A person could
offer a bed to someone traveling. A Trustee of alms could receive money
for their role as a Trustee. There is no rigid practice. One may honor
God as they please.

\emph{Reflection: If giving is meant to be regular rather than
spectacular, what kind of giving would you actually sustain?}

With giving established as a steady rhythm, we now turn to sacred time
itself and the calendar that shapes communal devotion.

\section{The Bayanic Calendar}\label{the-bayanic-calendar}

The Bayanic calendar was first introduced by the Báb in the Book of the
Reckoning. In it, He creates a system of divine time with 19 months of
19 days. Each month is named after a name of God, and the days of the
week are named after spiritual qualities. Below is a numbered list
describing the months:

\begin{enumerate}
\def\labelenumi{\arabic{enumi}.}
\tightlist
\item
  Month \#1: Arabic Name: Bahá'; English Name: Splendor; Elemental
  Phase: Fire; Thematic Phase: Glorification; Spiritual Function: The
  Point; essence of all months; attributed to ``He Whom God Shall Make
  Manifest''; Naw-Rúz = Day of ``There is no God but God''
\item
  Month \#2: Arabic Name: Jalál; English Name: Glory; Elemental Phase:
  Fire; Thematic Phase: Glorification; Spiritual Function: Kindles the
  fire of hearts; glory of divine majesty
\item
  Month \#3: Arabic Name: Jamál; English Name: Beauty; Elemental Phase:
  Fire; Thematic Phase: Glorification; Spiritual Function: Radiates
  divine beauty; fuels yearning and love
\item
  Month \#4: Arabic Name: 'Aẓamat; English Name: Grandeur; Elemental
  Phase: Air; Thematic Phase: Praise; Spiritual Function: Spirits are
  created and sustained through grandeur
\item
  Month \#5: Arabic Name: Núr; English Name: Light; Elemental Phase:
  Air; Thematic Phase: Praise; Spiritual Function: Light of divine
  guidance permeates all realities
\item
  Month \#6: Arabic Name: Raḥmat; English Name: Mercy; Elemental Phase:
  Air; Thematic Phase: Praise; Spiritual Function: Mercy as the
  sustaining breath of spiritual life
\item
  Month \#7: Arabic Name: Kalimát; English Name: Words; Elemental Phase:
  Air; Thematic Phase: Praise; Spiritual Function: Divine Word as
  sustaining truth
\item
  Month \#8: Arabic Name: Kamál; English Name: Perfection; Elemental
  Phase: Water; Thematic Phase: Oneness; Spiritual Function: Dies to
  imperfection, lives in affirmation of truth
\item
  Month \#9: Arabic Name: Asmá'; English Name: Names; Elemental Phase:
  Water; Thematic Phase: Oneness; Spiritual Function: Realization of
  divine Names beyond negation
\item
  Month \#10: Arabic Name: 'Izzat; English Name: Might; Elemental Phase:
  Water; Thematic Phase: Oneness; Spiritual Function: Strength comes
  through detachment from ego
\item
  Month \#11: Arabic Name: Mashíyyat; English Name: Will; Elemental
  Phase: Water; Thematic Phase: Oneness; Spiritual Function: Submission
  to the divine Will negates self-will
\item
  Month \#12: Arabic Name: 'Ilm; English Name: Knowledge; Elemental
  Phase: Water; Thematic Phase: Oneness; Spiritual Function: True
  knowledge arises through divine affirmation
\item
  Month \#13: Arabic Name: Qudrat; English Name: Power; Elemental Phase:
  Water; Thematic Phase: Oneness; Spiritual Function: Divine power is
  manifest through surrender
\item
  Month \#14: Arabic Name: Qawl; English Name: Speech; Elemental Phase:
  Dust; Thematic Phase: Magnification; Spiritual Function: Exalted
  speech gives life to souls purified from falsehood
\item
  Month \#15: Arabic Name: Masá'il; English Name: Questions; Elemental
  Phase: Dust; Thematic Phase: Magnification; Spiritual Function:
  Questioning as a means of spiritual revival
\item
  Month \#16: Arabic Name: Sharaf; English Name: Honor; Elemental Phase:
  Dust; Thematic Phase: Magnification; Spiritual Function: Honor gained
  by steadfastness in God's love
\item
  Month \#17: Arabic Name: Sultán; English Name: Sovereignty; Elemental
  Phase: Dust; Thematic Phase: Magnification; Spiritual Function: Life
  granted by renouncing all but divine sovereignty
\item
  Month \#18: Arabic Name: Mulk; English Name: Dominion; Elemental
  Phase: Dust; Thematic Phase: Magnification; Spiritual Function: All
  divine elements are gathered into earthly dominion
\item
  Ayyám-i-Há: English Name: The Days of Ha; Thematic Phase: Intercalary;
  Spiritual Function: Days outside the calendar cycle; devoted to
  hospitality, charity, and spiritual preparation before the Fast.
\item
  Month \#19: Arabic Name: `Alá'; English Name: Loftiness; Elemental
  Phase: Dust; Thematic Phase: Magnification; Spiritual Function: Final
  culmination; fruit of divine creation realized
\end{enumerate}

Here are the days of the week:

\begin{enumerate}
\def\labelenumi{\arabic{enumi}.}
\tightlist
\item
  Arabic Name: Jalál; English Meaning: Glory; Julian Counterpart:
  Saturday
\item
  Arabic Name: Jamál; English Meaning: Beauty; Julian Counterpart:
  Sunday
\item
  Arabic Name: Kamál; English Meaning: Perfection; Julian Counterpart:
  Monday
\item
  Arabic Name: Fidál; English Meaning: Grace; Julian Counterpart:
  Tuesday
\item
  Arabic Name: 'Idál; English Meaning: Justice; Julian Counterpart:
  Wednesday
\item
  Arabic Name: Istijlál; English Meaning: Majesty; Julian Counterpart:
  Thursday
\item
  Arabic Name: Istiqlál; English Meaning: Independence; Julian
  Counterpart: Friday
\end{enumerate}

Bahá'u'lláh uses this calendar in the Kitáb-i-Aqdas to describe ways to
honor God with regularity, based on the sacred pulse of the calendar.
These acts are opportunities for all People of Baha to share
experiences, memories, and develop stronger bonds of love, friendship,
and kinship. The calendar also helps express the potential spiritual
worlds we might pass through in greater cosmic order. The next sections
will be chronological events and acts Bahá'u'lláh prescribes.

\emph{Reflection: If time itself can be sacred, what would it mean for
your year to have a spiritual ``pulse'' rather than only deadlines?}

We now move through key days and seasons of the calendar, beginning with
the year's first day, Naw-Rúz.

\subsection{Naw-Rúz, 1 Bahá}\label{naw-ruxfaz-1-bahuxe1}

The Bayanic calendar starts with the festival of Naw-Ruz. This holiday
dates back to Zoroastrian times and has been a long-time cultural norm
for Persia. The day has now been elevated as the first day in the sacred
cycle of time. 1 Baha, when we overlap the names of the months for as
the days, could be called the Baha of Baha, or the Splendor of All
Splendor. Bahá'u'lláh says ``in it the breath of life passes over all
created things. Blessed is the one who encounters it with spirit and
joy, for we bear witness that they are among the victorious.'' The Báb
called it the Day of God.

The day is astronomically defined by when the sun transitions from
Pisces to Aries. This also correlates to the spring equinox, when there
are 12 hours of day and 12 hours of night for every location in the
world. Thus, it is also a day which honors the Oneness of God and the
unity of humankind.

Naw-Ruz is a festival for those who observed the fast out of love for
God. The festival begins at sunrise. Although Naw-Ruz originated in
classical Persia, there are no rules specifying how the festival should
be conducted. It could blend with any culture as long as the festival is
observed in a way which abides by the Kitáb-i-Aqdas. All that is
required is spirit and joy.

\emph{Reflection: If a holy day is meant to renew life, what would
renewal actually look like in your own habits and relationships?}

From Naw-Rúz we move into the Most Great Festival, Ridván, which centers
proclamation, joy, and community.

\subsection{Ridván, 13 Jalál to 5
Jamál}\label{ridvuxe1n-13-jaluxe1l-to-5-jamuxe1l}

Ridvan is ``the Most Great Festival, which is the king of festivals,
being those days when the Ancient Beauty emerged from the Most Great
House and illumined the Najibiyyih Garden with the light of His
countenance. This is also called the Festival of Ridvan and the Festival
of Roses, comprising twelve days from the Ancient Beauty's first entry
into that garden at the afternoon hour of the thirty-second day after
Naw-Ruz, until His departure from the garden at noon of the forty-third
day after Naw-Ruz. Work is forbidden on three days of this blessed
festival: the first day, from afternoon to afternoon; the ninth day; and
the final day. On the remaining days they may engage in their
occupations.''

The days where work is prohibited are 13 Jalál, 2 Jamál, and 5 Jamál.

Bahá'u'lláh manifested the most beautiful names and the highest
attributes. All things have been immersed in the ocean of purity from
the 1st day of Ridván. It is a bounty so that you may associate with the
followers of other religions and proclaim the Cause of your Lord. It is
the crown of deeds. It is a period characterized by ``boundless joy,
gladness, and delight.''

It appears this festival's focus would be to proclaim the Cause to the
followers of other religions. Here are some possible ideas to observe
Ridván. One would be to hold the festival in a public place which is not
necessarily a Baha'i space. If it is a Baha'i space, there should be
adequate space to welcome those who are not the People of Baha. There
could be marketing materials welcoming the city or village to
participate, particularly on the days all People of Baha are available
to serve at the festival. A program could be Day 1, to celebrate the
Announcement and who Bahá'u'lláh is. Day 9 could be to focus on sharing
some of the new teachings and what makes the Revelation relevant and
impactful. Day 12 could close with a shared spiritual practice of
prayer, remembrance, and recitation, welcoming new believers and giving
thanks to God for the bounty received. How would you and your community
observe Ridván?

\emph{Reflection: How might boundless joy be expressed in a Ridván
Festival?}

From festivals of proclamation, we now turn to days of giving that
prepare the soul for detachment.

\subsection{Ayyám-i-Há (Days of
Giving)}\label{ayyuxe1m-i-huxe1-days-of-giving}

Ayyam-i-Ha are the the intercalary days, which are not bound by the
limits of the year and its months. Most years are 4 days of Ha, and on
leap years, there are 5. They are considered the days of giving. The
giving is for yourself, relatives, the poor, and the needy and the
purpose of the giving is to exalt, magnify, and glorify the Lord with
joy and gladness. There is deep spiritual meaning behind these days.
Bahá'u'lláh says the Days of the Manifestation of Ha are eleven, with
six being associated with creation. The remaining 5 are were revealed on
the Night of Power through the letter ``Ba.'' These days are like a day
of sacrifice, where creation itself is completed.

I feel the placement of the Days of Ha after the month of Mulk is highly
intentional given the purpose of the Bayanic calendar. Mulk means
dominion and is to symbolize the period when we consolidate the
spiritual elements within our earthly realm. These spiritual elements
are a bounty, as much as what we earn through our occupations are also a
bounty. If Mulk is for consolidation, the Days of Ha is for giving the
excess with joy. This giving prepares us for the detachment from the
earthly means we express through the fast immediately after the Days of
Ha.

This would not have to be the only times to give the Rights of God and
zakát, but this could be great times to do so. Zakát does not have to be
financial, and could be how we give a portion of our time in service to
another. The community of the People of Baha, after accounting for what
they have during Mulk, can assist those who struggled with a kind act.
Maybe they consult on a plan to make the next year fruitful for those in
need. It's a great time to take care of all the People of Baha as well
as those who are not. Ayyam-i-Ha is not considered a festival by
Bahá'u'lláh.

\emph{Reflection: What do you most resist giving away, and what do you
most need to give away to be freer?}

From giving as preparation, we move into the Fast as the yearly
discipline of detachment.

\subsection{The Fast, the Month of
`Alá'}\label{the-fast-the-month-of-aluxe1}

The fast is every year for the entire month of `Alá'. The People of Baha
are to refrain from eating and drinking from sunrise until sunset. Also
forbidden are desires, which the Arabic Bayan includes intimate
relations with your spouse. The fast is required for any person who has
attained the age of maturity for spiritual practice (age 11), as it was
for the obligatory prayer. For those who are mature but are traveling,
sick, pregnant, nursing, menstruating, or old (age 42+), they are
excused from fasting. This does not mean the exclusion from
responsibility denies a person a right to blessings. Any person who is
not required to fast may still fast if they choose to do so.

With the fast coming immediately after the Days of Ha and prior to Naw
Ruz, it is the culmination of a year of spiritual attainment and
practice. The fast is a cause of refinement and reformation of souls.

\emph{Reflection: When your body is restrained, what becomes louder,
your ego or your soul?}

From restraint, we now turn to expansion, hospitality as a repeated act
of unity.

\subsection{Monthly Hospitality}\label{monthly-hospitality}

Every Bayanic month, hospitality has been prescribed, even if it is only
with water. The purpose is to bring hearts together. Bahá'u'lláh and the
Sacred Household regularly provided hospitality to visitors and pilgrims
when they were able to. Hospitality is the generous reception of guests,
without expecting anything in return. This act of hospitality does not
need to be a feast. This brings hearts together because it is an act of
service.

This is my personal perspective: ``My home is your home. In my home, I
serve your needs. In my home, we are friends. In my home, we may serve
the Cause of God. In my home, we are One.''

\emph{Reflection: What is the difference between welcoming someone and
approving of everything they believe or do?}

Having established the Bayanic rhythm, we now turn briefly to two
festivals preserved through the Hijri calendar.

\section{Holy Days Using the Hijri
Calendar}\label{holy-days-using-the-hijri-calendar}

There are two other festivals to honor God which are based on the Hijri
calendar, the calendar used by Muslims. In BH02278, written after the
Kitáb-i-Aqdas, Bahá'u'lláh clarifies the dates of the festivals without
using the Bayanic calendar, despite its revelation. I believe the
purpose is to tie in the significance of these dates prior to the
existence of the Bayanic calendar, as a transition from the revelation
of the Qur'an towards the fulfillment of the Qur'an.

\subsection{5 Jamadiyu'l-Avval}\label{jamadiyul-avval}

This day in the Hijri year 1260, is when the Primal Point made His
declaration to the Babu'l-Bab (Mulla Husayn). This correlates to the
year May 23, 1844 in the Gregorian calendar. This night and day are of
supreme greatness before God. This day also marks the beginning of the
Bayanic calendar, with 1844 being 0 BE.

\subsection{1 to 2 Muharram}\label{to-2-muharram}

The Primal Point was born on 1 Muharram 1235 and Bahá'u'lláh was born on
2 Muharram 1233. These two days are to be treated as one single
festival. 1 Muharram is also the first day of the Hijri calendar.
Muharram is considered in Islam to be a sacred month and is now
associated with the birth of the most recent Dawnbreakers of Revelation,
the Dawnbreakers who eventually made Naw Ruz the new 1st day of the
Bayanic calendar.

\emph{Reflection: What might be the purpose of having Holy Days on two
calendars? If time is sacred, how do we demonstrate time is sacred
within our lives?}

We now return to the story, illuminating how honoring God can shape the
executor's interior posture.

\section{Illuminations of The Unknown Sister - Emergent Virtues from
Trustworthiness}\label{illuminations-of-the-unknown-sister---emergent-virtues-from-trustworthiness}

\subsection{Justice}\label{justice}

Honoring God turns the inheritance from a private dilemma into a trust
held before a higher standard. Justice illuminates the rights embedded
in the will without letting familiarity decide whose claim feels real.
It presses her to see that ``due'' is not measured by closeness alone,
and that she must not let comfort or fear quietly tilt the scales.
Justice does not yet require movement; it requires that her inner
weighing be clean.

\subsection{Mindfulness}\label{mindfulness}

Mindfulness makes her attentive to the spiritual weight of small
choices, what she thinks, what she rehearses, what she avoids. The room,
the envelope, the date of birth, her daughter's warmth in her lap become
reminders that nothing here is casual. Honoring God through mindfulness
means she does not numb herself or dramatize the moment. She stays
awake, listening inwardly before any word is spoken outwardly.

\subsection{Patience}\label{patience}

Patience teaches her to let time serve truth rather than rushing to
relieve tension. Honoring God stretches her trust across uncertainty:
she may not know the full story, the consequences, or the best manner of
disclosure, but she can refuse haste. Patience does not delay out of
cowardice; it delays out of reverence for timing, so that whatever comes
later is not merely quick, but worthy.

\subsection{Repentance}\label{repentance}

Repentance illuminates her without accusing her. As she honors God, she
becomes aware of subtle places where her heart might harden: resentment,
possessiveness, the desire to protect her father's image, the wish that
the unknown girl did not exist. Repentance is the courage to return from
those impulses before they shape her. It keeps her honest with God about
what she feels, so she can be purified from what she does not want to
become.

\subsection{Submission}\label{submission}

Submission steadies her against the illusion of control. Honoring God
reminds her that this situation cannot be arranged into a painless shape
by force of will. Submission does not erase her agency; it reorders it.
She accepts that she must be guided, that she may be corrected, that she
may have to do what is right even if it disrupts what feels safe. In
that yielding, she becomes strong enough to carry what she did not
choose.

\subsection{Trustworthiness (Innate
Virtue)}\label{trustworthiness-innate-virtue}

In this moment, trustworthiness is her recognition that she has been
entrusted by law, by family, and by God with something sacred: the
rights of others, the integrity of truth, and the unseen consequences of
her choices. Honoring God makes that trust feel real. It calls her to
protect what has been placed in her hands without bending it toward
self-interest, and to become someone God could rely upon even when no
one else would ever know the difference.

\emph{Reflection: If God alone knew your motives, what would you still
choose to do next?}

We close by gathering the five practices together and pointing toward
the final theme that completes Part 2.

\section{Conclusion}\label{conclusion-1}

This ends the portion of the book which discusses the spiritual
practices of the Kitáb-i-Aqdas. Prayer, recitation, remembrance,
reflection, and honoring God are each pathways for our liberation and
pathways to develop and navigate our constellation of virtues. It is my
hope you have been inspired to perform these spiritual practices and
further your relationship with God. There is one more chapter to close
Part 2. We will discuss one important topic which exists throughout the
writings of Bahá'u'lláh. Unity.

\newpage{}

\chapter{Unity}\label{unity}

\section{From One Comes Many, From Many Comes
One}\label{from-one-comes-many-from-many-comes-one}

Unity is a surprisingly complex idea. I sometimes feel unity is defined
by many as meaning sameness, like an orchestra of clarinets each playing
the same melody. It might seem good in theory, but in implementation, it
is a highly unnatural arrangement which no person would pay to attend
nor critically acclaim it. To others unity means a minority should
submit to the will of the majority, merely so there is no conflict. This
would be like an orchestra where the clarinets outnumber the horns and
are in front of microphones. The horns are not allowed any contribution
and once again, it would not be an arrangement anyone would want to be a
part of.

Instead, the Báb opens the Bayán with a different vision of unity
derived from the name of God, the One:

\begin{quote}
The names and attributes are manifestations of the multiplicity of that
primal unity. Reflect upon the verbal letters of the Bayán: all
multiplicity originates from the first unity, even if it extends
infinitely. And in the multiplicity of the universal manifestations,
there arises a strength in the manifestation surpassing that of the
primal unity. Yet, all things are realized through Him, and all return
to Him, just as they originate from Him.\footnote{The Persian Bayan
  Vahid 3 Gate 10}
\end{quote}

If I had to summarize this theme here and elsewhere in the Bayán or the
teachings of Bahá'u'lláh, it would be that from One comes many, and many
returns to One. It is the very essence of the Báb being called the
Primal Point. So unity is not a policy, and it is not a mood. It is a
pattern that repeats everywhere, in God's names, in creation's worlds,
in the soul's practices, and in the virtues that shape character. If
that pattern is real, it should be traceable. The rest of this chapter
follows that trail, moving from the most abstract form of unity (God's
Oneness) into the most personal form: how a single human heart tries to
hold many truths without breaking.

\subsection{Unity in the Names of God}\label{unity-in-the-names-of-god}

In Chapter 1, we explored the various names and attributes of God. With
God being One, all the other names are multiplied. They describe various
attributes and these names exist in thousands of languages, which are
also derived from one tongue and one breath. All of these names and
attributes eventually return to the point of Oneness.

\subsection{Unity in the Worlds of
God}\label{unity-in-the-worlds-of-god}

In Chapter 3, we explored the various worlds of God. Within a single
creation, everything seen and unseen was multiplied. There are infinite
worlds and planes of existence. Yet, when creation is ready to be made
new, it is rolled all into one.\footnote{BH00103} All of these worlds
eventually return to the point of Oneness.

\subsection{Unity in the Spiritual
Practices}\label{unity-in-the-spiritual-practices}

Throughout Chapters 8-12, we explored the various spiritual practices.
From one soul are multiple pathways for the soul to be more fully
expressed, while exploring not only the multiplicity of the names,
attributes, and worlds of God, but also exploring the essential unity
underlying it all. From one God are a multiplication of souls throughout
the world, and from this multiplication of souls comes One. All souls
return to God.

The spiritual practices are not separate lanes. They are a single road
approached from different angles. Prayer turns the will toward God.
Recitation gives the Word a living voice. Remembrance keeps the heart
oriented. Reflection integrates what the soul receives. Honoring God
makes that inner order visible in action. When these practices converge,
they form a kind of inner unity. And from that inner unity, virtues do
not appear as isolated traits, but as interlocking lights.

\emph{Reflection: In what ways do you see oneness through many, or
manyness from one?}

Now we will explore those interlocking lights.

\section{Unity in the Virtues}\label{unity-in-the-virtues}

We also explored several virtues. With God being the essence of all
virtue, these virtues are the lights guiding each of us. For example,
from one light of trustworthiness can come multiple expressions of
trustworthiness, and from these multiple expressions of trustworthiness
can we return to the source of all virtue.

All these virtues are like stars of a greater constellation. The
constellation itself can be called Unity. When all the stars are within
view, they are all interconnected and work as One. If one virtue shines
too strongly, there is risk nearby stars cannot be seen. The
constellation is broken. If one virtue is never developed, the
constellation is also broken. The stars may not shine equally at all
times, but they must all be seen in order for the constellation of unity
to form. The constellation endures only when each virtue is allowed its
rightful place.

The shape of the constellation may be seen differently. One
constellation may appear as a Scorpio while to another, it may appear as
an Orion. These ways to interpret what we see can be limitless, as our
experiences are unique, our souls are unique, and our place in the
spiritual worlds may always be in a unique position. From these
multitudes of constellations of unity return to one, which is also
unity. These constellations will be seen by others. These constellations
are the evidence of sincere belief, being signs of the spirit of God for
those who seek God.

As we see, unity is not sameness. Instead, it is like a well-rehearsed
orchestra with hundreds of instruments. These instruments play different
melodies, different notes, different tones, yet harmonize into the most
beautiful sound ever heard. This harmonized sound passes through one
ear, and it is heard and felt. This is unity.

\emph{Reflection: What do your constellation of virtues look like? Do
any shine brightly more than others? If so, why?}

Unity becomes difficult when life becomes complicated. What happens when
truth introduces a new person, when loyalty meets justice, when courtesy
meets disclosure, when piety meets consequence? This is why the unknown
sister matters in this chapter. She is not a metaphor for sameness. She
is a test of whether many can still return to One without anyone being
denied.

\section{Illuminations of The Unknown Sister - The Constellation of
Unity}\label{illuminations-of-the-unknown-sister---the-constellation-of-unity}

Unity allows her to hold the whole without forcing it into sameness. The
family she thought she knew has become many. Some are known and unknown,
near and distant, yet all still originate from one source and return to
one truth. Unity does not require her to silence any part of this
reality or elevate one voice over another. Instead, it reveals how fear,
love, restraint, truthfulness, and trustworthiness can sound together
without collapsing into noise. The situation remains unresolved, but no
longer fragmented. Like an orchestra mid-rehearsal, each element now has
a place, and from their tension a deeper harmony becomes possible, even
before a single note is played.

Unity is not only what we aim for. It is also what we can lose. The
constellation fractures most easily when a single virtue becomes so
luminous that it starts to behave like a sun. When that happens, the
rest of the sky goes dark---even if we still call it light.

\subsection{When the Constellation Fractures - 5
Examples}\label{when-the-constellation-fractures---5-examples}

We will now explore what are the risks involved when we treat a virtue
as an absolute destination, focusing so much on one virtue we do not
allow the other virtues to shine. A virtue as a destination can serve as
an idol, distracting us from the total vision God counsels us to. Here
are some examples of risks the executor of the will could face if a
single virtue is potentially replacing God as our object of devotion.

\subsection{Piety as an Idol}\label{piety-as-an-idol}

She reads the will with disciplined attention, taking her role seriously
and letting duty govern her posture and pace. The moment is held inside
obligation, what must be carried, what must be executed, what must be
borne without display. While piety feels good in this case, there are
risks. Without pure truthfulness, piety can retreat into procedure,
using ``responsibility'' to avoid fully naming what has been revealed.
Without courtesy, piety can become spiritually correct yet relationally
blunt, honoring duties while failing to protect how others must receive
them. Without loyalty, piety can become impersonal righteousness, so
``doing what is right'' loses the warmth of standing with those who will
be affected.\\
Without trustworthiness, piety risks becoming image or intention without
follow-through, where devotion to duty is felt but cannot be depended
upon.

\subsection{Pure Truthfulness as an
Idol}\label{pure-truthfulness-as-an-idol}

The facts appear with clean force. There is one unfamiliar name, a
birthdate, an address, and one truth laid down without explanation, as
neutral and sharp as the legal language itself. Her mind computes
meaning instantly, as if truth's first duty is simply to be seen. While
pure truthfulness feels good in this case, there are risks. Without
piety, pure truthfulness can shrink into mere disclosure, detached from
the reverence of obligation and the moral weight of consequences.
Without courtesy, pure truthfulness can land like a blow, accurate but
unreceived, wounding the soul it claims to respect. Without loyalty,
pure truthfulness can become isolating clarity, where saying what is
real forgets the shared covenant of relationship and care over time.
Without trustworthiness, pure truthfulness can become momentary candor
without dependability, leaving others unsure whether the truth will be
carried consistently into action.

\subsection{Courtesy as an Idol}\label{courtesy-as-an-idol}

Her voice stays even and measured. Her soft answers to her daughter,
calm stillness with her brother, no sudden gestures that would rupture
the room. Courtesy preserves dignity in the air, keeping fear from
spilling outward, maintaining a stable surface while something immense
shifts underneath. While courtesy feels good in this case, there are
risks. Without piety, courtesy can become social smoothness that
preserves comfort while sidestepping the deeper obligations the moment
demands. Without pure truthfulness, courtesy can become
concealment-by-polish, where calm language quietly delays honest
recognition. Without loyalty, courtesy can become neutral distance,
respectful to everyone yet committed to no one when commitment is most
needed. Without trustworthiness, courtesy can become performance,
offering reassurance that feels stable while failing to secure what
others will need to rely on.

\subsection{Loyalty as an Idol}\label{loyalty-as-an-idol}

Her brother remains steady, hands folded, offering a quiet pledge: ``I
trust you; we will handle it.'' Loyalty takes the form of
presence---nonintrusive, watchful, ready---holding the room together by
refusing to pull anything from her before she is ready to speak. While
loyalty feels good in this case, there are risks. Without piety, loyalty
can drift into allegiance to the familiar, protecting the old family
story rather than yielding to higher obligation. Without pure
truthfulness, loyalty can become protective silence, preserving bonds by
keeping reality unspoken until truth arrives too late or too harshly.
Without courtesy, loyalty can turn into blunt partisanship, defending
``ours'' in a way that disregards the dignity of those newly implicated.
Without trustworthiness, loyalty can become sentiment without endurance,
pledged in feeling but unreliable when sustained costs begin.

\subsection{Trustworthiness as an
Idol}\label{trustworthiness-as-an-idol}

She aligns the pages, refolds them cleanly, returns them to the
envelope---executor-minded, controlled, careful not to let the room
fracture. Trustworthiness shows itself as containment: she holds the
truth first so others do not have to, managing the moment with
steadiness that looks like strength. While trustworthiness feels good in
this case, there are risks. Without piety, trustworthiness can become
mere competence, faithful to process but not necessarily faithful to the
moral meaning of what must be done. Without pure truthfulness,
trustworthiness can become quiet management, keeping things ``handled''
while the essential truth remains unspoken. Without courtesy,
trustworthiness can become hard reliability, efficient but emotionally
unsafe for those who must eventually enter the reality with her. Without
loyalty, trustworthiness can become cold trusteeship, fulfilling duties
without the enduring devotion that assures others they are not alone in
the outcome.

\subsection{Reflections of the Unknown
Sister}\label{reflections-of-the-unknown-sister}

While we did not advance the plot of the story, you may have explored
what is next for all characters involved. The constellation of virtues
developed through spiritual practice leads to a profound sense of
empathy for all involved, an awareness each are on their own spiritual
paths intertwined with that of the executor of the will.

\emph{Reflection: If you were any one of these characters, how would you
proceed?}

I also hope this story and the illuminations we explored throughout
offer an intuitive framework to help guide your inner transformation
regardless of the difficult situations we inevitably are involved in.
Most importantly, I hope you are willing to use the teachings of the Báb
and Bahá'u'lláh as primary sources for this framework. This leads us to
the final unity explored in this chapter, that of the Kitáb-i-Aqdas.

These examples show why unity cannot be reduced to tone, consensus, or
comfort. Unity is not achieved by avoiding tension, but by holding
tension within a higher order. The Book is what names that order.
Without revelation, ``unity'' becomes whatever the strongest voice needs
it to mean. With revelation, unity becomes a path: not sameness, not
coercion, but orientation to the One.

\section{Unity in the Kitab-i-Aqdas}\label{unity-in-the-kitab-i-aqdas}

When the Kitáb-i-Aqdas commands us to recognize the Manifestation of God
for this age, this is the foundation of all unity. You are committing
yourself to belief in God, the worlds of God, the revelations of God,
and all Manifestations of God. You are committing yourself to various
spiritual practices and the development of your inner virtues. You are
committing yourself to the liberation of your soul. You are also
committing yourself to the liberation of souls near to you. These
commitments can only arise through a commitment to unity, to the Oneness
of God, to the essential Oneness of everything.

The Kitáb-i-Aqdas is a book of unity. Because of this, as we proceed
from belief and spiritual practice to practical application, we are
constantly reminded of this essential unity. Every law, counsel,
boundary, and exhortation exists for unity. Unity is the path of glory.
Unity is the path of liberation. However, if we attempt to achieve unity
without the entire Book guiding us, unity becomes a mere illusion. If
unity is the only virtue, goal, or law, the religion of God becomes a
deadly weapon which oppresses the souls God desires to liberate.

\emph{Reflection: If you claim belief, yet pick and choose what to obey,
what then do you believe? Can one achieve unity ignoring the many which
composes One?}

The Kitáb-i-Aqdas is not a book of sameness. While it prescribes the
same Book to all, how the Book is expressed will have infinite forms.
This is by design. From these infinite forms, and from each and every
one of us, we have the opportunity to all walk a singular path of unity.

\section{A Transition to Part 3}\label{a-transition-to-part-3}

Unity is not something we announce. It is something we become. The proof
is not in our claims, but in what our lives can hold without breaking:
difference without contempt, truth without cruelty, loyalty without
blindness, justice without coldness, and reverence without denial. From
One comes many, and from many comes one. If this is the pattern of God,
then unity is not the flattening of souls, but the harmonizing of
them---until the world can finally sound like what it was always meant
to be. This leads us to Part 3, the rights and responsibilities the
Kitab-i-Aqdas enjoins. They are the framework of how virtues become
manifest in the social life, social systems, and institutions we
participate in. What good is virtue if it is only contained within?

\newpage{}

\part{Part 3: Rights and Responsibilities}

\chapter{From Birth to Maturity}\label{from-birth-to-maturity}

\section{Introduction}\label{introduction-6}

The first two parts of this book used the Kitab-i-Aqdas as a foundation
for the theology and beliefs enshrined in Baha'u'llah's teachings, as
well as the practices which immerse a soul into the spiritual worlds,
developing a relationship with God and the soul's virtues. The rest of
this book will be more legalistic, leaning heavily into the idea the
Kitab-i-Aqdas is a book of laws.

Instead of approaching this from a perspective of saying do not do this,
do not do that, I want to present the laws within a new framework. Every
law and counsel is an opportunity to express the rights of an individual
as well as an expression of their responsibilities. Baha'u'llah calls
the laws and counsels as boundaries not to be exceeded\footnote{Kitab-i-Aqdas
  \#29} while also saying we must exceed the boundaries of self and
desire.\footnote{Kitab-i-Aqdas \#2} There is this free space to operate
in, where a person may do as they feel best. I imagine this free space
to be as if you are living in a nation, able to travel to any city,
village, state, province, or natural area up to the boundaries of the
next nation. Within these boundaries, you have certain rights and you
have certain responsibilities. This space between the boundaries, the
spiritual nation of Baha'u'llah and of the People of Baha, are places
for the beliefs and virtues to most fully express themselves. It is a
space entrusted to us by God.

An example of how to frame rights and responsibilities could be from the
spiritual practice to pray daily. Any person has the right to pray if
they so choose to do so. A believer accepts the responsibility to pray
daily. No person or institution is allowed to coerce another into
praying, nor are they allowed to deny a person the right to pray. If a
person does not believe, they do not have the responsibility. This is
the spirit of divine law, trusteeship instead of control.

The entirety of Part 3 will discuss all the rights and responsibilities
afforded to all people as expressed within the Kitab-i-Aqdas. As you
proceed, imagine how each right and responsibility interacts with the
Unity Constellation. Imagine what a nation of Baha'u'llah could look
like, living your life where the citizens live within the boundaries you
share, even if they do not all believe the same.

This chapter will start with the rights of all people from conception
throughout their life. No responsibilities will be addressed here, as
babies, children, or others who have not attained maturity have
responsibilities in the faith of God.

\section{The Right to Life}\label{the-right-to-life}

In Chapter 2 Belief in the Human Soul, we discussed how all life is
animated by the spirit of God, with the human soul being created once
the developing body has reached a certain stage. Baha'u'llah prohibits
taking a life twice in the Kitab-i-Aqdas in verses 29 and 73. With the
prohibition against taking a life, comes the right to life for all. This
would include from the very moment of conception, even if the soul has
yet to be expressed. At no stage of a human life can this right be
deprived.

\section{The Right to Identity}\label{the-right-to-identity}

Verse 120 explicitly states the People of Baha have the opportunity for
``their stations to be revealed, your names will be established, and the
ranks and remembrances will be elevated in a well-guarded Tablet.'' This
implies a right to be named, a right to establish a unique identity, and
a right to be treated as an individual. No person has the right to
restrict the potentials of identity. No person has the right to force
any type of identity on another, to include broad group identities.

While verse 120 does state the People of Baha has the opportunity
stated, the right to identity is for every person from the moment they
exist. The identity ``People of Baha'' cannot be forced onto another nor
can it be denied.

\section{The Right to Purity}\label{the-right-to-purity}

Verse 74 decreed the water of semen as pure. Water is used throughout
the Kitab-i-Aqdas as a purifying agent. This means from the moment of
creation, a person is created pure and they are born pure. This applies
regardless of the circumstances a person is created. Every person has a
right to spiritual purity. This means they also have a right to be free
from corruption. Verse 64 says ``corruption is not of Us.''

\section{The Right to Dignified
Appearance}\label{the-right-to-dignified-appearance}

Verse 74 builds further on the theme of purity, cleanliness, and
refinement. Baha'u'llah says there should be no traces of dirt on their
garments, unless there is a reason. The water's essential properties
cannot be changed. This implies every person has a right to clothing
which has been cleansed with water prior to being worn. There is also no
restriction on clothing in Verse 159. Each person has a right to wear
clothing without being judged for their clothing.

In addition, Verse 106 requires at least weekly baths or pouring water
over yourself (showering) using fresh water, and Verse 152 saying the
feet being cleaned each day in the summer and every three days in the
winter Every person has a right to being physically clean. Verse 76
expresses the use of rosewater and pure perfume so one's fragrance is
pleasing. These types of perfumes could have floral extracts, resins,
essential oils, citrus, and natural spices. Synthetic chemicals or
artificial musk may not be as pure. Every person has a right to be
fragrant as a manifestation of paradise on earth. The purpose is to
gladden those nearby, not to distract, distort, or intoxicate.

The Kitab-i-Aqdas also says for all people to have trimmed nails. Verse
44 says to never shave your head, that hair is a sign of the natural
order of creation. For males, the hair should not pass the limits of the
ears. Each person has a right to trimmed nails and hair. Each person
also has a right for their hair to be displayed or presented.

\section{The Right to Love and
Kindness}\label{the-right-to-love-and-kindness}

Throughout the Kitab-i-Aqdas, Baha'u'llah enjoins kindness and love for
all.\footnote{Kitab-i-Aqdas 48} All people have a right to love and
kindness. These are not conditional nor are they earned. This helps
foster a sense of emotional well-being and security.
Affection\footnote{Kitab-i-Aqdas 65} is a limitless tool, as long as no
other rights are violated.

\section{The Right to Be Free of
Oppression}\label{the-right-to-be-free-of-oppression}

The Kitab-i-Aqdas forbids oppression.\footnote{Kitab-i-Aqdas 71}
Oppression is used in the context of lewdness, which opposes the virtues
of righteousness and piety.\footnote{Kitab-i-Aqdas \#64} Oppression
causes corruption. It is associated with deception and wrongdoing.
Oppression can be associated with tyranny and opposes the fear of God.
This type of oppression destroys what God creates. Oppression causes
humiliation and is caused by the heedless. Oppression opposes justice
and healing. Oppression can be caused by the intoxication of desire and
hatred. Let's explore the types of oppression and what all people have a
right to from conception.

\subsection{Maturity}\label{maturity}

Baha'u'llah describes various acts dependent upon reaching
maturity.\footnote{Kitab-i-Aqdas 10 \& 27} Every person has a right to
mature and also a right to be free from being forced to be mature before
they are mature. Preventing maturity is an act of oppression. This can
be done by withholding education, preventing skill development, or by
shielding from small responsibilities. Forcing a person to be mature
before they are mature is also oppression. This can be done by forcing
responsibility they are yet able to handle spiritually, physically, and
emotionally. A child should be a child. A person with a developmental
handicap should be treated compassionately within the context of their
handicap.

\subsection{Lewdness}\label{lewdness}

All people have a right to be free from lewdness.\footnote{Kitab-i-Aqdas
  74} Lewdness can be defined as the intent to cause sexual desire
through crude or obscene acts and words. Lewdness can also be the intent
to shock or humiliate someone through those sexual acts or words. These
acts can be unwelcome, or forced upon a person who does not consent. If
a person lacks maturity, consent can never be provided.

\subsection{Pederasty}\label{pederasty}

Pederasty\footnote{Kitab-i-Aqdas 107} is the practice of a mature man
having sex with an immature male. While the Kitab-i-Aqdas does not
explicitly apply the law to females, the primary aspect of pederasty
would apply regardless of gender. There is the right for an immature
person to be free from having sex with a mature person. This is another
act of oppression which is associated with the right to be free from
maturing before they are ready.

\subsection{Marriage}\label{marriage}

All people have a right to be free from a forced marriage.\footnote{Kitab-i-Aqdas
  65} Marriage is conditioned upon maturity and consent. Anyone under
the age of maturity is protected from marriage. Between pederasty,
lewdness, and marriage, every person has a right to sexual autonomy and
safety.

\subsection{Physical and Emotional
Harm}\label{physical-and-emotional-harm}

All people have a right to protection from physical and emotional
harm.\footnote{Kitab-i-Aqdas 148} The threat of physical harm to coerce
consent would also be prohibited. Emotional harm can include contention,
disputes, backbiting, and slander.\footnote{Kitab-i-Aqdas 19} Instead,
all people have a right to consultation and a right to walk away and
leave emotional harm.

\subsection{Slavery}\label{slavery}

All people have a right from being purchased or sold as slaves or
servants.

\subsection{Provision}\label{provision}

All people have a right to provision, even if they are unable to work.
This would include all people who are not mature and have no
responsibility to work. Provision can include food, shelter, clothing,
and other basic necessities. For those who work, their provision would
be fair wages or compensation which ensures food, shelter, clothing, and
basic necessities.

As all people have a right to provision, all people have a right against
being forced to beg. Those who are not mature cannot be forced to beg by
their families. A person who begs has either been denied their right to
provision, or has chosen not to exercise their responsibility to work
(post maturity).

Finally, provision can also include basic environmental necessities,
such as clean air, clean water, and food free from undisclosed toxins
and chemicals.

\subsection{Hatred}\label{hatred}

All people have a right to be free from hatred. Hatred can be expressed
verbally, physically, and even through exclusionary acts such as
shunning. Baha'u'llah associates hatred with an intoxication of desire.
Some of these desires can be vain imaginings, such as race, superiority
ideology, nationality, religion, or other aspects of identity. Hatred
can also be an expression of unfulfilled desire, where a person takes
what they want from another.

\subsection{Corruption}\label{corruption}

All people have a right to be free from corruption. This is not only
political corruption, but all acts of dishonesty, fraud, or the
degradation of virtues. Corruption can tie into some of the other
rights, such as with lewdness as lewdness corrupts the virtue of purity.
Corruption can be utilized by people in power, but also those who are
compelling specific actions or material gain.

There is a right to be free from consuming that which robs a person of
reason.\footnote{Kitab-i-Aqdas 119} This will cause a person to become
heedless and suspicious. Opium is explicitly mentioned\footnote{Kitab-i-Aqdas
  155 \& 190} along with gambling. Reason is a virtue all people have a
right to.

\subsection{Tyranny}\label{tyranny}

Tyranny is the use of power which is cruel, unfair, harsh, or unjust.
This can apply in any relationship where there is a power dynamic
involved. Every person has a right to be free from tyranny.\footnote{Kitab-i-Aqdas
  70}

\subsection{Justice}\label{justice-1}

All people have a right to justice. Justice can include the fair
implementation of a legal system by a government, but it can also be a
general fairness in how you are treated, especially when compared with
others. Rules, regulations, rewards, punishments, and rights are
executed in the same manner for one as they are for another.

\section{The Right to a Skilled
Physician}\label{the-right-to-a-skilled-physician}

Baha'u'llah says for those who fall ill, skilled physicians should be
referred to.\footnote{Kitab-i-Aqdas 113} Every person has a right to a
skilled physician who is capable of treating the illness. This does not
guarantee any person the best treatment money can buy, but a minimal
baseline. This does not guarantee every illness will be cured. The right
is solely for the ability to refer to a skilled physician, not to any
particular course of treatment nor to any particular outcome.

For those who have yet reached maturity, this could be considered under
the right to provision. If a skilled physician's treatment plan is not
followed, it could be considered a denial of provision and an act of
oppression.

\section{The Right to Inheritance}\label{the-right-to-inheritance}

All people have a right to receive any inheritance provided to them
through the legal contract of a will and testament.\footnote{Kitab-i-Aqdas
  20 and 109} This right can not be deprived under any circumstance. If
a person has not reached maturity, their inheritance must be protected
through a trust. This trust is managed professionally and responsibly
until the youth reaches maturity.

\section{The Right to Worship}\label{the-right-to-worship}

All people have a right to worship. They also have a right to be free
from being forced to worship, which is an act of oppression. Associated
with this right is the right to music and being able to recite the
verses of God in a melodious way. The right to music cannot be
restricted although it should be governed by the rights afforded to all
people. Baha'u'llah warns music should not lead anyone from a path of
dignity.

All people have a right to a Mashriq'ul-Adhkar. These are to be built in
every village and city. Every child has the right to attend a
Mashriq'ul-Adhkar and recite the verses of God therein. They also have
the right to be enraptured by the love of God. To deny this right is an
act of oppression. To restrict the building of the Mashriq is an act of
neglect.

\section{The Right to Education}\label{the-right-to-education}

Baha'u'llah emphasizes the right to education.\footnote{Kitab-i-Aqdas 48}
Education includes at least two languages\footnote{Kitab-i-Aqdas 118}
(the primary language of home and a secondary language),
science,\footnote{Kitab-i-Aqdas 77} arts, crafts,\footnote{Kitab-i-Aqdas
  33} and the sacred words of God.\footnote{Kitab-i-Aqdas 150} This
education involves the skills required to learn these subjects, such as
reading, writing, arithmetic, and critical thinking. Education involves
the virtues and rights enjoined by Baha'u'llah as well as the gradual
introduction of responsibilities to ensure maturity. Education also
includes a variety of methodologies, and in particular, children should
have a right to play. All people have a right to education.

As all have a right to be free from oppression, the right to education
also includes a right to be free from indoctrination. While being taught
all of these subjects, it must be done with fact, balanced opinions, and
without any coercion of belief. All rights must be ensured during
education.

\section{Responsibilities}\label{responsibilities}

People who have not reached maturity have no responsibility in the
command of God, other than what is necessary to mature.

\newpage{}

\chapter{From Maturity to Devotion}\label{from-maturity-to-devotion}

\section{The First Responsibility}\label{the-first-responsibility}

In the prior chapter we established some basic rights all people have
from conception until maturity. In this chapter, we will begin a series
of chapters of life once maturity is reached and the devotional identity
expected of a Person of Baha. This devotional identity will guide and
inform all rights and responsibilities for the remainder of Part 3.

Before explaining what maturity is, I want to introduce what I believe
is the first responsibility we have at maturity. Baha'u'llah says ``no
one should object to those who rule over the people.''\footnote{Kitab-i-Aqdas
  \#95} Instead of objecting to rule, we should leave leaders to what
they have, such as their power, and focus our attention on the hearts of
people. This is also a right. Every person has the right to peacefully
be led. They also have the right and responsibility to guide hearts,
through their conduct and speech. Every responsibility a person has from
maturity must be exercised in a way which does not overstep the bounds
of law, no matter where you live. To change or influence law and
leadership, the changing of hearts is the core of the devotional
identity.

Maturity will be a good way to help understand how to frame this
concept.

\section{What is Maturity?}\label{what-is-maturity}

The Kitab-i-Aqdas mentions maturity twice. The first is when prayer and
fasting become obligatory\footnote{Kitab-i-Aqdas 10} and when to receive
inheritance.\footnote{Kitab-i-Aqdas 27} On the latter right, Baha'u'llah
conditions trusteeship if the descendant is weak instead of being
mature. What might weakness be related to?

Weakness is usually associated with physical strength, but strength is
not a sole determination of maturity. I believe weakness in this case is
associated with two major characteristics. The first would be a person
who lacks firmness in character, or maybe someone who is still
understanding who they are as a person. The second characteristic would
be the inability to function normally or fully. This might be a level of
codependency, whether it be financial or emotional, which does not exist
when someone is mature.

To support these ideas, we can look further in Baha'u'llah's revelation.
In the Lawh-i-Rais (A Tablet to a Chief), Baha'u'llah closes with an
instruction for Ali Pasha. He says ``Ask God to help you reach maturity
so that you become aware of the beauty and ugliness of deeds and
actions.'' To reach maturity, one must understand the consequences of
their actions.

Finally, the Kitab-i-Aqdas also conditions marriage upon the consent of
both potential spouses. This would be the final qualification for the
determination of maturity, the ability to consent.

\section{The Laws of Maturity and Consent in
Nations}\label{the-laws-of-maturity-and-consent-in-nations}

With the first responsibility stated earlier, it would be wise to
consult the laws of the land you live in. In Persia and the Ottoman
Empire of the late 19th century, consent and maturity were determined to
be 15 years of age. In the Qajar dynasty, the prerequisite age to be a
King was 18 years. While Baha'u'llah never explicitly stated a person
needs to reach a prerequisite age to be mature, the Bab established the
age of 19.\footnote{Arabic Bayan Vahid 10, Gate 14} Prior to 19, there
are 2 other stages of maturation described in BB00083. At age 11, one is
to start prayer and fasting. At age 15 is the second stage of
maturation, with the completion of reason and moral awareness. Age 19 is
the final stage of full spiritual and legal maturity.

Where the laws which have jurisdiction do explicitly say, those laws
need to be followed if they are more strict. For example, as of this
writing Bahrain's law is age 21 and South Korea is 20. These ages of
maturity and consent would need to be followed instead of the 19
prescribed by the Bab.

The Bab provided flexibility. If someone shows capacity for any of these
stages, they should not be delayed. For example, if a child wants to
pray and fast before age 11, allow them if they have the ability to
understand why they are praying and fasting.

There are times when a person suffers from developmental handicaps which
prevent their ability to be fully mature. To determine is a person is
fully mature at the age 19 threshold requires careful consultation and
consideration of certain qualities. These qualities would be financial
and emotional independence, firmness in character and identity, and is
able to understand the consequences of their actions. Remember, it is
the right of every person to mature, but they also have a right to be
free from being forced to be mature before they are ready.

\section{The Devotional Identity of a Mature
Adult}\label{the-devotional-identity-of-a-mature-adult}

Maturity marks the official beginning of the devotional identity of an
adult, if they choose to do so. Baha'u'llah says in tablet BH00528:

\begin{quote}
When man attaineth the age of maturity, he must investigate and, putting
his trust in God and sanctified from love and hate, reflect upon that
whereunto the people adhere. He must hear with his own ears and see with
his own eyes, for if he looketh through the eyes of another, he will be
deprived of beholding the effulgent rays of the Sun of Divine Knowledge.
Various parties exist in the world, and each hath regarded itself as
being in the right, as He, exalted be He, hath said: ``Each party
rejoiceth in what it possesseth.''
\end{quote}

He goes on to say the ``understanding of every soul must\ldots. behold
itself independent.''\footnote{BH00601} This establishes the next set of
rights and responsibilities of every person after they reach maturity.
We have the right to independently seek truth and the responsibility to
be independent as we seek truth. To be a Person of Baha means to
actively believe in and see the truth of Baha'u'llah, not to do so
blindly. If there is a criticism or argument against Baha'u'llah or the
People of Baha, do not turn your back on it. Investigate. Always be
diligent in understanding each side of an argument or perspective, much
like a judge allowing both a prosecutor and defender in cross examining
evidence while presenting their cases. We have the responsibility to do
the spiritual work ourselves, not to defer it upon others.

\section{Right to Belief}\label{right-to-belief}

Every mature person has a right to believe as they wish, even if
Baha'u'llah says this comes with consequences. You may not believe in
anything in Part 1 of this book, or maybe have partial belief. You might
even believe in Baha'u'llah but are part of an organization which claims
infallibility. This is your right. Baha'u'llah says it is the
responsibility of every mature adult to reflect on what people adhere,
see with your own eyes, and react from there. His hope is to be one with
the People of Baha, but this hope comes without coercion.

A person of Baha has the right and responsibility to every spiritual
practice described in Part 2, and the right and responsibility to
develop their own Unity Constellation of virtues. Those who are not a
person of Baha has the right to practice any or all of these practices
and develop any or all of those virtues according to their belief. None
should ever be deprived, no matter who they are.

\section{Spiritual Accountability}\label{spiritual-accountability}

There are various guidelines in the Kitab-i-Aqdas which focus on aspects
of spiritual accountability. I will provide a list of these without
added explanation.

\begin{enumerate}
\def\labelenumi{\arabic{enumi}.}
\item
  A responsibility not to destroy what God has built.\footnote{Kitab-i-Aqdas
    73} A right to what God has built and created.
\item
  A responsibility not to oppose or object to Baha'u'llah.\footnote{Kitab-i-Aqdas
    141 \& 162} A right to support Baha'u'llah and His Cause.
\item
  A responsibility not to hesitate in following Baha'u'llah's
  command.\footnote{Kitab-i-Aqdas 132, 162, \& 182} A right to trust in
  and act sincerely in implementing the Kitab-i-Aqdas and related
  commands.
\item
  A responsibility not to question Baha'u'llah's actions.\footnote{Kitab-i-Aqdas
    7, 49, \& 126} A right to believe Baha'u'llah's actions reflect the
  Will of God, who represents all names to include Most Subtle.
\item
  A responsibility not to corrupt the Cause of God.\footnote{Kitab-i-Aqdas
    64 \& 165} A right to allow the Cause of God to reform the Earth.
\item
  A responsibility not to measure the Book of God by your own
  desires.\footnote{Kitab-i-Aqdas 165} A right to measure your desires
  according to the Book of God.
\item
  A responsibility not to object to the fragrance of God, which is His
  love.\footnote{Kitab-i-Aqdas 179} A right to receive God's love.
\item
  A responsibility not to withhold yourself from the bounties and grace
  of God.\footnote{Kitab-i-Aqdas 179} A right receiving the bounties and
  grace of God.
\item
  A responsibility not to deny what God has permitted.\footnote{Kitab-i-Aqdas
    36} A right to be free from the tyranny of those who deny what God
  has permitted or allow which God has forbade.
\item
  A responsibility to raise up the Sacred Houses and the places
  whereupon the Throne of the Lord (whereever the Bab and Baha'u'llah
  had lived and possibly imprisoned).\footnote{Kitab-i-Aqdas 133} A
  right to raise up and access the Sacred Houses and Thrones of the
  Lord.
\item
  A responsibility to ask about the Cause of God and what your souls
  need.\footnote{Kitab-i-Aqdas 126} A right to be able to ask about the
  Cause of God and what your souls need.
\item
  A responsibility to never deviate from a verses outward
  meaning.\footnote{Kitab-i-Aqdas 105} A right to interpret the outward
  meaning according to its explicit intent. A right to interpret inward
  meanings without restriction. There is also a responsibility to
  interpret both the inward and outward meanings.\footnote{The Tafsir of
    the Surah of the Sun} This is a right of every mature person.
\end{enumerate}

\section{The Right to Spiritual
Maturity}\label{the-right-to-spiritual-maturity}

The writings of Baha'u'llah often refer to another kind of maturity,
spiritual maturity. Even if you have reached maturity in the traditional
sense, it does not mean you have reached spiritual maturity. To help
describe what spiritual maturity is, Baha'u'llah does use the
traditional descriptions of maturity as symbols for the spiritual side.

The Kitab-i-Badi is the largest single book by Baha'u'llah, written in
response to questions from a man who followed Mirza Yahya, denying
belief in Baha'u'llah as a Manifestation of God. This section will not
go into a discussion of Mirza Yahya (see Chapter {[}unspecified yet{]}.
A significant portion of this book discusses the conditions and reasons
for lack of belief in individuals, especially those who are led by those
who claim belief in God yet are spiritually corrupt.

The first right of spiritual maturity is for a spiritual infant to be
from receiving the knowledge of spiritual maturity. There is a
corresponding responsibility for a teacher or other person to only give
the knowledge of spiritual maturity to those who are capable and ready.
This requires a high level for discernment. The Kitab-i-Badi says ``most
today are considered infants before God.'' Part of this lack of maturity
is due to their constellation of virtues are not valued, such as the
virtue of heedfulness. A sign of maturation is belief in the
Manifestation of God.

The Kitab-i-Iqan discusses extensively why people lack this development.
Baha'u'llah places considerable blame on religious leaders and
institutions who inhibit spiritual maturity. They do this by distorting
the Words of God by changing meanings, saying God said things God did
not, or by interpreting verses against their explicit intent. These
distortions introduce new doctrines, traditions, superstitions, and laws
which God had not intended. A spiritually immature person struggles to
understand what God actually wants in their life, such as these rights
and responsibilities being discussed. The weak and immature are easily
led astray by others who fabricate such distortions.\footnote{Lawh-i-Sirraj}

The rights and responsibilities of spiritual accountability from the
Kitab-i-Aqdas is the foundation to spiritual maturity. Even if you do
not believe in Baha'u'llah, these principles are incredibly vital to
your spiritual health. I personally believe if any person who is part of
any religion is able to see what is from God (the source), they have a
great chance to reach spiritual maturity. What I mean is if a Christian
were to focus on the actual teachings and example of Jesus Christ as the
primary source of their religious practice and understanding, while
choosing to use all other sources such as the teachings of Saint Paul or
their local pastor as secondary, they may have a great opportunity to
mature in their God consciousness. ``Today, knowledge and ignorance,
high and low, nearness and distance, truth and falsehood, life and
death,~maturity and infancy, wisdom and heedlessness, are all
distinguished by the confirmation of the manifestation.''\footnote{Kitab-i-Badi}
With this said, I do believe in what the Kitab-i-Aqdas says in its
opening in believing in the Manifestation of God in this age. God says
this is the source of all spiritual maturity.

The next right is to be allowed to attain the station of spiritual
maturity. Each person has a responsibility to attain spiritual maturity
and each person has a responsibility to allow others the ability to
reach spiritual maturity, according to their capacities. Baha'u'llah
says the children of the age must be nurtured with milk that is delicate
and subtle until they attain maturity.\footnote{BH00302} An associated
responsibility is for a person to no longer be moved by desires and
illusions.\footnote{Words of Paradise}

\section{Responsibilities of Spiritual
Maturity}\label{responsibilities-of-spiritual-maturity}

There are two important responsibilities for a person who has attained
spiritual maturity. These responsibilities do not come with associated
rights. The first responsibility of spiritual maturity is to never
exceed your station.\footnote{Kitab-i-Badi} Baha'u'llah says a person
can do this by purifying your soul from the world, not speaking of what
you do not know, and refraining from mentioning what you do not
understand. It is acceptable not to know everything. It is also
acceptable to stay away from being placed on a pedestal. It would be
worst if you placed yourself, or allowed others to place you in a
position which replaces the Manifestation of God.

The next responsibility for a spiritually mature person is to convey
God's Cause.\footnote{BH00035} This responsibility is reserved only for
the spiritually mature. A person who does not place the Manifestation of
God as the primary source of God's Will is unable to convey God's Cause.
They will distort God's Cause. The purpose of conveying God's Cause is
for others to reach spiritual maturity. Spiritual maturity is the
station where the soul has reached liberation. The liberation of the
soul is a right afforded to all people.

\begin{quote}
This is the day that the Most Great Spirit foretold and proclaimed in
the wilderness of yearning by His name. Those who drink of the wine must
act with wisdom, engage in the proclamation of the Cause, and, in all
matters, cling to the cord of consultation and hold fast to the hem of
compassion, so that the children of the age may attain~maturity through
the kindness and mercy of the divine sages and be nourished in complete
health.\footnote{BH00123}
\end{quote}

\section{The Bayanic Calendar's Rhythm of
Maturity}\label{the-bayanic-calendars-rhythm-of-maturity}

The numbers 11, 15, and 19 are not arbitrary. They correspond to the
months Masá'il (Questions), Sultán (Sovereignty), and 'Alá (Loftiness)
in the Bayánic calendar. This is the divine rhythm by which the world
itself matures. Thus, even time testifies that the awakening of
awareness, the assumption of will, and the perfection of trust are one
continuous act of God's creation. By the time a person turns 19, they
would have completed 19 cycles of 19 months, a complete Vahid of life.
This is the essence of unity.

\newpage{}

\chapter{The Private Self}\label{the-private-self}

This chapter will discuss the rights and responsibilities of a person
when they have privacy. The Kitab-i-Aqdas does not explicitly describe a
right to privacy, but it is heavily implied. The soul itself seems
hidden within the flesh, its qualities not always obvious. The practices
to refine this hidden entity are primarily private practices. We pray at
home and in the hidden chambers of the Mashriq'ul-Adhkar. Our
remembrance can occur in public, but in a way which is not meant to be
seen except for the Mashriq'ul-Adhkar.

These are the times when no one knows their actions except for
themselves and God. The private self could also be the thoughts and
feelings which we do not freely show to others. What we think, feel, and
do in private directly influences how our public self interacts in the
communities near to us, public social media, or anonymous online spaces
where the private and public self are merged closely together. Even when
we are in public, we maintain the right to privacy. These rights from
before you were mature still apply, as do the rights and
responsibilities from the prior chapter. These rights and
responsibilities of the private self help develop a robust
pyscho-spiritual framework for the liberation of the soul.

\section{Responsibilities Associated With Premature
Rights}\label{responsibilities-associated-with-premature-rights}

With becoming mature, there are some responsibilities a person will
carry into their private life. The associated rights were described
earlier in Chapter 14's From Birth to Maturity. I will list these in
this section, and as this chapter unfolds, some of these will be more
fully discussed.

\begin{enumerate}
\def\labelenumi{\arabic{enumi}.}
\item
  The responsibility of life
\item
  The responsibility of identity
\item
  The responsibility of purity
\item
  The responsibility of a dignified appearance
\item
  The responsibility of love and kindness
\item
  The responsibility to not oppress, to include lewdness, pederasty,
  physical harm, emotional harm, provision, hatred, corruption, tyranny,
  and justice.
\item
  The responsibility to have a skilled physician
\item
  The responsibility to education
\end{enumerate}

\section{The Right and Responsibility to Be Free From
Illusions}\label{the-right-and-responsibility-to-be-free-from-illusions}

Baha'u'llah discusses the need to be free from illusions. This is one of
the most repeated commandments throughout His writings. In the
Kitab-i-Aqdas, the warning exists in verses 17, 35, 37, 41, and 167.
Illusions are deceptions. They alter our sense of reality and can
erroneously shape our beliefs, feelings, thoughts, and actions.
Illusions can also interfere with our faith. If God is the source of all
creation, anything else we find more important or influential than God
might end up shaping our illusions. There are illusions which are
unintentional, some which are intentional, some created for us by
others, and some created by ourselves.

The Kitab-i-Aqdas describes illusions as idols of our
desires.\footnote{Kitab-i-Aqdas Verse 41} One cause of illusions are
conjectures, which are opinions and conclusions made by inference,
without evidence. We are certain this conjecture is true, even if it is
not. Conjecture can take many forms. There could be superstitions passed
on from prior generations, or even fictional stories of old treated as
truth of today. Conjecture could be saying that God wants us to do
something, but there is no evidence of this in any Scripture. Conjecture
could be treating traditions and religious jurisprudence as revelation.
Other forms of conjecture could be entertainment shows acting as
fact-based news, political discussions that exclude legislation or
voting records of representatives, conspiracy theories, gossip, or
things we create within our own imaginations. We have a right to be from
the illusions of others, with a responsibility to discern what is
evidence and what is not. We have a responsibility not to consume media
which create these illusions. If we consider our constellation of
virtues, conjecture hides reason. What is more damaging is feeling joy
or happiness from such illusions. What if we create illusions about a
loved one which negatively effects how we perceive that love?

What we create illusions within our own imaginations, it can be quite
devastating for us spiritually, emotionally, and psychologically.
Psychologically there are many disorders where illusions alter the sense
of reality when compared to a more rational person.
Suspicions\footnote{Lawh-i-Dunya (Tablet of the World)} are one way
illusions effect how we approach truth, especially if it reaches to a
level of paranoia. This means we have a right to be free from illusions
including a right for us to be free from illusions we create for
ourselves. We have a responsibility to be free from illusions we create
for ourselves, even if such illusions are derived from past traumas or
other false narratives. Illusions such as ``I am unworthy of love,'' ``I
am always a victim,'' or ``I deserve shame and humiliation'' are
incredible obstacles on our spiritual journey and life in this world.
They violate are right to kindness, where we have a responsibility to be
kind to ourselves. If such illusions interfere with your right to
identity or other rights, remember you also have a right and
responsibility to seek skilled physicians. The spiritual practices from
Part 2 are also a vital pathway to burning away such illusions, which
are considered veils between you and God.

\section{The Right and Responsibility To Not Be
Wretched}\label{the-right-and-responsibility-to-not-be-wretched}

The Kitab-i-Aqdas tells us not to follow a wretched one.\footnote{Kitab-i-Aqdas
  Verse 70} A wretched person is miserable, has a poor character, or
maybe is regularly claiming distress or misfortune. When we are
discerning what is true or real, we should also try to discern if we are
allowing a wretched person create illusions, such as the ones described
in the earlier section. We have a right and responsibility to never
follow a wretched person, even if this person claims to be an authority
in the subject they discuss. Is the talk show host miserable? Is this
friend focused on all the things which are wrong in their life, without
showing gratitude? When we are alone or having private thoughts, what
type of person has influenced this precious time and mental resources?

We also have the right and responsibility to now be wretched ourselves.
It would be a clear injustice if we are privately miserable but act with
charisma to gain influence or followers. What if we look in the mirror
and focus on changing our outer appearance with cosmetic surgery? This
is not true to our identity and is a manipulative illusion. We must be
careful to find pathways to avoid wretchedness in private so we do not
create woe elsewhere.

\section{The Right and Responsibility To Not Be a
Tyrant}\label{the-right-and-responsibility-to-not-be-a-tyrant}

The same verse of the Kitab-i-Aqdas tells us not to follow a tyrant.
This also means we also have a right and a responsibility not to be a
tyrant. A tyrant is someone who is cruel, harsh, and applies rules more
strictly for others than themselves. This starts in private just as
wretchedness does. What if we create an illusion a person does not
deserve a right merely because of another illusion such as race? This
betrays the very Cause of God. Imagining any person can be deprived of
rights you want to keep is tyranny, even if you do not have the power to
act on it. Baha'u'llah says ``Do not be preoccupied with
yourself,\footnote{Lawh-i-Dunya (Tablet of the World)} be in the thought
of improving the world and refining nations.'' He further adds to
``desire not for anyone what you would not desire for
yourselves.''\footnote{Kitab-i-Aqdas Verse 148} Any thought or feeling
which interferes with this responsibility is an illusion.

Sometimes illusions which cause tyranny are unintentional. Other times
they can be intentional. Illusions which are not conjecture-based are
delusions. A delusion is a purposeful deception. It can cause a false
belief or even interfere with the hopes and plans of yourself or
another. Baha'u'llah says it is important for scholars\footnote{Kitab-i-Aqdas
  Verse 165} and sovereign leaders\footnote{Kitab-i-Aqdas Verse 184} to
stop being the source of delusions. However, this is not limited to
scholars and sovereign leaders. The virtue of trustworthiness requires
all of us to be free of delusions. We have a right and responsibility to
avoid the delusions of others, and the right and responsibility to never
be the source of delusions. Baha'u'llah says to break the chains of
delusions in the name of the Lord of all mankind.\footnote{Kitab-i-Aqdas
  Verse 165}

\section{Break the Boundaries of Self and
Desire}\label{break-the-boundaries-of-self-and-desire}

Earlier in the chapter we shared Baha'u'llah's teaching how illusions
are the idols of desires. Idols are often considered as objects
worshiped as a false god, such as the golden calf created be the Hebrews
while Moses was on a sojourn with God. Idols can also be things which
are excessively adored to the point of being the primary driver of
thoughts, feelings, and belief. Ultimately, all idols detract from God
or in developing a personal relationship with all the names of God.

Like delusions, we are to break the boundaries of self and
desire.\footnote{Kitab-i-Aqdas Verse 2} If the command of God is the
true boundary, self and desire are more restrictive boundaries. We have
a right to break through them to attain the actual boundaries, which are
represented by the Sidrat-ul-Muntaha, the boundary of the furthest
horizon. No one has a right to keep you bounded by self and desire, nor
do they have a right to shape them for you. On the other hand, we each
have a responsibility to yourselves to put forth the effort to be free
of self and desire. Self and desire is a source of division among
people.\footnote{Kitab-i-Aqdas Verse 58} Division and unity start with
who we are in private.

\section{The Self}\label{the-self}

The American Psychological Association dictionary, dated November 2023,
says the self is the totality of the individual, consisting of all
characteristic attributes, conscious and unconscious, mental and
physical. Baha'u'llah wants us to break the boundaries of our
characteristic attributes.

This can include personality traits such as being introvert or
extrovert, your placement on a personality profile like a Jung typology,
a zodiac profile, or even something simple as saying you are a happy
person. These are habitual ways of how we shape our identity. These
habits could vary between our private self and the adapted social self,
or they can remain consistent in both settings. To break a boundary of
self, we must be willing to break these habits which do not benefit our
soul or the souls of others.

Cognitive attributes can be part of the characteristic attributes of
self, such as our typical styles of thinking. This can include our
styles of logic, inference, intuition, or how we see the strategic
versus the tactical. To break the boundary of self, we must be willing
to break our style of thinking.

Our emotional disposition such as baseline moods and tendencies help
shape the self. Optimism and pessimism, stability and instability, and
empathetic and narcissism are all various emotional conditions which
have a full spectrum of unique expression. They shape our responses to
relationships, stress, important events, and the unpredictable nature of
life. To break the boundary of self, we must be willing to break the
base emotions which guide our soul.

Values, morals, and ethical commitments shape are a part of our
characteristic attributes. These can be influenced by philosophies,
politics, cultural values, other religions, or even unique family
situations. They shape deep-seated goals, desires, and meanings. They
shape our normative self, the person who we hope to be and the person
who we hope to present to others. To break the boundary of self, we must
be willing to break these values, morals, and ethical commitments.

Our physical characteristics are a part of our self. This can include
observable characteristics such as our body composition, shape, height,
skin and hair color, gender, or even how the parts of the body function.
Less visible features such as chronic conditions and genetic profiles
shape our experiences. To break the boundary of self, we must be willing
to view the body as a temporary vessel for the soul, like discussed in
Chapter 2.

Our social and relational attributes also comprise our characteristics
of self. This can include interaction patterns such as dominance or
submissiveness, giving or receiving, apathy or caring, or even
cooperative or disruptive. These attributes also include the roles we
serve or identify with. Being a parent, leader, teacher, artist,
athlete, or inclusion in groups such as religious membership, clubs,
corporations, or NGOs become a part of our self-perception. To break the
boundary of self, we must be willing to break the boundaries of social
and relational attributes.

\section{Desires}\label{desires}

Desires are a subject Baha'u'llah was deeply concerned about in the
Kitab-i-Aqdas. The idea of desires interfering in the spiritual and
social pathways are everywhere. Desires themselves are not prohibited,
but their expression must be in a way which does not interfere with your
spiritual journey nor with the spiritual journey of others. Desires
which inhibit these things start with our private self and can become
intentional and unintentional illusions of self. The Kitab-i-Aqdas has
at least 17 different types of desires we are responsible for
controlling. The next section will discuss these according to theme.

\subsection{Desires of Status}\label{desires-of-status}

Desires of status include a few aspects. The first would be in our
station,\footnote{Kitab-i-Aqdas Verse 86 \& 170} meaning the position or
status we hold within this world. How highly do we desire to be exalted?
How special do we see ourselves when we consider our ambitions or look
in the mirror? We have a responsibility to never exceed our stations,
with an explicit prohibition on believing we can take the place of God
in essence, quality, attributes, functions, or status. We also have a
responsibility to be aware we are created from the same dust.\footnote{Hidden
  Words of Arabic \#68}

We also have a responsibility in our desire of might.\footnote{Kitab-i-Aqdas
  Verse 86} What is the purpose in wanting power over another? Is it to
dominate or is it to protect those who are vulnerable? Baha'u'llah warns
leaders such as Napoleon the most mighty rulers on Earth have passed on,
losing their entire kingdoms in an instant. Those kingdoms are God's,
even when God allows a person to have sovereignty. This might is not
limited to the rule of nations, though. There are power dynamics in
families, business, councils, schools, and at local levels of
government. Desiring might in any of these settings must be tempered
with a desire to reflect other names of God. No matter if our kingdom is
great or small, we have a responsibility in desiring them only if we
desire to serve God.\footnote{Kitab-i-Aqdas Verse 83}

One way we often demonstrate our status is through our wealth, or
obvious presentation of wealth. Baha'u'llah warns us in desiring the
treasures of the world.\footnote{Kitab-i-Aqdas Verse 39} We have a
responsibility to earn and have some wealth, but we cannot deprive
others of wealth in the process. Like sovereignty, the treasures you
possess will be possessed by others when you pass from this world. We
also have a responsibility in how we present our outward
adornment,\footnote{Kitab-i-Aqdas Verse 89} which can be fine clothing,
jewelry, luxury goods, lavish homes, and other symbols. We have a right
and responsibility to be refined in appearance, but not to be lavish.

One final way we express the desire of wealth status is through
inheritance. When we consider what we have and what we might leave
behind, how do we consider others? We have a right to inherit\footnote{Kitab-i-Aqdas
  Verse 29} and a responsibility to give inheritance. Baha'u'llah
describes in great detail to whom, without any prerequisites. For
example, a descendant is always a descendant. We do not have a right to
alter their station, even if our desires for our descendants do not
align with their desires. A will is the final way to express one's
desires and is a legacy we choose to leave behind. They can alter the
entire course of families and others who fulfill the station of kin.

All of these desires of status can serve as illusions if we shape our
identity by our status, might, and wealth. We have a right to desire
status, might, and wealth, but being detached from them when we shape
our identity helps ensure we keep God exalted and others as our equal.

\subsection{Desires of Knowledge}\label{desires-of-knowledge}

There are three types of desires regarding knowledge we have a
responsibility for. Baha'u'llah does not want us to be proud in our
knowledge.\footnote{Kitab-i-Aqdas Verse 41} This can be derived in our
desire of status, to be known as always right. This can also be derived
from a fear of being wrong. When we desire to be considered the source
of knowledge, we stop being open to learning the mysteries and sciences
of the world. We have a right to knowledge, and a responsibility to
always be open to receiving knowledge.

Our knowledge helps shape how we respond within creation, especially our
beliefs. When we are exercising our right to knowledge, we have the
responsibility in prioritizing our sources. No source is forbidden. Yet,
Baha'u'llah does challenge us by asking which Qiblih will we turn
towards?\footnote{Kitab-i-Aqdas Verse 137} Whichever source is our most
adored source also becomes our source of belief and religion. Our
responsibility is to ensure God and the Manifestation of God is our
point of adoration for knowledge. To desire other sources more, we may
struggle in our liberation of the soul.

One source of knowledge often attributed is the one called luck or
chance. Even when we have knowledge of what likely outcomes are, we have
a desire to beat those odds. Pride or desperation can motivate these
actions, even when they are hidden. Baha'u'llah addresses this with a
clear prohibition against gambling. We have no right to gambling and a
responsibility to refrain. Gambling is often associated with games of
chance whose goal is financial gain. Gambling is any action where we
purposefully take on high probability of risk for the low possibility of
reward. This can include games, but it can also include many other
aspects of life. The desire for gambling stems from the idea we have a
secret knowledge which can beat those odds. The desire also replaces
putting forth the effort to take actions with lower risk and better
possibilities of long-term reward.

These desires of pride in your knowledge, adoration in sources other
than God, and gambling against known odds can all create illusions which
keep us from learning what we need to know and applying it in
responsible ways. When we are in private, do we feel there is knowledge
to gain? When we are given knowledge, do we say outwardly we know but
privately we are dismissing it because it is not our most trusted
source? What are we willing to risk just to prove we are right or
capable? These are all the types of questions the spiritual practices
help answer, especially with self-reflection. They also help ensure our
knowledge do not lead to harmful delusions which lead us and others
astray.

\subsection{Desires of Self-Interest}\label{desires-of-self-interest}

The Kitab-i-Aqdas describes how we should view ourselves as the fingers
of one hand and the limbs of one body.\footnote{Kitab-i-Aqdas Verse 58}
This perspective starts in private. We have a right to self-interest,
but there is a responsibility to the interests of those near to you.
Mutual well-being requires a mutually shared interest in each other's
goals, dreams, aspirations, and health.

Baha'u'llah says the affairs of self and desires can cause division.
Some of these aspects have already been discussed earlier in this
chapter. In addition to those ways, the desires of self-interest can
also manifest in many other ways.

We might get caught up in the shame or guilt of past shortfalls. Our
desires cause us to hold onto that pain and possibly over-correct into
ways which create new shortfalls. A healthy practice of reflection,
remembrance, and repentance can keep the desire to be pure from becoming
a disillusioned fanatic.

We might get caught up in our careers or personal achievements,
constantly planning what the next step is while overlooking friends,
neighbors, coworkers, and others connected to us. These desires of
self-interest can also make make us hunger for affirmation and other
forms of praise, while privately we lack sincerity. For example, in
private do we avoid prayer but desire to be seen as pious and righteous?
Do we desire the other fingers of the hand to only serve one finger,
with a desire to possess or control that which those fingers have? This
can create desires of conformity instead of a unity through diversity.

Baha'u'llah says we should not desire for others what we would not
desire for yourself.\footnote{Kitab-i-Aqdas Verse 148} This is a great
expression of the Golden Rule. We have a right to desire something for
ourselves, but we have a responsibility to desire the similar thing for
others. This is not a desire of sameness or conformity. For example, you
might desire to be the CEO of a company. This does not mean you desire
others to be the CEO of a company. Not everyone can be a CEO. But,
another might desire to be an effective history teacher. These are not
uniform goals. Yet, the potential CEO can still desire the fulfillment
of the prospective teacher's professional goals and vice versa. Each
person has their own path and their own desires we need to respect,
honor, and desire as our own.

To close this section on self-interest, we have a right to be free from
hatred and a responsibility to never hold hatred within us.\footnote{Kitab-i-Aqdas
  Verse 89} The desires of hatred cause the beings of the spiritual
worlds, the concourse on high, to lament, which is to express deep
sorrow, anguish, and regret. The love of self should never cause the
hatred of another. There is absolutely no reason to ever hate another.
The desires of self-interest can create powerful illusions which when
held within us, can destroy that which God desires.

\subsection{Desires of the Body}\label{desires-of-the-body}

The final section about desire will discuss the desires of the body.
Baha'u'llah expresses specific commands regarding food and
drink,\footnote{Kitab-i-Aqdas Verse 17} drugs such as opium,\footnote{Kitab-i-Aqdas
  Verse 155} and certain aspects regarding sex and sexuality. With food
and drink, we should be mindful of our desires during fasting.
Baha'u'llah goes into more detail regarding our responsibilities with
food in the Tablet of Medicine. The first guidance in this tablet
regards food. If food is medicine, our desires are associated with
medicine and health. As we have a right to skilled physicians, we also
have a right to health when physicians are not present. Some of the
associated responsibilities which help temper desires are:

\begin{enumerate}
\def\labelenumi{\arabic{enumi}.}
\item
  Do not eat except when hungry
\item
  Do not drink except when thirsty
\item
  Eat only after digestion
\item
  Swallow only after thorough chewing
\item
  Use foods to initially treat ailments
\item
  Do not mix opposites at the table. Begin with soft before hard, and
  liquid before solid.
\item
  Walk a little after eating
\item
  Avoid foods which are hard to chew
\item
  Eat only a little in the morning
\item
  Being excessive in eating aggravates illnesses
\end{enumerate}

These guidances help temper the common desires associated with food so
we do not eat excessively, nor eat foods which can cause us to become
ill.~Sometimes the desire of food makes us eat too quickly, or mix foods
together to make eating more efficient. We have a right to be free from
eating when we are not hungry, or even eating foods which harm us.

We also have a responsibility not to deprive ourselves of certain kinds
of food as an ascetic practice, such as meats.\footnote{Suriy-i-Haykal}
Our desire to seem pious does not give a right to prohibit what God
allows for others. Our desires for ourselves cannot be desires imposed
onto others, even with food and drink.

We have a right to be free from opium and from consuming anything which
robs us of reason. Our responsibility is to be free from the desires
associated with their temporary benefits. We may desire to be free from
pain, so we may consume a drug. The pain could be from the body, or it
could be from the heart. We may have a desire to express a different
personality, so we seek alcohol or a stimulant. We may desire a powerful
spiritual experience without putting forth the effort, so we take a
psychedelic. All of these items are like forms of gambling, where there
is considerable risk for an immediate reward. The immediate reward is an
illusion and could potentially lead to delusions.

The final desire for this chapter will be regarding sexual intercourse
and associated desires. This was saved for last as this tends to be the
most influential desire and interlinks with all other desires in some
way or another. The Kitab-i-Aqdas mentions a few items regarding the
desires of sex. We have responsibilities not to desire sex with
children,\footnote{Kitab-i-Aqdas 107} nor with our parents and
step-parents.\footnote{Kitab--i-Aqdas Verse 107} Children have a right
to be free from being desired sexually, even to be considered for
engagement into marriage which is basically a promise of future sexual
desire being expressed. Sex with another's spouse (adultery) is also
prohibited.\footnote{Kitab-i-Aqdas Verse 19} This also means a married
person is a right to be free from the desire of another person who is
not their spouse.

We are also warned about allowing certain music and melodies\footnote{Kitab-i-Aqdas
  Verse 51} causing desire. There is a right to music and melodies, but
a responsibility to be virtuous in its use.

The Kitab-i-Aqdas also describes other sexual themes which are born from
desires. Lewdness, oppression, corruption, tyranny, and physical and
emotional harm can all be associated with sexual desire. Is it possible
to sexually desire someone in a way which is not lewd, oppressive,
corrupt, tyrannical, nor harmful? I would say yes, if we allow the other
desires to also be managed. For example, some people seek power merely
so they can have access to more sexual partners. Yet, coercive sexual
desires are tyrannical. Some might seek wealth to purchase a comfortable
life in exchange for sex. Even desiring marriage solely for a lawful
sexual outlet can be a burden, when Baha'u'llah describes the purpose of
marriage is tranquility and the creation of a life who knows God. The
creation of life does include sex, but the purpose is not sex in itself.

Many consider sexuality to be a core part of their identity, which could
be the biggest illusion for a soul to face. Sex is a phenomena which
exists only with the body. When the body perishes, so to does the
ability and desire for sex. If we allow sexual identity of any kind to
shape our sense of self, consume our private thoughts and emotions, we
end up trapping our soul in a prison which is difficult to escape from.
When all the desires mentioned in this chapter exist to serve the sexual
identity, the soul and associated psychology will suffer. Remembering
the nature of the soul and its purpose is vital to being free from the
illusions of the body and the self.

\section{The Power of Moderation}\label{the-power-of-moderation}

This chapter might seem as though all desire is harmful. It is not.
Remember the foundational virtues of Chapter 7. We have the innate
virtues of piety, pure truthfulness, courtesy, loyalty, and
trustworthiness. These can inform how we navigate our desires. For
example with our sexual desires, can they be rooted in the innate
virtues? They can be if we are immersed within them. These innate
virtues can govern every single desire we feel so they can be desired in
a healthy way.

Chapter 7 also described moderation, with the fear and love of God being
the ultimate guiding principles of moderation. Moderation can take many
forms when it comes to understanding the self. For example, where there
is a desire for affirmation might also come with a fear of rejection.
Every desire has an opposing fear. When our spiritual perspective is to
fear God, we may learn not to fear these other illusions such as
rejection. Without the fear of rejection, our desire for affirmation
might moderate and take a healthier perspective. When our spiritual
perspective is to love God, we may express the desire in a way which
loves creation or even those we desire.

Moderation is the ultimate guide to ensuring our desires conform to the
desires of God. Moderation is the fire which burns away the veils of
illusions. Moderation is the pathway to guiding our self to be aligned
with the divine Self.

\section{Freedom}\label{freedom}

To conclude the discussion about the private self, there is one more
desire I want to close with. Baha'u'llah warns in desiring freedom and
taking pride in it.\footnote{Kitab-i-Aqdas Verse 122} We each have a
right to freedom, where true freedom is following God's commands through
Baha'u'llah. Our private actions, thoughts, and feelings require
freedom. Freedom is part of the human experience. For the soul to be
liberated and free, our private self must desire the freedom of those in
our spiritual journey. All of the illusions we create imprison the soul.
Even private illusions can imprison the souls of others who care about
our souls.

The next chapter will move on to those other souls. We will move past
illusions and desires and cover actual actions we can take with others
to govern this freedom. Can the fingers of each hand reach towards
heaven together? They can with our rights and responsibilities of the
private self being our true boundary.

\newpage{}

\chapter{The Constructive Social
Life}\label{the-constructive-social-life}

The next two chapters will move from the private self to the social
life. This will focus on interactions with all people, regardless of
their status in your life. This chapter will discuss the constructive
social life which is framed as the Cause of God. These are a framework
of positive actions designed to be the foundations and instruments of
change within our communities.

Throughout the book, there have been allusions to our responsibilities
towards others. The names of God shed light on the potential
perspectives we can strive to. The soul and human station teaches us the
special and noble character all people are capable of. The spiritual
worlds allows us to creatively imagine the unique journeys all will
take. The urgency of the Day of Resurrection ensures we act with the
best interest of all. The command through the Manifestation of God
inspires us with what is best in a world which sorely needs our best.
The spiritual practices refine our constellation of virtues so we are
increasingly an example to whomever we cross paths with. We established
foundational rights all people deserve, regardless of any status. We
also ensured we are aware of responsibilities of maturity while burning
the veils of illusions and desires we have privately held.

All of the prior rights and responsibilities expressed so far in Part 3
still apply in this chapter. What follows builds upon this foundation.

\section{The Cause of God}\label{the-cause-of-god}

Throughout the book, we have several times mentioned the Cause of God in
association to Huquq'u'llah, reflection, and even the spiritual worlds.
My personal opinion is the Cause of God fully manifests itself in the
constructive social life of a person, where we go beyond the individual
and consider the community. To lead this concept, Kitab-i-Aqdas \#58
presents a great framework to express the Cause of God and how to view
community.

\begin{quote}
Beware lest the affairs of the self and desire cause division among you.
Be like the fingers of one hand and the limbs of one body. Thus does the
Pen of Revelation counsel you, if you are of those who believe with
certainty.
\end{quote}

\#58 largely explains itself, but I do want to add the extra visual.
What happens when a finger is injured, weak, or severed? The hand might
still function but not as effectively as before. The same is true about
the limbs of a body. It means we must make utmost care to ensure each
aspect is strong, healthy, and purposeful in function. To do this,
Baha'u'llah offers various guidance associated with the Cause of God.

All believers have the responsibility to exalt God's Word.\footnote{Kitab-i-Aqdas
  \#38} This also places other words from other sources as secondary.
This does not mean to exclude other words, but by keeping God's Word in
its rightful station, all fingers can be mindful of what the Cause is.
All people have a right to know what the Cause of God is and to know
God's Word as exalted, even if they do not believe it. They deserve the
opportunity to know. To clarify what God's Word is, Baha'u'llah says to
use this Book only\footnote{Kitab-i-Aqdas \#168} as prior scripture
avails nothing today. While we acknowledge the truth of scriptures past,
we must be completely understand they are relevant today as historical
context, not sources of God's living will. This will remain true until
the next Manifestation of God appears when the Kitab-i-Aqdas is replaced
by a new Book.

We have responsibilities to associate\footnote{Kitab-i-Aqdas \#75} and
consort\footnote{Kitab-i-Aqdas \#144} with the followers of other
religions in a spirit of friendliness and fellowship. The followers of
other religions have a right to be associated with, without being
excluded nor shunned on the basis of belief. This allows us to
proclaim\footnote{Kitab-i-Aqdas \#75} and assist\footnote{Kitab-i-Aqdas
  \#38} the Cause in a positive and healthy way. To help with
proclamation, Baha'u'llah also advises us to learn different
languages\footnote{Kitab-i-Aqdas \#118} so we can communicate with
different people. This is not a requirement, though. We should serve the
Cause in all conditions.\footnote{Kitab-i-Aqdas \#74} We are promised
God strengthens us and He is with us and will aid us. With this promise,
there is no need to hesitate in serving this Cause\footnote{Kitab-i-Aqdas
  \#132} or to let disbelievers' sorrow to overcome us.\footnote{Kitab-i-Aqdas
  \#35} With God's Word being exalted, there is no reason to dispute
about God and His Cause\footnote{Kitab-i-Aqdas \#177} even if there are
interpretive differences and various understandings. We have right to be
free from disputes about God and His Cause. The fear of God will help to
keep veils away\footnote{Kitab-i-Aqdas \#132} which can lead to
disputes, much like they helped with the prior chapter's illusions and
desires. God's love burns away those veils, a love which is the
foundation of the friendliness and fellowship to show others.

We have a responsibility to support His chosen ones.\footnote{Kitab-i-Aqdas
  \#117} Throughout the revelation, Baha'u'llah refers to Manifestations
of God as Chosen, but He also refers to others as chosen. Baha'u'llah in
the Kitab-i-Badi offers an allegory of who chosen ones are. Those who
rejected the idol of the golden calf, an illusion to the days of Moses,
were chosen. Much like how the Hebrew people who believed in God and
Moses were the Chosen Ones of God in those early days of Israel,
Baha'u'llah affirms this criteria for today. In the Surah of the Temple,
he says these chosen ones end up in God's Kingdom. In the Kitab-i-Aqdas,
He says these chosen ones elevate His Word (exalting) and are the stars
of heaven. To support one of these people could be in a variety of ways.
It could be through prayer, financial assistance, giving food and other
provisions, printing a book, or other logistical need. We have a
responsibility to support them. Baha'u'llah regularly used Huquq'u'llah
to support His chosen ones. We must be careful, though, from promoting
ourselves as chosen.

For example, I have no right to claim I am chosen by God or Baha'u'llah.
But, maybe we observe someone who exalts God's Word, is friendly and
kind to people of varying religions, is learning languages to travel to
a new land to proclaim the Cause, and exemplify this other criteria. We
have a responsibility to support them. My personal belief is any person
who says they are chosen by God or Baha'u'llah most likely is exalting
themselves and should be approached cautiously. Now if a community feels
I am chosen, they have every right to support me. Even then, all chosen
ones are still fingers of the hand and limbs of the body. Never should
we exalt anyone chosen, just merely support them in the Cause and the
Cause only.

\subsection{Community Building}\label{community-building}

Another purpose with the Cause of God is to ensure the development of
our communities. The only guaranteed method of socioeconomic development
which lifts up all fingers of a hand is through working together with
shared or complimentary goals and roles. Baha'u'llah ties development as
a responsibility for everyone.

Baha'u'llah says to build up the cities and lands of God\footnote{Kitab-i-Aqdas
  \#160} then remember Him within them. This building up includes not
only specific buildings of faith, but also the infrastructure to support
the entire Cause. In the chapter ``Honoring God'' we learned how having
an occupation is a requirement for those who are able to. Each business
and supporting jobs must have their buildings, markets, and networks
such as communications and transportation. Infrastructure must exist to
support the right to provision, even if provision is governed by market
pricing. This provision as stated in Chapter 14.7 includes environmental
necessities. Every person has a right to socioeconomic development, and
it is a responsibility of all believers. Other infrastructure would be
focused on what is needed to fulfill the full spectrum of rights
afforded to all.

Outside of the business sphere, we are also to build your own houses
(dwellings) as perfectly as possible in the lands.\footnote{Kitab-i-Aqdas
  \#31} This would include the outside and the inside such as
furnishings. This interplays highly with the constellation of virtues,
and like all of our rights, we must be mindful in how we express them. A
house must not need to be extravagant to be as perfect as possible, but
within the means you have, keep it as perfect as possible. Baha'u'llah
also tells us to renew our furnishings after the passing of nineteen
years.\footnote{Kitab-i-Aqdas \#151} All people have a right to secure
dignified housing.

On the religious side, the command to build Mashriq-ul-Adhkars in every
city and village is a key component of the socioeconomic development of
the lands. It is the right of every person to have access to one, even
if they do not have a responsibility to enter if they have different
religious practices. As every person has a right to belief, there should
be no restrictions in allowing the people of other religions from
building whatever places of worship and faith. Remember we are to be
friendly and offer fellowship. Yet, all believers of Baha'u'llah have a
responsibility to build Mashriq-ul-Adhkars.

The final aspect of the socioeconomic development of the lands are to
build Houses of Justice in every city\footnote{Kitab-i-Aqdas \#30} when
believing individuals reach 9 or more. The Houses of Justice have
clearly defined roles, functions, and funding which will be explained in
its own chapter due to its uniqueness and importance. While being tied
to religious revelation, these Houses of Justice exist as
representatives for all who dwell on Earth\footnote{Kitab-i-Aqdas \#30}
regardless of which city they are in. The implication is every city
should receive the benefits of a House of Justice within them. The
individuals who arise to serve on the Houses of Justice do so to serve
the Cause.\footnote{Lawh-i-Bisarat (Glad Tidings) - 13th Glad Tiding}
Houses of Justice do not need to be built in villages, but there is no
exclusion if there are the minimum 9 individuals.

The Cause of God cannot be in progress without any of these aspects of
socioeconomic development. If one is missing or excluded, the limbs of
the body of the community will have weakness. Perhaps they will grow
unequally. The Cause of God is the pathway of liberation for all people.
In order to ensure this pathway is truly beneficial for all in a
community, Baha'u'llah enjoins specific responsibilities for all
believers and rights for all people. The following section will briefly
list the positive social actions required to support the Cause of God.

\subsection{Positive Social Actions}\label{positive-social-actions}

A responsibility to act with good deeds\footnote{Kitab-i-Aqdas \#73} and
a right to be free from bad deeds.

A responsibility to engage in activities which benefit yourself and
others\footnote{Kitab-i-Aqdas \#33} and a right to be free from harmful
activities by others. Associated with this is a prohibition of idleness,
the wasting of time without any meaningful activity. Sloth is also
prohibited, which is an apathetic reaction to opportunities and duties.
With this, we have a right from being forced into idleness.

A responsibility to provide for the poor and needy\footnote{Kitab-i-Aqdas
  \#16} and a right to be free of poverty.

A responsibility to provide hospitality\footnote{Kitab-i-Aqdas Verse 57}
once per Bayani month.

A responsibility to fulfill promises\footnote{Kitab-i-Aqdas \#156} and a
right for promises received to be fulfilled. Promises can be informal
such as verbal agreements with friends or the proverbial pinky promise,
or they can be formalized through contracts.

A responsibility to purify every disliked thing with water\footnote{Kitab-i-Aqdas
  \#74} and a right to purified things. The layers of this could be
garments, surfaces, buildings, and publicly shared spaces. As there is a
prohibition on consuming intoxicants, as we purify things we should be
careful in using chemicals which can act as toxins inside and outside
the body.

A responsibility to respond with joy when invited to banquets and
gatherings\footnote{Kitab-i-Aqdas \#156} and a right to receive a
response to such invitations. The response can be a no, but if it is a
yes, the promise should be fulfilled. If the response is a joyless yes,
consider making it a no.

A responsibility from being a plaything for the ignorant,\footnote{Kitab-i-Aqdas
  \#159} a responsibility born from good manners. This can be associated
with what we say, do, or present ourselves. All has a right from being
considered as playthings.

A responsibility to seek knowledge for the recognition of the
known\footnote{Kitab-i-Aqdas \#102} and a right to seek knowledge.
Relatedly we have a responsibility to read from sciences which benefit
you\footnote{Kitab-i-Aqdas \#77} with a right to be able to read these
sciences. This could be rather expansive but would require discernment.
In general, the pursuit of facts, the wisdom to use these facts to
benefit yourself and others, and caution towards opinions which disguise
facts and harms.

A responsibility to respond in kindness if angered.\footnote{Kitab-i-Aqdas
  \#153} Note this is not a prohibition of anger, but a guidance to
ensure anger does not end up betraying our constellation of virtues.
There is a right to be responded to without anger.

A responsibilty to respond to being rebuked without rebuking. This can
imply a prohibition on rebuking altogether. Instead, in the spirit of
friendliness and kindness, consider using reminders, counsel, or advice.
Baha'u'llah advises Napoleon III and the people of France to cover up
mistakes instead of arguing.\footnote{The Suriy-i-Haykal 150 \& 151}
This does not mean hide injustice or the breaking of laws.

These sets of responsibilities and rights are the pivotal framework of a
healthy social life no matter who we cross paths with. In a perfect
world where everyone believed in and abided by the Kitab-i-Aqdas, these
positive social actions would be enough to create heaven on Earth.
Realistically this may never be the case. For example, a person may
consider implementing what they consider to be good deeds, but do so
without the beliefs of part 1, the spiritual practices of part 2, and
without refining their private life. This is the essence of the very
first paragraph of the Kitab-i-Aqdas, where Baha'u'llah says ``whoever
is deprived of it is indeed among the people of error, even if he
performs every good deed.'' We should be careful from taking shortcuts.

In the event of shortcuts within this framework, Baha'u'llah introduced
protections to help ensure the rights of all people. These are all clear
prohibitions which set boundaries we should never cross in our social
life. Most are non-punitive and some are punitive. These are the first
laws of the Kitab-i-Aqdas which Baha'u'llah designed to be enforceable
by a justice system which this book covers. The next chapter will
discuss the protections for the social life.

\newpage{}

\chapter{The Protective Social Life}\label{the-protective-social-life}

When looking at the framework of the constructive social life and the
Cause of God, that by itself could seem like a utopia. Yet, we know life
is much more complex. In Chapter 2, we learned how each soul is created
noble, is unique, and is embodied in the flesh with different
experiences and capacities. Everyone lives their life in their own way.
While the framework for the Cause of God with the underlying beliefs,
spiritual practices, virtues, protected rights, and related
responsibilities by themselves should be suitable for a society which
liberates souls, it must also have measures to protect people, protect
their rights, and ensure one person's expression of rights does not
infringe upon another's rights. Everything discussed so far are goals to
constantly be striving for.

Being a mirror reflecting God's names is not an automatic on-switch and
Baha'u'llah understands this. Belief does not mean you will
automatically treat others as we should, spiritual practice does not
automatically mean you act with God consciousness, and being in front of
another does not mean you see them as equal in soul capacity. All of
these things are practices which need refinement, The mirror needs
constant polish and reorientation. The limbs of the body need regular
exercise.

This is where the protective social life comes in. These are a
collection of clear prohibitions designed to protect the community. Some
are punitive and many are not. On the punitive prohibitions, the
prescribed punishment is designed to be implemented by communities with
Houses of Justice. These are not retributive. In communities with a
different code of laws, those laws are to be honored and respected. What
follows are the protective measures Baha'u'llah ordained which safeguard
dignity, harmony, and rights of all.

\section{Protecting Against Those Who Exalt
Themselves}\label{protecting-against-those-who-exalt-themselves}

The first danger I want to discuss is those who exalt themselves over
another. The most egregious example would be someone who declares they
have revelation of God before the completion of a full thousand years
(2873 AD minimum).\footnote{Kitab-i-Aqdas \#37} At some point, the
period of the Kitab-i-Aqdas will end with a new Day of Resurrection,
Manifestation of God, and a new Book.

\subsection{Regarding Interpretation}\label{regarding-interpretation}

A person does not need to declare themselves to have new revelation or
to be a Manifestation to take steps to act like one. Baha'u'llah warns
against interpreting what has been revealed and deviates from its
outward meaning.\footnote{Kitab-i-Aqdas \#105} This causes distortion of
God's Word and whoever does this is considered a loser in the eyes of
God. These distortions caused by certain interpretive acts can
fundamentally change what people believe are the Words of God and the
true commands. Another way to cause distortion is by claiming these
interpretations are equal to the Word, to the point a believer struggles
to distinguish between Baha'u'llah and the interpreter.

Interpretation itself is not prohibited. Baha'u'llah advises to
interpret both the outward and the inward.\footnote{Tafsir on the Surah
  of the Sun} To neglect either will cause deviation. If Baha'u'llah
interprets anything Himself, this takes precedence. He says ``if you
differ on a matter, refer to what has been revealed by Him''\footnote{Kitab-i-Aqdas
  Verse 53} The issue regarding interpretation is using interpretation
to self-exalt or exalt another. For example, this entire book is how I
personally interpret the Kitab-i-Aqdas and related writings. If you feel
I error, let me know and allow me an opportunity to correct the error.
This book should never be viewed above the Kitab-i-Aqdas. For this
matter, I highly recommend reading the Kitab-i-Aqdas first and reference
the verses highlighted in the footnotes. If you have an interpretation,
even if it differs from mine, it is equal in station. Diverse views can
lead to expanded knowledge and better understandings of truth. If we
limit access to knowledge and understanding by limiting ourselves to
only one interpretation, we actually limit our ability to learn, to gain
wisdom, and to reflect God's name the All-Knowing.

When interpretation becomes a tool for elevating one person's voice
above others as if it has divine authority, the step into claiming
infallibility is dangerously close.

\subsection{Regarding Infallibility}\label{regarding-infallibility}

Another way to act like a Manifestation without declaring one is by
claiming to be infallible. Baha'u'llah says there is no partner for the
Source of the Cause in the greatest infallibility, that no one else has
been given a share.\footnote{Kitab-i-Aqdas \#47} By claiming
infallibility, even if you do not claim to be a Manifestation, you are
claiming the same station and role. Infallibility means a person is
exempt from error or have an inability to make any mistakes. In the
Lawh-i-Ishraqat (Tablet of Splendors), Baha'u'llah does describe a
lesser infallibility where there are various stations. These stations
include a protection by God from mistakes, disobedience, turning away,
disbelief, polytheism, and the like. Each one can be referred to as
lesser infallibility. If anyone deviates, they cannot be infallible.
Lesser infallibility can only be conferred by God, not declared by a
person. Much like the idea behind being a chosen one, you nor I can make
this determination on our own.

Baha'u'llah gives an example in His sermon on infallibility in the
Ishraqat. He declares

\begin{quote}
Look at the blessed, revealed verse that made pilgrimage to the House
obligatory for everyone. Those who stood after Him in command are
obliged to act according to what they were commanded in the Book. No one
may transgress the limits of God and His traditions. He who transgresses
is among the erring in the Book of God, Lord of the Great Throne.
\end{quote}

In this example, if someone declares pilgrimage to another place or site
as lawful from God, they could never be infallible. Anyone who
transgresses is in error. He includes another such message in the
Kitab-i-Badi describing those who believe without hesitation as the
people of infallibility as long as they remain under the shade of God.
The Lawh-i-Sarraj confirms this. Yet, not a single individual was ever
named infallible in any tablet of Baha'u'llah. There are countless He
praises, yet never said any individual on their own was infallible. I
believe the purpose of this was to protect those He praised from the
dangers of self-exaltation.

\subsection{Abolition of Prior Religious
Practices}\label{abolition-of-prior-religious-practices}

There are some other practices mentioned to help prevent the
self-exaltation of another, or to keep us from exalting any other
charismatic person. Baha'u'llah prohibits the kissing of
hands\footnote{Kitab-i-Aqdas Verse 34} as a sign of inferiority. We are
all prohibited from ascending pulpits\footnote{Kitab-i-Aqdas \#154}
which physically visualizes one being above the others.

One of the most important in my view is the prohibition from seeking
forgiveness from another.\footnote{Kitab-i-Aqdas \#34} This is not a
prohibition on apologizing for a mistake. What it is a command to only
repent to God. If a person demands forgiveness from another, they have
asserted an authority of superiority over another. If a person requests
to be forgiven, they have submitted their soul symbolically to the one
they feel they have wronged. Forgiveness can only happen when it is
freely given without anything in return.

\subsection{Begging}\label{begging}

In addition, we are forbidden from begging and from giving to those who
do beg.\footnote{Kitab-i-Aqdas \#147} Begging is not solely an act of
financial destitution. Begging can happen in the context of requesting
mercy in a punishment. It can happen when we desire a secret to remain
hidden. There are many aspects of begging, which can lead to blackmail,
bribes, and other manipulative tactics. Begging completely compromises
one's soul and risks causing the beggar to make compromises they would
not normally make. This rule does not prevent a person from making a
request, or stating their needs or wishes. Asking involves mutual
respect and reciprocity. Begging requires one person to be seen as above
another in a dangerous power dynamic.

\subsection{Boasting}\label{boasting}

The final act of self-exaltation would be boasting.\footnote{Kitab-i-Aqdas
  Verse 72} Boasting is prohibited. This can also take many forms
involving pride and self-satisfaction, designed to convey a sense of
superiority. These forms can include commonly used reasons for
prejudice, such as skin-color, nationality, religious label, ethnicity,
gender, and even body composition. It can include bragging about your
wealth, skills, achievements, or other aspects of identity. To place
yourself on a pedestal simultaneously involves placing another in a
lower status than you. The only true way to increase your own station is
to simultaneously help others increase theirs.

There is no right to self-exaltation, and all believers have a
responsibility from self-exaltation and from exalting others. If anyone
does, the best response would be to remind them we are ``all created
from water, and unto dust (we) shall return.''\footnote{Kitab-i-Aqdas
  \#148}

\section{Prohibited Speech}\label{prohibited-speech}

The next set of prohibitions are associated with what we say and how we
say it. Everything in this section is non-punitive, meaning Baha'u'llah
did not create any enforceable fines or punishments. All of these are
responsibilities of those who believe in Baha'u'llah, but none of these
affect a legal right to speech unless later determined by the Houses of
Justice.

\subsection{Causing Sadness}\label{causing-sadness}

The foundation of this section is the responsibility from section \#148
of the Kitab-i-Aqdas. In the Persian Bayan's Vahid 7 Gate 18, the Bab
instituted a fine of nineteen mithqals to anyone who caused sadness to
another's heart and soul with their speech and conduct. Baha'u'llah
restated this exact punitive law, but repealed the fine. Instead He
requires us be kind and demonstrate piety, reminding us of the golden
rule. There is no exception based on whether a person intended to cause
sadness or if it was unintentional. We have a responsibility to show
empathy. Mindfulness goes a long way, even if we can never control how
others receive our speech. A spirit of friendship and fellowship is
better for us. All people have a right to be free from sadness caused by
another.

\subsection{Slander}\label{slander}

Slander is prohibited.\footnote{Kitab-i-Aqdas Verse 19} Slander,
according to Merriam Webster and LegalZoom, is a form of spoken
communication that is both false and malicious designed to injure the
reputation of another. If the speech is true, it is not slander. If it
is false but makes another look good, it is not slander. This could
differentiate from libel which is written, but as of now I am uncertain
if Baha'u'llah differentiated between oral and written speech. My gut
feeling is both would be included, but I have no evidence of this for
now other than Baha'u'llah's stern responses to written statements which
seemed like libel.

\subsection{Backbiting}\label{backbiting}

Backbiting is prohibited.\footnote{Kitab-i-Aqdas Verse 19} Backbiting is
basically slander without a person's knowledge. This means the speech
must be false, malicious, and without the person's knowledge. All three
criteria must be met for it to be backbiting. If the speech is true,
malicious, and in secret, it is neither slander nor backbiting. If the
speech is false, favorable, and in secret, it is neither slander nor
backbiting.

What does it mean for speech to be true? The standard I would propose is
for the speech to be fully accurate of what was said, whether it be
verbatim quotes or paraphrasing the same effect. The speech must be
fully contextualized. Any omission of relevant context would not meet
the standard of truth and could end up being slander. Whoever recounts
what another says has a right to share how the speech made them feel,
even if this seems as though it is opinion. Feelings are real and thus
true.

Finally, slander and backbiting are tools used to boast. By trying to
injure someone's reputation through slander, you are also trying to
boost your own reputation in comparison.

\subsection{Contention and Disputing}\label{contention-and-disputing}

Contention and disputing\footnote{Kitab-i-Aqdas \#148} are forbidden. A
dispute is basically a failure to agree. The basic reason why a dispute
can persist is if one side or both sides are unwilling to adjust their
original contention. Disagreement is not prohibited, but the persistent
act of disagreement is. Contention ends up causing an environment of
rivalry and competition. To resolve a dispute, both sides have to be
willing to refer their difference to Baha'u'llah's revelation\footnote{Kitab-i-Aqdas
  \#53} and to be willing to let go of their original contention, even
if they feel they are most right. To let go of their original contention
requires a moving of your position to the point of moderation (a virtue)
or by deciding you have no control over the other's position and letting
it go. Having a detachment from outcomes is vital to resolution.

\subsection{Objecting}\label{objecting}

Objecting is prohibited.\footnote{Kitab-i-Aqdas \#73} This is a
seemingly broad concept and in my opinion, one that would be the most
difficult to practice. An objection is a feeling (private) or expression
(public) of disapproval. I could have included this in the private self
discussion, but I felt this would be more relevant for the social life.
However, all objection does start with the private self. The primary
reason a person objects is because you feel something is wrong and it
needs to be corrected. This can include individual objection or
group-based objection such as assembling to protest.

What if you feel you are right? I see two pathways to answer this
question. The first pathway would be to explore why you object, the
consequences of your objecting for yourself and others, and deciding to
let go of your objection. The second pathway is after exploring why you
object, you decide to express it once in a constructive method. One
constructive method I feel is useful is the Socratic Method. According
to Wikipedia,

\begin{quote}
``the Socratic method, named after the Greek philosopher Socrates, is a
form of inquiry and dialogue based on asking and answering questions.
The goal is not to present a definitive answer or lecture but to help
someone discover a deeper understanding of a topic on their own. Instead
of telling you what to think, I've asked you a series of probing
questions to help you examine your own beliefs and the logical
consistency of your arguments.''
\end{quote}

One key aspect of this method is it also gives the questioner an ability
to learn from the answers. In the Bayanic calendar, the month of
Questions (\#15) comes immediately after the month of Speech (\#14),
which leads to the month of Honor (\#16). The Bab's method of organizing
time seems to integrate the importance of asking questions to help
develop honor. Now, someone who is boastful may not take kindly to being
asked questions designed for a deeper understanding. Be mindful in how
this method is used to prevent further conflict.

Another constructive method would be to offer a solution to which you
feel is a problem. If the solution is accepted, you played a positive
part in social change. If the solution is not accepted, allow it to be
and use the experience to inform your own future thoughts, feelings,
actions, and speech. As you get one shot to present your case, present
your case to the best of your ability and in a manner which fits within
the positive social life.

\subsection{Raising Voices in Public
Remembrance}\label{raising-voices-in-public-remembrance}

Baha'u'llah prohibits raising voices in streets or marketplaces in
remembrance.\footnote{Kitab-i-Aqdas \#108} He says the proper place is
within your own home or in places designated for remembrance, such as a
Mashriq'ul-Adhkar or any other place of worship you choose to visit,
regardless of religion. This will help keep people from trying to seem
pious even if they are not, and it also will help prevent conflict with
those who do not believe as you do. This also will help prevent
uncomfortable situations of proselytizing. Yes, we are allowed to
proclaim the Cause, but there are boundaries to be respected. Everyone
has a right not to be pressured or made uncomfortable in public by loud
religious expression.

\subsection{Regarding Books and Other
Writings}\label{regarding-books-and-other-writings}

We are prohibited from reading sciences which lead to
disputes.\footnote{Kitab-i-Aqdas \#77} This could be expanded to any new
types of media which has been introduced since 1873, when the
Kitab-i-Aqdas was written. This can be quite expansive, but can include
opinion pieces, political treatises, ideological statements,
jurisprudence based on traditions, or even commentary which goes against
the Kitab-i-Aqdas (as all disputes should be referred to Baha'u'llah).
However, we are also prohibited from obliterating those
books.\footnote{Kitab-i-Aqdas \#77} Everyone has a right to produce
whatever content they want. Censorship is forbidden. We are responsible
for what we consume.

\section{Dehumanization}\label{dehumanization}

Self-exaltation and the types of prohibited speech are quite harmful to
souls. In Chapter 2.3, we learned how the souls are created noble and
all souls are the temples of a singular mankind. The protections above
ensure the souls can continue to be treated as noble and as sacred
temples. When those protections are violated, we start down a dangerous
and slippery road. This is the road towards dehumanization.

Imagine everything which makes you be you. Part of this you is how you
reflect God's names (even if you do not believe in God), your soul, your
constellation of virtues, your identity, your rights, your private self,
your social self, and all of the loved ones you experience. Each one of
the issues regarding self-exaltation and prohibited speech can be used
to deny you or another a targeted part of yourself, even if it seems
small or insignificant. Repeated violations accumulate, expand, and over
time, works to deprive the very things which make you human. This is the
process of dehumanization. Another places you less human, and maybe more
animal. What are the consequences? The consequences are inevitably
violent and oppressive, towards individuals and target groups of people.
These consequences include the targeted deprivation of rights, harm to
the body, harm to the psychological and emotional self, and potentially
the loss of life.

We have a right to be free from dehumanization and a responsibility to
view every soul as capable of nobility, even if it feels they are
falling short of that standard. The right hand always has the choice of
helping the left hand, and our prayers and deeds can help restore
temples which have fallen in a state of disrepair.

\section{Violent and Oppressive Acts}\label{violent-and-oppressive-acts}

To protect us within the social life, Baha'u'llah specified clear
prohibitions for violent and oppressive acts, along with associated
punishments for many of them. For those without specific punishments or
severities, those are to be determined by Houses of Justice within their
jurisdictions. There are only 8, but the intersection of these 9 aspects
should over most situations if anyone were to appreciate a small code of
laws.

These will be listed in no particular order:

Law \#1: Carrying weapons.\footnote{Kitab-i-Aqdas \#159} The only
exception is during a time of necessity. Ownership is not prohibited. No
punishment is specified. I would consider expanding this to include
objects not typically used as weapons but carried with the intention of
being used as a weapon.

Law \#2: Striking another (assault \& batter)\footnote{Kitab-i-Aqdas
  \#148} For injuries to the head and body\footnote{Kitab-i-Aqdas Verse
  56} there are unspecified fines for each level of severity. The fines
are called diyah, which means blood money. 33\% of all diyah is paid to
the Houses of Justice\footnote{Kitab-i-Aqdas \#52} and the other 66\% is
offered as compensation to the injured person.

Law \#3: There is a fine for manslaughter and other accidental
killing.\footnote{Kitab-i-Aqdas \#188} There are no exceptions. The
diyah payment is 100 Bayánic Mithqáls of gold (See Appendix 4) for the
family of the deceased.

Law \#4: Murder (Homicide).\footnote{Kitab-i-Aqdas Verse 19} The
punishment is execution or life imprisonment.\footnote{Kitab-i-Aqdas
  Verse 62}

Law \#5: Trespassing.\footnote{Kitab-i-Aqdas \#145} There is no
punishment specified.

Law \#6: Theft (stealing).\footnote{Kitab-i-Aqdas Verse 45} The 1st and
2nd offenses includes banishment and imprisonment. Banishment means they
are not allowed to live in their home and will be banished to the
prison. After the 3rd offense, the thief loses their anonymity and will
be permanantly marked on the forehead along with the banishment and
imprisonment.

Law \#7: Arson\footnote{Kitab-i-Aqdas Verse 62} has a punishment of
either execution or life imprisonment. I do want to note this seems
harsh compared to certain legal standards, but arson is impactful. It
deprives a person of shelter, wealth, and potentially life. According to
the US Fire Administration, in 2023 there were 344,600 residential fires
which caused 2,890 deaths, 10,400 injuries, and over \$11billion USD in
property losses. Even if loss of life is unintentional, the arson is an
intentional act.

Law \#8: Selling and buying people\footnote{Kitab-i-Aqdas Verse 72}
without any stated punishment.

\subsection{Intersection of Laws: An Example of
Rape}\label{intersection-of-laws-an-example-of-rape}

One might read the Kitab-i-Aqdas and wonder why a crime such as rape is
not specified. I believe it is because rape includes many aspects of
these laws. Rape often includes the use of a weapon, in this case
whatever which was used to commit the act such as a man's penis. Rape
often requires a physical assault, or the threat of physical assault.
The victim's body was trespassed against without consent and the
victim's body was used against their will, which is a theft. More could
be stolen if virginity is a consideration.

This example of rape includes violations of laws \#1, 2, 5, and 6 by the
perpetrator. With laws 5 and 6 being determined by things such as
severity, this could be a more severe punishment than is typical for
rape. I feel intersectionality could be a way to simplify a legal code
to make the mechanisms of justice fairly quick and straightforward.
However, it is up to each House of Justice in how they refine and
specify such laws. It would also make sense for a House of Justice to
specifically specify rape.

This concludes the basic foundations of the social life. We will
transition towards building and maintaining loving relationships,
marriage, parenting, and divorce.

\newpage{}

\chapter{Affectionate Relationships}\label{affectionate-relationships}

This chapter is going to look at various kinds of affectionate
relationships. Traditionally marriage is the type of relationship most
think of when they imagine religion. It is true marriage is the
foundation of affectionate relationships, but marriage is not the only
kind. Marriage does not magically happen in a vacuum. There are
affectionate relationships before marriage, some people choose never to
marry, and then there are the situations where these affections end.

Marriage will be the foundation for this chapter. When we understand its
purpose, we can consider how affectionate relationships are a deep
reflection of our beliefs, regular spiritual practices, our unique
constellation of values. They are shaped by our fidelity to the rights
of all souls, and the rights and responsibilities we attain after
maturity, whether privately or socially.

Much of the marriage law is from the Persian Bayan, with modifications
within the Kitab-i-Aqdas. The Bayan is integral to the understanding of
Baha'u'llah's marriage law. Symbolically, this is where I would say both
Books find union, in the verses regarding the union of two people in
wedlock.

\section{Marriage}\label{marriage-1}

Marriage is enjoined upon us.\footnote{Kitab-i-Aqdas \#63} Enjoined is
also used in the counsel to pray. This means it is part of the divine
Doctor's prescription for us to be healthy. The Bab, when he described
marriage being enjoined, reinforced the fact it is a firm
obligation.\footnote{Persian Bayan Vahid 8, Gate 15} Baha'u'llah
instructed monks to leave their seclusion so they may bear fruits and
enter into wedlock.\footnote{Suriy-i-Haykal} To a particular believer,
Baha'u'llah said marriage is not permitted to be passed by as it is
God's tradition.\footnote{BH10796}

\subsection{Purpose of Marriage}\label{purpose-of-marriage}

From the second paragraph of the Marriage Prayer (I am unable to find
the first), Baha'u'llah says marriage is a ``fortress for well-being and
salvation'' and ``that from you may appear he who will remember Me
amongst My servants.''\footnote{BH03181} These are the fruits both the
Bab described as ``a letter of their own being may remain to proclaim
the unity of God.''

The Bayan had recommended believers to separate if either partner is
able to fulfill the obligation of having a baby. In the Suriy-i-Haykal,
Baha'u'llah says ``But for man, who, on My earth, would remember Me, and
how could My attributes and My names be revealed?'' Yet, in the
Kitab-i-Badi as Baha'u'llah was addressing Azali Bayanis, He rebukes
those who say a woman who is barren has no value. Baha'u'llah says those
who make such claims have barren hearts, minds, eyes and are deprived of
the mercy of God. The implication is every person has a right to create
a child, with a responsibility to do so if you are able to. Still, every
person maintains their right to love, kindness, and all the other rights
if they are unable to create a child. Cruelty is forbidden.

What do you imagine when you hear the phrase a fortress for well-being
and salvation? In my imagination, I like to relate the journey of an
affectionate relationship and marriage to be similar to the journey of
the soul to God the Beloved in Baha'u'llah's ``The Seven Valleys'' with
the valley of unity being where a couple truly feels they are one. This
unity helps provide emotional and spiritual stability through such
methods as affection, trust, and loyalty. Each spouse is able and
willing to shield each other, being each others fortress wall. As each
acts in remembrance of each other, they can also reinforce a deeper
remembrance of God. This unity helps each spouse develop their
individual constellations of virtues, where they help compliment,
balance, and merge with their loved one's constellation. These virtues
are tested and refined, in good times and in bad. Finally, this fortress
for salvation extends to the spouse's fruit, their children and to the
community they live in. For the child, it provides stability to nurture
their noble birthright. For the community, it serves as a lamp leading
by example. In both ways, the married couple ensures God is remembered
across generations.

Marriage, in this sense, is the very foundation of the constructive
social life. Every mature person has a right to a fortress of well-being
and salvation. No person has a responsibility to deny this right under
any circumstances.

\subsection{Who May Marry and How
Many?}\label{who-may-marry-and-how-many}

Everywhere Baha'u'llah describes marriage, he does define it within the
male and female dynamic. BH03181 says ``the Lord loves that His
handmaidens be joined in marriage to His monotheistic servants.'' In the
Aqdas He says ``Beware that you do not exceed two wives. He who is
content with one maid will ensure his own tranquility and hers as
well.'' These verses explicitly says marriage is between a man and a
woman.

These verses also allow for two wives (bigamy) with a promise of
tranquility for a man who prefers monogamy. When looking at the divorce
verses, such as in Kitab-i-Aqdas \#68, it suggests a woman could also
have a second husband. If she takes another husband, a new union is
required to restore the first marriage. Baha'u'llah does not specify she
must divorce the second husband, although traditionally religious
institutions would not allow this.

I will explain within the context of marriage being a form of contract
law. The 1st marriage with husband \#1 was under Contract \#1. They
decided to end Contract \#1. Therefore, the woman enters into Contract
\#2 with husband \#2. If ex-husband \#1 wants to be married again to the
woman, they must mutually agree to a new Contract \#3. The Kitab-i-Aqdas
does not say Contract \#2 must end.

Thus, every mature person has the right to marry, and if the law allows,
the right to have up to two spouses. Yet this right is tempered by the
responsibility to be content with one, for tranquility rests upon the
foundation of simplicity and fidelity. The fortress of well-being may
stand with two, but its walls are strongest when founded upon unity with
one.

\subsection{Consent}\label{consent}

Marriage requires the consent of both parties.\footnote{Kitab-i-Aqdas
  \#65} Consent can only be obtained by those who have attained
maturity. Consent cannot be obtained through coercion or any type of
threat. Consent is formerly recognized for both the man and woman with a
phrase signifying their devotion to God.\footnote{Persian Bayan Vahid 6,
  Gate 7}

\begin{quote}
``Indeed, I am for God, the Lord of the heavens and the Lord of the
earth, the Lord of all things, the Lord of what is seen and unseen, the
Lord of the worlds.''
\end{quote}

Consent also involves payment of a dowry from the groom to the bride. If
the man is living in a city, he must offer between 19 and 95 Bayanic
mithqals of gold. For a man living in a village, dowry is the same
weight in silver.\footnote{Kitab-i-Aqdas \#65} This continues the
Bayan's law, which also says the dowry must be in incremental steps of
19, and nothing in between. The groom offers what he is willing or able
to pay and the bride chooses to accept. See Appendix 4 regarding the
calculation of a Bayanic mithqal.

I feel it should be noted dowry is not a bride price paid to family, nor
is symbolic in any way of the bride being property. The dowry is a gift
for the bride. It is a symbol of affection towards her while helping
provide financial security and stability. Currently, in most countries,
women have a lower average income than men, are more likely to be single
parents, and face the most risk. I believe those risks are less if the
soul's mirror is sincerely facing the Sun of Truth, but things happen,
people change, and faith can waiver. While women are equal in the eyes
of God, it is also true they often face unequal outcomes. Every
believing groom has a responsibility to pay his bride the dowry, and
every bride has a right to the dowry.

Baha'u'llah also enjoins consent of both parents of the bride and groom.
This helps bring families together. He reminds us of the Qur'an and how
that Holy Book relates kindness to parents to belief in the oneness of
God.\footnote{BH02956} He says there are hidden purposes with this
command. At best, I believe this serves a purpose of reconciliation for
any parent and child who might be estranged, and an opportunity for
parents who do not believe in God or Baha'u'llah to learn more about the
Cause of God. At worst, two people who love each other may not get
married. This can be a profound spiritual test for all involved.

In a different tablet, Baha'u'llah discusses how since the Day of
Resurrection, families and friendships have struggled when one believes
and one does not. Often, the one who believes in Baha'u'llah becomes
estranged with the other renouncing them.\footnote{BH00091} He says as a
consequence, relationships are not considered. ``Accepting souls are
both fathers and sons, for they walk upon one path.''

In one example, Baha'u'llah nominated himself to be in place of the
father of Mirza Asadu'llah Isfani in his marriage with Gawhar in
1882.\footnote{BH00093} Baha'u'llah could not attend in person, but it
does suggest the consent of parents carries a great deal of nuance.
Ultimately, the Houses of Justice can appoint individuals to oversee
marriages and determine such matters of consent. I think any House of
Justice should be careful to focus on love, affection, and unity in such
a way discord and resentment is not created.\footnote{Kitab-i-Aqdas \#65}

\subsection{Virginity}\label{virginity}

Marriage is not dependent on virginity. In a traditional sense, this
helps keep marriage as an option for divorcees, widows, and single
parents. Outside of the traditional sense, this also helps prevent
purity standards from being established. Baha'u'llah in His counsels
does regularly refer to chastity.\footnote{BH00726} There is regular
counsel towards women to be chaste when He says ``The greatest ornament
for the handmaidens hath ever been and still is chastity and virtue. By
God! The light of chastity illumineth the horizons of the spiritual
worlds and its fragrance reacheth unto the Most Exalted
Paradise.''\footnote{BH02573}

While there is a right to purity and a right to be free from corruption,
purity cannot be forced upon anyone. There cannot be any forms of
oppression such as virginity inspections imposed upon women. With all of
the teachings of Baha'u'llah, especially within the constellation of
virtues, premarital sex would be strongly discouraged. This
discouragement is based solely on purity, but it is to ensure the bonds
of affection are not ruled by lust or other biological impulses of the
body. The bonds of affection are to be ruled by spiritual
considerations.

If one is solely attached to the concept of virginity, but overlooks
other spiritual qualities, a marriage is going to have a difficult time.
If a person has premarital sex, but later decides through their
spiritual practice to develop chastity, this person can be purified. All
of the spiritual virtues have opportunities to shine more brightly and
are never completely lost.

To demonstrate this idea, I want to show two examples. In one,
Baha'u'llah references Fatimah al-Zahra, the daughter of the Seal of the
Prophets Muhammad. She was married to 'Ali ibn Abi Talib and they had at
least 4 children. They are known in Shi'a Islam as the Ahl al-Bayt,
where the Imams were descended. Baha'u'llah says ``O Land of Sad, the
Sun of Hijaz asks thee about His family, and the Virgin, the Chosen One,
about her sons and daughter.''\footnote{BH02352} How is it a mother is
given the title Virgin? This demonstrates virginity is not solely a
physical attribute but a spiritual condition for one who acts purely for
the sake of God.

In the second example, Baha'u'llah is discussing Mirza Yahya, the leader
of the Bayani people who did not believe in Baha'u'llah. He says ``Every
year he would summon a virgin from the outlying regions, \ldots{} veiled
by the imaginary veils of these idolatrous temples whose reality is
evident from their deeds.''\footnote{BH00357} In this example, the
coveting of virgins is a form of idolatry which is a veil between you
and God.

\subsection{The Marriage Contract}\label{the-marriage-contract}

In the Bayan, once the consent verse is recited by both parties, they
affix their seals (signatures) on a document. This is the marriage
contract. The Bab requires witnesses representing both spouses, at least
ten in total, to attest to dowry, consent, and signing the contract. The
purpose of the witnesses is so that both spouses are protected, and
neither can deny their marriage nor claim a fake marriage.\footnote{BH00035}
This particularly protects the rights of women from coercion. The Houses
of Justice are to ensure marriage is conducted according to the divine
law. Each couple can add anything to the contract they feel is
important.

One addition could be to state if the marriage is monogamous or
polygamous. If it is monogamous, there is no possibility of future
spouses within that marriage. If it is polygamous, consent would be a
vital foundation. Both spouses and the newest spouse would have to each
consent to this new union, along with the parents. If any person does
not consent, the second marriage cannot be created. It is acceptable if
the fortress is open to another, but never in such a way the fortress
imprisons one spouse. This ensures each spouse retains the right to be
free from coercion, with each bearing responsibility to honor the terms
agreed upon.

The marriage contract becomes nullified if dowry is not paid, if one
does not return home from travel by an agreed upon time, and if one of
the spouses passes away.

\subsection{Divorce}\label{divorce}

This contract can be terminated through divorce. Divorce is an option if
any discord or aversion arises between the two, and marriage can
continue if there is reconciliation and the fragrance of love emanates
between them.\footnote{Kitab-i-Aqdas \#68} Divorce requires a one-year
waiting period for an opportunity for reconciliation and love to be
reestablished. Divorce is only allowed three times, which also means a
person is only allowed to be married three times, even if you believe
that fourth marriage is guaranteed to never end. During the waiting
period, they are still considered legally married and the marriage
contract is in effect.

\subsection{Discord}\label{discord}

Discord in a marriage is any kind of persistent conflict, hostility,
aggression, a breach of trust, or even irreconcilable worldviews. If a
marriage ever leads to the rights of one or both being violated
regularly, it might be good to separate. The responsibility to remain
married should never infringe upon the rights to safety, love, kindness,
and freedom from oppression. Individual goals which are not mutually
beneficial could be a source of argument.

\subsection{Aversion}\label{aversion}

Aversion can be fairly broad in scope. It could be as simple as
disliking each other, such as a loss of affection where bonds of
intimacy or respect are lost. Aversion could be apathy or other
emotional alienation where marriage feels cold or forced. If closeness
feels forced or requires a sense of degradation, there is aversion.
Aversion could also be from a situation where each others constellation
of virtues are no longer aligned, compatible, or changed so
significantly, the spiritual bond is severed.

\subsection{Adultery}\label{adultery}

Adultery is a breach of the marriage contract which is enforceable by
Houses of Justice. For each man and woman who commits adultery, a fine
must be paid of nine Bayanic mithqals of gold.\footnote{Kitab-i-Aqdas
  \#49} For each offense, the fine is doubled from the prior offense.
This goes from 9 to 18 to 36 to 72 and so on. Baha'u'llah also specifies
a humiliating punishment for a third offense, which would be similar to
the punishment for theft, where a mark is placed on the thief's
forehead. How the third punishment would take shape is up to the Houses
of Justice. The purpose would be to let other people know someone is
violating their marriage.

Adultery has also traditionally been treated as including all forms of
extra-marital sex, to include people who are not married. One way I have
personally considered this is within the scope of the age of maturity.
What would happen if two people who are not mature nor independent
decide to have consensual sex? It would most likely be financially
impossible for both of them to pay 9 Bayanic mithqals of gold. The
reason I view adultery as being more about contract law is because of
the seriousness when a contract is violated. The Bab and Baha'u'llah
repeatedly commanded people to honor their commitments, in both their
personal and business lives, reflecting the virtues trustworthiness and
fidelity.

\section{Affectionate Relationships Leading to
Marriage}\label{affectionate-relationships-leading-to-marriage}

With marriage being a fortress of well-being and salvation, we
understand it is a core institution of our social life. Marriage does
not happen in a vacuum. It requires the ability to build the bonds of
affection and to feel the fragrance of love. Even if these bonds are
built, marriage may not be a goal a person has. While there is a
traditional ideal, we should be mindful Baha'u'llah says that ``God
loves unity and concord.''\footnote{Kitab-i-Aqdas \#70} In seeking an
ideal, we shouldn't risk limiting the ways unity and concord can
manifest itself in every day life.

\subsection{Stage 1 - Friendship}\label{stage-1---friendship}

As discussed in Chapter 18, our first foundation is friendliness and
fellowship\footnote{Kitab-i-Aqdas \#148} even with the followers of
other religion.\footnote{Kitab-i-Aqdas \#144} We show friendliness
through casual interactions, enjoyable shared experiences, and building
connections based on mutual interests. Friendliness is characterized by
warmth and kindness, not merely politeness. This starts from the moment
we greet someone, such as by smiling and showing they are welcome. In
our interactions, we engage in light conversation and show genuine
interest in their lives, such as who they are, what they do, and what
they enjoy. We offer invitations in social activities and accept
invitations when possible.

Friendship requires mutual effort. Every person you come across may not
be a friend. Baha'u'llah warns us not to associate with the wicked, nor
seek companionship with them.\footnote{Hidden Words of Persian \#57} If
someone comes up to you and does not start with a smile and a warm
greeting, be cautious. If a stranger is focused on fulfilling their
immediate wants, they may not be a friend. Friendship is not a
transaction.

If a relationship only remains a friendship, this friendship is still a
vital component in the Cause of God. This is the foundation of all other
affectionate relationships

\subsection{Stage 2 - Fellowship}\label{stage-2---fellowship}

Fellowship is a type of relationship which grows from friendship. You
take everything you have been doing with your friends and add in some
components of commitment, support, sacrifice, and growth. Fellowship
adds a strong spiritual component, where you might meet together for
spiritual practice, the study of scripture, or acts of service. When
there are difficult times, you are there offering encouragement and
accountability. You are willing to help, even if it requires a level of
self-sacrifice such as sharing time or resources. The conversation can
be more challenging, where discussions promote spiritual growth and
understanding. This type of challenge can help each other deepen in
faith.

There are times where one person desires fellowship, but the other
friend or friends may not share this desire. A person should make the
effort to achieve fellowship if they desire it, but it should not be
done in any oppressive way. If fellowship is not returned, be content to
in your friendship.

\subsection{Stage 3 - Spiritual
Kinship}\label{stage-3---spiritual-kinship}

From fellowship can come a relationship I am calling spiritual kinship.
The foundation of this relationship is a shared belief in Baha'u'llah
and living in the same Day of God. Distant souls are drawn
closer\footnote{BH00091} by walking one path together, drinking the same
nectar, gazing towards the same horizon, and are secluded together in
one abode. In spiritual kinship, believers feel as though they are
parents, children, and siblings to each other even if this is untrue
biologically and legally. This brings a new layer when the Kitab-i-Aqdas
says to show love and kindness to your kindred.\footnote{Kitab-i-Aqdas
  \#61}

The Hidden Words is, in my opinion, a pillar of spiritual kinship.
Throughout the Arabic and Persian versions, Baha'u'llah enjoins us with
a declaration ``O Son\ldots{}'' where each of us are both a Son of God,
a Son of Baha'u'llah in His role as Father, and thus we are siblings of
each other. For example, when the Arabic Hidden Word \#38 says ``grieve
not in your distance from us,'' we can share in our grief as a single
family who longs for nearness to God.

\subsection{Stage 4 - Courtship}\label{stage-4---courtship}

Courtship takes what we have established in friendship, fellowship, and
spiritual kinship and adds the fragrance of love. There is a sense of
desire and longing to be near all the virtues the other person
exemplifies. You love their piety, pure truthfulness, courtesy, loyalty,
and trustworthiness, even if the person does not perfectly demonstrate
them. You feel safe and secure with them. You can visualize building
that fortress together, a fortress which has potential room for
descendants. You can see your souls shining within the same mirror, even
if they have not yet merged into one.

\subsection{Stage 5 - Engagement}\label{stage-5---engagement}

If courtship is successful, you are on the path towards marriage. When
discussing marriage, Baha'u'llah counsels regarding the matter of
marriage by saying ``it behooves whosoever desires to enter into any
matter to first hold fast to consultation and to act according to what
appears therefrom, placing his trust in God, the Protector, the
Self-Subsisting.''\footnote{BH03310} This applies to attaining the
consent of not only each other, but of the parents (or representatives
of the parents). If the consultations are successful, you are engaged.
If they are unsuccessful, you are not engaged. Thereupon you can either
continue courtship or see if it is best to go back to spiritual kinship,
and move on. At no point should this consultation lead to discord or
enmity.

Once engaged, Baha'u'llah advises delaying the wedding is not
recommended.\footnote{BH01039} He did not define a hard limit though and
in the same advice, said the timing of the wedding is entirely up to the
couple's discretion. This might be something discussed by the couple and
parents in their consultation.

\subsection{Skipping Stages}\label{skipping-stages}

Life can be rather complex and messy, and sometimes a person can be
ruled by their passions more than their relationship with God. In modern
life, it is common to start straight at courtship or a simulated
marriage before the foundations of friendship were even established. The
more stages which are skipped, the more difficult it is to establish an
enduring fortress for salvation and well-being. It is not impossible,
but if there is a sense of commitment without fully knowing what is
being committed, tranquility may seem elusive.

Spiritual kinship might be the most difficult stage to achieve,
especially if the relationship is an interfaith relationship. While
Baha'u'llah enjoins friendship and fellowship with those of other
religions, it takes a special discernment to identify if you are kin to
one of another faith. Can you share in each others spiritual practice?
Can you view each other as sharing soul-building experiences even with
having different outward labels? Navigating this carefully and
intentionally should open the doors of unity without neither having to
compromise who they are. God is the Lord of all religions, and is the
God to liberate all souls.

\section{Non-Traditional Relationship
Choices}\label{non-traditional-relationship-choices}

To close this chapter, I want to briefly discuss two concepts which I
feel are important for well-functioning communities to understand and
accept. The first concept is when a person chooses to remain unmarried,
regardless of the relationships they form. They may prefer spiritual
kinship or courtship as representing their best way to show love.
Celibacy, while not being enjoined upon believers, is a personal option
which should be respected.

The second concept is those who express their right to identity in ways
which are not easily defined by traditional religious labels and
discourse. This can include gender identification, gender-role
identification, and preferences in who the feel the fragrance of love
for. This is where we should be reminded that God prefers unity and
concord, the same God who created all souls noble. Communities need to
ensure they are not the cause for discord or enmity, especially for an
affectionate relationship to end. On the same token, as expressed in
Chapter 16 ``The Private Self,'' we also need to be aware as individuals
what is an illusion and what is real. This is a delicate balance
individuals and communities must navigate with care and consultation.

I am of the belief the fortress of salvation and well-being should be
accessible to all in an inclusive way. All of these affectionate
relationships, friendship, fellowship, spiritual kinship, courtship,
engagement, and marriage are rights for all mature people, with
friendship, fellowship, and spiritual kinship a right of all people from
birth. Every person has a responsibility not to deny these rights. God
prefers unity and concord.

\newpage{}

\chapter{Trusteeship}\label{trusteeship}

So far in Part 3, we have looked at how the Kitab-i-Aqdas guides us
through the rights of all people and the responsibilities we have as
peers. The Cause of God has first been introduced in our social life,
which is an integral baseline in our interactions with others. This is
not all the Cause is. I feel the most important aspect of the Cause of
God is trusteeship.

A trust is a type of custodial relationship in which a person, people,
or organization are charged to care for another. It is usually
considered a legal term, but surprisingly describes responsibilities of
trustees regularly in His revelation. A trustee is person, organization,
or legal entity who is designated to be responsible for a trust.
Typically a trustee is responsible for property or any other
contractually obligated purpose. Trusteeship is how we describe the
position, function, and duties of a trustee in relationship to the
trust.

There are many kinds of trusteeships expressed in the Kitab-i-Aqdas.
Baha'u'llah describes trusts for orphans, widows,\footnote{Kitab-i-Aqdas
  \#21} weak descendants,\footnote{Kitab-i-Aqdas \#27} the
destitute\footnote{Kitab-i-Aqdas \#147}, and the kingdom.\footnote{Kitab-i-Aqdas
  \#172} The trustees are the Houses of Justice, fathers, the wealthy,
and God. To frame the rest of this chapter, let us take a look at the
sacred trust of God with God as the Trustee.

\section{God as Trustee - Kingdom as
Trust}\label{god-as-trustee---kingdom-as-trust}

In verse 172, Baha'u'llah outlines the Sacred Trust. The trust includes
three closely interconnected realities. They are dominion, the kingdom,
and the realm of power. A dominion is an area or territory where
sovereignty is exercised. Baha'u'llah associates dominion with the role
of God the Witness. A kingdom is a specific type of dominion ruled by a
monarch and the law the monarch governs by. Baha'u'llah associates the
kingdom with the role of God the Trustee. The realm of power is a more
broad concept which can include any area or subject where authority is
exercised. The realm of power includes any place where there is a power
dynamic within the political, social, and spiritual spheres. It is
associated with the role of God the Helper.

Knowing God is sufficient to be a Witness, we understand He is the Judge
in all of creation. He does observe how sovereignty is used and for what
ends sovereignty achieves. We need to be constantly diligent in how we
use sovereignty. Knowing God is sufficient to be a Trustee, we can trust
God will be a steward of us, sharing what the Cause of God is for the
liberation of our souls and ensuring the Kingdom of God is a sovereignty
deep within our hearts. Knowing God is sufficient to be a Helper, we can
be assured of support and sustenance as we live our lives in remembrance
of Him.

These three roles, Witness, Trustee, and Helper are three roles we can
also serve as we align our souls with God's Names. If we serve as
witness without serving as trustee or helper, we are merely judgemental.
We exalt over others and use power to subdue and illicitly gain, even if
those gains are according to the rules of society. If we merely try to
be a helper, but we do not witness a root cause nor understand our roles
as a steward, our efforts might only be temporary or cause further harm,
even if unintentional. Trusteeship requires both witnessing and helping
to be fully dynamic, just as God serves as Trustee of us. Most of the
rest of the book will be framed within the context of Trusteeship, as we
have moved into the realm of power and the dynamics power causes. This
chapter will look at Trusteeship with weak descendants, widows, and the
destitute.

\section{Weak Descendants as Trust - Parents as
Trust}\label{weak-descendants-as-trust---parents-as-trust}

Weak descendants include any children from the time they are conceived
until they reach maturity and any adult descendants who are unable to
attain maturity due to circumstances such as a developmental handicap.
The Kitab-i-Aqdas first associates them with a trust related to
inheritance. All rules regarding inheritance will be discussed at the
end of this chapter. A trustee is recommended to manage their
inheritance until they reach maturity, with an agreed upon rightful
share of the profits paid to the trustee. The inheritance is to be
invested with the purpose of gaining profit, also known as acting as a
fiduciary.

Every father is enjoined to educate their sons and daughters in learning
and writing, and also what has been prescribed.\footnote{Kitab-i-Aqdas
  \#48} This includes the teachings of Baha'u'llah and other subjects,
such as sciences which profit mankind and the ability to recite the
verses in Mashriq'ul-Adhkars in the best of melodies. Baha'u'llah says
any father who fulfills this duty for their children or even other
children is the same as educating one of Baha'u'llah's children. It is
an honor. If the father fails to educate their children, the trustees of
the House of Justice are to take from the father whatever is necessary
for their education if the father is wealthy. If the father is not
wealthy, the matter is to be referred to the House of Justice. A
negligent father has lost the right of fatherhood.\footnote{BH09698}

In Chapter 14, many rights were defined for every person. The parents
are entrusted with the responsibility in ensuring these rights as the
child's trustee. The trust includes the right to life, identity, purity,
dignified appearance, love and kindness, freedom from oppression, to
skilled physicians, to inheritance, to worship, and to education. This
responsibility lasts until the child reaches the age of 19. If there are
periods when the mature child needs assistance, the parents must do so
if they are able to.\footnote{Arabic Bayan Vahid 10, Gate 14}

The specifics does not have to be the same for every family, as each
family has different environmental, cultural, and economic contexts.
However, sincerity to the framework is vital. With the purpose of
marriage being to create one who makes mention of God, the foundation of
this could be teaching the names of God and nurturing souls who desire
to mirror those names. Subtlety through Purity and the way they reflect
will be unique as the child's iris's. Their right to identity includes
their ability to shape this identity through these Names, which all lead
back to One.

In fulfilling this trust, how does acting as a witness and helper help
the parents? To a young child, a parent already seems so powerful and
capable. The parent has full dominion. They are a witness to their
child, their welfare, their personality, their environment, and
everything else which affects the child. While the parent seems so
powerful and capable, they witness the weakness and vulnerability of
their child. They use their power in trust to nurture within that
framework of responsibilities and belief. The purpose is to help the
child become strong, mature, and not only a noble soul at birth, but an
adult who retained their nobility. The framework of spiritual,
educational, and full legal maturity at ages 11, 15, and 19 offer goals
of refinement and targets for parents to achieve. Finally, the framework
of witness, trustee, and helper also aligns with the Bayanic calendar's
rhythm of maturity.

\begin{longtable}[]{@{}
  >{\raggedright\arraybackslash}p{(\columnwidth - 6\tabcolsep) * \real{0.2083}}
  >{\raggedright\arraybackslash}p{(\columnwidth - 6\tabcolsep) * \real{0.2083}}
  >{\raggedright\arraybackslash}p{(\columnwidth - 6\tabcolsep) * \real{0.3750}}
  >{\raggedright\arraybackslash}p{(\columnwidth - 6\tabcolsep) * \real{0.2083}}@{}}
\toprule\noalign{}
\begin{minipage}[b]{\linewidth}\raggedright
Age
\end{minipage} & \begin{minipage}[b]{\linewidth}\raggedright
Bayánic Month
\end{minipage} & \begin{minipage}[b]{\linewidth}\raggedright
Meaning
\end{minipage} & \begin{minipage}[b]{\linewidth}\raggedright
Spiritual Station
\end{minipage} \\
\midrule\noalign{}
\endhead
\bottomrule\noalign{}
\endlastfoot
\textbf{11} & \emph{Masá'il} & ``Questions'' --- awakening of
understanding & Witness \\
\textbf{15} & \emph{Sultán} & ``Sovereignty'' --- command and
responsibility & Trustee \\
\textbf{19} & \emph{'Alá} & ``Loftiness'' --- fulfillment and unity &
Helper \\
\end{longtable}

If the child for some reason is unable to become mature and independent
due to unplanned circumstances, these responsibilities for the parents
never end.

\section{Orphans as Trust - Houses of Justice as
Trustee}\label{orphans-as-trust---houses-of-justice-as-trustee}

Sometimes a child loses both parents. Throughout history, this has
caused undue hardship upon children. Baha'u'llah laments this in the
Lawh-i-Sultan, describing how the rulings of Naser al-Din Shah Qajar is
causing new orphans to be made. The Kitab-i-Aqdas designates the
Trustees of the Houses of Justice to be in trust of orphans.\footnote{Kitab-i-Aqdas
  \#21} The funds are to be derived from any inheritance which is
supposed to go to descendants when there are no dependents. This money
is also to be used for widows and the general benefit of others.

Baha'u'llah served as the trustee for orphans in His lifetime. For
example, he provided fifty tumens for the orphans of Ḥájí 'Alí
Qarabagí.\footnote{BH00332} While orphans are being care for in trust by
the Houses of Justice, it would also be good if a person chose to adopt
an orphaned child. As Baha'u'llah had described the honor it is for a
father to educate a child that is not his own, imagine the glory in
raising a child as if it were Baha'u'llah's? Orphans have a right to be
taken care of. Their rights are not lost when their parents are lost.

\section{Aging Parents as Trust - Mature Children as
Trustee}\label{aging-parents-as-trust---mature-children-as-trustee}

For parents who have not forfeited their rights as parents due to
neglect of their children, Baha'u'llah says the second greatest fruit is
regard for the rights of one's parents.\footnote{BH02374} The Bab says
we are to provide for our parents if they are not
self-sufficient\footnote{Arabic Bayan Vahid 10, Gate 14} once we have
reached maturity at age 19.

This helps establish a lifelong bond of mutual assistance between
parents and children. All mature people are obligated to earn a living
if they are able to, so this would mostly be assistance as parents age
or random circumstances of life impact their ability to earn, even if
temporarily. The parents are a trust with mature children serving as
trustee.

Now, even as parents age, they must still be treated with the full
dignity, respect, and rights of every person. Authority over their
possessions cannot be denied unless a skilled physician has determined
there is an onset of true mental incapacity.\footnote{BH00035} If an
adult has lost their mental capacity but does not have mature children,
the authority over their possessions, if not clearly defined by a legal
contract, would be entrusted to the Houses of Justice.

In regards to aging, there are some rights and responsibilities
associated with decaying health and the potential of dying. We had prior
established the right to skilled physicians. If there is illness, seek a
physician.\footnote{Kitab-i-Aqdas \#113} Baha'u'llah uses a candle as an
instructive metaphor. He says ``Consider the candle and its light. If
you protect it from outward elements through means such as a lantern or
otherwise, then as long as the wax and wick remain, it is possible to
preserve it from temporal vicissitudes. However, if it reaches its end -
that is, if the wax and wick are exhausted - in this case the
continuation of light is impossible. This latter state is determined,
while the former state is conditional.'' He also says the preservation
of human life is greater than any other matter.\footnote{BH00093}

Forcing a person to artificially be alive would be a violation of the
trust. Medical care has to be focused on healing a condition. It would
be vital for parents and children to understand their rights and
responsibilities in a spirit of consultation, especially in stressful
and painful times.

\section{Widows as Trust - Houses of Justice as
Trustee}\label{widows-as-trust---houses-of-justice-as-trustee}

Sometimes a wife loses her husband. Like orphans, this has often caused
undue hardship to the widow. While she has the ability to work and earn
a living, she may have also been provided for by her deceased husband.
She may still be nurturing a child. We do not know her circumstances.
While she has full autonomy of her life without losing any rights she
had before and during her marriage, her care is entrusted to the Houses
of Justice.

I want to explore the witness, trustee, and helper model with widows.
How would witnessing manifest itself while respecting the rights and
autonomy of the widow? Witnessing does not equal surveillance or
invading privacy. That is contrary to our constellation of virtues such
as courtesy. The House is not to serve as her master. Witnessing
involves seeing their dominion clearly. This means ensuring she is safe
from exploitation in a vulnerable time and she receives what is due to
her from inheritance. If she is economically or socially vulnerable, it
should be acknowledged without stripping her of her agency.

As trustee, the Houses are stewards of fairness. They do not act as her
decision maker. They ensure systems of support are fairly distributed
when requested, and they are freely available to be requested. They do
not manage her choices, relationships, or property. Instead they oversee
any process of communal support without favoritism or corruption. The
purpose of this trusteeship is to offer help and assistance when needed,
without coercion. They provide resources, which does not need to be
limited to money. These resources could be access to a grief counselor,
support groups, protection from social pressures, and introductions to
new opportunities. The process of witness, trustee, and helper is an
extremely simple yet thorough model to exercise power as an act of
service.

\section{The Destitute as Trust - The Wealthy as
Trustee}\label{the-destitute-as-trust---the-wealthy-as-trustee}

In Chapter 18, we described the prohibition on begging. We also have a
responsibility to never give to beggars, no matter what is desperately
being asked of. This seems contrary to generosity, but we have outlined
such responsibilities as Huquq'u'llah and Zakat. The goal is to create a
community where no person would need to beg. No person deserves to be
destitute nor desperate. Every person has a right to provision.

There are two trustees of the destitute. The first are the Houses of
Justice.\footnote{Kitab-i-Aqdas \#147} The second are the wealthy. Each
are to provide what is necessary. The Houses of Justice can consult on
what is necessary, but with it could fall within the right of provision
from Chapter 14 Section 7.7. A wealthy individual could also make this
determination, according to how generous their spirit is.

Who is wealthy? I would define wealthy as any person who has wealth.
Wealth is an abundance of resources which can be used to invest or make
transactions without worry of running out of resources. The wealthy have
plenty. The destitute or impoverished have nothing or very little. They
are a trust and the wealthy are meant to serve as their trustees.

As a potential conflict of interest, a community might want to consider
limiting how many wealthy people serve on Houses of Justice. Trusts
struggle with ethical issues when there are conflicts of interest.

\section{Animals and the Earth as Trust - Believers as
Trustee}\label{animals-and-the-earth-as-trust---believers-as-trustee}

The Kitab-i-Aqdas does not explicitly tell us to take care of the Earth
and all that is contained within the environment. There are allusions
throughout. So far we have learned God has created everything, with the
Earth made the vessel of our provision. While the parents have a
responsibility to provide provision to their children, God has provided
the provision for everyone. We also described how all people have a
right to provision. I would say these provisions themselves have rights,
and the living things of earth have a right to provision.

One of these provisions is water, which we have described as a purifying
agent. How can water purify if the water is polluted? Baha'u'llah says
rivers of fresh water gush forth from the stones (a natural spring) due
to the sweetness of our Lord.\footnote{Kitab-i-Aqdas \#54} The water is
created pure, just as we are. Yet, when God discusses our purity with
lewdness and oppression, He says ``beware that you do not corrupt the
Earth after it has been reformed.''\footnote{Kitab-i-Aqdas \#64} What if
one can be lewd and oppressive to water, which causes it to become
impure? If oppression defiles the heart, pollution defiles the spring.
Both are violations of purity, and both begin with heedlessness. The Bab
describes the four elements of fire, air, water, and earth as purifying
agents\footnote{Persian Bayan Vahid 5 Gate 14} and are integrated within
the Bayanic Calendar.\footnote{Persian Bayan Vahid 5 Gate 3} Chapter
12.7 includes these elements in the table describing the calendar.

Creation is where we recognize the names of God and develop our virtues,
a corrupted Earth would greatly increase the difficulty in achieving
this spiritual progress. Thus, we have a moral ecology in which we
operate.

\subsection{Moral Ecology}\label{moral-ecology}

Moral ecology can refer to the pragmatic evolution of ethics over time
as if it were an ecosystem. It can also refer to the study of human
interactions with the natural environment and the ethics of these
interactions. Much like the water example, both of these definitions
seem to be rather integrated. The environment shapes our morals, and our
morals help shape the environment. None can actually ever be divorced
from the other.

In numerous instances, Baha'u'llah describes Earth, or parts of the
Earth, as having feelings, expressing feelings, and having spiritual
experiences. Usually this is used as metaphor to describe feelings,
expressions, and spiritual experiences people should be having. What if
Earth does have feelings, expresses those feelings, and have spiritual
experiences? ``Every stone and tree has cried out with the loudest
voice, Bethlehem has been moved by the breath of God, and the trembling
of reunion has seized Mount Sinai.\footnote{The Lawh-i-Aqdas (Most Holy
  Tablet)} The Land of T (Tehran) is instructed not to grieve the
injustice perpetuated upon it.\footnote{Kitab-i-Aqdas \#91} The Land of
Ba rejoiced when Baha'u'llah arrived after being released from the land
of prison, which the land of prison became saddened.\footnote{Lawh-i-Ard-i-Ba
  (Tablet of the Land of Ba)} Flowers, fruits, trees, leaves, and rivers
are pleasing examples of divine power and craftsmanship in honor of the
passing of a believer.\footnote{BH00010} In the same tablet He says the
trees, stones, clay, and pebbles serve as witnesses.

Every instance we observe Baha'u'llah's metaphors and descriptions of
the earthly creation, maybe we should also consider the idea Baha'u'llah
is describing that which is real in metaphorical ways. When He says
``plant nothing in the garden of the heart except the flower of
love''\footnote{The Hidden Words of Persian \#2} maybe we have two
things to consider. The obvious is the heart should have love inside of
it, so that love is what courses through our body and is able to be
freely expressed outwardly, seen and admired by those who are able to
observe this flower of love in the garden of the heart. The less obvious
might be we should sometimes plant flowers out of love for the garden
itself. Why not both? The flower we plant may inspire love by a person
who witnesses it, whether the garden is in our heart or made from the
Earth we were created from, and the Earth we will return to.

Trusteeship within moral ecology would be to view the ecology of the
Earth, its minerals, its air, its water, its living things as an
integral part of who we are. We witness it, we preside over it as
trustees instead of as masters, and serve to help it thrive. When we are
instructed to build up the lands and cities, this is definitely not an
act of destruction. It is an act of construction, but in a way which is
reverent to the idea the Earth is sacred, our provision for ourselves
and future generations must be provided, and we are not the only things
on Earth which has feelings and spiritual journeys. Imagine when the
next future Manifestation of God appears hundreds of years from now.
Will they witness an Earth and people who are more purified than when
Baha'u'llah lived among us?

The Bab even included a prohibition from buying or selling the four
elements of fire, air, water, and earth.\footnote{The Arabic Bayan Vahid
  9, Gate 10} I do not see this prohibition continued by Baha'u'llah,
nor do I see Baha'u'llah expressing a permission. In consideration of
what the Bab intended, how might we adapt our economies to be mindful of
how we use those four purifying elements in commerce? I personally
witness how water is extracted from aquifers which cannot be
replenished, bottled into plastic, and the plastic is thrown back into
surface water. The profit margins of bottled water bring large,
multinational corporations significant profits, yet it comes at
potentially significant long-term externalized costs elsewhere. It may
not reflect God's vision of justice for a few to profit at the cost of
many. A trusteeship of these four elements could develop methods to
account for externalized costs, or find ways for markets which believers
operate in to guide the Bab's vision at various levels. Helping would
find ways to ensure extraction does not exclude giving, that there are
pathways to achieving a true accounting balance, tied to the virtue of
moderation.

\subsection{Animals}\label{animals}

Baha'u'llah does tell us not to load an animal with more than it can
bear.\footnote{Kitab-i-Aqdas \#187} He associates justice and fairness
between the heavens and Earth. Loading an animal would explicitly mean
examples such as ensuring a pack animal is not carrying excessive weight
for an excessive duration. This also implies the burden is undue
suffering, physical pain, and in this example, cruelty. When I consider
other ways living creatures can suffer or face physical pain due to
human activity, I also consider habitat loss where animals lose their
shelter, food, and water. I consider pollution which makes a creature
sick, disoriented, or in places like Chernobyl, mutated.

Baha'u'llah also says we should not be excessive in hunting\footnote{Kitab-i-Aqdas
  \#60} for prey. Again, this is an act of provision without excess.
This again is associated with justice and fairness. If animals are used
to hunt, the names of God should be mentioned. If we are taking the life
for provision, it comes with a spiritual act of reverence towards that
which is lost, that which was provided by God. Should we hunt for sport
and entertainment? If so, how does this reflect the names of God? Should
we kill anything which attempts to eat our food sources? Should we use
pesticides to spray yards, gardens, and fields and killing every insect
which ventures onto that land? If we are viewing Earth as our dominion,
do we have the power to do anything we wish for entertainment and
comfort, or do we have a role as trustees? These are all types of
questions we should consult on as we build up these lands and cities for
the Cause of God.

In considering the witness, trustee, and helper model, we can be guided
by the virtues of moderation, courtesy, and thankfulness. These can
ensure that for anything we take, we are willing to give back something
equal or more to that which we have taken from.

\section{Wills and Testaments - Houses of Justice as
Trustees}\label{wills-and-testaments---houses-of-justice-as-trustees}

Every soul has been ordained to write a will.\footnote{Kitab-i-Aqdas
  \#109} The will must have certain features:

\begin{enumerate}
\def\labelenumi{\arabic{enumi}.}
\item
  Heading with the Greatest Name,
\item
  Declaration of the oneness of God in the Manifestation of His
  appearance,
\item
  Good deeds they wish to be remembered for,
\item
  How they wish to be buried, and
\item
  The distribution of inheritance according to their wishes.
\end{enumerate}

A will and testament is a legal contract, to be executed upon the death
of a person, who is known as the testator. They entrust the Houses of
Justice to act as the trustee of the contract, also known as the
executor of the will. A testator can designate another person to
execute, but the Houses of Justice still have the final responsibility
to ensure the contract is fulfilled. If a will is not made prior to a
person's death, then the local House of Justice will act according to
the Kitab-i-Aqdas and a few amendments to the law afterwards.

When distributing the estate, there is an order of priorities. The
funeral expenses are to be paid first, then any outstanding debts. If
any money remains, the payment of Huquq'u'llah for any portions of
wealth Huquq'u'llah was not paid on. If there is any money after
Huquq'u'llah, then the estate will be distributed as
inheritance.\footnote{BH00984}

\subsection{Inheritance Without a
Will}\label{inheritance-without-a-will}

The Arabic Bayan introduces the inheritance distribution with 2520
shares divided between seven categories (Books) of
inheritors.\footnote{Arabic Bayan Vahid 10, Gate 14} Baha'u'llah kept
this exactly the same in the Kitab-i-Aqdas but later changed it in
BH01964. Below is the distribution. I listed the categories as described
in the Bayan and Kitab-i-Aqdas, with the original number struck out. The
changes from BH01964 are added or subtracted with the final total and
percentage calculated.

Split from the Letter Z (Number 7)

\begin{enumerate}
\def\labelenumi{\Roman{enumi}.}
\tightlist
\item
  Descendants - Book Ṭ (9) according to number of M, Q, T -- \st{540} +
  540 = 1080 (42.8\%)
\item
  Spouses - Book Ḥ (8) according to number T and F -- \st{480} - 90 =
  390 (15.5\%)
\item
  Fathers - Book Z (7) according to number T and K -- \st{420} - 90 =
  330 (13.1\%)
\item
  Mothers - Book W (6) according to number R, F, Y, A -- \st{360} - 90 =
  270 (10.7\%)
\item
  Brothers - Book H (5) according to number Sh -- \st{300} - 90 = 210
  (8.3\%)
\item
  Sisters - Book D (4) according to number R and M -- \st{240} - 90 =
  150 (6.0\%)
\item
  Teachers - Book J (3) according to number Q and F -- \st{180} - 90 =
  90 (3.6\%)
\end{enumerate}

\subsection{Items Excluded From Sale}\label{items-excluded-from-sale}

If the house is inhabited by descendants, the male descendants inherit
the house.\footnote{Kitab-i-Aqdas \#25}

Specific garments are to be given to the male descendants. This may not
exactly mean common clothing defined by gender roles. Instead, this
could mean garments which have a significant meaning or purpose within
the family. Ceremonial, spiritual, and culturally significant garments
would be included. This also implies these types of garments are not to
be sold as part of the estate, and remain a part of the family. An
example of how a ceremonial garment might be designed could take
inspiration from the Persian Bayan Vahid 5, Gate 10.

\subsection{When an Inheritor Does Not
Exist}\label{when-an-inheritor-does-not-exist}

It might not be guaranteed all seven categories of inheritors were born
or are still alive when the deceased passes away. The Kitab-i-Aqdas
provides guidance in these situations.

If there are no descendants, their portion will go to the House of
Justice.\footnote{Kitab-i-Aqdas \#21} (1080 of 2520 shares)

If the deceased has descendants but no other specified heirs, 2/3rds
will go to the descendants and 1/3rd will go to the House of
Justice.\footnote{Kitab-i-Aqdas \#22} For example, if there is no
spouse, 260 of 2520 shares will be added to the descendants for a total
of 1340 shares. 130 shares will go to the House of Justice.

If there are no direct heirs from the first 6 categories but there are
nephews, nieces, or their children, 2/3rds will go to them and 1/3rd to
the House of Justice.\footnote{Kitab-i-Aqdas \#23}

If there are no direct heirs and no nephews, nieces, or their children,
the inheritance from the first 6 categories will go to the House of
Justice.\footnote{Kitab-i-Aqdas \#24}

\subsection{The Funeral}\label{the-funeral}

The funeral is mostly derived from the Persian Bayan Vahid 5, Gate 12
and Vahid 8, Gate 11 with some adjustments in the Kitab-i-Aqdas
\#128-130. The Bayan describes the cleansing and burial process as
preparing the temple of the body for purification at the return of all
things (resurrection). It is a process which much be done with the
deepest honor of the person who passed. The following are the steps for
the full funeral.

\subsubsection{Preparation}\label{preparation}

\begin{enumerate}
\def\labelenumi{\arabic{enumi}.}
\tightlist
\item
  Handle the body with dignity and stillness.
\item
  Repeat the six Names of God---or ``God'' alone---from the moment of
  death until burial.
\item
  Prepare pure water, optionally mixed with camphor and lotus leaves.
\item
  Ensure the washing is performed by the righteous.
\end{enumerate}

\subsubsection{Washing of the Body}\label{washing-of-the-body}

\begin{enumerate}
\def\labelenumi{\arabic{enumi}.}
\setcounter{enumi}{4}
\tightlist
\item
  Wash the head --- ``O Singular One.''
\item
  Wash the abdomen --- ``O Living One.''
\item
  Wash the right side --- ``O Self-Subsisting One.''
\item
  Wash the left side --- ``O Wise One.''
\item
  Wash the right foot --- ``O Just One.''
\item
  Wash the left foot --- ``O Powerful One.''
\item
  Perform one washing; up to three or five permitted.
\item
  Use water (warm or cool) suitable to the condition of the body.
\item
  After washing, perfume the body with fresh fragrance.
\end{enumerate}

\subsubsection{Shrouding}\label{shrouding}

\begin{enumerate}
\def\labelenumi{\arabic{enumi}.}
\setcounter{enumi}{13}
\tightlist
\item
  Shroud the body in five layers of silk or fine cotton.
\item
  Up to nineteen Names of God may be inscribed on the shroud.
\item
  Wrap the body with calm and reverence.
\end{enumerate}

\subsubsection{The Ring}\label{the-ring}

\begin{enumerate}
\def\labelenumi{\arabic{enumi}.}
\setcounter{enumi}{16}
\tightlist
\item
  Place on the right hand:
\end{enumerate}

Men: ``To God belongs whatever is in the heavens and the earth and what
is between them, and God is All-Knowing of all things.'' Women: ``To God
belongs the dominion of the heavens and the earth and what is between
them, and God is over all things Powerful.''

\subsubsection{The Coffin}\label{the-coffin}

\begin{enumerate}
\def\labelenumi{\arabic{enumi}.}
\setcounter{enumi}{17}
\tightlist
\item
  Place the body in a coffin of crystal, stone resistant to decay, or
  fine, hard wood.
\item
  Place fragrance within the coffin before closing it.
\end{enumerate}

\subsubsection{The Funeral Prayer (Ṣalát
al-Janázah)}\label{the-funeral-prayer-ux1e63aluxe1t-al-januxe1zah}

\begin{enumerate}
\def\labelenumi{\arabic{enumi}.}
\setcounter{enumi}{19}
\tightlist
\item
  The congregation stands; no bowing or prostration is performed.
\item
  The prayer is said once only, facing toward God.
\end{enumerate}

\begin{quote}
O my God, this is Your servant (maidservant) and the son (daughter) of
Your servant who has believed in You and Your signs, and has turned
towards You, detached from all else but You. You are indeed the Most
Merciful of the merciful. I beseech You, O Forgiver of sins and
Concealer of faults, to deal with him in accordance with Your heavenly
grace and ocean of bounty, and to admit him into the shelter of Your
supreme mercy, which has preceded the earth and the heavens. There is no
God but You, the Forgiving, the Generous.

Then recite:

\begin{enumerate}
\def\labelenumi{\arabic{enumi}.}
\tightlist
\item
  ``Allah-u-Abhá'', then ``We all, verily, worship God'' --- 19 times.
\item
  ``Allah-u-Abhá'', then ``We all, verily, bow down to God'' --- 19
  times.
\item
  ``Allah-u-Abhá'', then ``We all, verily, are devoted to God'' --- 19
  times.
\item
  ``Allah-u-Abhá'', then ``We all, verily, remember God'' --- 19 times.
\item
  ``Allah-u-Abhá'', then ``We all, verily, are grateful to God'' --- 19
  times.
\item
  ``Allah-u-Abhá'', then ``We all, verily, are patient for God'' --- 19
  times.
\end{enumerate}
\end{quote}

\subsubsection{Burial}\label{burial}

\begin{enumerate}
\def\labelenumi{\arabic{enumi}.}
\setcounter{enumi}{21}
\tightlist
\item
  Following the prayer, the coffin is carried reverently to the grave.
\item
  Bury the body within one hour's distance from the place of death.
\item
  Bury the body with spirit and fragrance in a nearby place.
\item
  As the body is lowered, one may say:
\end{enumerate}

\begin{quote}
``I began from God and returned to Him, detached from all else, and
clinging to His Name, the Most Merciful, the Most Compassionate.''
\end{quote}

\begin{enumerate}
\def\labelenumi{\arabic{enumi}.}
\setcounter{enumi}{25}
\tightlist
\item
  Maintain silence and remembrance until the burial is complete.
\end{enumerate}

\section{Closing Thoughts About
Trusteeship}\label{closing-thoughts-about-trusteeship}

There are many aspects of life which could be covered regarding
trusteeship, especially within our social lives and relationships. I
hope covering these major themes helps provide a vision and framework in
how to apply these ideas in the various complicated situations which
arise in life. Trusteeship, as when it is applied to the raising of
children in relation to the Bayanic Calendar, is an enriching way to add
to the practice of Honoring God.

The key foundation of trusteeship, outside of viewing it as a spiritual
or legal obligation, is the act of consultation. Everything regarding
trusteeship, marriage, and other important matters requires the ability
to consult. The last chapter of Part 3 is going to discuss consultation.
With unity being the bridge between spiritual practices and virtue
development and our rights and responsibilities, consultation is the
bridge between unity and trusteeship. Consultation is the practical
application of unity. Consultation is the engine that drives witnessing,
the mechanism that enacts trusteeship, and the means through which we
learn how best to help.

\newpage{}

\chapter{Consultation}\label{consultation}

In Chapter 15, we learned the close association of consultation with
spiritual maturity. Baha'u'llah said to ``cling to the cord of
consultation.'' Chapter 14 describes a right to consultation as an
opposing force to oppression. Consultation is also a key part of
affectionate relationships and trusteeship. Baha'u'llah describes
consultation, when combined with compassion, are two radiant lights in
the sky of wisdom.\footnote{Lawh-i-Ishraqat (Tablet of Illumination)}

\section{Types of Consultation}\label{types-of-consultation}

There are two types of consultation. One is when we refer a matter to an
expert. The trained professional is capable on consulting their
expertise to be a trustee to whom sought their expertise. They are
skilled in being a witness of the problem and in providing the necessary
help. In an example regarding seeking medical care, we are commanded to
heed a doctor's prescriptions and orders.\footnote{BH00093} We are to
seek expertise where we lack expertise, and follow the expert's commands
in the subject they are an expert in. This is the same as following God.
The only condition is the expert must actually be skilled. Standards in
licensing, education, and other professional certifications can help a
person discern who is an expert and how is not. We cannot follow every
claimant.

The second type of consultation is among a group of peers. The
Kitab-i-Aqdas mentions consultation only once, but it is directly in
regards to trusteeship.

\begin{quote}
God has ordained that in every city a House of Justice should be
established, where individuals shall gather in the number of Bahá (9),
or more if desired. They should regard themselves as entering the
presence of the Exalted One, observing the unseen. These members must be
the trusted ones of the Merciful among people and the representatives of
God for all who dwell on earth. They should~consult~on the welfare of
the servants of God for His sake, just as they~consult~on their own
affairs, and choose what is best.\footnote{Kitab-i-Aqdas \#30}
\end{quote}

The Houses of Justice are to be the exemplars of trusteeship and
consultation, but consultation is a binding command for all.\footnote{BH00046}
There is no power except through unity, and no well-being except through
consultation.\footnote{BH00083} Consultation is necessary in all
matters.\footnote{BH00083} It increases awareness and turns conjecture
into certitude.\footnote{BH00123} It is the cause and means of
vigilance, welfare, and safety.\footnote{BH00083}

The Holy Spirit confirms consultation.\footnote{BH00040} Consultation is
such an act of power, Baha'u'llah even allowed acts of consultation to
override his directives to community members when He was acting as
administrator to the communities, such as which city to travel
next.\footnote{BH00093} Consultation does not override revelation from
God, and works within the framework and boundaries provided to us.

All power mechanisms after Baha'u'llah are subordinate to consultation,
no matter who has power and no matter the context of power. All
consultation is guided by God and affirmed as part of the Cause of God.

\section{Consultation Process}\label{consultation-process}

For the Houses of Justice, there was no specific consultation process
outlined. For the general consultation for everyone else, there are a
few guidelines Baha'u'llah provided.

\begin{quote}
As for what you wrote concerning consultation, after presentation they
said that if disagreement arises among the first group assembled, new
souls should be added and then by drawing lots select the number of the
Greatest Name, or less or more than that, and consult anew. Whatever
emerges therefrom is to be obeyed. And if disagreement still persists,
``adorn the two with a third and take the strongest.'' Verily, He
guideth whom He willeth to the straight path. Thus hath the Lord of all
beings decreed in this station.\footnote{BH00059}
\end{quote}

When looking at this paragraph, there is no minimum number of people to
consult. It could be between 2 people or more. Consultation does not
need to be with a House of Justice. Since all matters are to be
consulted upon with compassion, it would not be physically possible for
a House of Justice to be involved in each one. This social
responsibility is for all, and all have a right to consultation in
matters which affect them.

Ideally, the consultation goes well, a solution is agreed upon, and a
course of action is taken. If the consultation ends in disagreement,
then an unspecified number of random people are added to the
consultation, chosen by lot. Drawing lots is completely random, such as
putting names in a box and picking from the names. This helps bring
fresh perspectives and more knowledge. If the consultation is
successful, it must be obeyed.

Yet, if there is still disagreement, there is one more process. Adorning
the two would refer to the two strongest remaining proposals from the
prior two rounds of consultation. This could be two people, two ideas,
two hypothesis, or two premises. The third might be a mediator
introducing a third path or a different perspective applicable. The
third could be the prior consultative group taking the two positions and
discovering a middle path. Basically the third represents a stage of
evaluation and refinement, a synthesis which extracts the highest
quality result from the prior divergence.

Baha'u'llah does not necessarily express a majority rules framework. I
could see a community adopting a majority rules perspective in some
consultations, but there might be other factors to consider. Truth,
justice, unity, and compassion might be more important than a simple
majority. As Baha'u'llah says, consultation brings certitude, welfare,
and safety to us through a depth of understanding. It is more than
counting votes.

To evaluate consultation, I want to take a pass of the witness, trustee,
and helper model through some of the prior concepts of this book.

\section{Matters to Consult On}\label{matters-to-consult-on}

The following is a list of some potential subjects we could consult on:

\textbf{Personal and Ethical} -- conscience, health, vocation,
discipline, temperament, friendship, conduct, aspiration, repentance

\textbf{Family and Household} -- marriage, parenting, inheritance,
education, shelter, nutrition, caregiving, celebration, mourning

\textbf{Community and Social} -- service, fellowship, conflict, culture,
recreation, safety, hospitality, communication, reputation

\textbf{Professional and Vocational} -- ethics, training, mentorship,
innovation, employment, leadership, compensation, scheduling, evaluation

\textbf{Economic and Financial} -- trade, investment, taxation, charity,
property, production, distribution, consumption, stewardship

\textbf{Religious and Spiritual} -- worship, doctrine, pilgrimage,
ritual, study, repentance, translation, guidance, devotion

\textbf{Civil and Political Governance} -- legislation, justice,
security, planning, infrastructure, diplomacy, representation, welfare,
education

\section{Witnessing}\label{witnessing}

In consultation, witnessing is the first role. To be a witness is
directly tied to our capacities and orientation of a soul as a mirror of
God. Witnessing is tied directly to our belief in God's names, as these
names reflect within our souls and illuminates how we see truth. No
person reflects all of God's names perfectly, and some names may appear
in a soul more brightly than others. For example, a person may primarily
reflect the name of Power more than Knowledge and another in the counsel
may reflect Love more than Justice. Each person will witness
differently, even if each person who witnesses are functionally equal in
the consultation. Even if a person does not believe in God, they are
capable of reflecting God's names.

With the mirrors illuminating truth in their own capacities, we describe
what we observe the best we can. These observations include facts and
how we feel about these facts. Opinions are equally important as facts,
as opinions are true to the person who has them. We listen to how
another in the counsel describes their own truth and observations. The
mirrors of the names of God, when acting together, can merge their
lights together. This illuminates the truth with a more radiant light
which includes all Names each reflect. If Knowledge was missing with one
person, another in the counsel might shine that light. We might not see
all names of God as we serve as witness, but the chances to witness with
an increasing number of names is much greater than if a person were to
witness alone.

Witnessing requires the ability to ask questions. If a person feels they
only have answers or their testimony is the only valid testimony, it
will show by the inability of the person to ask questions. Consultation
requires a perspective you can learn from another, even if you disagree
with their perspective or observation. This is because to consult, each
person must be willing to view their collective testimony as a single
mirror illuminating by the Names of God, not merely illuminating with
their own birth name.

\section{Trustee}\label{trustee}

Consultation requires us to view us a trustees. We are trustees of what
we witness, curated by our spiritual practices. These spiritual
practices develop our innate and emergent virtues which govern how we
treat our collective testimony. Much as the Holy Spirit moves through
spiritual practice, It also moves through our constellations of virtue
which guide our movements.

Every person has developed their virtues and is currently in the process
of developing their virtues. This process needs to be respected. For
example, one might be guided most strongly by humility and sincerity and
another might be guided most strongly by righteousness and dignity. When
combined together, we have four guiding lights instead of two each.

We are also trustees of truth. We treat what is witnessed as sacred, and
we allow our spiritual practice as sacred to the process. This does not
necessarily mean the counsel pray in the moment, but to allow our
prayers, remembrance, recitation, reflection, and honoring God to also
act in trust of our consultation. How does our conversations with God
move us in this situation? What does God's Word say? Are there laws and
counsels available which directly address our situation? How have my
experiences influenced what I observed? Do I view the others experiences
as relevant as mine? Does the situation affect how we honor God
together, such as in a later festival? There are many ways we can be
trustees of our spiritual practice, and to allow our spiritual practice
serve as trustees to consultation. Allow the Holy Spirit to move through
you and the counsel. The Holy Spirit animates your practice and virtues.

\section{Helper}\label{helper}

Consultation requires us to be able to help each other during
consultation and after consultation. Whether in agreement or
disagreement, the counsel needs to ensure all individuals involved and
anyone affected are supported and encouraged. As consultation operates
within the social life, it is the key instrument to the Cause of God.
The result of all consultation must be treated as fruits of the Cause of
God. Fruits not only nurture, but they sew seeds for future trees and in
theory, an exponentially increasing amount of fruit. Fruits will only
emerge through help.

In helping, we might need to adapt how we viewed our role in the
situation or adapt our understanding of what we witnessed. We may have
to understand a prior result of consultation may not apply to every
situation, so the fruits of consultation could vary in a case-by-case
basis. We need to be able to affirm the positive aspect of each person's
role and where difficulties arise, be willing to serve the person
struggling with the consultation itself or the situation the
consultation is addressing.

Each person has their own capacities to help. One might be adept at
referencing Baha'u'llah's words, another might have a well-developed
empathy. One might have skills to make a task easier, and another might
have resources available to ease a burden. Help is additive and
potentially multiplicative, being greater than the sum of its parts.

Finally, help does not equal coercion even in disagreement. Any concern
in disagreement needs to be viewed as legitimate and addressed as best
as possible. Consultation cannot be effective if it is missing
compassion. Consultation, when well-assisted by the counsel and people
of the community, is a process which renews unity. If the constellation
of virtues is unity, how these constellations guide a counsel is also
also unity.

\section{Conclusion}\label{conclusion-2}

I believe consultation, as confirmed by the Holy Spirit, the animating
extension of revelation. It keeps the Book living and is the continual
process which will liberate us. When we approach consultation as a
divine process, we are utilizing all the skills we have learned through
the Kitab-i-Aqdas itself. All souls will achieve their greatest degree
of liberty through this process.

This concludes Part 3 of this book. The fourth and final part of this
Book will progress to the Houses of Justice and how Baha'u'llah
envisioned the period of time after Him and before the appearance of the
next Manifestation of God. What is the vision after 1892 for the next
1,000+ years?

\newpage{}

\part{Part 4: Leadership After Baha'u'llah}

\chapter{Houses of Justice}\label{houses-of-justice}

\section{Introduction}\label{introduction-7}

The Kitab-i-Aqdas \#30 says:

\begin{quote}
God has ordained that in every city a House of Justice should be
established, where individuals shall gather in the number of Bahá (9),
or more if desired. They should regard themselves as entering the
presence of the Exalted One, observing the unseen. These members must be
the trusted ones of the Merciful among people and the representatives of
God for all who dwell on earth. They should consult on the welfare of
the servants of God for His sake, just as they consult on their own
affairs, and choose what is best. Thus has your Lord, the Mighty, the
Forgiving, decreed. Beware not to neglect what is explicitly stated in
the Tablet. Fear God, O people of insight.
\end{quote}

This chapter is the beginning of Part 4: Leadership After Baha'u'llah.
Baha'u'llah extensively discussed spiritual leadership and political
governance throughout His writings. The Houses of Justice are the new
institution Baha'u'llah created which serves as a link between both
spheres. We will discuss who may serve on a House of Justice, their
roles, and how they are the foundation of a distinct governance model
designed to liberate everyone within their jurisdiction. It should be
noted I may refer to the singular House of Justice instead of the plural
Houses at times. I view the singular House to be the name of the
institution, while many Houses comprise the institution.

\section{The Selection Process}\label{the-selection-process}

Baha'u'llah instructed a Council of Consultation to be formed for the
appointment of trustees at a time when the formation of a House of
Justice does not cause injustice upon the trustees or those they
serve.\footnote{BH06839} Baha'u'llah refrained from appointing anyone
Himself. The time for appointment was too early due to kindled fire
within the lands. He says ``if today in various cities institutions
specifically known by the name of House of Justice and the like become
known, there is danger for all, as the people are immature. Leave them
until their backs are strengthened and they attain their
maturity.''\footnote{BH04495}

The Council's results are confirmed by the signatures of believers of
the lands. This suggests there could be a separation in duties. The
trustees are different than the Council of Consultation, who are
different than the signatories of the results. The signatories would act
as a notary and witness, testifying to the integrity of the selection
process. The Council does not need to function through election,
although that could be an option. The consultation would follow the
rules set forth in Chapter 21, which would be guided by the Holy Spirit.
Baha'u'llah reminds the Council to ``let them hold fast unto whatsoever
leads to exaltation, elevation, dignity, composure, goodly deeds,
spiritual qualities, words of counsel, and the reformation of heedless
souls, according to the requirements of the days.''

The Council process does not need to be rigid, as the consultation needs
to be according to the requirements of the days. What might be relevant
in 2026 may not be relevant five hundred years later. These Councils and
the Houses of Justice act ``until God shall come with His command.''
Once the next Manifestation of God appears, this process ends.

In the Kitab-i-Aqdas and a few other tablets, Baha'u'llah does
specifically say men serve on the Houses of Justice. There are some
people who say only men can serve on the Houses of Justice due to how
gender language is used. In a letter contained in BH00158 addressed to
Narjis and Sakinih Khatun, he tells them any women who have partaken of
the Choice Sealed Wine (The Kitab-i-Aqdas) are men, knights of the
field. He warns of those who are deprived of certitude due to vain
imaginings (illusions), while women such as them bring the light of
certitude. BH00158 provides many statements regarding the station of
women.

Thus, gendered terminology does not limit spiritual capacity. The
measure of one's fitness for the House is not outward form but inward
reality. Those who have drunk from the Choice Sealed Wine are men in
spirit. They are steadfast, detached, and radiant in faith. The Council
must therefore discern the measure of this inner knightliness when
selecting members, whether male or female in body.

\section{Roles of the Houses of
Justice}\label{roles-of-the-houses-of-justice}

The Houses of Justice are to serve various interlinked roles. They
include being trustees of the Merciful,\footnote{Kitab-i-Aqdas \#21} the
representatives of God,\footnote{Kitab-i-Aqdas \#30} the shepherds of
the sheep of God,\footnote{Kitab-i-Aqdas \#52} and the dawning places of
His command.\footnote{Tablet of Splendors, 8th Illumination} These roles
are broad in scope but they are also quite well-defined. In further
sections, I will clarify each of the four roles.

In each role, there is no room to expand the scope beyond the limits
what Baha'u'llah stated. There is no room to add additional roles,
either. In the Tablet of the Houses of Justice, Baha'u'llah tells the
members of the Houses of Justice to ``be mindful not to act contrary to
what has been revealed in the divine verses in this mighty, eternal
Manifestation, for whatever the True One---exalted is His station---has
decreed is indeed what is best for the servants.'' He includes another
restriction; religious practices should follow what God has revealed in
His Book.

He says in the same tablet these roles exist with the authority of the
people. This means a House of Justice cannot be formed without the
authority of the people, as we saw with the selection process. This also
means the people must consult on what roles the House of Justice will
serve. The House of Justice does not establish its own authority. In
receiving the authority of the people, Baha'u'llah offers a first
priority, followed by the other matters they should consult on. I
interpret this to be a potential ranking, or a progression of the roles
the House of Justice may take on over time.\footnote{Tablet of
  Splendors, 8th Illumination} The next sections will discuss five
separate authorities derived from the Tablet of the House of Justice,
the four roles of the Houses of Justice in fulfilling what they have
been authorized to do, and the progressive nature of these authorities.

\subsection{Authority \#1: The Propagation of the Cause of
God}\label{authority-1-the-propagation-of-the-cause-of-god}

The first authority is the propagation of the Cause of God. If we were
to assign a virtue, this would be the illumination phase of the House of
Justice. The members of the House of Justice shine as one Mirror,
manifesting divine light through their service. The Cause of God would
include all aspects of a person within their social life, their
affectionate relationships, and the various roles they fulfill. It also
includes all of the preparatory steps to enhance the public sphere, such
as the belief in God, the soul, the spiritual worlds, the Day of
Resurrection, the Manifestations of God, the spiritual practices, the
development of virtues, the recognition of rights, the ownership of
responsibilities, and refinement within our personal lives.

To propagate means to reproduce, multiply, and to be fruitful. There are
two main ways the Cause of God can expand. The first is for the Cause to
be newly embraced by others whose conscious has been awakened. The
second is for the Cause to inspire a healthy family life, where children
are born to parents who holistically embrace the Cause. Both pathways
would fall under the first authority of the House of Justice.

Role \#1 Trustees of the Merciful: The foundation is for the trustees to
consult on the welfare of others, just as they consult on their
own.\footnote{Kitab-i-Aqdas \#30} This is where the witness, trustee,
and helper model is first institutionalized. They see where people are
struggling and with what resources they do have, consult on a solution,
and help. To assist, they manage endowments dedicated to charitable
purposes.\footnote{Kitab-i-Aqdas \#42} These charitable purposes can
include being used towards elevated places of the Cause, such as the
maintenance of the pilgrimage sites in Shiraz and Baghdad and
Mashriq'ul-Adhkars. The trustees elevate the Cause, but not themselves.
Finally, these endowments could be used to safeguard Baha'u'llah's Word,
promote translation efforts, and publish Baha'u'llah's Word.

Role \#2 Representatives of God: The foundation is to be an advocate on
behalf of God, with full accountability for their actions and decisions.
This would be the public-facing role. An example would be to share the
nine Illuminations from the Tablet of Ishraqat to leaders, organize or
support the organization of public dialogue focused on the teachings of
Baha'u'llah in relevant context. Representatives act as exemplars,
actively demonstrating through their lives how to live the Cause.
Finally, they show full accountability. Maybe they provide transcripts
of their consultations and public accounting of the endowments they
receive and how endowments are spent.

Role \#3 Shepherds of the sheep of God: As shepherds, they help guide
the community through consultation. They protect by ensuring the
propagation does not bring harm, Baha'u'llah's Word is not altered, nor
that anyone's own words do not become equal to or greater than
Baha'u'llah's. They promote an environment where affectionate
relationships are encouraged and supported. They encourage community
members to consult together and have a healthy social life. Protection
does not equal policing, but by being vigilant.

Role \#4 Dawning places of His command: This role is not about
enforcement, legislation, or membership as propagation can only be had
through invitation, not coercion. To be dawning places of His command,
think of the role of the Mashriq'ul-Adhkar in the life of a community.
They serve as the dawning places of the remembrance of God. To be a
dawning place, one must be seen as a symbol of God's light rising from
the horizon in fellowship. A House of Justice could lead or support
festivals, holy days, remembrance services, or monthly hospitality. They
celebrate marriages and births, without officiating over them. They help
inspire artistic endeavors or scholarly research into Baha'u'llah's
teachings, sciences, and other forms of knowledge which shed light on
truth.

There are many ways a House of Justice can consult on and help propagate
the Cause of God. The most important aspect is those who are chosen to
serve must be diligent in their virtues, and understand the community
does not exist to serve them. They exist to serve the community. The
authority is given to the House through the people's trust. Should that
trust be withdrawn through consultation by the people, the House's
mandate ceases. As long the House consults in the spirit Baha'u'llah
ordained, it is confirmed by the Holy Spirit. Thus, the people and the
House are co-trustees of one Cause, each guarding the other from
injustice.

When propagation has been successful to a degree the people are pleased
with, the people may choose to consult on whether to give the House its
second authority.

\subsection{Authority \#2: The Morals of
Souls}\label{authority-2-the-morals-of-souls}

The second authority is the morals of souls. Where authority one was
focused on illumination through propagation, the House of Justice will
add refinement to its responsibilities. They will nurture virtue and
conscience to assist inner transformation of the individual and
transformation of the communities.

Role \#1 Trustees of the Merciful: They view morality as a trust, but it
is not imposed. The members of the House of Justice offer moral
companionship. We should always be reminded the members of the House of
Justice are also on their own spiritual journey. While selected for
their leadership in how they express virtue and good deeds, they are
still striving for moral improvement. Being being entrusted to morals,
they are more importantly entrusted to souls. The House of Justice
consults in how to educate, mentor, and counsel youth, families, and
adults of all ages. Voluntary programs might be established for moral
education or developing spiritual habits.

Role \#2 Representatives of God: The House members are the moral
exemplars, embodying the morality of souls publicly. When they consult
on a matter of ethics or morality, not only do they record their
consultation, but do so in a way which can be learned from by anyone who
reads or watches. They may participate in local or civic dialogues on
ethical or moral issues raised in public life, and do so humbly and
gently. Through them, the public is aware God is active.

Role \#3 Shepherds of the sheep of God: They consult on what nearby
services are available to help others overcome moral struggles. However,
services can never be imposed. Companionship is sharing a moral journey
together. The members are not judges in this authority, but are Mirrors
reflecting the names of God until others can see their own soulful
nobility. Members offer consultation to help others learn how to repent
and to forgive helps relationships move forward. The House may mediate
moral disputes and guard against gossip or public humiliation.

Role \#4 Dawning places of His command: The House consults on ways to
transform Baha'u'llah's Word into the moral fabric of culture. Moral
transformation is not viewed as austerity or deprivation, but as a
radiant beauty all are attracted to. They support the development of
arts, literature, and scholarship that ennoble human character and the
development of the constellation of virtues. The members reinforce the
connection between spiritual practice and virtues. Finally, if a
Mashriq'ul-Adhkar has not been established, the House would need to find
a pathway for its establishment. The Mashriq'ul-Adhkar is the foundation
of the devotional life of the community, where remembrance of God
inspires the culture.

When the moral character of the community is sufficiently reflected in
their deeds and well-being, the people may decide to consult on granting
the House of Justice the third authority.

\subsection{Authority \#3: The Preservation of
Honor}\label{authority-3-the-preservation-of-honor}

The third authority is the preservation of honor. From here, the House
moves past refinement towards the protection of human dignity, the
protection of a soul created noble. We take the authorities of
propagation, which teaches what the Cause of God is, and the authorities
of refining morals, which transforms the people who believe, and apply
them towards social action.

Preserving honor would be focused on taking steps and measures to
protect the rights of all people, and to help heal those whose rights
have been violated.

Role \#1 Trustees of the Merciful: The honor of a person is the trust,
and the House of Justice is the trustee. Baha'u'llah's instruction to
consult on what benefits others as if they are consulting on what
benefits each individual member is the foundation of this trusteeship.
They witness the right of a person, without regarding whether the person
is a believer of Baha'u'llah or not, and considers the best remedy.

Role \#2 Representatives of God: The members of the House are the
representatives of honor. They may choose to host or attend forums,
media, and other public-facing outlets which emphasize the preservation
of honor. The House may assist or sponsor those, whether or not they are
believers, who are actively working to preserve honor and protect the
rights of people.

Role \#3 Shepherds of the sheep of God: In guiding believers and others,
the members of the House will consult on matters which ensure people are
honoring others and being honored. They educate people, without
exceptions, on the nature of the soul, the rights all are created with,
the rights we mature into, our responsibilities, and how to express
these rights while honoring the rights of others. The House also ensures
those whose rights are compromised are protected, counseled, and not
subjected to further degradation.

Role \#4 Dawning places of His command: The House develops or sponsors
educational, cultural, and civic programs designed to preserve the
people's honor. These would be tied to the teachings of the Bab and
Baha'u'llah, inspiring the various levels of outreach and social action.
Education, culture, and civic programs can be completely interlinked, as
each inform the other. Imagine a culture where Baha'u'llah's teachings
about leading with good deeds and righteous acts starts to be integrated
at every level of society, even in the political sphere. While the
members of the House of Justice do not have the authority at this stage
to participate in roles of government, they can be the guiding lights of
those who do.

When the honor of the people are increasingly and sufficiently
preserved, the people may decide to consult on granting the House of
Justice the fourth authority.

\subsection{Authority \#4: The Development of
Cities}\label{authority-4-the-development-of-cities}

The fourth authority is the development of cities. Through this point,
the House of Justice has continued its successful trusteeship of
propagating the Cause of God, the morals of souls, and the preservation
of honor. The members of the House of Justice view the cities as
bestowals from God, being treated with reverence and love. The purpose
is to take the lands which were created by God and refine them so they
reach a higher station towards perfection. This is not about achieving
perfection, but the constant striving towards it.

The development of cities would include the physical, economic,
educational, and environmental components by engaging with social
institutions, endowments, urban design, educational systems, and civic
planning. There is wisdom in delaying this authority until now. With the
House of Justice experienced in preserving the honor of all people in
the city, they will be better capable to ensure development also
preserves and enhances honor.

Role \#1 Trustees of the Merciful: The city is the trust, and the House
of Justice is the trustee. The members of the House consult regarding
services, infrastructure, zoning, and other related matters regarding
development. The House is not the government. They do not enact
ordinances, laws, taxation, or other legislative, executive, and
judicial acts as a government entity. Instead, they consult.
Consultation could include development firms, business leaders,
non-profit organizations. The Houses could create or support
organizations and individuals who are active in these roles.

Role \#2 Representatives of God: This would be the role where the House
of Justice shapes policy through example, not domination. They serve as
advisers and experts in the public sphere, ensuring Baha'u'llah's
principles are active in civic affairs. The members may advocate for
policies which align with the Cause, publish or present statements on
ethics in economics, sustainable development, or community planning.
They do this through their own integrity, transparency, and wisdom.

Role \#3 Shepherds of the sheep of God: By this stage, being shepherds
has become increasingly complex. Pastoral care continues, ensuring
urbanization and its inherent materialism does not erode the soul of the
community. The House promotes neighborhood-level consultation and may
support projects which alleviate the suffering of vulnerable people. The
diversity of the city is navigated as pathways of unity. A city will
have multiple voices and organizations working towards various causes
and goals. The House of Justice can encourage their support, refinement,
and utilize resources to fill-in any potential needs being under-served.
All of this is to ensure all within the city have access to the
resources and opportunities required to support their rights and
responsibilities to each other.

Role \#4 Dawning places of His command: Within this role, the House of
Justice can find ways to integrate spirituality into city design, to
include Mashriq-ul-Adhkars and other places of worship. They can promote
public festivals and other cultural events which celebrate service,
knowledge, and beauty. Parks, gardens, and buildings could be encouraged
to be designed in ways which reflect or symbolize divine virtues.

When the city's institutions embody justice, when culture and commerce
serve human welfare, and when the people see divine beauty reflected in
their civic life, the community is ready for the fifth authority:
governance itself. At that stage, the House of Justice becomes not only
the moral compass but the axis of order for the land.

\subsection{Authority \#5: The Governance For the Lands and Protection
For the
Servants}\label{authority-5-the-governance-for-the-lands-and-protection-for-the-servants}

This is the final authority which can be granted to the House of
Justice. There are two authorities intertwined which serve each other.
They are the governance for the lands and the protection for the
servants. One aspect cannot exist without the other. Unlike the other
authorities, this one must be granted by the entire city, not just those
who believe in Baha'u'llah. A successful House of Justice has laid the
groundwork to be a trusted institution within the city through the prior
four authorities. Once the House is responsible for governance and
protection, they have been granted their final tools to continue the
development of cities, the preservation of honor, refinement of morals,
and the propagation of the Cause.

Role \#1 Trustees of the Merciful: The government is the trust and the
House of Justice is the trustee. The government's first role is the
protection of the servants. As trustees, they are not focused on the
protection of the government or the House of Justice itself. The House
has no inherent right to exist except through the authority of the
people. They are not only advocates for the rights of others, but they
ensure legislation is enacted which protect the people's rights, the
laws are enforced fairly, and any judgments reflect the principles of
consultation Baha'u'llah taught us. Once again, consultation is the
foundation of trusteeship and when done correctly, is guided by the Holy
Spirit from God.

Role \#2 Representatives of God: As representative of God, the members
of the House act as though the their city is God's, establishing a
kingdom on Earth as it is in Heaven. The House strives to reflect the
names of God, even as they represent people of various faiths and those
without belief in God. Everything they do, and those who work for them
must constantly remember they are representatives of God. Their
leadership nurtures a culture which better reflects those names and
attributes more than prior to the House of Justice being created.

Role \#3 Shepherds of the sheep of God: The House is responsible in all
aspects of the nurturing and protection of all people in their cities.
The protective measures of the Cause of God, such as expressed in
Chapter 18, are the foundation of the House's role as shepherds. There
is a delicate balance in being able to identify threats to one's safety,
protecting it, while also ensuring the rights of all are also fully
guaranteed. This may require continued educational efforts which promote
healthy private and social habits as much as it does through legislative
or criminal justice system actions.

Role \#4 Dawning places of His command: This final role, in my opinion,
is the pinnacle of the entire revelation of God through Baha'u'llah.
Every law revealed in the Kitab-i-Aqdas is in effect. The implementation
of this divine law is not only a dawning place of His command, but the
dawning place of a new era for the city, and potentially humankind.

\section{What is Not Mentioned}\label{what-is-not-mentioned}

Baha'u'llah is quite clear in expressing what rights, responsibilities,
and authorities people have in various aspects of life. This is true for
those who serve various leadership positions, especially in government
and religion. The House of Justice does have roles in both spheres. It
could seem quite alarming in today's age to read about an institution
born from religious decree to lead a city's government. I share those
same concerns, which is why I find it useful to consider what is not
mentioned by Baha'u'llah. My perspective is the House is only granted
the authorities given to it, first as outlined by God and later as
permitted by the people. These are two exceptionally strong guardrails,
if honored.

One authority not granted to the House of Justice is the implementation
of religious practice. The eighth splendor mentioned earlier in the
chapter explicitly denies the House of Justice any authority in
religious practice. Religious practice is entirely a responsibility of
the individual, unenforceable by other people. This means the House of
Justice, while decreed in the Most Holy Book, is not a religious
institution. I believe this is why there are no laws with any punishment
which exist solely in the private life of a person. We only see such
laws to protect the social life, to prevent harmful acts towards others.

Given they have no role in religious practice, when the members of the
House of Justice propagate the Cause of God, they also have no authority
in interpretation, translation, or other aspects in the unveiling and
understanding of Baha'u'llah's writings. Their authority is in
propagation, as described earlier.

The House of Justice has no authority to be a monarch of a kingdom,
although a monarchy can have a House of Justice. This is not an
extensive list of what is not authorized, but I do hope these examples
encourage reflection and creative thinking in understanding the unique
and divine institution. With God being the source of all authority,
authority can only be granted by God and His Manifestation. This
authority cannot be taken as one wills.

\section{Scope of Authority Beyond the
Cities}\label{scope-of-authority-beyond-the-cities}

Baha'u'llah says ``all matters of State should be referred to the House
of Justice.''\footnote{Lawh-i-Bisarat} He also says ``Although a
republican form of government profits all the peoples of the world, yet
the majesty of kingship is one of the signs of God. We do not wish that
the countries of the world should remain deprived thereof. If the
sagacious combine the two forms into one, great will be their reward in
the presence of God.''

\subsection{State-Level}\label{state-level}

A House of Justice can be established which has the authority of the
matters of State and operates within a republican form of government. A
republican form of government is a system where the people choose
representatives for governance, legislation, and public decision-making.
As with the cities, this is a consultative body whose authority is
granted by the people.

A state-level House of Justice with the preceding authorities could
theoretically be nominated the people of a city which has a House of
Justice where the city is a city-state. Where the city is but a part of
a larger jurisdiction, cities with Houses of Justice could authorize the
creation of a state-level House of Justice to assist in the propagation
of the Cause. The pattern to expand authority would follow the
guidelines of the rest of this chapter. Other jurisdictions such as
county, province, parish, and their equivalents in other cultures would
also follow a similar pattern.

A singular city who has Authority \#1 could not declare other
authorities for other jurisdictions. This would mean jurisdictions must
be adjoining. Within a state, it could also be possible to have multiple
Houses of Justice operating at different authorities as they are
dependent on those authorities being granted by the people of their
locality. This is purposefully decentralized.

The only way to by-pass this is if a monarch creates a new republican
monarchy with a House of Justice established within the new government.
This scenario will be covered more in the next chapter.

\subsection{World-Wide}\label{world-wide}

Baha'u'llah also teaches the trustees of the House of Justice to adopt
one language (or create one) and choose one script, so that the children
of the world may appear in one homeland.\footnote{Lawh-i-Ishraqat.} This
is a foundation in loving the world instead of having pride in their
homeland. He adds the ministers of the House of Justice must implement
the Most Great Peace.\footnote{Lawh-i-Dunya (Tablet of the World)} The
Most Great Peace is the unification of the world's parties or the
unification of the world's religion.\footnote{Kalimát-i-Firdawsíyyih
  (Words of Paradise)} This would fall under Authority \#1.

These two authorities are universal in nature. There are two ways for
this to be achieved, or a hybrid of the two ways. The first could be the
Houses of Justice established throughout the lands consult together as
one institution called the House of Justice. The second could be a
world-wide House of Justice is established with these sole two
authorities. This would be for the people to decide in due time. I
believe either would be viable and effective.

The entire focus of this world-wide House of Justice would be to
establish the Most Great Peace, nothing more and nothing less. They
would work with political leaders, NGOs, and other organizations to
progress towards the adoption of one language and script. There would be
a number of smaller goals to work towards to achieve this, while also
ensuring people maintain their right to language (especially their
native language). This world-wide House would also be active in the
political sphere in conflict-resolution and mediation, maybe supporting
forums and educational efforts which works towards peace and diplomacy,
and helping political parties have a vision greater than their own
party.

Finally, the world-wide House of Justice would also work towards the
unification of the world's religion. There could also be multiple
pathways towards this goal. One would be nurturing the belief of all
people towards Baha'u'llah. Another would be help reform other religions
in such ways as their people recognize there is one God who has guided
each of them. This pathway would indirectly lead people towards
Baha'u'llah. Another pathway could simply be the leaders of the world's
religions to consult together and determine all should live in peace
together. All three of these pathways could be independent of each other
or be closely linked, depending on the result of the House's
consultations.

If there did end-up being a Most Great Peace established, it could also
be the world's nations have decided to operate under a single
constitution. If so, it might be possible for a House of Justice to have
worked its way to Authority \#5. However, a world-wide House of Justice
cannot be established prior to the other preceding jurisdictions. This
is never a top-down mechanism.

\section{The Rest of Part 4}\label{the-rest-of-part-4}

We are concluding this most important chapter of Part 4. I highly
believe this model for the entire institution for the House of Justice
is our best pathway in the liberation of humankind. Up to this point, we
defined the ultimate structure from the ground-up. The rest of the
chapters are going to slowly start from the tops of the political and
religious power dynamics and work our way back down to the grassroots.
To start at the top of the political structure, we will next look at
Baha'u'llah's guidance to the monarchs of the world, His hope for a
future monarch who believes in Him, and how the He counseled monarchs
and other state actors.

These teachings make up a large part of the Kitab-i-Aqdas and many of
His unsolicited teachings.

\newpage{}

\chapter{Political Leadership}\label{political-leadership}

\section{Introduction}\label{introduction-8}

The Houses of Justice were not meant to lead on their own. Paragraphs 77
through 97 of the Kitab-i-Aqdas includes a large portion addressing the
assembly of kings, particular political leaders of the time, and various
lands. Many of Baha'u'llah's teachings are about the responsibilities of
political leaders, the rights they are to protect, and our
responsibility to those leaders. Baha'u'llah says ``Now, what seems good
in the British nation's constitution, which is adorned with both the
light of sovereignty and the consultation of the nation \ldots{} but the
matter that is a cause of preservation and prohibition in both the outer
and inner aspects, is the fear of God.''\footnote{Lawh-i-Dunya (Tablet
  of the World)}

The ``Epistle to the Son of the Wolf'' provides a great example of
Baha'u'llah's vision. In addressing a Shayhk who serves the Ottoman
Sultan, He offers this short sermon:

\begin{quote}
Every nation should consider the position of its ruler, be submissive to
his command, act by his decree, and hold fast to his judgment. Kings are
the manifestations of the power, elevation, and grandeur of God. This
oppressed one has never flattered anyone; all bear witness to this fact.
However, considering the status of kings is from God, and it is clear
and known from the words of the Prophets and saints.

In the presence of the Spirit (Jesus), it was asked: ``O Spirit of God,
is it lawful to give tribute to Caesar or not?'' He said: ``Yes, render
unto Caesar what is Caesar's and unto God what is God's.'' He did not
forbid it, and these two words are one to those who perceive, for what
is Caesar's would not be lawful if it were not from God. Likewise, in
the blessed verse: ``Obey God and obey the Messenger and those in
authority among you.'' The primary and foremost meaning of ``those in
authority'' are the Imams (may the peace of God be upon them), who are
the manifestations of power, the sources of command, the treasuries of
knowledge, and the dawning places of divine wisdom. In the secondary
rank, it refers to the kings and rulers whose light of justice
illuminates and brightens the horizons of the world. It is hoped that
from the Sultan (may God preserve him) a light of justice will shine
that will encompass all the parties of the nations. All should ask God
for what is befitting today for His sake.
\end{quote}

This chapter will consider the derived authority from God a monarch may
express, how the government derives its authority from the monarch,
their responsibilities, potential opportunities leaders had in
Baha'u'llah's time, and promises and prophecy God makes to certain lands
(and the people therein). As seen in Baha'u'llah quoting both Jesus
Christ and Muhammad, this has been a consistent teaching of God for
thousands of years. Baha'u'llah gives us the framework to achieve this.

\section{Opportunities of Sovereign Leaders in Baha'u'llah's
Time}\label{opportunities-of-sovereign-leaders-in-bahaullahs-time}

Baha'u'llah addressed several sovereign leaders and the opportunities
presented to them, in the Kitab-i-Aqdas and elsewhere. This section will
briefly share some of these opportunities.

\subsection{To the Emperor of Austria (Franz Joseph
I)}\label{to-the-emperor-of-austria-franz-joseph-i}

He visited Al-Aqsa Mosque in Jerusalem in 1869 but did not enquire about
Baha'u'llah, nor sense Him.\footnote{Kitab-i-Aqdas \#85} Baha'u'llah was
with the Emperor in all conditions, but the Emporer was clinging to the
branch but heedless of the root. Baha'u'llah invited the Emperor to
recognize Him instead of clinging to a prior Branch. Emperor Franz
Joseph I missed the opportunity to express his belief in Jesus Christ by
recognizing Him in a new name. Franz was a well-respected sovereign and
reigned for 68 years until his death in 1916. The empire of Austria
ended with his death, an outcome of World War I which started with his
declaration of war against Serbia after the assassination of his heir.

\subsection{To the King of Berlin (Wilhelm
I)}\label{to-the-king-of-berlin-wilhelm-i}

The Second Reich of the German Empire was formed in 1870, just 3 years
prior to the revealing of the Kitab-i-Aqdas. Baha'u'llah addresses
Emperor Wilhelm I\footnote{Kitab-i-Aqdas \#86} warning him about pride,
and a king prior who sought to dominate the lands and rule over the
people. He had originally resisted a constitutional monarchy but
relented upon Otto von Bismarck's recommendation. Nevertheless, Germany
remained militaristic and authoritarian with designs on restoring the
Holy Roman Empire of Germany to its former glory. Wilhelm I passed away
at the age of 90 in 1888.

\subsection{Regarding Napoleon III of
France}\label{regarding-napoleon-iii-of-france}

When Baha'u'llah reminded Wilhelm I of the prior king, He was
referencing a fulfilled prophecy to Napoleon III of France. In the
Suriy-i-Haykal, Baha'u'llah addressed Pope Pius IX of Rome, Czar
Alexander III of Russia, Queen Victoria of Great Britian, and Sultan
Nasiri'd-din-Shah of Persia in addition to Napoleon III. Baha'u'llah in
1868 told Napoleon III after he claimed to fight against oppression:

\begin{quote}
``Due to your actions, affairs in your kingdom will differ, and dominion
will slip from your hand as a result. Then, you'll find yourself in
clear loss, and earthquakes will affect all tribes unless you stand in
support of this cause and follow the spirit on this straight path. The
honor you value won't last, it will fade unless you hold onto this
strong rope. We see humiliation following you while you are among the
heedless.''
\end{quote}

By the time the KItab-i-Aqdas was revealed 5 years later, Napoleon III
had been captured by Germany, and died a prisoner. He was remembered in
France as a deserter of his army and was publicly humiliated. This
context places the warning to Emporer Wilhelm I as an urgent warning.
Pride in yourself cannot come at the cost of your country and its
citizens.

\subsection{To the Kings and Presidents of the
Americas}\label{to-the-kings-and-presidents-of-the-americas}

By 1873, the western hemisphere had largely become comprised of several
new republics, starting with the United States of America in 1776. There
was one independent King, Pedro II of Brazil, although descended from
Austria's House of Habsburg. Europe also exercised colonial rule over
many other lands.

Baha'u'llah instructs the Kings and Presidents to adorn the temple of
dominion with the raiment of justice and piety, and its head with the
crown of the remembrance of your Lord.\footnote{Kitab-i-Aqdas \#88}
Baha'u'llah did not promise the Americas would be the light of justice
and piety, but it was their responsibility to do so.

Appendix 7 offers a table outlining the leaders of the world in 1873.
The idea is to provide context for the Kitab-i-Aqdas's place in both
time and space. Maybe you may ask yourself, did these nations seize the
opportunity or let the opportunity pass away? Do these opportunities
still exist? I believe where these opportunities were not acted on,
these opportunities still exist. It is never too late to adorn the
temple of dominion with the raiment of justice and piety.

\section{Roles and Responsibilities of
Monarchs}\label{roles-and-responsibilities-of-monarchs}

\subsection{We Share the Same
Foundations}\label{we-share-the-same-foundations}

Monarchs have various roles and responsibilities derived from their
position. Before we get into the roles and responsibilities unique to
monarchs, let's start from the beginning.

In the beginning, God created the worlds and everything within them.
Over time humans emerged and God gave them a unique soul. Every human's
souls is created noble, regardless of their social or economic status. A
person who emerges as a monarch has a soul created by God, just like
you, myself, and others we know. The monarch's soul has opportunities to
pass through the spiritual worlds, and believe in God, their own soul,
these spiritual worlds, the Day of Resurrection, the Command of God, and
the Manifestations of God.

The monarch may not believe in God, believe in Baha'u'llah, or believe
in these things. Yet, these opportunities exist. They may choose to
pray, remember God in worship, recite the verses of God, reflect on
themselves, and honor God through various practices. They may not do
these things, or even publicly say they do but in their private life do
not. The monarch has opportunities to develop and refine their virtues.

The monarch is also born with the same rights we have. As they attain
maturity at age 19, they have additional responsibilities and rights.
They have a private life, like we do, and may struggle with illusions,
delusions and desires. They have opportunities to overcome them.
Monarchs have a constructive social life, and need protections of this
social life. Monarchs have affectionate relationships and have
responsibilities towards those in their personal life to include being
trustees and participating in consultation. As you can see, for every
aspect of their existence, we share the same foundations.

\subsection{Beyond Shared Foundations}\label{beyond-shared-foundations}

The Kitab-i-Aqdas and Suriy-i-Muluk (The Tablet to the Kings) outline
specific responsibilities monarchs have. This will section will list
these responsibilities:

\subsubsection{Spiritual Foundations of
Leadership}\label{spiritual-foundations-of-leadership}

\begin{itemize}
\item
  Purify yourself from the wealth of the world, do not be preoccupied
  with wealth\footnote{Kitab-i-Aqdas \#79}
\item
  Take from the world only what is sufficient, leaving what is
  excessive\footnote{Suriy-i-Muluk \#21}
\item
  Do not let the love of others enter your heart, let the love of God
  rule the heart so you may know Oneness\footnote{Suriy-i-Muluk \#62}
\item
  Act as vassals of God and rise to serve the Purpose for which you were
  created\footnote{Kitab-i-Aqdas \#82}
\item
  You are the shadow of God on earth\footnote{Suriy-i-Muluk \#61}
\item
  Leave your houses (might refer to any dwelling places of monarchs to
  include palaces) and turn to the Kingdom of God\footnote{Kitab-i-Aqdas
    \#84}
\item
  Leave your laws and follow the law of God\footnote{Suriy-i-Muluk \#23}
\end{itemize}

\subsubsection{Ethical Foundations of
Leadership}\label{ethical-foundations-of-leadership}

\begin{itemize}
\item
  Do not wrong anyone\footnote{Suriy-i-Muluk \#10}
\item
  Prevent oppressors from their oppression\footnote{Suriy-i-Muluk \#14}
\item
  Secure the rights of the oppressed\footnote{Suriy-i-Muluk \#14}
\item
  Examine the affairs of people before issuing judgments or
  punishments\footnote{Suriy-i-Muluk \#19}
\item
  Do not punish those who do not disobey the your laws\footnote{Suriy-i-Muluk
    \#30}
\item
  Recompense the debts of those wrongfully punished\footnote{Suriy-i-Muluk
    \#32}
\item
  Do not take people's money unjustly (bribes, blackmail,
  etc)\footnote{Suriy-i-Muluk \#83}
\item
  If a wicked person brings you news, verify it\footnote{Suriy-i-Muluk
    \#35}
\item
  Beware of listening to words of malice and hypocrisy\footnote{Suriy-i-Muluk
    \#37}
\item
  Do not impose upon others what you cannot bear yourselves\footnote{Suriy-i-Muluk
    \#42}
\end{itemize}

\subsubsection{Economic Foundations of
Leadership}\label{economic-foundations-of-leadership}

\begin{itemize}
\item
  Do not impose your expenses on your subjects beyond their capacity (do
  not over tax)\footnote{Suriy-i-Muluk \#11}
\item
  Do not betray your trusteeship to the poor\footnote{Suriy-i-Muluk \#12}
\item
  Spend on servants according to their need, not according to what they
  hoard or use for adornment, so that none are in need nor
  hoarding\footnote{Suriy-i-Muluk \#}
\item
  It is not fitting to have some in abundance and others in clear
  poverty and humiliation\footnote{Suriy-i-Muluk \#59}
\end{itemize}

\subsubsection{Safety and Security Foundations of
Leadership}\label{safety-and-security-foundations-of-leadership}

\begin{itemize}
\item
  Reduce your armies so that your expenses decrease\footnote{Suriy-i-Muluk
    \#10}
\item
  Maintain an army to protect the lands and kingdoms\footnote{Suriy-i-Muluk
    \#10}
\item
  Do not wrong those who have migrated to you and protect
  them\footnote{Suriy-i-Muluk \#11}
\item
  Reconcile among yourselves (sovereign leaders reconcile with sovereign
  leaders)\footnote{Suriy-i-Muluk \#10}
\end{itemize}

\subsubsection{Bureaucratic Administrative Foundations of
Leadership}\label{bureaucratic-administrative-foundations-of-leadership}

\begin{itemize}
\item
  Respect scholars who act according to their knowledge, follow the
  limits set by God, and judge according to what God has decreed in His
  Book\footnote{Suriy-i-Muluk \#43}
\item
  Do not gather around you agents who abandon their trusts and follow
  their desires\footnote{Suriy-i-Muluk \#54}
\item
  Consult with agents who exude the scent of faith and
  justice\footnote{Suriy-i-Muluk \#54}
\item
  Do not relinquish control of your affairs to others and do not become
  complacent\footnote{Suriy-i-Muluk \#56}
\end{itemize}

\subsection{Additional Rights of
Monarchs}\label{additional-rights-of-monarchs}

Baha'u'llah did not outline any additional rights of monarchs. Once
again, what is excluded is quite informative. Baha'u'llah does not say
the monarch should be concerned about the protection of their
government, but only the citizens and lands. This is one example and I
invite you to consider what other traditional notions of governance are
excluded from Baha'u'llah's directives.

\section{To Various Lands and Cities}\label{to-various-lands-and-cities}

Baha'u'llah had addressed various lands in the Kitab-i-Aqdas and in
other writings. This section will only cover what is in the
Kitab-i-Aqdas. We will see admonitions, prophecies, and encouragement.
All of these are related to how the governments of these lands treat
their citizens and their adversaries.

\subsection{To the Company of Rome (Byzantine
Rome)}\label{to-the-company-of-rome-byzantine-rome}

Baha'u'llah hears the sound of the owl among them.\footnote{Kitab-i-Aqdas
  \#89} He asks ``Has the intoxication of desire seized you, or are you
among the heedless?'' In 1873, the Eastern Orthodox Church was a
semi-autonomous religious entity governed under the Rum millet within
the Ottoman Empire. Istanbul at the time was about 50\% Sunni Muslim
(versus over 90\% today) with the Eastern Orthodox Catholics forming a
significant population. Istanbul was the home of the former Eastern
Roman Empire from the 4th through 15th centuries, the home of the first
Christian monarch, and nearby Anatolia was the home of the first Pauline
churches. The Greek population formed an important economic, social, and
political base in the Ottoman Empire, even if politically they were
considered second-class citizens.

When Baha'u'llah addresses the company of Rome, Baha'u'llah is directly
referring to Orthodox Christians governed under the Rum millet, and the
historical significance Rome had in representing the community who
believed in Jesus Christ. When He mentions the owl, I believe
Baha'u'llah is evoking Psalms 102 from King David of Israel. I will
share the full text and allow the Psalms to carry the weight of this
section and the next:

\begin{quote}
Hear my prayer, Lord; let my cry for help come to you. Do not hide your
face from me when I am in distress. Turn your ear to me; when I call,
answer me quickly. For my days vanish like smoke; my bones burn like
glowing embers. My heart is blighted and withered like grass; I forget
to eat my food. In my distress I groan aloud and am reduced to skin and
bones. I am like a desert owl, like an owl among the ruins. I lie awake;
I have become like a bird alone on a roof. All day long my enemies taunt
me; those who rail against me use my name as a curse. For I eat ashes as
my food and mingle my drink with tears because of your great wrath, for
you have taken me up and thrown me aside. My days are like the evening
shadow; I wither away like grass. But you, Lord, sit enthroned forever;
your renown endures through all generations. You will arise and have
compassion on Zion, for it is time to show favor to her; the appointed
time has come. For her stones are dear to your servants; her very dust
moves them to pity. The nations will fear the name of the Lord, all the
kings of the earth will revere your glory. For the Lord will rebuild
Zion and appear in his glory. He will respond to the prayer of the
destitute; he will not despise their plea. Let this be written for a
future generation, that a people not yet created may praise the Lord:

``The Lord looked down from his sanctuary on high, from heaven he viewed
the earth, to hear the groans of the prisoners and release those
condemned to death.''

So the name of the Lord will be declared in Zion and his praise in
Jerusalem when the peoples and the kingdoms assemble to worship the
Lord. In the course of my life he broke my strength; he cut short my
days. So I said:

``Do not take me away, my God, in the midst of my days; your years go on
through all generations. In the beginning you laid the foundations of
the earth, and the heavens are the work of your hands. They will perish,
but you remain; they will all wear out like a garment. Like clothing you
will change them and they will be discarded. But you remain the same,
and your years will never end. The children of your servants will live
in your presence; their descendants will be established before you.''
\end{quote}

The Eastern Orthodox Church suffered greatly in Istanbul and Anatolia
not long after 1873. Massacres of Armenian Christians in 1894 and 1909,
the Armenian Genocide during World War I, and the aftermath of the
Greco-Turkish war led to a decline the Church has never recovered from
in its historical homeland. These losses were tragic and painful and
still form a painful scar for many in eastern Europe.

\subsection{To the Point on the Shore of Two Seas
(Istanbul)}\label{to-the-point-on-the-shore-of-two-seas-istanbul}

Baha'u'llah turns His attention to the point on the shore of the two
seas, which is Istanbul. He says:

\begin{quote}
The throne of oppression has been established upon you, and the fire of
hatred has been ignited within you to such an extent that it has caused
lamentation in the Concourse on high and those who circle around the
exalted Throne. We see within you the ignorant ruling over the wise, and
darkness boasting over the light, while you are in manifest delusion.
Your outward adornment has deceived you. By the Lord of creation! It
will perish, and the daughters and widows, and the tribes among you,
shall weep. Thus does the All-Knowing, the All-Informed give you
tidings.
\end{quote}

In 1876, the Ottoman Empire did adopt a constitutional monarchy much
like how Baha'u'llah had prescribed. In 1878, Sultan Abdul Hamid II
suspended the constitution. The Ottoman Empire quickly declined and by
1922, after 620 years, the Ottoman Empire was no more. Istanbul, the
former Constantinople, stopped being the capital of Christendom and
Sunni Islam after 1600 years.

\subsection{To the Banks of the Rhine
River}\label{to-the-banks-of-the-rhine-river}

Baha'u'llah addresses the rising Second Reich of the German Empire,
formed in 1871 by the Prussian Wilhelm I. He says ``We have seen you
covered with blood, as the swords of retribution were drawn against you,
and again you shall have it.''\footnote{Kitab-i-Aqdas \#90} The
Franco-Prussian War was fought in 1870, another of several wars between
France, Prussia (later Germany) which started in the late 1790s. The
Napoleonic Wars themselves had an estimated 4-6 million casualties, and
many battles were fought along the Rhine River. These wars seemed to be
a never-ending cycle of retribution.

Baha'u'llah includes a prophecy stating that after 1873, the Rhine River
banks will again be covered with blood. The Battles of Marne (1914 \&
1918) and Verdun (1916) were fought during World War I. About 1.5
million casualties were reported in these three battles.

\subsection{The Lamentation of Berlin}\label{the-lamentation-of-berlin}

Baha'u'llah then addresses Berlin, the capitol of the German Empire.
``We hear the lamentation of Berlin, though she is now in manifest
glory.'' Lamentation is an expression of deep sadness and regret, often
due to a sense of wrongdoing. Berlin was the capitol of the most
industrialized, most modern, the most militaristic, and potentially the
most authoritarian government in Western Europe. The First Reich was
known as the Holy Roman Empire, when Pope Leo III crowned Charlemagne as
the first emperor of the symbolic Western Roman Empire in the year 800.
This decentralized kingdom lasted until the Napoleonic War in 1806,
nearly a thousand years of rule between the Roman Catholic Church and
Frankish (a German ethnicity) monarchs.

I believe the lamentation of Berlin alludes to the lost moral leadership
of the Holy Roman Empire amid the increasingly violent retributive wars
throughout Europe, plus the repercussions throughout the world through
colonialism and imperialism. The centralized Second Reich was not the
same as the decentralized First Reich. The first was moderated by the
influence of Christianity, while the second was no longer restrained by
religious authority. The retributive wars in Europe did not end until
World War 2, once the Third Reich was defeated.

Sadly, retribution endured as a consequence of the unraveling of moral
authority once diffused and restrained across Eastern Rome, Western
Rome, and the Sunni Caliphate. These traditions, despite their failures
and imperfections, governed in reference to God rather than
unconstrained force. As centralization displaced trusteeship and power
slipped free from humility and law, the lands once oriented toward
Jerusalem inherited cycles of violence instead of peace.

\subsection{To the Lands Within
Persia}\label{to-the-lands-within-persia}

The next section addresses the people and lands of Persia, where
Baha'u'llah was born, raised, imprisoned, and exiled from.

\subsubsection{To The Land of Ṭā
(Tehran)}\label{to-the-land-of-ux1e6dux101-tehran}

Baha'u'llah describes Tehran as the place where the Dawn of
Manifestation was born\footnote{Kitab-i-Aqdas \#91 - \#93}, referencing
the appearance of the Holy Maiden to Him in the Siyah-Chal prison of
Tehran in 1852. Tehran is the source of the world's joy, so do not
grieve over anything. He then offers two promises, one is conditional
and the other is not.

The conditional promise relies on whether God will it or not. This
promise is Tehran will be blessed by a sovereign leader who will rule
with justice and gather the scattered sheep (believers) from the wolves.
He will greet the people of Baha (those who believe in Baha'u'llah) with
joy and gladness.

The unconditional promise has two parts. The first part is the masses
will pass judgment upon Tehran, with things being overturned within
Tehran. Yet, when this happens, be at peace because the divine bounty
will not cease from you. After this period, tranquility will follow the
turmoil. This promise has been decreed by God.

The unconditional promise does not require the conditional promise to be
fulfilled. The period of turmoil and the period of tranquility are
inevitable. It is hard to pinpoint any moment as being the specific
turmoil. There was the civil war of 1908 to 1910 which ushered a
constitutional revolution, the American-led coup in 1953 which replaced
the constitution with absolute monarchy, and then the Islamic Revolution
of 1979 which overthrew the monarchy with a constitutional theocracy
based on Shi'a Islam. Each of these could be the turmoil depending on
perspective, the entire period could be a part of a greater turmoil, or
maybe it has not happened yet. Depending on what you believe the turmoil
is, might inform what you believe the period of tranquility will be
thereafter. No matter what, I do hope Tehran can become illuminated by
that light from the Dawn of Revelation.

\subsubsection{To the Land of Khá
(Khurasan)}\label{to-the-land-of-khuxe1-khurasan}

Baha'u'llah says ``O Land of Khá! We hear within you the voices of men
extolling your Lord, the Self-Sufficient, the Most Exalted. Blessed is
the day when the banners of the Names are raised in the Kingdom of
Creation in My Most Glorious Name. On that day, the sincere will rejoice
in the victory of God, while the disbelievers will lament.''\footnote{Kitab-i-Aqdas
  \#94}

I have wondered if this could mean Khurasan as the current province of
Iran, or the greater historical area of Khurasan which includes modern
cities such as Herat (Afghanistan), Bukhara (Uzbekistan), and Ashgabat
(Turkmenistan). These areas were first included with the name Khurasan
during the Zoroastrian Sasanian Empire in the 6th century. Where might a
Kingdom begin which establishes the banners of the Names of God in His
Most Glorious Name (Baha'u'llah)?

\subsubsection{To the Land of Kāf and
Rā}\label{to-the-land-of-kux101f-and-rux101}

The final land from Persia mentioned by Baha'u'llah is the land of Kāf
and Rā (Kirman). This paragraph \#164 is immediately part of a series of
paragraphs addressed to religious scholars. Baha'u'llah has witnessed
that which He has chosen to remain a secret. However, I infer the later
clause regarding the insinuations of the learned and the doubts of the
skeptics to be potentially related to this secret.

One potential aspect of history which this may allude to is related to
the husbands of two of Baha'u'llah's nieces, the daughters of Mirza
Yahya (Subh-i-Azal). Mirza Aqa Khan Kermani and Sheikh Ahmad Rouhi were
Azali Babi's who were influential in the revolutionary movements in the
late 1800's Persia. In 1875, just 2 years after this paragraph was
revealed by Baha'u'llah, they went to various cities of Persia and later
to Constantinople to garner support for greater freedoms for Persians.
In 1896, Kermani assassinated Naser al-Din Shah and both were beheaded.

The followers of Baha'u'llah have long been associated with the
assassination of the Shah and other revolutionary movements of Persia,
despite Kermani and Rouhi writing public treatises against Baha'u'llah.
Baha'u'llah says ``no one should object to those who rule over the
people''\footnote{Kitab-i-Aqdas \#95} which is a clear instruction
against revolution and rebellion.

\section{Conclusion}\label{conclusion-3}

As we can see, Baha'u'llah has a clear vision about the roles and
responsibilities of kings and queens. This vision is direct from God. We
can also look back into history and see the issues caused when kings,
queens, and various leaders act contrary to this vision. The people
under their rule and guidance suffer greatly. Early in this book I had
briefly discussed suffering and ways to be liberated from suffering.
While spiritual practice is the foundation, the pathways of suffering
and the liberation therefrom also exist in the ways we organize and
govern ourselves. John Locke had argued in 1689's ``Two Treatises of
Government'' that political authority is based on the consent of the
people. Do we consent to suffer or do we consent to be liberated from
suffering?

To answer this question, we realize common mechanisms of change are not
true pathways. Revolution, rebellion, and even protest do not
consistently ease suffering or cause progress without consequences. The
next chapter will help us look at this question with a different lens.
We will look at spiritual and religious leadership in general in Chapter
24 and specifically at Baha'u'llah's vision for the Cause of God after
His passing. These are all components of what we consent to, how we
express consent, and reduce the suffering we and the generations after
us experience.

\newpage{}

\chapter{Spiritual Leadership}\label{spiritual-leadership}

After discussing the Houses of Justice and civil sovereignty, a
necessary question remains unresolved. If no institution governs belief,
worship, or conscience, what form, if any, does spiritual leadership
take?

Bahá'u'lláh is unambiguous. Authority over the soul belongs to God
alone. No House of Justice, no ruler, and no scholar possesses the right
to command belief, regulate spiritual practice, or act as an
intermediary between a person and God. Yet this absence of coercive
authority does not imply the absence of leadership. On the contrary,
Bahá'u'lláh speaks repeatedly and with great intensity about those who
influence souls.

This chapter will examine spiritual leadership as moral authority
instead of institutional power, as service rather than control, and as
illumination rather than governance. Before we identify the role of
spiritual leaders, let us recount some of the firm prohibitions we have
stated earlier in this book. Baha'u'llah says none possess authority
over belief, over spiritual practices, over interpretation, nor from
using names, persons, and traditions to veil a person from accessing
God.

\section{To the Concourse of Ulama (Religious
Scholars)}\label{to-the-concourse-of-ulama-religious-scholars}

Both the Bab and Baha'u'llah were considerably critical towards the
Muslim ulama. The ulama are people who fulfilled a wide range of roles
within the religion, such as scholarship, theological law, judges,
interpreters, transmitters, government officials of Islamic states,
teaching, community leaders, and other scholarly roles. They can be
incredibly influential at all levels of society. Most of these roles are
not exactly unique to Islam, though. A person can see aspects of these
roles in all religions. Thus, the guidance Baha'u'llah offers in the
Kitab-i-Aqdas can be applicable to many aspects of spiritual leadership
of all religions, to include the religions the Bab and Baha'u'llah led.

This section will be a summary of the guidance for the ulama from
paragraphs 165-172:

\begin{itemize}
\item
  Do not view yourself as greater than God. This can manifest itself in
  actions, not just words.
\item
  Do not worship the idols of your own desires.
\item
  Abandon your illusions and turn towards God.
\item
  Do not weigh the Book of God with the rules and sciences you possess.
  This means old techniques may not actually be useful or relevant to
  understanding the Book of God, even Books such as the Qur'an.
\item
  Do not weigh the Book of God with your own desires. Let God guide you
  instead of you trying to guide God.
\item
  Do not corrupt the Cause of God.
\item
  Do not create veils which hinder people's recognition of God.
\item
  Do not let names keep you from God, such as the names of prior
  Prophets or the names of prior Books.
\item
  Do not let your allegiance to a prior Prophet veil you from the
  authority of God.
\item
  Do not let the spiritual practice of remembrance veil you. This refers
  more about performative worship instead of transformative.
\item
  Do not become the cause of discord.
\item
  Do not be the cause for others to reject God.
\end{itemize}

Many of these prohibitions may seem as though they had been discussed in
various aspects of this book. This is because, like sovereign leaders,
spiritual leaders are also normal people, like you and I. They have the
same rights, the same responsibilities, their own spiritual journey,
beliefs, spiritual life, social life, affectionate relationships, and
goals. But, as we can see with some of Baha'u'llah's counsels, He was
concerned with their influence over others. For various reasons, people
who take on the various roles of the ulama chose these roles. They chose
to take on certain responsibilities, and these responsibilities have
real consequences within this world, especially to those individuals who
relied on the ulama for guidance and salvation.

One of Baha'u'llah's missions was to liberate the souls from corrupt
spiritual leaders. Yet, He did not abolish spiritual leadership.
Instead, Baha'u'llah offers His own vision of spiritual leadership.

\section{Scholars in the Cause of
Baha}\label{scholars-in-the-cause-of-baha}

In the Kitab-i-Aqdas \#173, Baha'u'llah offers these praises for
scholars in the Cause of Baha:

\begin{quote}
Blessed are you, O concourse of scholars in the Cause of Bahá! By God,
you are the waves of the Most Great Ocean, the stars of the heavens of
bounty, and the banners of victory between the heavens and the earth.
You are the dawning-places of steadfastness among the people and the
rising-points of utterance for those in existence.
\end{quote}

He also offers blessings to those who turn towards the scholars, and woe
to the headless. These scholars and those who turn to them, and are
nourished by the Revelation of God through Baha'u'llah.

\subsection{Turning To}\label{turning-to}

What does it mean to turn to? In the most simple terms, to turn to
someone means to give your attention. The responsibility to turn towards
scholars in the cause of Baha is a responsibility for anyone who
believes in Baha'u'llah as a Manifestation of God. Yet, Baha'u'llah does
not express any rights for scholars. Instead, the prohibitions and
admonitions described for the Muslim Ulama would also apply to scholars
in the Cause of God.

Without any additional rights, what do we give to scholars? We give our
attention and support. Support does not necessarily mean financial
support, but any type of support we want to offer. We may share their
explanations, insights, and arguments. We may support unfiltered access
to historical sources. We may pray for them and their success. And, if a
community has the resources to do so, offer financial support for their
work or the logistics necessary to complete the work.

Yet, scholars do not carry authority. They have no right nor claim. Any
person we turn towards is less authoritative then Baha'u'llah. While a
scholar's work is valuable, it can never supplant the Word of God.
Baha'u'llah says after His passing, to refer to what has been revealed
from Him. Baha'u'llah is always the primary source until the next
Manifestation of God comes. Everything scholars may unveil are portions,
never the completeness. We should be careful never to make any scholar a
primary source. For example, anyone who reads this book should never
refer to this book as a primary source about Baha'u'llah. This is why I
provide footnotes throughout, so a person can refer to the revelation
itself. Your understanding might be different than mine, you may have
better insights, and perhaps the totality of our insights helps others
approach Baha'u'llah. It can never replace Baha'u'llah. If we focus on
secondary sources and develop scholarship of secondary sources, every
level thereafter will get farther and farther from God's intentions.

None can match Baha'u'llah in the field of mystical insight, knowledge,
wisdom, and expression.\footnote{Kitab-i-Aqdas \#101}

\section{Examples of Spiritual Leadership Gone
Wrong}\label{examples-of-spiritual-leadership-gone-wrong}

\subsection{Shaykh Muhammad Hasan
al-Najafi}\label{shaykh-muhammad-hasan-al-najafi}

Baha'u'llah says this about Shaykh Muhammad Hasan al-Najafi:

\begin{quote}
Mention the Shaykh who was called Muhammad before Hasan, who was among
the most learned of scholars in his time.\footnote{Kitab-i-Aqdas \#166}
When the truth appeared, he and others like him turned away, while one
who winnowed wheat and barley turned toward God. He spent his nights and
days writing, as he claimed, the judgments of God, but when the Chosen
One appeared, not a single letter of his work availed him. Had it
benefited him, he would not have turned away from the face through which
the countenances of the near ones are illumined. If you had believed in
God at His appearance, the people would not have turned away from Him,
and what has befallen Us today would not have occurred. Fear God and do
not be among the heedless.
\end{quote}

He was a the author of ``Jawāhir al-kalām fī sharḥ sharāʾiʿ al-islām,''
a book about Shi'a jurisprudence. By this point, jurisprudence sometimes
had used the Qur'an as a source, but typically secondary sources were
used to determine legal and ethical standards. The emphasis of sources
other than God created a veil for al-Najafi and those who regarded
al-Najafi as a primary source of understanding. Al-Najafi failed his
followers.

\subsection{Hajjī Mirza Muhammad Karim Khān
Kirmānī}\label{hajjux12b-mirza-muhammad-karim-khux101n-kirmux101nux12b}

Baha'u'llah mentions Karim:

\begin{quote}
Recall Karim\footnote{Kitab-i-Aqdas \#170}, when We summoned him to God,
and he grew arrogant, following his own desires, even after We sent to
him that which delighted the eye of proof in the realms of existence and
completed the argument of God upon all in the heavens and the earth. We
commanded him to turn in acceptance, as a favor from the
Self-Sufficient, the Exalted. Yet he turned away, retreating, until the
hosts of torment seized him, as a just decree from God. Verily, We were
witnesses.
\end{quote}

The following information is a summary from the Hurqalya Publications
from the University of California Merced website as of 2025.\footnote{\url{https://hurqalya.ucmerced.edu/node/24}}
Karim was the self-appointed third leader of the Twelver Shi'a Shaykhi
movement.

The Shaykhi movement had started with Shaykh Ahmad al-Ahsa'i, who became
a mujtahid who studied in Karbala at around age 40. A mujtahid was a
scholar who was recognized to perform ijtahid, deriving legal rulings
based on the Qur'an, Hadith, and other secondary sources. Shayhk Ahmad
eventually had visions that the Twelfth Imam would return in 1260AH
(late 1843 or early 1844 AD), after 1,000 years of occultation. After he
passed away in 1826, his appointed successor Siyyid Kazim Rashti
continued this teaching through a newly formalized school called the
Shaykhi School. Prior to passing away in 1843, Siyyid Kazim had
instructed his students to search for the Qa'im (the returning Twelfth
Imam).

Karim did not participate in the search. When one of the Shaykhi
students, Mulla Husayn, first recognized the Bab on the evening of May
22, 1844, the Bab became the nominal leader of the Shaykhi School as the
Qa'im and Twelfth Imam. Karim did not follow the Bab. Instead, he wrote
a refutation called ``The Crushing of Falsehood in Refutation of
Babism.'' The same year the Kitab-i-Aqdas was penned, Karim had passed
away. His last act was the writing of a Will and Testament appointing
his hereditary descendants as leaders. The Shaykhi School still exists
and is currently led by Hajji Sayyid Ali Musawi al-Hifzih.

\subsection{Mirza Yahya Nuri
(Subh-i-Azal)}\label{mirza-yahya-nuri-subh-i-azal}

Baha'u'llah addresses Mirza Yahya:

\begin{quote}
Say: O Manifestation of rejection, abandon concealment and then speak
the truth among the people. By God, My tears have flowed down My cheeks
upon seeing you inclining toward your desires and turning away from the
One who created and fashioned you. Recall the favor of your Lord when We
nurtured you through nights and days for the service of the Cause. Fear
God and be among the repentant. Suppose your matter has been mistaken by
the people---can it be mistaken by yourself? Fear God, and then remember
when you stood in the presence of the Throne and wrote what We revealed
to you of the verses of God, the Almighty, the All-Powerful, the
All-Capable.

Beware lest pride prevent you from turning toward the One True God. Turn
to Him and do not fear your deeds, for He forgives whomever He wills by
His grace. There is no God but Him, the Forgiving, the Generous. We
counsel you for the sake of God; if you accept, it is for your own soul,
and if you turn away, your Lord is independent of you and those who
follow you with a clear delusion. God has taken the one who misled you.
Return to Him, humbled, submissive, and lowly, for He will forgive your
misdeeds. Your Lord is indeed the Oft-Forgiving, the Almighty, the
Merciful.\footnote{Kitab-i-Aqdas \#184}
\end{quote}

Mirza Yahya Nuri is Baha'u'llah's half-brother. When he completed his
first Bayanic cycle at 19 years of age, the Primal Point had instructed
him to execute His Will and Testament.\footnote{The Testament of the
  Honored Primal Point (BB00210)} Yahya, who was designated with the
name Subh-i-Azal by the Bab, was the appointed custodian or steward. By
this point in time in 1850, he was considered as a mirror, reflecting
the light of God through the Bab. Yahya was not a Manifestation of God.

The testament instructs Subh-i-Azal to do specific acts. The custodian
is instructed to safeguard and transmit the writings of the Báb without
alteration, to distribute entrusted texts and responsibilities among
designated individuals and regions, and to ensure unity, reverence, and
fidelity to what has been revealed. He must act only within the limits
given, neither adding nor subtracting anything, and preserve the Cause
in its existing form until God reveals the next Manifestation. He is to
coordinate the dissemination of the writings, protect them from misuse,
and ensure that no individual claims excessive authority or possession.
His role is custodial rather than sovereign: to maintain order, prevent
division, uphold spiritual integrity, and ultimately surrender all
authority the moment God makes His will manifest through another. The
testament expressed the possibility God could manifest Himself again
within the lifetime of Azal.

8 years later in 1858, Baha'u'llah wrote the Tafsir-i-Hu, an explanation
of the Name of God ``He.'' In it, Baha'u'llah affirms Yahya's status as
a Mirror, whose purpose is to reflect divine names and attributes. Yahya
is fully capable of reflecting the light of God if he remains faithful
to God's command. It seems Yahya was at the threshold of remaining
aligned and potentially turning away. In the same tablet, Baha'u'llah
warns Yahya of being proud and exalting himself beyond his role. The
Mirror must not mistake itself as the Source.

In 1863, Baha'u'llah had publicly announced that He is ``He Whom God
Shall Make Manifest.'' This was 19 years (a full Bayanic cycle) after
the dawning of the Bab, and 13 years after His instruction to Mirza
Yahya. Yahya did not recognize Baha'u'llah as a Manifestation of God. By
1868, when both had been taken to Constantinople and Edirne, Baha'u'llah
revealed the Kitab-i-Badi. In it, Baha'u'llah describes how Mirza Yahya
had violated the trust given to him by the Bab, how he was no longer
acting as a custodian. He left the Writings behind in Persia when he
migrated to Baghdad. Baha'u'llah says Yahya focused more on concealment
than custodianship, withholding the Bab's Texts, obstructing their
circulation, and centralizing authority through himself instead of
allowing the Bayani communities to be led by the Bayan.

Outside of custodianship, Baha'u'llah describes Yahya as acting contrary
to what God desired of him. Yahya may have been ruled more by fear of
others or a fear of losing prestige instead of fearing God. He acted
with jealousy, attachment to status, and allowed these feelings to
impair his moral conduct. Baha'u'llah accuses him of sowing suspicion,
manipulating relationships, and permitting and encouraging harm to
others. Yahya struggled with restraint, especially towards women and
potentially drugs such as opium and hashish, which may have further
multiplied his erratic behaviors. Baha'u'llah instructs Yahya's
followers to read Yahya's Mustayqiz as evidence of Baha'u'llah's
observations.

In the last year of Baha'u'llah's life on Earth in 1891, He continued to
offer love and forgiveness towards Yahya. He instructs His followers to
send someone to Cyprus to look after Yahya with the hope Yahya will
recognize God's love. Still, at this time it was observed some of
Yahya's followers put Yahya's image above others. His self-exaltation
led to others exalting him as an idol.

\section{Protecting Against Corrupt Spiritual
Leaders}\label{protecting-against-corrupt-spiritual-leaders}

In the examples of Hasan al-Najafi, Karim Kirmani, and Mirza Yahya, we
see different ways how spiritual leadership was used to self-exalt
themselves and mislead others into believing in their exaltation. Each
leader claimed rights for themselves which God did not allow, and failed
their responsibilities as trustees over God's trust (the believer).
Al-Najafi relied on secondary sources to interpret, Karim self-appointed
himself and a hereditary successorship, and Yahya failed in his
trusteeship. These three examples are as relevant today as they were
when the Kitab-i-Aqdas was written.

By the time each had passed away, their souls were chained by their
self-passion and self-indulgence. They failed to liberate themselves and
worse yet, they inhibited the liberation of souls who turned towards
them. As they pass through the spiritual worlds of God, my hope is they
are able to be free of their constraints and can truly be set free.

Chapter 18.1 offers teachings which help protect the community from
these types of leaders. These teachings include any claims of
authoritative interpretations, of any type of infallibility, boasting,
leading others to beg, and other prior religious practices such as
ascending pulpits. People must constantly be wary of charismatic
personalities who seek to use such claims to secure more rights than God
allows and from being responsible over which God had commanded.

\subsection{Qualities of Disbelievers}\label{qualities-of-disbelievers}

To help discern which people may be corrupt spiritual leaders,
Baha'u'llah does offer much counsel throughout the Kitab-i-Aqdas. This
will be a list of qualities attributed to disbelievers, to include those
who claim belief and desire leadership.

\begin{itemize}
\item
  They seek prestige to any degree.\footnote{Kitab-i-Aqdas \#36}
\item
  They claim hidden knowledge and esoteric understanding.
\item
  They cling to their own principles.\footnote{Kitab-i-Aqdas \#17}
\item
  They are sorrowful.\footnote{Kitab-i-Aqdas \#35}
\item
  They deny what God has permitted.\footnote{Kitab-i-Aqdas \#36}
\item
  They treat outward actions as a substitute for inner sincerity
  (Example: ``I pray in public, therefore I am righteous'').
\item
  They hesitate in the Cause of God.\footnote{Kitab-i-Aqdas \#162}
\item
  They are wolves in sheep's garments.\footnote{Kitab-i-Aqdas \#52}
\item
  They are intoxicated by desire.\footnote{Kitab-i-Aqdas \#39}
\item
  They distort the Word of God.\footnote{Kitab-i-Aqdas \#105}
\end{itemize}

\subsection{Your Responsibility to Remain
Free}\label{your-responsibility-to-remain-free}

The foundation to ensuring you do not follow corrupt spiritual leaders
is to adhere to believe in the rights you have. Believe these rights
extend to everyone. Believe in your responsibilities derived in your
belief in God. Believe in the responsibilities of those who do lead in
any manner. Learn how the trusteeship model can be applied in any
spiritual leadership role. Baha'u'llah did not prohibit clergy, nor
religious scholarship. Be willing to lead if you feel you are capable
and can withstand the temptations of disbelief. Be willing to allow
others to lead if another is capable and can withstand the temptations
of disbelief. Finally, use your spiritual practice and your
constellation of virtues to help you discern the various ways spiritual
leadership can be manifested.

As we have a responsibility to turn towards scholars, Baha'u'llah also
describes what should happen once He was no longer among us. After His
passing in 1892, what was supposed to happen? The next chapter will
describe spiritual successorship.

\newpage{}

\chapter{Spiritual Successorship}\label{spiritual-successorship}

Spiritual successorship describes Baha'u'llah's plans for when He is no
longer with us in person. The Kitab-i-Aqdas has a few paragraphs which
outline His vision. This chapter will discuss these plans, as well as
Baha'u'llah's explanations of these plans in His other writings. What
was the future supposed to look like from 1892 up to today? How will we
shape the future in Baha'u'llah's vision after today?

\section{BH11278 (The Book of My Testament After
Me)}\label{bh11278-the-book-of-my-testament-after-me}

I want to introduce The Book of My Testament After Me, designated as
BH11278. I am unsure exactly when it was written, but as Baha'u'llah
Himself named it the ``Testament After Me,'' I feel it would still be
relevant at any point after Him.

The book is uncompromising in its nature of spiritual successorship.
Authority passes through God to the Manifestation and is
non-transferable. Baha'u'llah says ``Can anyone share with Him in
authority? Nay, by the Lord, though all may claim such for themselves.''
The true focal point of authority is the Book itself. No one may alter,
interpret, or legislate apart from what God has revealed. He asks ``Do
you find any changer besides God or any interpreter apart from Him?''

Succession is about recognition, not an inheritance. This recognition is
borne of belief and submission to the Divine command, even when the
Manifestation is no longer among us. Spiritual legitamancy is measured
by fidelity to the divine light. Baha'u'llah says ``The Remnant of God
in these days is the Manifestation of Light.'' Without this light,
claims of authority are like shadows, they don't reflect the light in
the mirror of their soul. It may have form, but it lacks substance.

Finally, BH11278 establishes a model of successorship where all
authority after the Manifestation is custodial, conditional, and
answerable to God alone. It is not absolute, it is not infallible, and
it does not grant any rights to any leader unless explicitly granted by
God.

\section{Endowments}\label{endowments}

Baha'u'llah says ``no one is permitted to manage them except with
permission of the Source of Revelation.''\footnote{Kitab-i-Aqdas \#42}
He continues by saying endowments is passed to the Aghsán, after them
the institution of the House of Justice if it is established in the
land. If not, endowments revert to the people of Baha. These endowments
can only be used as specified in the Book. The specifications are for
charitable purposes and in elevated places for the Cause.

In BH08767, Baha'u'llah says all endowments revert to the Huquq'u'llah.
Chapter 12.4 describes the various ways Baha'u'llah had utilized these
endowments. This was reiterated by Baha'u'llah in BH00979.

Baha'u'llah begins the Kitab-i-Ahd (The Book of My Covenant) with a
sermon regarding wealth. The Kitab-i-Ahd, which is reflects His Will and
Trust says He did not leave treasures. ``There is hidden fear and
concealed danger in wealth\ldots{} The wealth of the world is
untrustworthy.'' Baha'u'llah adds specifically for the Aghsan that ``God
has not granted them any right to the property of others.'' Outside of
collecting Huquq'u'llah as an endowment for charity and the Cause of
God, the Aghsan had to still earn a living. They have no right to
collect any additional property from anyone. They are not the Cause of
God.

\section{Aghsán (The Branches of
Baha'u'llah)}\label{aghsuxe1n-the-branches-of-bahaullah}

There are 2 collections of letters Baha'u'llah wrote which are unnamed.
One is BH00023 which are letters to the Yazd community through Varqa, a
well-respected follower in the time of Baha'u'llah. The other is
BH00057, addressed to believers in Zanjan. Each has a portion which is
entirely identical to each other, and both were written in 1881. They
are explanations of certain parts of the Kitab-i-Aqdas, and will be
referenced regularly in this chapter.

With the revelation of the Kitab-i-Aqdas and thereafter, Baha'u'llah
would refer to branches which were not Himself or a Manifestation of
God. Baha'u'llah defines the Aghsan as ``the present branches.'' Among
them are two Great Branches. After the present branches, souls are
considered fruits and leaves. The branches are His sons. The Great
Branches are the Ghusn-i-Azam (Abbas Effendi) which means Most Great
Branch and then Ghusn-i-Akbar (Mirza Muhammad Ali) which means the Most
Mighty Branch. Both are titles given by Baha'u'llah are relate to names
of God. There were two other present branches, Ghusn-i-Anwar
(Badi'u'llah) whose title means Luminous. There was also Diya'u'llah,
who I am unable to locate a given Ghusn title. BH00017 (unnamed) does
describe Diya as a Branch. Diya means Radiant. Another son,
Ghusn-i-Athar (Mirza Mihdi) had passed away while in the prison in Akka
in 1870. His title meant Pure.

For every son, there was either a title or name which related to a name
of God. In Chapter 1, we described many of the names of God as well as
potential dangers if we remove a name of God from our belief. I consider
the entire tree of God, the Sundrat-ul-Muntaha, would bear the most
fruits if all the branches were allowed to manifest these names of God
fully.

In terms of endowments, after Baha'u'llah passed away, the Aghsán were
responsible for endowments such as Huquq'u'llah. Baha'u'llah did not say
one Branch, but all were responsible until the Houses of Justice are
created. The Aghsán were to operate as a consultative body, where once
again, consultation is guided by the Holy Spirit. They were to serve as
trustees and examples of the future Houses of Justice. They were not to
profit from these endowments. This did not happen. As one Branch tool
sole responsibility of endowments, the Holy Spirit was not able to guide
their use towards charity and the elevated places of God, such as
Mashriq'ul-Adhkars, and the Sacred Houses in Shiraz and Baghdad.

After Ghusn-i-Akbar (Mirza Muhammad Ali) passed away, the Aghsán were no
longer responsible for endowments. Until there are Houses of Justice,
such as in your city, these endowments are to be managed by the people
of Baha. The people of Baha are not to profit from them.

\subsection{Serving on the Throne}\label{serving-on-the-throne}

After the Kitab-i-Aqdas was written and Baha'u'llah and His followers
gained more freedoms in the Akka and Haifa areas, He would receive
believers as pilgrims and letters from across Persia and the Ottoman
Empire. In the context of receiving believers, their news, requests, and
giving decisions, support, and guidance, Baha'u'llah would refer to His
place as the Sacred Throne with the Sacred Court.

The Aghsán served this Throne and Court. Abbas Effendi and Mirza
Muhammad Ali during the entire period from 1873 - 1892. As Diya'u'llah
and Badi'u'llah aged into maturity, they also served. The entire family
was in support in various capacities. Unrelated followers, such as Mirza
Aqa Jan, served as an amenuencis who recorded Baha'u'llah's words in His
pen. For each Aghsán, I will provide an example of their service and
another example of the praise Baha'u'llah gave them.

\subsection{In Honor of Abbas Effendi}\label{in-honor-of-abbas-effendi}

In respect to his service:

\begin{quote}
``O My Name! Upon thee be My Glory! The~Most Great Branch~hath presented
thy letter and mentioned it before the face of the Wronged One. We have
answered thee with that which no treasuries of utterance can equal, and
when it was sent down, the necks of the just bowed before
it.''\footnote{BH00314}
\end{quote}

In honor of his station:

\begin{quote}
``He is the One Who speaks as He pleases

O My Most Great Branch! Before the Wronged One has appeared thy letter,
and We have heard that which thou didst commune with God, the Lord of
the worlds. We have made thee a protection for all the worlds, and a
guardian for all who are in the heavens and on earth, and a fortress for
those who have believed in God, the One, the All-Knowing. We beseech God
to protect them through thee, to enrich them through thee, to nourish
them through thee, and to inspire thee with that which will make thee
the Dawning-Place of wealth for all created things, the Ocean of bounty
for all in the world, and the Dayspring of grace unto all peoples. He,
verily, is the All-Powerful, the All-Knowing, the All-Wise. We beseech
Him to water through thee the earth and all that is therein, that there
may grow from it the grass of wisdom and utterance, and the ears of
knowledge and understanding. He, verily, is the Protector of whosoever
turneth unto Him and the Helper of whosoever calleth upon Him. There is
none other God but Him, the Mighty, the Praised One.''\footnote{BH09144}
\end{quote}

\subsection{In Honor of Mirza Muhammad
Ali}\label{in-honor-of-mirza-muhammad-ali}

In respect to his service:

\begin{quote}
``O Jamalu'd-Din! Upon thee be the Glory of God, the True, the Manifest
King. The letters which thou didst send were presented before the Most
Holy Court by the Greater Branch. We beseech God to assist thee, to
confirm thee, and to aid thee in that which will draw thee closer to Him
in all conditions, and to exalt through thee His mention among His
servants. In these days certain passages have been revealed under the
name of ``Traveler'' - there would be no harm if they were sent to
European lands.''\footnote{BH07364}
\end{quote}

In honor of his station:

\begin{quote}
``Blessed is he who seeketh shelter beneath the shade of the Branch of
God, his Lord and the Lord of the Throne and the Lord of all the worlds.
O My Branch! Be thou as the spring cloud of My loving-kindness, then
shower upon all things in My wondrous Name. O My Branch! We have chosen
thee by virtue of Him Who is the Chosen One choosing thee. Say: Praise
be unto Thee, O God of the worlds! O Greater Branch! We have chosen thee
for the triumph of My Cause. Arise to achieve a wondrous triumph. Subdue
the cities of names through My Name, the Sovereign over all that He
willeth. O thou Ocean! Surge by My Name, the All-Compelling, the Most
Great. Open thou the cities of hearts through My Name, the Best-Beloved,
the Mighty, the Impregnable. Every deed hath been made dependent upon
thy love. Blessed is he who hath attained unto that which his Lord, the
All-Knowing, hath willed. Blessed is he who hath heard thy call and
turned to thee out of love for God, the Lord of the worlds.''\footnote{BH09434}
\end{quote}

\subsection{In Honor of Diya'u'llah (Also known as
Ziya'u'llah)}\label{in-honor-of-diyaullah-also-known-as-ziyaullah}

In respect to his service:

\begin{quote}
``O people of Manshad! Upon you be the Glory of God, His grace, the
mercy of God and His peace. Verily the Kingdom of Utterance hath turned
toward you and made mention of you in such wise that all mention in the
world hath bowed low at its revelation, that ye might rejoice and be
numbered among the thankful. His honor Diya, one of My Branches, hath
presented your letter and made mention of you before Our
presence.''\footnote{BH06516}
\end{quote}

In honor of his station:

\begin{quote}
``At this moment, one of the Branches of the Blessed Tree, His Holiness
Diya, is present before Us. We have commanded him to pen this Tablet,
from whose horizon's heaven hath shone forth the light of God's mercy,
the Lord of the worlds.''\footnote{BH03971}
\end{quote}

\subsection{In Honor of Badi'u'llah}\label{in-honor-of-badiullah}

In respect to his service:

\begin{quote}
``O Mirza 'Ali-Akbar-i-Milani, upon him be the Glory of God!

In My Name, the Caller between earth and heaven! One of My Branches, who
hath been named Badi'u'llah, hath presented thy letter before Our Face.
We answer thee through this perspicuous Tablet.''\footnote{BH11823}
\end{quote}

In honor of his station:

\begin{quote}
``He is the Mighty. The fire of separation, after the comfort of the
days of union and communion, has burned the soul and melted the heart.
The heart is with you - ask it about the state of affairs. We hope
through God's grace that He will soon grant nearness and reunion. We
long to see and yearn to speak with the Spirit of Life, Ziya'u'r-Rahman,
that we may hear sweet words from those sugar-sweet lips in Samarkand.
We remember Badi'u'llah with our tongue, seek him with our heart, and
yearn for him with our soul. May my soul be sacrificed for thy nearness
and thy words, O Badi'.''\footnote{BH10221}
\end{quote}

\section{Answering Questions About the
Book}\label{answering-questions-about-the-book}

Baha'u'llah instructs us to ``refer what you do not understand from the
Book to the Branch that has branched forth from this mighty
Root.''\footnote{Kitab-i-Aqdas \#174} He further explains this in the
letters to Yazd and Zanjan. He says the Book refers to the Kitab-i-Aqdas
and the Branch that has branched forth from this mighty Root refers to
the Aghsán.

This is an extremely specific responsibility for the Aghsán as a
consultative body. As we have learned throughout this book, Baha'u'llah
repeatedly nurtures this sense of self-discovery, these opportunities to
immerse yourself within God's consciousness on your own terms. This is
your experience. This is your soul. This is your free-will. Liberation
is your right, and liberation is the responsibility we have to ourselves
and others. With this said, let's look closely at this verse.

Baha'u'llah starts the counsel with a conditional statement. ``Refer
what you do not understand.'' Baha'u'llah invites us in paragraph \#53
to refer what is revealed from Him. We have a responsibility to attempt
understanding through the spiritual practices of recitation, reflection,
and where possible, actions which embody these teachings. Sometimes
understanding unfolds over time as we pass through the spiritual worlds.
Sometimes we feel stuck on an idea. To refer what we do not understand
does not absolve our responsibility to attempt understanding on our own.
We refer only once we have tried.

The second phrase says ``from the Book.'' This phrase purposefully
narrows the scope even further. In BH00023 and BH00057, the Book is
explicitly defined as the Kitab-i-Aqdas and the Kitab-i-Aqdas only. This
means we refer only what we do not understand of the Kitab-i-Aqdas. This
does not include any other book or writing.

As it appears, the Aghsán were to consult together where any particular
believer attempted to understand the Kitab-i-Aqdas but were unable to.
The result of this consultation would be guided by the Holy Spirit.
Baha'u'llah did not express any teaching the understanding of the Aghsan
would replace other believer's understandings. The consultation was for
the individual who did not understand, completely within the context of
what they asked and understood already. I envision the Aghsán's role to
be like that of scholars Baha'u'llah commanded us to turn to. While
guided by the Holy Spirit, it may not encompass the entirety of the Most
Great Ocean. This consultation would be incredibly valuable for the
community.

Again, this did not happen.

\section{The Branch to Turn Towards}\label{the-branch-to-turn-towards}

The last part of the Kitab-i-Aqdas which describes spiritual
successorship is from the paragraph 121.\\
``When the sea of reunion is stilled, and the Book of Origin reaches its
end, turn towards Him whom God has willed, the one who branched from
this ancient Root.'' The letters to Yazd and Zanjan also explain this.
Baha'u'llah responds clearly to the inquiry:

\begin{quote}
``The divine intention refers to His Holiness, the Most Great Branch,
and after him, His Holiness, the Most Mighty Branch. My spirit, essence,
and being are a sacrifice for the dust of their feet.''
\end{quote}

Baha'u'llah bestows great honor upon Abbas Effendi and Mirza Muhammad
Ali. Baha'u'llah used the phrase ``turn towards.'' I understand this as
Abbas Effendi has a responsibility, but he is not given any particular
rights. While holding a high station, Baha'u'llah's leadership model
does not change. Throughout the book, we describe how Baha'u'llah taught
trusteeship in every level of society. Parents and monarchs and every
level in between are to lead as trustees over that which they are
entrusted to.

The Aghsán still exist to manage endowments and to explain what is not
understood from the Kitab-i-Aqdas. I understand Abbas Effendi's role to
be custodial, much as Mirza Yahya's was to be. Baha'u'llah from 1873
until His passing in 1892 strove to teach the Kitab-i-Aqdas, show love
towards His family and kinsmen (all believers), and nurture the
liberation of souls in an environment which was quite oppressive. The
Aghsán were to be an example of the consultation all were capable of, if
they believed.

None of the Branches had the authority to deprive any person of their
rights, to deny anyone of their responsibilities, nor to change the law
of Baha'u'llah. Again, this did not happen.

\section{The Kitab-i-Ahd (The Book of My
Covenant)}\label{the-kitab-i-ahd-the-book-of-my-covenant}

After Baha'u'llah passed away, this document was read and shared with
the community. We shared the very first teaching within it regarding
wealth. Baha'u'llah teaches the purpose of His revelation was ``to
extinguish the flames of malice and hatred, so that the horizons of the
hearts of the people of the world may be illuminated with the light of
unity and attain true tranquility.'' He promises our human station is
great. He reminds us to hold firm to the Divine Command.

Baha'ullah describes all religion to be for love and unity. Strife and
contention are forbidden. He offers blessings for those who lead,
especially those who believe in the Divine Command. He purpose is for
the Kitab-i-Aqdas to shine radiantly and rise in the horizon. He warns
us not to make the Cause of God a tool for disorder and division. We are
to say ``All are from God.''

Baha'u'llah reiterates His explanations from the letters to Zanjan and
Yazd. He says ``We have chosen the Most Mighty Branch after the Most
Great Branch as a matter from the All-Knowing, the All-Aware.'' He
counsels love for the branches is incumbent upon everyone, without
exceptions. He later says respect and regard for the branches are
required, as well as for the entire family of Baha'u'llah, the Afnán
(the Bab's family). Finally, all are counseled to serve the nations and
reform the world. I want to clearly state one thing about the family. In
the letters which have been mostly been unavailable to the People of
Baha, the Sacred Family is honored repeatedly for their service and
devotion to the Cause of God. Their respect and love is well-earned.

Sadly, these things in the Kitab-i-Ahd did not happen.

\subsection{The Testament Requires
Trusteeship}\label{the-testament-requires-trusteeship}

The Kitab-i-Ahd and BH11278 were both testaments from Baha'u'llah. Abbas
Effendi was entrusted to execute the final testament of Baha'u'llah. He
was to witness, to be a trustee, and to help. Nothing more and nothing
less. It was an incredible honor and reflection of the high hopes
Baha'u'llah had for his eldest son. It was also a difficult position to
be in. I will not explain further, but I can only say Baha'u'llah's
final testament was not fulfilled. God's All-Knowing nature, while aware
of all the probabilities any outcome could have, knew there was a chance
of success and a chance of failure. I feel the greatest failure stemmed
from self-exaltation and the proceeding inability to consult. Remember,
if consultation is guided by the Holy Spirit, what is guiding
authoritarian rule?

I want to close this section on the Aghsn with an excerpt from the
Tablet of Khalil, written sometime when Baha'u'llah was in Edirne. When
asked about His sons (who are not called Branches yet), He says:

\begin{quote}
As for what you asked about my son, know that if my sons follow God's
laws and do not exceed what has been specified in God's book, the
prevailing, the Ever-Existing, and they command themselves and the
servants to do good, and they forbid evil, and they testify to what God
has testified in His decisive verses, the conclusive, the definitive,
and they believe in whoever God reveals on the day in which the times of
the former and the latter are counted, and on it, everyone presents
themselves to their Lord, and they will not disagree on God's command
and will not stray from His ordained, written law. Then know that they
are leaves of the tree of monotheism and its fruits, and with them, the
clouds rain and the clouds lift with grace if you truly believe. They
are God's household among you and His family in your midst, and His
mercy upon the worlds if you know. From them, the breeze of God blows on
you, and the winds of dignity and love pass over those close. They are
God's pen, His command, and His word among His creatures, and with them,
He takes and gives if you understand. Through them, the earth has shone
with the light of your Lord, and the signs of His grace have appeared to
those who do not deny God's signs. However, those who hurt them have
hurt me, and those who hurt me have deviated from God's path, the
prevailing, the Ever-Existing. So, you will find the deviation of the
deviators and their arrogance towards us and their transgression against
ourselves without clear evidence or a preserved book.

Say, O people, they are God's signs among you, beware of arguing with
them, or killing them, or be among those who oppress and do not realize.
They are God's secret on earth and returned under the hands of the
oppressors on this earth that fell behind the elevated mountains. All of
that was returned to them at the time when they were young in the
kingdom, and they had no sin but in the path of God, the Capable, the
Powerful, the Mighty, the Beloved, and those from them who appear
naturally and God runs from their tongues signs of His power, and he is
among those God has chosen for His command. There is no god but He, to
Him belongs creation and command, and we are all commanded by His
command. We ask God to make them successful in obeying Him and to
provide them with what pleases their hearts and the hearts of those who
inherit Paradise from God, the Mighty, the Prevailing, the
Ever-Existing.
\end{quote}

It seems the sons were our greatest test.

\section{The Path Forward}\label{the-path-forward}

While the Aghsán were unable to fulfill their purpose, we still have a
path to move on. Baha'u'llah taught after the Aghsán, if the Houses of
Justice are not yet established, the People of Baha are the spiritual
successors. As you read this, you might be a Person of Baha. I invite
you to this path forward.

There are many roles we may take on as trustees. We can serve as
scholars, we can organize remembrance services and the melodious
recitation of the verses of God, we can promote our Holy Days and
festivals, and we can nurture each other's path. We can manage
endowments for the needy, and strive for the elevated places of the
Cause. The possibilities are endless as long as you believe.

I believe this all starts with honoring not only the Kitab-i-Aqdas, but
also Baha'u'llah's Testaments. If you hold any malice towards any of
Baha'u'llah's family, let it go. It is in the past. Pray for their
spiritual well-being and union with God. If you hold onto any notions of
infallibility, let it go. Infallibility is reserved for the next
Manifestation of God which will exist no earlier than 1,000 years after
Baha'u'llah, and no later than the Bab's concept of Mustagith (about
1,500 years). Today is today and tomorrow is tomorrow.

The next chapter will focus in on Trusteeship Governance, where we put
the various concepts of Part 4 together. This is a path we can all walk
on together in the name of Baha, and in honor of His ministry.

\newpage{}

\chapter{Peace}\label{peace}

\section{The Most Great Peace}\label{the-most-great-peace}

In Chapter 22, we describe the world-wide House of Justice's role in
establishing the Most Great Peace, as described in the Ninth Leaf of
Paradise in the Words of Paradise. The Most Great Peace is the
unification of the world's affairs towards a common goal or the
unification of the world's religions as one. Holy war is strictly
prohibited, even if another religion or religious institution prescribes
it.\footnote{Lawh-i-Bisarat (The Tablet of Glad-Tidings)}

\subsection{The Lesser Peace}\label{the-lesser-peace}

Baha'u'llah's Epistle to the Son of the Wolf is a great review of many
overarching themes Baha'u'llah had taught throughout His ministry. One
of these themes is the Lesser Peace, which He describes as the greatest
cause of the preservation of the world. It is not the only cause for the
preservation of the world, but I do read this as there is risk the world
may not be preserved if the Lesser Peace is not pursued.

The first responsibility to establish the Lesser Peace are the kings of
the world, who are the dawning places of wealth and power. Baha'u'llah
describes they should establish an assembly where they or the ministers
who act on their behalf, to create a decree of unity between nations.
They should also agree to limiting weapons to only that which is
necessary for the defense of their nations. Baha'u'llah says if any
sovereign leader rises against another, a declaration of war, all the
other nations should prevent this offense. Instead of weapons, resources
will be used for the reformation of the world.

The purpose of the Lesser Peace is not solely for the preservation of
the world. The purpose also includes the establishment and preservation
of the human rights of all, which is a way to ensure liberation while
the soul is in the physical domain of Earth. The people who live under
the Lesser Peace will be engaged in comfort and joy, and the
lamentations of most will cease.

\subsection{Trusteeship Is Required for
Peace}\label{trusteeship-is-required-for-peace}

Trusteeship is required in order to achieve either form of peace. World,
religious, and even business leaders need to be able to view their
wealth, power, and influence not within terms of domination, extraction,
extortion, nor status but instead to view their uncanny abilities to
lead as a trusteeship towards all. This concept is affirmed in the
Kitab-i-Ahd's final paragraph, where Baha'u'llah states His Will and
Testament. Those who lived when Baha'u'llah lived would hear his final
Words being read, where He says:

\begin{quote}
``I counsel you to serve the nations and reform the world. From the
Kingdom of the Bayán, what has been revealed is intended for the life of
the world and the salvation of the nations.''
\end{quote}

Every single member of humankind is to be a trustee, if they are
capable. To have liberation, we need to be responsible for another's
liberation. This does not happen automatically. This book has
established all of the foundations for us to achieve the Lesser or the
Most Great Peace. Let's put everything we have learned together and see
what we can do to achieve God's vision through Baha'u'llah.

\section{A World Assembly}\label{a-world-assembly}

The Lesser Peace is conditioned upon a world assembly. An assembly is
any group of people who have chosen to meet for a common purpose. In
everyday life, assemblies can be related to legislative, religious,
educational, and social purposes. A world assembly would not be a world
government. Nations do not lose their sovereignty.

There is no prerequisite required to convene a Lesser Peace assembly.
This means this is not limited to the type of government a nation has.
The participating governments can be decentralized or centralized power
structures, capitalist or socialist economies, single-party or
multi-party, and led by civilians or the military. Baha'u'llah did not
describe any qualifications other than the attendees are sovereign
leaders or the ministers acting on their behalf.

The assembly is not the United Nations. The United Nations is a
commendable attempt at creating a more peaceful world, but it was
designed with 5 nations have veto power on any security actions. Two of
these nations have attacked other nations preemptively in the last 20
years without any consequences by member nations of the UN. Another has
several disputed borders, and has recently claimed marine territory
other nations had controlled. The Lesser Peace assembly is only for
nations to agree to no longer attack each other. Once again, there are
no exceptions. The Lesser Peace assembly is for nations to disarm to
only what is necessary to defend their nations. While some nations
follow this guidance, a majority do not. The United Nations is not the
Lesser Peace assembly.

A world-wide House of Justice's sole purpose is to work towards this
assembly. This House of Justice does not have any authority at the
assembly, but they can be the organizers. The members do everything they
can to teach and spread the idea of a Lesser Peace and this type of
world assembly. Local Houses of Justice can also promote the Lesser
Peace within their jurisdictions, regardless of the authority level
granted to them by the people. Baha'u'llah also does not require there
to be a world-wide House of Justice for a Lesser Assembly to be
convened. Any sovereign leader can call the assembly, if the choose to
do so.

\section{A Common Affair}\label{a-common-affair}

The Most Great Peace has two modes. The first mode of achievement is to
work towards a common affair. An affair is a transaction or other
matters of public state business. These common affairs could be
identified through a world assembly similar to the Lesser Peace
assembly. One really good example of a successful common affair the
world has achieved since Baha'u'llah's call is when the nations of the
world banned chlorofluorocarbons (CFCs).

Common affairs could be a wide variety of issues. A good starting point
could be the rights discussed throughout this book, such as a right to
education. Nations might agree to promote various virtues associated
with education, such as pure truthfulness which is not affected by
ideology, state interest, or other power dynamics. Knowledge is not
withheld nor is opinion treated as fact.

Nations retain their sovereignty and cannot be forced. An affair must
genuinely be common. This means nations with relative strength cannot
bully nor coerce another state. The people of all nations have the same
rights listed in this book, even if the nation they live in does not
honor those rights.

Houses of Justice can at the city level can help be public figures of
these rights and causes. If a House of Justice is able to attain the
fifth level of authority, they can guide their nation towards the common
affairs of all nations. However, a sovereign leader can also call for
assemblies of common affairs as we progress to the Most Great Peace.

\section{Unification of Religion}\label{unification-of-religion}

The second mode of the Most Great Peace is the unification of religion.
I feel it might be common to view this principle as all people are part
of the same religion or a dominate majority. I see three potential
parallel pathways unification can take. The key is remember unity is
more about how our constellation of virtues guide each of us. The more
stars that shine brightly, the more light available to guide the right
paths.

Parallel Pathway \#1: The world increasingly becomes more aware of
Baha'u'llah and the Kitab-i-Aqdas. They learn about the Cause of God,
the liberation of souls and societies, and believe Baha'u'llah is from
God.

Parallel Pathway \#2: The religions of the world start to naturally
adopt many of the teachings and beliefs of Baha'u'llah, such as the
abolition of holy war and the full equality of women. These religions
retain their names, their rituals, and traditions but evolve.

Parallel Pathway \#3: The leaders of religions find points of common
belief between each other, choosing friendliness. They draw from each
others strengths. A good test of such pathway is how interfaith
marriages are perceived.

None of these pathways are mutually exclusive and each can occur
together. The world-wide House of Justice can work towards each of these
3 pathways, or any other pathway identified. Assemblies of religions can
help foster these pathways.

\section{A King Who Believes}\label{a-king-who-believes}

We have a responsibility towards any sovereign leader who arises to
serve the Cause of God. The Kitab-i-Aqdas \#84 says:

\begin{quote}
Blessed is the king who arises to support My Cause in My kingdom and
detaches himself from all else. He is one of the people of the Crimson
Ark, which God has made for the people of Bahá. It is fitting for
everyone to honor, revere, and assist him, that he may conquer cities
with the keys of My Name, the One Who rules over all in the realms of
the unseen and the seen. He is as the sight of mankind, the most radiant
crown upon the brow of creation, and the head of generosity for the body
of the world. Support him, O people of Bahá, with wealth and lives.
\end{quote}

\section{Belief, Spiritual Practice, Social Life and Affectionate
Relationships}\label{belief-spiritual-practice-social-life-and-affectionate-relationships}

We must be careful in how we pursue the Lesser and Most Great Peace. For
example, it is incredibly difficult for a person to proclaim peace and
unity if they struggle mightily to achieve such things in their personal
life. Likewise, a community which fails to provide a healthy social life
for its locality lacks the ability to prescribe more global measures.
While it is entirely possible for top-down centralized peace initiatives
to be formed, such as through a monarch who arises to serve the Cause,
every other institution Baha'u'llah described is highly decentralized.

Decentralization is fundamentally more difficult and throughout history,
decentralized institutions have regularly ceded increasing power and
control towards centralization. Crisis are especially noteworthy in
authorities seeking more control to avert the crisis, and out of fear,
people are willing to reduce their rights for an increased sense of
security. While rights are inherently popular, individual
responsibilities rarely are. I don't want to seem cynical, as I enjoy my
pleasures of life, but this has always been the most pressing
philosophical issue regarding governance. What is our responsibility,
and what am I willing to give up to pass that responsibility to a
government composed of individuals, who also have the same
self-interests I do?

This is why trusteeship is integral to every level of governance. For
each right we hope to have, Baha'u'llah gives us an equal
responsibility. Everything comes down to the individual. There is a
cyclical never-ending process we must practice to increase our chances
of creating a Lesser and Most Great Peace. First, we must believe such a
peace is possible. We believe in God, our souls, the spiritual worlds,
the Day of Resurrection, God's Command, and His Manifestation. We don't
have to believe perfectly, but we believe there is more than meets the
eye, and it inspires us. We perform our spiritual practices which help
inform the development of our virtues. These transform our personal
lives and prepare us for a healthy social life, full of potential for
affectionate relationships. Our belief is expressed in good deeds. These
deeds and experiences, even negative experiences, provide a feedback
look back into the belief. Every day is a new opportunity for
refinement. Practice may not lead to perfection, but it is the only
pathway towards perfection. Always keep practicing.

This feedback loop ensures we are as capable as any person to
participate in the emergence of peace. I recognize it can seem like hard
work, and potentially a significant reorientation of our lifestyles.
This work prepares us to be trustees. While being mindful Baha'u'llah
did not want us overburdened, we can actively pursue good deeds. This
can be as simple as helping an elderly person grab an item at a grocery
store or as complex as raising another's child. Being a trustee can also
utilizing your skills to actively work in social, civic, and political
roles.

If Baha'u'llah envisioned a King to arise to serve the Cause, it
requires not only an individual who believes, but an individual who
believes and is willing to act on that belief. While not everyone can be
a sovereign leader of a nation, there are countless ways we can serve as
trustees. Use your imagination. Baha'u'llah eliminated the
intermediaries who created veils over our imagination. Explore within
yourself, within your developing God consciousness. As peace grows
within you, that peace can help others attain peace.

\section{Towards the Last Chapter}\label{towards-the-last-chapter}

We cannot do any of this alone. The last chapter of this book is about
teaching the Cause. We will look at Baha'u'llah's guidance in teaching
others. How will people know there is a Kitab-i-Aqdas created to
liberate them starting today and for at least another 800 years?

\newpage{}

\chapter{Serving the Cause}\label{serving-the-cause}

Throughout the Kitab-i-Aqdas, Baha'u'llah enjoins upon us to serve the
Cause of God. This can be done through proclamation, teaching, and being
a good example. We are also to support those who serve, in whichever
ways we can.\footnote{Kitab-i-Aqdas \#117} This book has covered in
significant detail how to be a good example, or at least has shared the
framework the Kitab-i-Aqdas provides. Thus, this closing chapter will
focus on proclaiming and teaching.

\section{Proclaiming the Cause}\label{proclaiming-the-cause}

\subsection{The Abolition of Impurity}\label{the-abolition-of-impurity}

Baha'u'llah tells us to proclaim the Cause to the followers of other
religions.\footnote{Kitab-i-Aqdas \#75} This command comes immediately
after describing the abolition of the law of impurity from all things
and of other religions. This is a subtle but important foundation. The
first day of Ridván itself purifies. Remember how Ridván is an annual
festival whose purpose is to proclaim the Cause. Invite people to the
festival and proclaim the purifying effect of Ridván. It is a day when
people of all religions no longer have to view another as impure.

\subsection{Mention God Among Nations and
People}\label{mention-god-among-nations-and-people}

Baha'u'llah has permitted, but not required, people to learn various
languages to proclaim the Cause of God throughout both East and
West.\footnote{Kitab-i-Aqdas \#118} The Kitab-i-Aqdas also expresses a
couple of proclamations God had made Himself through Baha'u'llah. I
consider these proclamations to be examples of how to mention God among
the nations and people.

One example is for people to turn with radiant faces and hearts of joy
towards the Sidrat-ul Muntaha which proclaims ``There is no God but Me,
the All-Possessing, the Self-Subsisting.''\footnote{Kitab-i-Aqdas \#100}
Another example is when Baha'u'llah counsels us to be manifestations of
steadfastness when the Book proclaims ``There is no God but Me, the
Almighty, the All-Praised.''\footnote{Kitab-i-Aqdas \#134}

\subsection{The Great Announcement}\label{the-great-announcement}

These two simple instructions parallel most of Baha'u'llah's guidance in
what a proclamation consists of. When associated with the Day of
Resurrection, the proclamation is about the Great Announcement where God
has announced He has arrived to speak and guide humankind. Thus,
proclamation should be considered as an announcement. Ridván is the day
when Baha'u'llah first made His announcement to the world, and this
announcement continued throughout the rest of His ministry.

We must be steadfast, radiant, and joyous in our proclamation. If we
lack either trait or quality, the Great Announcement will not have the
full effect. There are only four things which need proclaimed. The first
is ``there is no God but Me'', the second is God has spoken again
through the name of Baha'u'llah, the Glory of God. The third
proclamation is the Cause of God. You do not even have to explain the
Cause of God in the proclamation, but that there is a Cause. A Cause can
also be known as an effect of God's Word. God exists, God speaks, and
God affects you. If you were to say what the Cause is, I would advise to
identify in one short sentence what the Cause is. For me, an effect
could be as simple as saying ``to usher in the Most Great Peace,'' ``to
liberate the souls from oppression,'' or ``to purify all religion.''
What is the Cause of God to you? If you believe in this Cause with
radiance, joy, and steadfastness, share it but share it simply. The
final part of the proclamation is the invitation. Invite the audience to
the Cause.

\section{Teaching the Cause}\label{teaching-the-cause}

The Kitab-i-Aqdas itself says very little about teaching, yet we know
teaching has a high station. With the inheritance law, teachers are the
only non-familial category to receive a share in the absence of a will.
If anyone chose to model their inheritance after the Aqdas, teachers
receive 3.6\%. This means not only are teachers entrusted with a task,
we are also entrusted to serve the teachers. There is no other
profession in the Kitab-i-Aqdas held with such a high responsibility.

Children are the first audience to teach, which is the responsibility of
parents and the Houses of Justice in the absence of parents being able
to fulfill this responsibility. Teaching is not limited to children, but
it is for anyone who is interested in learning. In this capacity, even
adults are children who are still developing their souls. Teaching comes
after proclamation, if anyone had responded to the proclamation. If no
one responded to the proclamation, we let go and move on.

Baha'u'llah instructs us adorn ourselves with good deeds, then through
wisdom and eloquence.\footnote{Kitab-i-Aqdas \#73} Baha'u'llah expands
upon this simple framework in an unnamed tablet BH02623. I'm personally
calling it ``Teaching with the Measure of Mercy.'' It is a thorough but
brief summary of all the ways Baha'u'llah Himself has taught others, and
how we can teach using His example. The entirety of this section is
derived from this tablet.

\section{Teaching With The Measure of
Mercy}\label{teaching-with-the-measure-of-mercy}

Teaching with the measure of mercy begins with orientation. Baha'u'llah
says the teacher is the ``spiritual physician and the true sage.'' A
sage is one who is acclaimed and respected for their experience,
judgment, and wisdom. Not every person who wants to teach is capable of
teaching, but through time, practice, and patience, one can be a true
sage if they desire to heal the spirit.

\subsection{Being a Spiritual
Physician}\label{being-a-spiritual-physician}

The second aspect of a spiritual physician being needed is there are
things in this world which have harmed the spirit of a soul. The teacher
must look towards two conditions as they seek to heal the spirit with
wisdom. The first condition is the requirements of the time and days.
This could be a good application of the helper, trustee, and helper
model of trusteeship. The teacher witnesses a common struggle people are
having. For example, in the United States it has been common enough for
religious leaders to sexually prey on children. There is considerable
distrust of religious institutions due to these oppressive corruptions.
Through witnessing such issues, a true sage will consider someone who
responds to the proclamation could be a person who experienced such
hardship in their youth. What might be the remedy for this issue of our
days? The second condition directly speaks to this, which is the
condition of the soul. If we witness a particular soul who has suffered
from this greater social issue, we can serve as their trustee. The
medicine we give them must be tailored to their condition in order to be
of help. If we lack the wisdom to give the proper medicine, find an
expert who can.

The best way to ensure we are qualified to be a spiritual physician is
to be immersed in the spiritual practices of the Kitab-i-Aqdas. Pray,
recite, remember, reflect, and honor God with regularity. This keeps the
mirror of the soul oriented and polished as well as it can be. As
spiritual practice is regular, without a rush, and with a sense of
detachment, teaching is the same way. When we seek to teach, we do not
give all treatments, all at once, without knowing the patient or
student. We take time to know them, for them to know us, and only
develop a treatment plan (teaching mode) once we know what needs
healing.

\subsection{Trauma of Bad Spiritual
Leadership}\label{trauma-of-bad-spiritual-leadership}

Baha'u'llah describes what happens when people are subjected to
spiritual leaders who are scholars of illusions. Remember the chapter on
the private self and all of the illusions one may face. Imagine a soul
who trusted a spiritual leader whose spirit was enraptured by their own
illusions and desires, not the salvation or well-being of a soul?
Baha'u'llah says this type of soul will be shaken by just the mention of
spiritual leaders, let alone the divinity of God. He says in cases such
as these, be wise by withholding teachings of command, power,
manifestation, and such. Perhaps the focus would emphasize God's names
of Mercy and Love, without expecting the soul to want to submit to any
type of spiritual leader. Guide the person to healing. Let them heal
from their spiritual trauma.

This trauma can be caused in a myriad of ways. I will not go in all the
ways, but know if trusteeship is meant to be practiced in every level of
human experience, so too have the expressions of oppression and tyranny.
Be receptive and know this world has a lot of pain to heal. Even the
teacher themselves could be healing as they heal.

\subsection{Milk Before Meat}\label{milk-before-meat}

Baha'u'llah uses the analogy that you do not give a child meat instead
of milk. They start with milk. The medicine of teaching is based on the
developmental capacity of the student. The milk begins with a goodly
character and pure deeds. The teacher is the example of this milk, and
teaches this goodly character and pure deeds. This duration we give milk
can vary, and is entirely dependent on the soul receiving the milk. The
person likes the ethics, but may not be ready for more difficult or
challenging teachings.

Once the soul's capacity has been nurtured and has grown, the next stage
is to give fruit. The fruit is guardianship (the Arabic word wilayat). A
wali could be someone viewed as a saint or friend of God. This would be
the stage where the teacher introduces Baha'u'llah as not merely a
teacher of good character, but also as a friend who loves them. They are
being prepared for maturity. They can view Baha'u'llah as an authority
without panic or with suspicion. They can start to trust Baha'u'llah
cares for them, and sees Him and their teacher as examples to follow.
The spiritual practices can be introduced, instructed, and practiced.
The soul explores these at their own pace and learn to develop a
personal relationship with God. Even if there is a challenging time, or
even a challenging teaching. Most importantly, the person feels safe to
be themselves, a more regularly refining version of themselves. The soul
may still wish God acted as they wanted, or that revelation more
conformed to their wishes. Yet, because of their trust in Baha'u'llah,
they still partake of their fruit. They remain friends.

Over time, the fruit has nurtured the soul and when they are ready, the
teacher may give them the meat. The meat would be the most difficult
aspects of belief. This could include expressing the full station of
Baha'u'llah. For many, just the idea there is not a final prophet is a
huge psychological challenge to overcome. Or in the case of someone
abused by a person who claimed to be a man of God, or acting on behalf
of God, this could be scary. It could take years to achieve the stage of
meat. Other forms of meat would be teachings which challenge the current
culture or commonly accepted notions of how things are. Meat can be
tough to chew, and has to be prepared just at the right temperature, for
the right duration, and with the right seasoning. Once the soul is
consuming this spiritual meat, they are at the stage where they are
willing to transform beyond their who they once were. They are
developing their own God consciousness. This is the stage of submission
to God's will.

\subsection{Every Soul is a Seeker}\label{every-soul-is-a-seeker}

Baha'u'llah explains that every soul is a seeker. They may not be
actively seeking in the moment, but at some point they will. The reason
why they stop seeking or turn away from God is due to what He calls the
``fancies arranged beforehand.'' These fancies are shaped by parents,
friends, religious leaders, non-spiritual leaders, cultural icons, and
other sources. Burning away these fancies is very difficult.

The only way to burn through these fancies is for the spiritual
physician to embrace mercy, kindness, and compassion. All souls were
created in stages, and even if we desire a soul to be in a different
stage for their own long-term well-being, we must never force upon the
soul more than they are ready for. The journey is not on the teacher's
terms. A seeker may seek, stop, seek, stop, and continue this cycle. The
true sage is detached from outcomes. All that is desired is the soul is
free from the oppression of prior institutions. A liberated soul can
love and navigate the spiritual worlds freely. Maybe this soul, nurtured
with milk, fruit, and then meat will help the next soul who suffered
with the ways of the world.

\section{The Kitab-i-Aqdas is the
Curriculum}\label{the-kitab-i-aqdas-is-the-curriculum}

I want to close this book with one more observation. In real life,
doctors work best when their top priority is the health of their
patients. In the United States, healthcare has many bureaucratic layers
which were originally designed to help facilitate the healthcare
process. Over time, the bureaucratic layers became the process and the
pathway to healing. The doctor has consumed more of their time in
checking the boxes and filling out the appropriate forms more than
actually being with their patients or leading care.

I believe this is why Baha'u'llah did not create any other curriculum or
any formal institution. The Teaching With Mercy method is purposefully
open-ended. Each stage of milk, fruit, and meat when working with a true
sage is specifically curated for the seeker's needs. There is no
standardized curriculum which works for every person. Yet, all potential
curriculum can be derived from the Kitab-i-Aqdas and the letters,
tablets, and books Baha'u'llah wrote to support the Kitab-i-Aqdas.

I hope this book had done a good job in demonstrating one way the
Kitab-i-Aqdas could be viewed. I also hope this book has inspired a
desire to explore the Kitab-i-Aqdas further, on your own terms. Finally,
I do also hope there was something in the book to inspire you. In all
truth, each chapter could have been a book in and of itself. There will
always be much more to explore. No matter what your future holds, I do
hope you believe you have a soul, its purpose is to be liberated, and
that you are willing to act as a trustee to your own soul. Thank you!

\newpage{}

\part{Appendices}

\chapter{Appendix 1: Names of God}\label{appendix-1-names-of-god}

The following is a list of the 19 groups of names of God and names which
fall within them

\begin{enumerate}
\def\labelenumi{\arabic{enumi}.}
\item
  Subtlety - Most Subtle
\item
  Manifestation - Manifest
\item
  Knowledge - All-Knowing, All-Informed, Wise, Knower of All Things
\item
  Creation - Dawning Place, Creator
\item
  Power - Almighty, All-Powerful, All-Capable, All-Subduing,
  All-Sufficient
\item
  Lordship - Lord
\item
  Justice - Judge, Just, Reckoner
\item
  Exaltation - Most Exalted, Most High, Great, Majestic, Most Glorious,
  Greatest Infallibility
\item
  Independence - Self-Sufficient, Self-Subsisting, Independent
\item
  Command - Commander, Ordainer, Fulfiller
\item
  Counsel - Counselor, Speaker, Source of Inspiration
\item
  Faithfulness - Trustworthy, Faithful Guardian
\item
  Praise - All-Praised, Praised, and Praiseworthy
\item
  Love - Beloved, Loving, Gracious
\item
  Forgiveness - All-Forgiving, Ever-Forgiving, Oft-Forgiving
\item
  Mercy - All-Merciful, Most Merciful, Most Compassionate
\item
  Generosity - All-Bountiful, Bestower, Most Generous, Most Bountiful
\item
  Sovereignty - Sovereign, Lord of Dominion, Ruler, Master
\item
  Purity - Purest, True, One
\end{enumerate}

\newpage{}

\chapter{Appendix 9: A Structural Map of the Worlds of
God}\label{appendix-9-a-structural-map-of-the-worlds-of-god}

This appendix provides a reference framework for the terms used
throughout this book when discussing the worlds of God. These terms
describe relationships, functions, and stations of existence rather than
physical locations or spatial realms. The purpose of this appendix is
clarity and consistency, allowing readers to orient themselves without
interrupting the narrative flow of the main chapters.

\section{Worlds of God}\label{worlds-of-god}

The worlds of God are broad realms of existence beyond the material
world. They represent levels of reality and meaning through which the
soul journeys. Although not necessarily perceptible in a physical or
spatial sense, these worlds are real and serve as the context for
spiritual growth, transformation, and continuation beyond earthly life.

\section{Kingdoms}\label{kingdoms}

Kingdoms are domains within the worlds of God associated with divine
attributes. A kingdom may reflect an attribute such as Justice, Mercy,
Knowledge, Power, or Command. Engagement with a kingdom shapes how an
attribute is understood, experienced, and expressed by the soul as it
matures.

\section{Cities}\label{cities}

Cities are shared spaces of meaning or experience within a kingdom. A
city represents a collective context in which souls encounter similar
conditions, lessons, or spiritual realities. Cities emphasize the
relational and communal dimensions of spiritual existence rather than
isolation or individualism.

\section{Schools}\label{schools}

Schools are contexts of learning and maturation within a city. They
represent processes through which the soul acquires understanding,
refinement, discipline, or detachment. Progress through schools is not
uniform and depends on the soul's capacity, receptivity, and spiritual
condition.

\section{The Celestial Concourse}\label{the-celestial-concourse}

The Celestial Concourse refers to the gathering or communion of exalted
souls and divine realities. It reflects unity, harmony, and cooperation
in service to the will of God. The Concourse is not confined to a single
world or kingdom but participates across realms of existence.

\section{Angels}\label{angels}

Angels are beings or realities associated with divine action, guidance,
and the execution of God's will. They may be understood as forces,
perfected souls, or expressions of divine attributes in action. Angels
are not necessarily bound to anthropomorphic form or physical
limitation.

\section{Sidrat al-Muntahá (The
Lote-Tree)}\label{sidrat-al-muntahuxe1-the-lote-tree}

The Sidrat al-Muntahá represents the boundary at which created knowledge
reaches its limit. It marks the meeting point between the created worlds
and divine Revelation. Beyond this horizon lies that which cannot be
fully grasped or comprehended by created beings.

\section{The Mother Book}\label{the-mother-book}

The Mother Book is the source from which divine knowledge and Revelation
proceed. It represents divine order, coherence, and originating wisdom.
Access to the Mother Book is mediated through Revelation rather than
independent human inquiry.

\section{The Holy Spirit and the Holy
Maiden}\label{the-holy-spirit-and-the-holy-maiden}

The Holy Spirit and the Holy Maiden are expressions of divine
inspiration, communication, and life-giving influence. They serve as
intermediaries through which Revelation, guidance, and confirmation are
conveyed to creation. These realities emphasize God's active and
sustaining presence within the worlds of God.

\chapter{Appendix 2: Spiritual
Practices}\label{appendix-2-spiritual-practices}

Here is a list of spiritual practices expressed in the Kitab-i-Aqdas:

\begin{itemize}
\item
  Prayer

  \begin{itemize}
  \item
    Daily Obligatory Prayer
  \item
    Prayer of the Signs
  \item
    Prayer for the Dead
  \item
    Personal Prayer (In Private)
  \end{itemize}
\item
  Recitation (of the verses of God)
\item
  Remembrance (Dhikr)

  \begin{itemize}
  \item
    Subconscious Remembrance
  \item
    Conscious Remembrance
  \end{itemize}
\item
  Reflection
\item
  Honoring God

  \begin{itemize}
  \item
    Building and Using Mashriq-ul-Adhkars
  \item
    Pilgrimage (Baghdad or Shiraz)
  \item
    Rights of God (Ḥuqúqu'lláh)
  \item
    Engaging in an Occupation
  \item
    Zakat
  \item
    Ayyam-i-Ha (Days of Giving)
  \item
    Fasting
  \item
    Monthly Hospitality
  \item
    Festival of Naw-Ruz
  \item
    Festival of Ridvan
  \item
    Festival for the Declaration of the Bab
  \item
    Festival for the Twin Birthdays
  \end{itemize}
\end{itemize}

\newpage{}

\chapter{Appendix 3: Virtues}\label{appendix-3-virtues}

Here is a list of the virtues identified in the Kitab-i-Aqdas and used
for the Constellation of Virtues called Unity:

\section{Foundational Virtues}\label{foundational-virtues-1}

\begin{enumerate}
\def\labelenumi{\arabic{enumi}.}
\item
  Fear of God
\item
  Love of God
\item
  Moderation
\end{enumerate}

\section{Innate Virtues}\label{innate-virtues}

\begin{enumerate}
\def\labelenumi{\arabic{enumi}.}
\item
  Piety
\item
  Pure Truthfulness
\item
  Courtesy
\item
  Loyalty
\item
  Trustworthiness
\end{enumerate}

\section{Emergent Virtues}\label{emergent-virtues}

\begin{enumerate}
\def\labelenumi{\arabic{enumi}.}
\item
  Emerging from Piety

  \begin{enumerate}
  \def\labelenumii{\arabic{enumii}.}
  \item
    Detachment
  \item
    Humility
  \item
    Lowly
  \item
    Reverence
  \item
    Thankfulness
  \end{enumerate}
\item
  Emerging from Pure Truthfulness

  \begin{enumerate}
  \def\labelenumii{\arabic{enumii}.}
  \item
    Eloquence
  \item
    Heedfulness
  \item
    Perception
  \item
    Reason
  \item
    Sincerity
  \item
    Wisdom
  \end{enumerate}
\item
  Emerging from Courtesy

  \begin{enumerate}
  \def\labelenumii{\arabic{enumii}.}
  \item
    Dignity
  \item
    Fairness
  \item
    Kindness
  \item
    Purity
  \item
    Radiance
  \item
    Refinement
  \end{enumerate}
\item
  Emerging from Loyalty

  \begin{enumerate}
  \def\labelenumii{\arabic{enumii}.}
  \item
    Fidelity
  \item
    Moral
  \item
    Righteousness
  \item
    Servitude
  \item
    Steadfast
  \end{enumerate}
\item
  Emerging from Trustworthiness

  \begin{enumerate}
  \def\labelenumii{\arabic{enumii}.}
  \item
    Justice
  \item
    Mindfulness
  \item
    Patience
  \item
    Repentant
  \item
    Submissive
  \end{enumerate}

  \newpage{}
\end{enumerate}

\chapter{Appendix 4: The Bayanic
Mithqal}\label{appendix-4-the-bayanic-mithqal}

\subsection{Bayánic Mithqál}\label{bayuxe1nic-mithquxe1l}

The Bayánic Mithqál is a unit of weight defined in the \emph{Persian
Bayán} (Vahid 5, Gate 19) as the measure of nineteen grains (nakhuds).

\begin{itemize}
\item
  1 nakhud ≈ 0.195 grams (Qajar-era standard)\footnote{Marcinkowski, M.
    Ismail, Measures and weights in the Islamic world : an English
    translation of Walther Hinz's hanbook ``Islamsche Masse und
    Gewichte'' (2003)

    \url{https://www.amazon.com/Measures-weights-Islamic-world-translation/dp/B0BB8CMY26}}
\item
  1 Bayánic Mithqál ≈ 3.705 grams
\end{itemize}

Conversions

\begin{itemize}
\item
  1 Bayánic Mithqál ≈ 0.119 troy ounces
\item
  9 Bayánic Mithqáls ≈ 33.345 grams ≈ 1.072 troy ounces
\item
  19 Bayánic Mithqáls ≈ 70.395 grams ≈ 2.263 troy ounces
\item
  95 Bayánic Mithqáls ≈ 352.975 grams ≈ 11.348 troy ounces
\item
  100 Bayánic Mithqáls ≈ 370.500 grams ≈ 11.909 troy ounces
\end{itemize}

\newpage{}

\chapter{Appendix 5: Letters and
Meanings}\label{appendix-5-letters-and-meanings}

The following is a list of Arabic letters, their associated numerical
values (based on the abjad system where utilized in the sources), and
the primary symbolism ascribed to them:

\begin{longtable}[]{@{}
  >{\raggedright\arraybackslash}p{(\columnwidth - 4\tabcolsep) * \real{0.2500}}
  >{\raggedright\arraybackslash}p{(\columnwidth - 4\tabcolsep) * \real{0.3750}}
  >{\raggedright\arraybackslash}p{(\columnwidth - 4\tabcolsep) * \real{0.3750}}@{}}
\toprule\noalign{}
\begin{minipage}[b]{\linewidth}\raggedright
Letter (Arabic)
\end{minipage} & \begin{minipage}[b]{\linewidth}\raggedright
Abjad Value (Standard/Contextual)
\end{minipage} & \begin{minipage}[b]{\linewidth}\raggedright
Symbolism and Significance
\end{minipage} \\
\midrule\noalign{}
\endhead
\bottomrule\noalign{}
\endlastfoot
\textbf{Alif} & 1 (Implicit in sequence/One) & Represents the Divine
Identity, blessings upon creation, and the letter of beginning. Symbol
of affirmation in the utterance ``but God''. The first letter that spoke
from God. Its total number of letters (A-L-F) is the number of the Most
Exalted Name. It is the origin of all contingent and existential
numbers. \\
- \emph{Hidden Alif} (Soft Alif) & - & Station of Destiny or Beauty,
reflects the countenance of 'Ali (reserved for his successor). \\
- \emph{Standing Alif} & - & Established all creation in the kingdoms of
creation and command. \\
\textbf{Ba} & 2 (Implicit) & The Throne upon which the All-Merciful is
established. First letter of the Book. The origin of all creation, as
the Point dissolved and the Point was found beneath it. The innermost
essence of Primary and Secondary Eternity, and multiplicity. \\
\textbf{Jim} & 3 (Implicit) & Derived Name: Holy (Quddús). The sum of
its similar numerical values in the Temple equals the number of
``Allah'' (36). \\
\textbf{Dal} & 4 (Implicit) & Letter of finitude and limitation in the
Muhammadan station. \\
\textbf{Ha} & 5 (Explicitly \$\textbackslash text\{Ha\}'\$) & Numerical
value is 5. Symbolizes unity in the heart and detachment in praise. The
outer form of the Temple of Man. The completion of man's creation in
five years. The spirit of the \emph{Surih}. The first line in the Temple
is the number of knowledge. \\
\textbf{Vav} & 6 (Explicitly \$\textbackslash text\{Vav\}\$) & Inner
form of the Temple of Man. Letter of Being. Symbolizes Universal
Guardianship and the spirit of the letter Ha. Its number is the number
of Truth. \\
\textbf{Za} & 7 (Implicit) & Similar Za's (in the ninth line of Temple
computation) total 84 (Aziz). Related to the Book of David (Book of
Za'). \\
\textbf{Ha (glutteral)} & 8 (Implicit) & One of the four letters of the
Hidden, Well-Guarded Name (Bearers of creation, provision, death,
life). \\
\textbf{Ta} & 9 (Implicit) & Related to the Pentateuch (Book of Ta').
Similar Ta's (in the eleventh line of Temple computation) total 108
(Haqq/Truth). \\
\textbf{Ya} & 10 (Implicit) & The final letter of the name of 'Ali.
Manifestation of God's Hand. The number of Ya is associated with the
recommended age for a woman to marry. \\
\textbf{Kaf} & 20 (Implicit) & First word of the Command (Kun/Be!). Rank
of the Primal Will. \\
\textbf{Lam} & 30 (Implicit) & Banner of Grandeur/Divine Unity. Standard
of Divine Unity. Its numerical value (30) is the number of nights God
promised Moses. The Lam of multiplicity joins the Alif of unity
(derivation of the cross). \\
\textbf{Mim} & 40 (Explicitly 40) & First letter of Will (Mashiyyat).
Completion of the appointed time/ranks (40 years). Letter of
Glory/essence of manifestation. \\
\textbf{Nun} & 50 (Implicit) & Rank of Purpose. Light of God in the
niche/Covenant. Light of creation/invention/origination/glory. \\
\textbf{Sad} & 60 (Implicit) & Glory of splendor, praise, Cloud, and
Destiny. \\
\textbf{'Ayn} & 70 (Implicit) & Inmost Reality of the Will. Completion
of the letters of the command ``Be!''. Transcendence of Divine Unity. \\
\textbf{Fa} & 80 (Implicit) & Uniqueness of unity, All-Merciful, signs,
and stations. Through its cleaving asunder, creation was wrought. \\
\textbf{Qaf} & 100 (Implicit) & Mention of Power/ocean of Oneness.
Manifestation of praise (in the Point). The Qaf mentioned in the
Glorious Qur'an. \\
\textbf{Ra} & 200 (Implicit) & Primary, Eternal Mercy. Universal
mercy/creation. \\
\textbf{Shin} & 300 (Implicit) & Associated with the inheritance portion
for brothers. \\
\textbf{Ta} & 400 (Implicit) & Soil of the graves (Husayn, his father,
the Imams, the Messenger of God). \\
\textbf{Kha} & 600 (Implicit) & Seclusion of oneness and imposition of
the separation of attributes. \\
\textbf{Dhal} & 700 (Implicit) & Pinnacle of the Throne, Paradise,
ranks, and everything named. \\
\textbf{Dad} & 800 (Implicit) & Associated with the 27th year of the
writer's age. \\
\textbf{Zha} & 900 (Implicit) & The letter that God created at the end
of the name of 'Ali. \\
\end{longtable}

\textbf{Key Numerical and Esoteric Terms:}

\begin{longtable}[]{@{}
  >{\raggedright\arraybackslash}p{(\columnwidth - 4\tabcolsep) * \real{0.3333}}
  >{\raggedright\arraybackslash}p{(\columnwidth - 4\tabcolsep) * \real{0.3333}}
  >{\raggedright\arraybackslash}p{(\columnwidth - 4\tabcolsep) * \real{0.3333}}@{}}
\toprule\noalign{}
\begin{minipage}[b]{\linewidth}\raggedright
Term
\end{minipage} & \begin{minipage}[b]{\linewidth}\raggedright
Numerical Value (if defined)
\end{minipage} & \begin{minipage}[b]{\linewidth}\raggedright
Meaning
\end{minipage} \\
\midrule\noalign{}
\endhead
\bottomrule\noalign{}
\endlastfoot
\textbf{Hayy (The Living)} & 18 & The number of the Letters of the
Living. \\
\textbf{Vahid (Unity)} & 19 & The number of the Signs of Unity,
Manifestations, and the 19 units that revolve. It is the numerical basis
for the Bayán (19 months, 19 days). \\
\textbf{Kull Shay' (All Things)} & 361 (19 x 19) & The totality of
creation, representing the number of the year. \\
\textbf{Mustaghath (One Invoked for Help)} & 2010 (Implicit) & The
ultimate number of the divine names. The number is related to the
maximum limit of names revealed. \\
\textbf{Baha (Splendor)} & 9 (Implicit) & A name of God. The number is
associated with the name of Tahirih and Tehran. The month of Baha is
singled out for Him Whom God shall make manifest. \\
\textbf{Huva (He)} & 11 (Implicit) & The number 11 is the age where
fasting becomes incumbent. The word Huva (He) is the essence of the
mystery of the Point. \\
\end{longtable}

\newpage{}

\chapter{Appendix 6: Child Development
Model}\label{appendix-6-child-development-model}

\section{From Birth to Age 19 (Bayánic Calendar
Alignment)}\label{from-birth-to-age-19-bayuxe1nic-calendar-alignment}

\begin{longtable}[]{@{}
  >{\raggedright\arraybackslash}p{(\columnwidth - 6\tabcolsep) * \real{0.1944}}
  >{\raggedright\arraybackslash}p{(\columnwidth - 6\tabcolsep) * \real{0.1944}}
  >{\raggedright\arraybackslash}p{(\columnwidth - 6\tabcolsep) * \real{0.1944}}
  >{\raggedright\arraybackslash}p{(\columnwidth - 6\tabcolsep) * \real{0.4167}}@{}}
\toprule\noalign{}
\begin{minipage}[b]{\linewidth}\raggedright
Approx. Age
\end{minipage} & \begin{minipage}[b]{\linewidth}\raggedright
Arabic Name
\end{minipage} & \begin{minipage}[b]{\linewidth}\raggedright
English Name
\end{minipage} & \begin{minipage}[b]{\linewidth}\raggedright
Developmental Theme
\end{minipage} \\
\midrule\noalign{}
\endhead
\bottomrule\noalign{}
\endlastfoot
0--1 & Bahá' (بهاء) & Splendor & The soul awakens to existence, radiant
and receptive, reflecting divine beauty. This is the first awareness of
life's light and the sacredness of being. \\
1--2 & Jalál (جلال) & Glory & The infant discovers the majesty of love
through attachment and trust, sensing protection and reverence in
caregivers, laying the foundation for awe before God. \\
2--3 & Jamál (جمال) & Beauty & Joy, play, and affection blossom. The
child learns harmony and attraction, feeling the beauty of creation and
the delight of being loved and loving. \\
3--4 & 'Aẓamat (عظمة) & Grandeur & Expanding curiosity, the child begins
to perceive vastness and order in the world. A sense of wonder and
respect for greatness takes root. \\
4--5 & Núr (نور) & Light & Awareness sharpens as understanding grows.
The child names, identifies, and learns through light---discovering
meaning and the joy of illumination and learning. \\
5--6 & Raḥmat (رحمة) & Mercy & Compassion awakens as the child begins to
care for others. Empathy, tenderness, and forgiveness become part of
emotional life and social connection. \\
6--7 & Kalimát (كلمات) & Words & Speech and comprehension deepen. The
child learns the creative power of words, recognizing truth, honesty,
and communication as spiritual acts. \\
7--8 & Kamál (كمال) & Perfection & Awareness of right and wrong matures.
The child seeks to act correctly, striving toward goodness and
self-improvement with growing self-discipline and sincerity. \\
8--9 & Asmá' (أسماء) & Names & The child discovers identity and
individuality, learning that every person reflects divine attributes.
Personal dignity and respect for others emerge. \\
9--10 & 'Izzat (عزة) & Might & Confidence and independence develop. The
child feels inner strength and begins to test limits, learning that true
might is tempered by humility. \\
10--11 & Mashíyyat (مشية) & Will & The sense of choice awakens. The
young mind begins to understand purpose, intention, and the difference
between impulse and deliberate action. \\
11--12 & 'Ilm (علم) & Knowledge & Curiosity matures into genuine
inquiry. The child learns to question, reason, and seek knowledge with
reverence, marking the dawn of moral reflection. \\
12--13 & Qudrat (قدرت) & Power & Self-confidence and courage increase.
The youth begins to act with conviction, recognizing personal strength
as a trust to be used with justice. \\
13--14 & Qawl (قول) & Speech & Expression becomes more purposeful. The
ability to articulate beliefs and values emerges, along with awareness
of how speech shapes truth and unity. \\
14--15 & Masá'il (مسائل) & Questions & The moral intellect awakens
fully. The youth questions deeply, seeking meaning and coherence,
preparing for responsibility and the trust of maturity. \\
15--16 & Sharaf (شرف) & Honor & A sense of duty and moral honor arises.
The young person strives for integrity, respect, and steadfastness in
upholding divine and personal principles. \\
16--17 & Sultán (سلطان) & Sovereignty & Personal sovereignty and
leadership appear. The youth learns to govern the self, balancing
independence with humility and care for others. \\
17--18 & Mulk (ملك) & Dominion & Wisdom consolidates as all capacities
harmonize. The person begins to act responsibly within community life,
exercising stewardship and social trust. \\
18--19 & `Alá' (علاء) & Loftiness & Spiritual adulthood is attained. The
individual embodies unity, love, and service --- living as a helper of
others and a conscious trustee of divine trust. \\
\end{longtable}

\subsection{Overview}\label{overview}

\begin{itemize}
\item
  Ages 0--11: Formation of awareness and empathy --- the world of
  Witnessing.
\item
  Ages 11--15: Moral and intellectual awakening --- the world of
  Trusteeship.
\item
  Ages 15--19: Social and spiritual integration --- the world of Helping
  and Unity.
\end{itemize}

\newpage{}

\chapter{Appendix 7: Political Leaders in
1873}\label{appendix-7-political-leaders-in-1873}

\section{Of the Americas:}\label{of-the-americas}

\begin{longtable}[]{@{}
  >{\raggedright\arraybackslash}p{(\columnwidth - 8\tabcolsep) * \real{0.2055}}
  >{\raggedright\arraybackslash}p{(\columnwidth - 8\tabcolsep) * \real{0.1644}}
  >{\raggedright\arraybackslash}p{(\columnwidth - 8\tabcolsep) * \real{0.1644}}
  >{\raggedright\arraybackslash}p{(\columnwidth - 8\tabcolsep) * \real{0.3288}}
  >{\raggedright\arraybackslash}p{(\columnwidth - 8\tabcolsep) * \real{0.1370}}@{}}
\toprule\noalign{}
\endhead
\bottomrule\noalign{}
\endlastfoot
\textbf{Nation / Power} & \textbf{Status (1873)} & \textbf{Independence
Year} & \textbf{Head of State (in 1873)} & \textbf{Title} \\
\textbf{United States} & Independent & 1776 & Ulysses S. Grant &
President \\
\textbf{Haiti} & Independent & 1804 & Michel Domingue & President \\
\textbf{Paraguay} & Independent & 1811 & Salvador Jovellanos &
President \\
\textbf{Venezuela} & Independent & 1811 & Antonio Guzmán Blanco &
President \\
\textbf{Argentina} & Independent & 1816 & Domingo Faustino Sarmiento &
President \\
\textbf{Chile} & Independent & 1818 & Federico Errázuriz Zañartu &
President \\
\textbf{Colombia} & Independent & 1819 & Manuel Murillo Toro &
President \\
\textbf{Mexico} & Independent & 1821 & Sebastián Lerdo de Tejada &
President \\
\textbf{Peru} & Independent & 1821 & Manuel Pardo y Lavalle &
President \\
\textbf{Guatemala} & Independent & 1821 & Justo Rufino Barrios &
President \\
\textbf{El Salvador} & Independent & 1821 & Santiago González &
President \\
\textbf{Honduras} & Independent & 1821 & Carlos Céleo Arias &
President \\
\textbf{Nicaragua} & Independent & 1821 & José Diriangén Dávila &
President \\
\textbf{Costa Rica} & Independent & 1821 & Tomás Guardia Gutiérrez &
President \\
\textbf{Brazil} & Independent & 1822 & Pedro II & Emperor \\
\textbf{Ecuador} & Independent & 1822 & Gabriel García Moreno &
President \\
\textbf{Bolivia} & Independent & 1825 & Adolfo Ballivián & President \\
\textbf{Uruguay} & Independent & 1825 & Tomás Gomensoro & Interim
President \\
\textbf{Dominican Republic} & Independent & 1844 & Ignacio González &
President \\
--- & --- & --- & --- & --- \\
\textbf{United Kingdom} & Colonial Power & N/A & Queen Victoria &
Queen \\
\textbf{Spain} & Colonial Power & N/A & Emilio Castelar &
President\footnote{The Spanish political situation was highly unstable
  in 1873. King Amadeo I abdicated in February, and the First Spanish
  Republic was proclaimed. Therefore, Spain's head of state in the
  latter half of 1873 (ruling Cuba and Puerto Rico) was the President of
  the Executive Power, Emilio Castelar (who was serving at the end of
  the year).} \\
\textbf{Netherlands} & Colonial Power & N/A & William III & King \\
\textbf{France} & Colonial Power & N/A & Patrice de MacMahon &
President\footnote{France was under the Third Republic in 1873. It was
  not ruled by a monarch, but by a President, Patrice de MacMahon, who
  governed the French colonies in the Americas (e.g., French Guiana,
  Martinique).} \\
\textbf{Denmark} & Colonial Power & N/A & Christian IX & King \\
\end{longtable}

\section{Of Europe:}\label{of-europe}

\begin{longtable}[]{@{}
  >{\raggedright\arraybackslash}p{(\columnwidth - 6\tabcolsep) * \real{0.2432}}
  >{\raggedright\arraybackslash}p{(\columnwidth - 6\tabcolsep) * \real{0.2568}}
  >{\raggedright\arraybackslash}p{(\columnwidth - 6\tabcolsep) * \real{0.3108}}
  >{\raggedright\arraybackslash}p{(\columnwidth - 6\tabcolsep) * \real{0.1892}}@{}}
\toprule\noalign{}
\endhead
\bottomrule\noalign{}
\endlastfoot
Nation & Status (1873) & Head of State (in 1873) & Title \\
German Empire & Empire & Wilhelm I & Emperor (Kaiser) \\
Austria-Hungary & Dual Monarchy & Franz Joseph I & Emperor-King \\
United Kingdom & Monarchy & Queen Victoria & Queen \\
Russian Empire & Empire & Alexander II & Emperor (Tsar) \\
French Republic & Republic & Patrice de MacMahon & President \\
Ottoman Empire & Empire & Sultan Abdülaziz & Sultan/Padishah \\
Kingdom of Italy & Monarchy & Victor Emmanuel II & King \\
Spain & Republic* & Emilio Castelar & President* \\
Kingdom of Portugal & Monarchy & Luís I & King \\
Kingdom of Sweden & Monarchy & Oscar II & King \\
Kingdom of Greece & Monarchy & George I & King \\
Kingdom of Belgium & Monarchy & Leopold II & King \\
Kingdom of the Netherlands & Monarchy & William III & King \\
\end{longtable}

\section{Of Africa:}\label{of-africa}

\begin{longtable}[]{@{}
  >{\raggedright\arraybackslash}p{(\columnwidth - 6\tabcolsep) * \real{0.2500}}
  >{\raggedright\arraybackslash}p{(\columnwidth - 6\tabcolsep) * \real{0.2500}}
  >{\raggedright\arraybackslash}p{(\columnwidth - 6\tabcolsep) * \real{0.2500}}
  >{\raggedright\arraybackslash}p{(\columnwidth - 6\tabcolsep) * \real{0.2500}}@{}}
\toprule\noalign{}
\endhead
\bottomrule\noalign{}
\endlastfoot
Nation / State & Political Status (1873) & Head of State (in 1873) &
Title \\
Ethiopian Empire & Independent Empire & Yohannes IV & Emperor (Nəgusä
Nägäst) \\
Republic of Liberia & Independent Republic & Joseph Jenkins Roberts &
President \\
Egypt (Khedivate) & Autonomous State (under Ottoman suzerainty) &
Isma'il Pasha & Khedive \\
Sultanate of Morocco & Independent Sultanate & Hassan I & Sultan \\
Tunis (Beylik) & Autonomous State (under Ottoman suzerainty) & Muhammad
III as-Sadiq & Bey \\
Oman \& Zanzibar Sultanate & Independent (split from Oman in 1861) &
Sayyid Barghash bin Said & Sultan \\
Asante Empire & Independent Empire & Kofi Karikari & Asantehene \\
Merina Kingdom (Madagascar) & Independent Kingdom & Ranavalona II &
Queen \\
\end{longtable}

\section{Of Asia:}\label{of-asia}

\begin{longtable}[]{@{}
  >{\raggedright\arraybackslash}p{(\columnwidth - 6\tabcolsep) * \real{0.2500}}
  >{\raggedright\arraybackslash}p{(\columnwidth - 6\tabcolsep) * \real{0.2500}}
  >{\raggedright\arraybackslash}p{(\columnwidth - 6\tabcolsep) * \real{0.2500}}
  >{\raggedright\arraybackslash}p{(\columnwidth - 6\tabcolsep) * \real{0.2500}}@{}}
\toprule\noalign{}
\endhead
\bottomrule\noalign{}
\endlastfoot
Nation / State & Political Status (1873) & Head of State (in 1873) &
Title \\
Qing Dynasty (China) & Empire & Tongzhi Emperor & Emperor \\
Meiji Japan & Empire & Emperor Meiji (Mutsuhito) & Emperor \\
Persia (Qajar Dynasty) & Empire & Nasser al-Din Shah Qajar & Shah (King
of Kings) \\
Siam (Thailand) & Kingdom & Chulalongkorn (Rama V) & King \\
Afghanistan & Emirate & Sher Ali Khan & Amir \\
Vietnam (Nguyễn Dynasty) & Empire (under increasing French pressure) &
Tự Đức & Emperor \\
British India & Colonial (Crown Rule established 1858) & Queen Victoria
& Empress of India (declared 1876, but ruling via Viceroy) \\
Dutch East Indies (Indonesia) & Colony & William III (King of
Netherlands) & King/Sovereign Power \\
Korean Empire (Joseon) & Kingdom (Chinese vassal, but effectively
independent) & Gojong & King \\
Khivan Khanate & Khanate (Conquered by Russia in 1873) & Muhammad Rahim
Khan II & Khan \\
\end{longtable}

\newpage{}

\chapter{Appendix 8: Trusteeship Levels and
Roles}\label{appendix-8-trusteeship-levels-and-roles}

\begin{longtable}[]{@{}
  >{\raggedright\arraybackslash}p{(\columnwidth - 10\tabcolsep) * \real{0.1667}}
  >{\raggedright\arraybackslash}p{(\columnwidth - 10\tabcolsep) * \real{0.1667}}
  >{\raggedright\arraybackslash}p{(\columnwidth - 10\tabcolsep) * \real{0.1667}}
  >{\raggedright\arraybackslash}p{(\columnwidth - 10\tabcolsep) * \real{0.1667}}
  >{\raggedright\arraybackslash}p{(\columnwidth - 10\tabcolsep) * \real{0.1667}}
  >{\raggedright\arraybackslash}p{(\columnwidth - 10\tabcolsep) * \real{0.1667}}@{}}
\toprule\noalign{}
\endhead
\bottomrule\noalign{}
\endlastfoot
Level of Authority & Objective & Functional Roles & Witness (The
Observation) & Trustee (The Responsibility) & Helper (The Action) \\
Level 1: Inspiration & Individual Integrity \& Truth & Artist,
Scientist, Friend & Identifying the ``Divine Names'' (Virtues) and laws
of reality. & Guarding the purity of truth, creativity, and personal
loyalty. & Inspiring radiance and ``Independent Investigation'' in
others. \\
Level 2: Education & Moral \& Spiritual Upbringing & Parent, Teacher,
Scholar, Religious Leader & Noticing the specific ``Milk, Fruit, or
Meat'' required for a soul's growth. & Acting as a ``Spiritual
Physician'' for those in one's care. & Healing trauma and nurturing
``God-consciousness'' through wisdom. \\
Level 3: Honor & Human Rights \& Social Identity & Social Work
Organizer, NGO Leader, Lawyer & Recognizing where the ``Right to
Identity'' or ``Honor'' is being oppressed. & Protecting the sacred
boundaries of the soul and the property of others. & Intervening against
``Slander'' and advocating for the vulnerable. \\
Level 4: Welfare & Provision \& Community Support & Business
Owner/Manager, Wealth Manager & Viewing wealth and resources not as
personal status, but as public trust. & Managing Endowments and capital
for the ``Salvation of Nations.'' & Generating ethical prosperity and
funding ``Elevated Places'' of service. \\
Level 5: Governance & Global \& Local Peace & Politician, Legislator,
Judge, Monarch, Diplomat, General & Discerning ``Common Affairs'' and
the collective security of the people. & Exercising authority only via
Consent and ``Consultation.'' & Establishing the Lesser Peace and
protecting the ``Rights of All.'' \\
\end{longtable}

\newpage{}


\backmatter

\end{document}
